\documentclass{article}

\usepackage[margin=1in]{geometry}
\usepackage{amsmath,amsthm,amssymb}
\usepackage{bbm, enumerate}

\newenvironment{problem}[2][Problem]{\begin{trivlist}
\item[\hskip \labelsep {\bfseries #1}\hskip \labelsep {\bfseries #2.}]}{\end{trivlist}}
\newenvironment{note}[1][Note.]{\begin{trivlist}
\item[\hskip \labelsep {\bfseries #1}]}{\end{trivlist}}

\begin{document}

\title{Complex Analysis: Homework 1}
\author{Peter Kagey}

\maketitle

% -----------------------------------------------------
% First problem
% -----------------------------------------------------
\begin{problem}{2} (page 2) \\
  If $z = x + iy$ ($x$ and $y$ real), find the real and imaginary parts of \[
    z^4, \hspace{30pt}
    \frac{1}{z}, \hspace{30pt}
    \frac{z - 1}{z + 1}, \hspace{30pt}
    \frac{1}{z^2}.
  \]
\end{problem}

\begin{proof}
  \begin{enumerate}[(a)]
    \item
    \begin{align*}
      z^4 &= ((x + iy)^2)^2 \\
      &= (x^2 - y^2 - 2xyi)^2 \\
      &= (x^2 - y^2)^2 - (2xy)^2 - 2(x^2 - y^2)(2xy)i \\
      &= x^4 - 6x^2y^2 + y^4 + 4xy(x^2-y^2)i
    \end{align*}
    Therefore the real and imaginary parts are \[
      \operatorname{Re}(z^4) = x^4 - 6x^2y^2 + y^4, \text { and }
      \operatorname{Im}(z^4) = 4xy(x^2-y^2).
    \]

    \item
    \begin{align*}
      \frac{1}{z}
      &= \frac{1}{x + iy} \\
      &= \frac{1}{x + iy} \cdot \frac{x - iy}{x - iy} \\
      &= \frac{x - iy}{x^2 + y^2} \\
      &= \frac{\bar{z}}{|z|^2}
    \end{align*}
    Therefore the real and imaginary parts are \[
      \operatorname{Re}\left(\frac{1}{z}\right)
        = x/(x^2 + y^2), \text { and }
      \operatorname{Im}\left(\frac{1}{z}\right)
        = -y/(x^2 + y^2).
    \]

    \item
    \begin{align*}
      \frac{z - 1}{z + 1}
      &= \frac{x - 1 + iy}{x + 1 + iy} \\
      &= \frac{x - 1 + iy}{x + 1 + iy} \cdot \frac{x + 1 - iy}{x + 1 - iy} \\
      &= \frac{x^2 + y^2 - 1+ 2yi}{(x + 1)^2 + y^2}
    \end{align*}
    Therefore the real and imaginary parts are \[
      \operatorname{Re}\left(\frac{z - 1}{z + 1}\right) =
        \frac{x^2 + y^2 - 1}{(x + 1)^2 + y^2}, \text { and }
      \operatorname{Im}\left(\frac{z - 1}{z + 1}\right) =
        \frac{2y}{(x + 1)^2 + y^2}.
    \]

    \item
    \begin{align*}
      \frac{1}{z^2}
      &= \left(\frac{x - iy}{x^2 + y^2}\right)^2 \\
      &= \frac{x^2 - y^2 -2xyi}{x^4 + 2x^2y^2 + y^4}
    \end{align*}
    Therefore the real and imaginary parts are \[
      \operatorname{Re}\left(\frac{1}{z^2}\right) =
        \frac{x^2 - y^2}{x^4 + 2x^2y^2 + y^4}, \text { and }
      \operatorname{Im}\left(\frac{1}{z^2}\right) =
      \frac{-2xy}{x^4 + 2x^2y^2 + y^4}.
    \]
  \end{enumerate}
\end{proof}

% -----------------------------------------------------
% Second problem
% -----------------------------------------------------
\pagebreak

\begin{problem}{1} (page 6) \\
  Show that the system of all matrices of the special form \[
    \begin{bmatrix}
      \alpha & \beta \\
      -\beta  & \alpha
    \end{bmatrix}
  \] combined by matrix addition and matrix multiplication, is isomorphic to the
  field of complex numbers.
\end{problem}

\begin{proof}
  Define the mapping $\phi$ by \[
    \phi\left(
      \begin{bmatrix}
        \alpha & \beta \\
        -\beta  & \alpha
      \end{bmatrix}
    \right) = \alpha + \beta i.
  \] Then $\phi$ a homomorphism with respect to matrix/complex addition.
  \begin{align*}
    \phi\left(\begin{bmatrix}
      \alpha_0 & \beta_0 \\
      -\beta_0  & \alpha_0
    \end{bmatrix} + \begin{bmatrix}
      \alpha_1 & \beta_1 \\
      -\beta_1  & \alpha_1
    \end{bmatrix}\right)
    &= \phi\left(\begin{bmatrix}
      \alpha_0 + \alpha_1 & \beta_0 + \beta_1 \\
      -(\beta_0 + \beta_1)  & \alpha_0 + \alpha_1
    \end{bmatrix}\right)\\
    &= \alpha_0 + \alpha_1 + (\beta_0 + \beta_1)i\\
    &= \alpha_0 + \beta_0i + \alpha_1 + \beta_1i\\
    &= \phi\left(\begin{bmatrix}
      \alpha_0 & \beta_0 \\
      -\beta_0  & \alpha_0
    \end{bmatrix}\right) + \phi\left(\begin{bmatrix}
      \alpha_1 & \beta_1 \\
      -\beta_1  & \alpha_1
    \end{bmatrix}\right).
  \end{align*} And similarly $\phi$ is a homomorphism with respect to
  matrix/complex multiplication. \begin{align*}
    \phi\left(\begin{bmatrix}
      \alpha_0 & \beta_0 \\
      -\beta_0  & \alpha_0
    \end{bmatrix} \begin{bmatrix}
      \alpha_1 & \beta_1 \\
      -\beta_1  & \alpha_1
    \end{bmatrix}\right)
    &= \phi\left(\begin{bmatrix}
      \alpha_0\alpha_1-\beta_0\beta_1 & \alpha_0\beta_1 + \beta_0\alpha_1 \\
      -(\alpha_0\beta_1 + \beta_0\alpha_1)  & \alpha_0\alpha_1-\beta_0\beta_1
    \end{bmatrix}\right)\\
    &= \alpha_0\alpha_1 - \beta_0\beta_1 + (\alpha_0\beta_1 + \beta_0\alpha_1)i\\
    &= (\alpha_0 + \beta_0i)(\alpha_1 + \beta_1i)\\
    &= \phi\left(\begin{bmatrix}
      \alpha_0 & \beta_0 \\
      -\beta_0  & \alpha_0
    \end{bmatrix}\right)\phi\left(\begin{bmatrix}
      \alpha_1 & \beta_1 \\
      -\beta_1  & \alpha_1
    \end{bmatrix}\right).
  \end{align*}
  Lastly, $\phi$ is clearly a bijection with \[
    \phi^{-1}(\alpha + \beta i) = \begin{bmatrix}
      \alpha & \beta \\
      -\beta  & \alpha
    \end{bmatrix}.
  \]
\end{proof}

% -----------------------------------------------------
% Third problem
% -----------------------------------------------------
\pagebreak

\begin{problem}{3} (page 8) \\
  Prove that \[
    \left|\frac{a - b}{1 - \bar{a}b}\right| = 1
  \] if either $|a| = 1$ or $|b| = 1$.
  What exception must be made if $|a| = |b| = 1$?
\end{problem}

\begin{proof}
  Because of the identity \[
    \left|\frac{a - b}{1 - \bar{a}b}\right| =
    \frac{|a - b|}{|1 - \bar{a}b|},
  \] it is sufficient to show that $|a - b| = |1 - \bar{a}b|$.\\
  \begin{enumerate}[\text{Case} 1:]
    \item Assume that $|a| = 1$. \[
      |1 - \bar{a}b|
      = \left|(1 - \bar{a}b)\cdot \frac{a}{a}\right|
      = \left|\frac{a - |a|^2b}{a}\right|
      = \left|\frac{a - b}{a}\right|
      = \frac{|a - b|}{|a|}
      = |a - b|.
    \]
    \item Assume that $|b| = 1$ (and so $|\bar{b}| = 1$.) \[
      |1 - \bar{a}b|
      = \left|(1 - \bar{a}b)\cdot \frac{\bar{b}}{\bar{b}}\right|
      = \left|\frac{\bar{b} - \bar{a}|b|^2}{\bar{b}}\right|
      = \left|\frac{\bar{a} - \bar{b}}{\bar{b}}\right|
      = \frac{|\bar{a} - \bar{b}|}{|\bar{b}|}
      = |\bar{a} - \bar{b}|
      = |\overline{a-b}|
      = |a - b|.
    \]
  \end{enumerate}

  Notice that if $\bar{a}b = 1$ (and thus $a = b$), then the quotient is not well-defined.
\end{proof}

% -----------------------------------------------------
% Fourth problem
% -----------------------------------------------------
\pagebreak

\begin{problem}{4} (page 8) \\
  Find the conditions under which the equation $az + b\bar{z} + c = 0$ in one
  complex unknown has exactly one solution, and compute that solution.
\end{problem}

\begin{proof}
  Assuming that $a, b, c \in \mathbb{R}$. Denote $z$ by $\alpha + \beta i$ with
  $\alpha, \beta \in \mathbb{R}$.\\
  Then \begin{align*}
    az + b\bar{z} + c
    &= a(\alpha + \beta i) + b(\alpha - \beta i) + c \\
    &= \alpha(a + b) + \beta i(a - b) + c \\
    &= 0
  \end{align*}
  So considering the real and imaginary parts separately \begin{align*}
  &\operatorname{Im}(\alpha(a + b) + \beta i(a - b) + c) &&= \beta(a - b)      &&= 0\\
  &\operatorname{Re}(\alpha(a + b) + \beta i(a - b) + c) &&= \alpha(a + b) + c &&= 0.
  \end{align*}
  In order for the imaginary part to vanish, either $a = b$ or $\beta = 0$.
  However if $a = b$ then $\beta$ can take on any value, so the equation has
  infinitely many solutions. Thus $\beta = 0$, $\alpha = -c/(a + b)$, and $z = \bar{z} = -c/(a + b)$.
\end{proof}

% -----------------------------------------------------
% Fifth
% -----------------------------------------------------
\pagebreak

\begin{problem}{1} (page 11) \\
  Prove that \[
    \left|\frac{a - b}{1 - \bar{a}b}\right| < 1
  \] if $|a| < 1$ and $|b| < 1$.
\end{problem}

\begin{proof}
  It is sufficient to show that \[
    \left|\frac{a - b}{1 - \bar{a}b}\right|^2
    = \frac{(a - b)\overline{(a - b)}}{(1 - \bar{a}b)(1 - a\bar{b})}
    = \frac{|a|^2 + |b|^2 - a\bar{b} - \bar{a}b}{1 + |a\bar{b}|^2 - a\bar{b} - \bar{a}b}
    = \frac{|a|^2 + |b|^2 - 2\operatorname{Re}(a\bar{b})}{1 + |a\bar{b}|^2 - 2\operatorname{Re}(a\bar{b})}
    < 1.
  \]
  The denominator is nonzero because $|\bar{a}b| < 1$,
  so $(1 - \bar{a}b)^2 \not= 0$.\\ \\
  Since $|a| < 1$ and $|b| < 1$, it follows that $1 - |a|^2 > 0$  and $1 - |b|^2 > 0$. Thus
  \begin{align*}
    0 < (1 - |a|^2)(1 - |b|^2) &= 1 - |a|^2 - |b|^2 + |ab|^2, \text { so} \\
    |a|^2 + |b|^2 &< 1 + |ab|^2.
  \end{align*} Therefore \[
    \left|\frac{a - b}{1 - \bar{a}b}\right|^2 < 1
  \] and the result follows.
\end{proof}

% -----------------------------------------------------
% Sixth
% -----------------------------------------------------
\pagebreak

\begin{problem}{1} (page 17) \\
  When does $az + b\bar{z} + c = 0$ represent a line?
\end{problem}

\begin{proof}
  As in problem 4 on page 8, let $z = \alpha + \beta i$ with
  $\alpha, \beta \in \mathbb{R}$, and as shown previously: \begin{align*}
    &\operatorname{Im}(\alpha(a + b) + \beta i(a - b) + c) &&= \beta(a - b)      &&= 0\\
    &\operatorname{Re}(\alpha(a + b) + \beta i(a - b) + c) &&= \alpha(a + b) + c &&= 0.
  \end{align*}

  In order to satisfy the imaginary part either $\beta = 0$ or $a = b$ .
  If $\beta = 0$ then $\alpha$ is determined by the equation: $\alpha = -c/(a + b)$.
  If $a = b$ then $\alpha = -c/(2a)$ and $\beta$ can take on any value, and so
  the equation describes a line.
\end{proof}

% -----------------------------------------------------
% Seventh
% -----------------------------------------------------
\pagebreak

\begin{problem}{1} (page 20) \\
  Show that $z$ and $z'$ correspond to diametrically opposite points on the
  Riemann sphere if and only if $z\bar{z}' = -1$.
\end{problem}

\begin{proof}
  ($\Longrightarrow$) Assume that $z$ and $z'$ correspond to diametrically opposite points on the Riemann sphere.\\
  Then \[
    z = \frac{x_1 + ix_2}{1 - x_3},
    z' = \frac{-x_1 - ix_2}{1 + x_3}, \text{ and } \\
    z\bar{z}' = \frac{x_1 + ix_2}{1 - x_3} \cdot \frac{-x_1 + ix_2}{1 + x_3} =
    \frac{x_1^2 + x_2^2}{x_3^2 - 1}.
  \] because $(x_1, x_2, x_3)$ is on the unit sphere \[
    x_1^2 + x_2^2 = -1(x_3^2 - 1) \Longrightarrow \frac{x_1^2 + x_2^2}{x_3^2 - 1} = -1.
  \]
  \\
  ($\Longleftarrow$) Assume that $z\bar{z}' = -1$.
  Then denote \[
    z = \frac{x_1 + ix_2}{1 - x_3} \text{ and } \bar{z}' = \frac{\alpha_1 - i\alpha_2}{1 - \alpha_3}.
  \] Then using the identity $z = 1/\bar{z}'$ \[
    |z|^2
    = \frac{1 + x_3}{1 - x_3}
    = \frac{1}{|\bar{z}'|}
    = \frac{1 - \alpha_3}{1 + \alpha_3}.
  \] and solving for $x_3$ \begin{align*}
    (1 + x_3)(1 + \alpha_3) &= (1 - x_3)(1 - \alpha_3) \\
    1 + x_3 + \alpha_3 + x_3\alpha_3 &= 1 - x_3 - \alpha_3 + x_3\alpha_3 \\
    2(x_3 + \alpha_3) &= 0 \\
    x_3 &= -\alpha_3.
  \end{align*}
  Then \begin{align*}
    z\bar{z}' & = -1 \\
    (x_1 + ix_2)(\alpha_1 - i\alpha_2) &= -(1 - x_3)(1 - \alpha_3) \\
    \operatorname{Im}((x_1 + ix_2)(\alpha_1 - i\alpha_2)) &= -\operatorname{Im}((1 - x_3)(1 - \alpha_3)) \\
    \alpha_1x_2 - x_1\alpha_2 &= 0 \\
    \frac{x_1}{\alpha_1} &= \frac{x_2}{\alpha_2}.
  \end{align*}
  So combining the above fact that $x_3 = -\alpha_3$ \begin{align*}
    x_1^2 + x_2^2 + x_3^2 &= \alpha_1^2 + \alpha_2^2 + \alpha_3^2 = 1\\
    x_1^2 + x_2^2 &= \alpha_1^2 + \alpha_2^2 \\
    \frac{1}{x_1^2}(x_1^2 + x_2^2)
      = 1 + \left(\frac{x_2}{x_1}\right)^2
      &= 1 + \left(\frac{\alpha_2}{\alpha_1}\right)^2
      = \frac{1}{\alpha_1^2}(\alpha_1^2 + \alpha_2^2) \\
    \frac{1}{x_1^2} &= \frac{1}{\alpha_1^2} \\
    x_1 &= \alpha_1 \text { or } -\alpha_1.
  \end{align*} Similarly \begin{align*}
  \frac{1}{x_2^2}(x_1^2 + x_2^2)
    = \left(\frac{x_1}{x_2}\right)^2 + 1
    &= \left(\frac{\alpha_1}{\alpha_2}\right)^2 + 1
    = \frac{1}{\alpha_2^2}(\alpha_1^2 + \alpha_2^2) \\
    \frac{1}{x_2^2} &= \frac{1}{\alpha_2^2} \\
    x_2 &= \alpha_2 \text{ or } -\alpha_2.
  \end{align*}
  By the identity $\alpha_1 x_2 = x_1 \alpha_2$,
  either $\alpha_1 = -x_1$ and $\alpha_2 = -x_2$,
  or $\alpha_1 = x_1$ and $\alpha_2 = x_2$.\\
  Since $x_3 \leq 1$, \[
    x_1\alpha_1 + x_2\alpha_2 = x_3^2 - 1 \leq 0
  \] and so $\alpha_1 = -x_1$, $\alpha_2 = -x_2$. Therefore $z$ and $z'$
  correspond to antipodal points on the Riemann sphere.
\end{proof}

% -----------------------------------------------------
% Eighth
% -----------------------------------------------------
\pagebreak

\begin{problem}{3} (page 28) \\
  Find the most general harmonic polynomial of the form
  $ax^3 + bx^2y + cxy^2 + dy^3$.
  Determine the conjugate harmonic function and the corresponding analytic
  function by integration and by the formal method.
\end{problem}

\begin{proof}
  I'm assuming that $z = x + yi$ with $x, y \in \mathbb{R}$,
  so the function is real valued. Thus \[
    f(x + yi) = ax^3 + bx^2y + cxy^2 + dy^3 = u(x + yi) + iv(x + yi)
  \] where $u = f$ and $v = 0$. Taking the derivative: \begin{align*}
    u_{xx} &= 6ax + 2by,\\
    u_{yy} &= 2cx + 6dy,\\
    v_{xx} &= 0, \\
    v_{yy} &= 0.
  \end{align*}
  Because $f$ is harmonic for all $x, y$, \begin{align*}
    6ax + 2by + 2cx + 6dy &= 0 \\
    (6a + 2c)x + (2b + 6d)y &= 0.
  \end{align*} Thus $c = -3a$ and $b = -3d$. So the most general form of $f$ is \[
    f(x + yi) = u(x + yi) = ax^3 -3dx^2y -3axy^2 + dy^3.
  \]\\
  To determine the conjugate harmonic function via integration \begin{align*}
    &\frac{\partial u}{\partial x} &&= 3ax^2 -6dxy - 3ay^2 &&= \frac{\partial v}{\partial y}\\
    &\frac{\partial u}{\partial y} &&= -3dx^2 - 6axy + 3dy^2 &&= -\frac{\partial v}{\partial x}
  \end{align*} so \begin{align*}
    v(x + yi) &= \int 3ax^2 -6dxy - 3ay^2\ \partial y \\
    &= 3ax^2y - 3dxy^2 - ay^3 + \psi(x)
  \end{align*} and similarly \begin{align*}
    v(x + yi) &= \int 3dx^2 + 6axy - 3dy^2\ \partial x \\
    &= dx^3 + 3ax^2y - 3dxy^2 + \xi(y).
  \end{align*} Thus $v(x + yi) = dx^3 + 3ax^2y - 3dxy^2 + ay^3$,
  and the corresponding analytic function $g$ is \[
    g(x + iy) = (ax^3 -3dx^2y -3axy^2 + dy^3) + (dx^3 + 3ax^2y - 3dxy^2 + ay^3)i.
  \]\\

  Using the formal method, \begin{align*}
    g(z) &= 2u(z/2, z/2i) - u(0, 0) \\
    &= 2\left(
      a\frac{z^3}{8}
      - 3d\frac{z^2}{4}\cdot\frac{z}{2i}
      - 3a\frac{z}{2} \cdot \frac{z^2}{-4}
      + d\frac{z^3}{-8i}
    \right) \\
    &= \frac{a}{4}{z^3} - \frac{3d}{4i}z^3 +\frac{3a}{4}z^3 - \frac{d}{4i}z^3 \\
    &= z^3\left(
      \frac{ai -3d + 3ai -d}{4i}
    \right)
    = z^3\left(
      \frac{ai -d}{i}
    \right)
    =z^3(a + di)
  \end{align*}

\end{proof}

% -----------------------------------------------------
% Ninth
% -----------------------------------------------------
\pagebreak

\begin{problem}{4} (page 28) \\
  Show that an analytic function cannot have a constant absolute value without
  reducing to a constant.
\end{problem}

\begin{proof}
  If $|f(z)| = 0$ then $f(z) = 0$, which is analytic, so suppose that $|f(z)| = c > 0$.
  Then \[
    \frac{1}{f(z)} = \frac{\overline{f(z)}}{|z|^2} = \frac{\overline{f(z)}}{c^2}.
  \] Clearly $1$ is analytic, and $f(z) \not= 0$ so $\frac{1}{f(z)}$ is analytic too.

  However, since $\overline{f(z)}$ is analytic, it has derivative $0$ with
  respect to $z$. therefore $\overline{f(z)}$ (and thus $f(z)$) is a constant.
\end{proof}

% -----------------------------------------------------
% Tenth
% -----------------------------------------------------
\pagebreak

\begin{problem}{5} (page 28) \\
  Prove rigorously that the functions $f(z)$ and $\overline{f(\bar{z})}$ are simultaneously
  analytic.
\end{problem}

\begin{proof}
  It is sufficient to show that $f(z)$ being analytic implies $\overline{f(\bar{z})}$
  is analytic because $\overline{\overline{f(\bar{\bar{z}})}} = f(z)$.\\
  From section 1.2, if $f(z) = u(z) + iv(z)$ is analytic,
  then $u(x, y)$ and $v(x, y)$ have partial derivatives
  which satisfy the Cauchy-Riemann equations, and the converse is also true.
  Thus it is sufficient to show that $u(x, -y)$ and $-v(x, -y)$ have partial
  derivatives which satisfy the Cauchy-Riemann equations. \begin{align*}
    \frac{\partial}{\partial x} u(x, -y) &= u_x \\
    \frac{\partial}{\partial y} u(x, -y) &= -u_y \\
    \frac{\partial}{\partial x} [-v(x, -y)] &= -v_x \\
    \frac{\partial}{\partial y} [-v(x, -y)] &= -(-v_y) = v_y
  \end{align*}
  Because $f$ is analytic \begin{align*}
    \frac{\partial}{\partial x} u(x, -y) = u_x &= v_y = \frac{\partial}{\partial y} [-v(x, -y)] \\
    -\frac{\partial}{\partial y} u(x, -y) = u_y &= -v_x = \frac{\partial}{\partial x} [-v(x, -y)]
  \end{align*}
  So $f(z)$ is analytic if and only if $\overline{f(\bar{z})}$ is analytic.

\end{proof}


\end{document}
