\documentclass{article}

\usepackage[margin=1in]{geometry}
\usepackage{amsmath,amsthm,amssymb}
\usepackage{bbm, enumerate}

\newenvironment{problem}[2][Problem]{\begin{trivlist}
\item[\hskip \labelsep {\bfseries #1}\hskip \labelsep {\bfseries #2.}]}{\end{trivlist}}
\newenvironment{note}[1][Note.]{\begin{trivlist}
\item[\hskip \labelsep {\bfseries #1}]}{\end{trivlist}}

\begin{document}

\title{Complex Analysis: Homework 2}
\author{Peter Kagey}

\maketitle

% -----------------------------------------------------
% First problem
% -----------------------------------------------------
\begin{problem}{6} (page 47) \\
  Determine all values of \begin{enumerate}[(a)]
    \item $2^i$,
    \item $i^i$,
    \item $(-1)^{2i}$.
  \end{enumerate}
\end{problem}

\begin{proof} \text{} \\
  \begin{enumerate}[(a)]
    \item $2^i = \exp(i\log(2)) = \cos(log(2)) + i\sin(log(2))$,
    \item $i^i = \exp(i\log(i))
      = \exp(i\pi/2 + 2\pi i k)^i
      = \exp(i\pi/2 + 2\pi i k)^i
      = \exp(-\pi/2 - 2\pi k)
    $,
    \item $(-1)^{2i} = \exp(\pi i)^{2i}
      = \exp(\pi i + 2\pi i k)^{2i}
      = \exp(-2\pi - 4\pi k)
    $.
  \end{enumerate}
\end{proof}
% -----------------------------------------------------
% Second problem
% -----------------------------------------------------
\pagebreak

\begin{problem}{7} (page 47) \\
  Determine the real and imaginary parts of $z^z$.
\end{problem}

\begin{proof}
  Let $z = x + iy$ where $x,y\in\mathbb{R}$
  \begin{align*}
    x^x &= \exp(\log(z))^z \\
    &= \exp(\log|z| + i\arg(z))^z \\
    &= \exp(\log|z|)^z + \exp(iz\arg(z)) \\
    &= |z|^z \cdot \exp(iz\arg(z)) \\
    &= |z|^x \cdot |z|^{iy} \cdot \exp(iz\arg(z)) \\
    &= |z|^x \cdot |z|^{iy} \cdot \exp((ix-y)\arg(z)) \\
    &= |z|^x \cdot |z|^{iy} \cdot \exp(ix\arg(z)) \cdot \exp(-y\arg(z)) \\
    &= |z|^x \cdot \exp(iy\log|z|) \cdot \exp(ix\arg(z)) \cdot \exp(-y\arg(z)) \\
    &= |z|^x \cdot \exp(-y\arg(z)) \cdot \exp(i(y\log|z| + x\arg(z)))
  \end{align*}
  Therefore \begin{align*}
    \operatorname{Re}(z^z) &= |z|^x e^{-y\arg(z)} \cos(y\log|z| + x\arg(z)) \text{ and}\\
    \operatorname{Im}(z^z) &= |z|^x e^{-y\arg(z)} \sin(y\log|z| + x\arg(z)).
  \end{align*}
  (Where the argument is up to integer multiples of $2\pi$.)
\end{proof}
% -----------------------------------------------------
% Third problem
% -----------------------------------------------------
\pagebreak

\begin{problem}{1} (page 72) \\
  Give a precise defintion of a single-valued branch of
  $\sqrt{1 + z} + \sqrt{1 - z}$
  in a suitable region and prove that it is analytic.
\end{problem}

\begin{proof}
\end{proof}


% -----------------------------------------------------
% Fourth problem
% -----------------------------------------------------
\pagebreak

\begin{problem}{3} (page 72) \\
  Suppose that $f(z)$ is analytic and satisfies the condition $|f(z)^2 - 1| < 1$
  in a region $\Omega$.
  Show that either $\operatorname{Re} f(z) > 0$ or $\operatorname{Re} f(z) < 0$
  throughout $\Omega$.
\end{problem}

\begin{proof}
\end{proof}

% -----------------------------------------------------
% Fifth
% -----------------------------------------------------
\pagebreak

\begin{problem}{3} (page 78) \\
  Prove that the most general transformation which leaves the origin fixed and
  preserves all distances is either a rotation or a rotation followed by a
  reflection in the real axis.
\end{problem}

\begin{proof} \text{}\\
  Let $S$ be a linear transformation that maps $0 \mapsto 0$,
  $\infty \mapsto \infty$, and $1 \mapsto e^{i\alpha}$
  for some $\alpha \in \mathbb{R}$ (in order to preserve the unit distance
  between $S(1)$ and the origin.)\\
  \\
  Then $S^{-1}$ maps $0 \mapsto 0$, $\infty \mapsto \infty$, and
  $e^{i\alpha} \mapsto 1$, so $S^{-1}$ is described the the cross ratio \[
    S^{-1}z = (z, e^{i\alpha}, 0, \infty) =
    \frac{z}{e^{i\alpha}} = ze^{-i\alpha}.
  \] and has inverse \[
    (S^{-1})^{-1}z = \frac{e^{i\alpha}z - 0}{-0z + 1} = e^{i\alpha}z = Sz.
  \]\\
  \\
  Therefore the corresponding matrix in $GL_2(\mathbb{C})$ is the rotation matrix
  \[
     \begin{bmatrix}
       e^{i\alpha} & 0 \\
       0 & 1
     \end{bmatrix}.
  \]
  Also because both $S$ and complex conjugation preserve distance, the map
  $x \mapsto S\bar{z}$ is also a (not linear) transformation that leaves the
  origin fixed and preserves all distances.
\end{proof}
% -----------------------------------------------------
% Sixth
% -----------------------------------------------------
\pagebreak

\begin{problem}{4} (page 78) \\
  Show that any linear transformation which transforms the real axis into itself
  can be written with real coefficients.
\end{problem}

\begin{proof} \text{} \\
  Let $S$ be the linear transformation that maps the real axis into itself.
  Thus $0 \mapsto r_0$, and $1 \mapsto r_1$ with $r_0, r_1 \in \mathbb{R}$.\\
  \\
  Thus $S^{-1}z$ can be written as the cross ratio $(z, r_1, r_0, z_3)$ for some
  $z_3 \in \mathbb{C}$. Let $z \in \mathbb{R} \setminus \{r_0, r_1\}$.
  Then $(z, r_1, r_0, z_4) \in \mathbb{R}$
  (because $S^{-1}$ maps the real axis into itself.) Then by Theorem 13, $z_4$ is
  on the same line as $z, r_1$, and $r_0$, that is, $z_4 = \infty$ or
  $z_4 \in \mathbb{R}$.\\
  \\
  \textbf{Case 1.} Assume $z_4 = \infty$. Then \[
    S^{-1}z = \frac{z - r_0}{r_1 - r_0}
  \] is written with real coefficients (so $S$ is too.)\\
  \\
  \textbf{Case 2.} Assume $z_4 \in \mathbb{R}$. Then \[
    S^{-1}z = \frac{z - r_0}{z - z_4} \cdot \frac{r_1 - z_4}{r_1 - r_0} = \frac{(r_1 - z_4)z - (r_1 - z_4)r_0}{(r_1 - r_0)z - (r_1 - r_0)z_4}
  \] is written with real coefficients (so $S$ is too.)\\
  \\
\end{proof}

% -----------------------------------------------------
% Seventh
% -----------------------------------------------------
\pagebreak

\begin{problem}{3} (page 80) \\
  If the consecutive vertices $z_1, z_2, z_3, z_4$ of a quadrilateral lie on a
  circle, prove that \[
    |z_1 - z_3| \cdot |z_2 - z_4|
      = |z_1 - z_2| \cdot |z_3 - z_4|
      + |z_2 - z_3| \cdot |z_1 - z_4|
  \] and interpret the result geometrically.
\end{problem}

\begin{proof}
  By Theorem 13, the cross ratio $(z_1, z_2, z_3, z_4) = r$ for some real $r$ because the
  points lie on a circle. \[
    (z_1, z_2, z_3, z_4)
    = \frac{z_1 - z_3}{z_1 - z_4} \cdot \frac{z_2 - z_4}{z_2 - z_3}
    = \frac{z_1z_2 -z_1z_4 - z_2z_3 + z_3z_4}{z_1z_2 - z_1z_3 -z_2z_4 + z_3z_4}
    = r \in \mathbb{R}
  \] Similarly \begin{align*}
    (z_1, z_3, z_2, z_4)
    &= \frac{z_1 - z_2}{z_1 - z_4} \cdot \frac{z_3 - z_4}{z_3 - z_2} \\
    &= \frac{z_1z_3 -z_1z_4 - z_2z_3 + z_2z_4}{z_1z_2 - z_1z_3 -z_2z_4 + z_3z_4}\\
    &= \frac{(z_1z_2 -z_1z_4 - z_2z_3 + z_3z_4)-(z_1z_2 - z_1z_3 -z_2z_4 + z_3z_4)}{z_1z_2 - z_1z_3 -z_2z_4 + z_3z_4} \\
    &= (z_1, z_2, z_3, z_4) - 1
  \end{align*} Therefore \[
    (z_1 - z_3)(z_2 - z_4) = r(z_1 - z_4)(z_2 - z_3)
  \] and \[
    (z_1 - z_2)(z_3 - z_4) = (r - 1)(z_1 - z_4)(z_3 - z_2)
  \] thus \begin{align*}
    (z_1 - z_3)(z_2 - z_4) &= (z_1 - z_4)(z_2 - z_3) + (r - 1)(z_1 - z_4)(z_3 - z_2) \\
    &= (z_1 - z_4)(z_2 - z_3) + (z_1 - z_2)(z_3 - z_4)
  \end{align*}
  The geometric interpretation is that the sum of the products of the lengths of
  the opposite sides is equal to the sum of the lengths of the diagonals. (This is Ptolemy's Theorem.)
\end{proof}

% -----------------------------------------------------
% Eight
% -----------------------------------------------------
\pagebreak

\begin{problem}{4} (page 80) \\
  Show that any four distinct points can be carried by a linear transformation
  to positions $1, -1, k, -k$, where the value of $k$ depends on the points.
  How many solutions are there? How are they related?
\end{problem}

\begin{proof}
  Let $c_1, c_2, c_3, c_4 \in \mathbb{C}$ be four arbitrary (distinct points).\\
  Construct a linear map $S$ that maps
  $c_1 \mapsto 1$, $c_2 \mapsto 0$, and $c_3 \mapsto \infty$. \[
    Sz = (z, c_1, c_2, c_3) = \frac{z-c_2}{z-c_3}\cdot\frac{c_1-c_3}{c_1-c_2}
    \text{ with corresponding matrix }
    \begin{bmatrix}
      c_1 - c_3 & c_2(c_3 - c_1) \\
      c_1 - c_2 & c_3(c_2 - c_1)
    \end{bmatrix}.
  \]
  The construct another map $T$ that maps
  $1 \mapsto 1$, $-1 \mapsto 0$, and $k \mapsto \infty$. \[
    Tz = (z, c_1, c_2, c_3) = \frac{z + 1}{z-k}\cdot\frac{1-k}{2}
    \text{ with corresponding matrix }
    \begin{bmatrix}
      1 - k & 1 - k \\
      2 & -2k
    \end{bmatrix}.
  \]

  These are constructed to so that $T^{-1} \circ S$ maps $c_1 \mapsto 1$,
  $c_2 \mapsto -1$, and $c_3 \mapsto k$.
  Now we must simply pick the value of $k$ so that $c_4 \mapsto -k$ under
  $T^{-1} \circ S$. \[
  \begin{bmatrix}
    -2k & k-1 \\
    -2 & 1 - k
  \end{bmatrix}\begin{bmatrix}
    c_1 - c_3 & c_2(c_3 - c_1) \\
    c_1 - c_2 & c_3(c_2 - c_1)
  \end{bmatrix}
  \] which corresponds to \[
    \frac{
      ((k - 1) (c_1 - c_2) - 2 k (c_1 - c_3))z +
      (k - 1) (c_1 - c_2) c_3 - 2 k c_2 (c_3 - c_1)
    }{
      ((1 - k) (c_1 - c_2) - 2 (c_1 - c_3))z +
      -(1 - k) (c_1 - c_2) c_3 - 2 c_2 (c_3 - c_1)
    }
  \]
  Therefore when $z = c_4$, solving for $k$ \[
  \frac{
    ((k - 1) (c_1 - c_2) - 2 k (c_1 - c_3))c_4 + (k - 1) (c_1 - c_2) c_3 - 2 k c_2 (c_3 - c_1)
  }{
    ((1 - k) (c_1 - c_2) - 2 (c_1 - c_3))c_4 +
    -(1 - k) (c_1 - c_2) c_3 - 2 c_2 (c_3 - c_1)
  } = -k
  \] multiplying by the denominator will result in a quadratic equation with two
  complex roots.Thus there are two solutions for $k$ related by the complex
  conjugate.
\end{proof}

\end{document}
