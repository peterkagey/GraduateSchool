\documentclass{article}

\usepackage[margin=1in]{geometry}
\usepackage{amsmath,amsthm,amssymb}
\usepackage{bbm, enumerate}

\newenvironment{problem}[2][Problem]{\begin{trivlist}
\item[\hskip \labelsep {\bfseries #1}\hskip \labelsep {\bfseries #2.}]}{\end{trivlist}}
\newenvironment{note}[1][Note.]{\begin{trivlist}
\item[\hskip \labelsep {\bfseries #1}]}{\end{trivlist}}

\begin{document}

\title{Complex Analysis: Homework 9}
\author{Peter Kagey}

\maketitle

% -----------------------------------------------------
% First problem
% -----------------------------------------------------
\begin{problem}{1} (page 166) \\
  If $u$ is harmonic and bounded in $0 < |z| < \rho$, show that the origin is a
  removable singularity in the sense that $u$ becomes harmonic in $|z| < \rho$
  when $u(0)$ is properly defined.
\end{problem}
\begin{proof} \text{} \\
  % https://math.stackexchange.com/q/459134/121988
  By Theorem 20, \[
    \frac{1}{2\pi}\int\displaylimits_{|z| = r} u\,d\theta = \alpha \log(r) + \beta
  \] but since $u$ is bounded the arithmetic mean is bounded for any $r$---in
  particular, the modulus is finite in the limit: \[
    \lim_{r\rightarrow 0^+} \left| \frac{1}{2\pi}\int\displaylimits_{|z| = r} u\,d\theta \right|
    = \lim_{r\rightarrow 0^+} \left| \alpha \log(r) + \beta \right|
    < \infty.
  \] Thus $\alpha = 0$, and $u(0) = \beta$.
  \\
  \\
  In general, \[
    \alpha = -\int\displaylimits_{|z| = r} r\,\frac{\partial u}{\partial r}\,d\theta
  \] so this means that \[
    \int\displaylimits_{|z| = r} r\,\frac{\partial u}{\partial r}\,d\theta = 0
  \] and therefore $u$ is harmonic.
\end{proof}
% -----------------------------------------------------
% Second problem
% -----------------------------------------------------
\pagebreak

\begin{problem}{1} (page 171) \\
  Assume that $U(\xi)$ is piecewise continuous and bounded for all real $\xi$.
  Show that \[
    P_U(z) = \frac{1}{\pi}\int_{-\infty}^\infty
      \frac{y}{(x - \xi)^2 + y^2}U(\xi)\,d\xi
  \] represents a harmonic function in the upper half plane with boundary values
  $U(\xi)$ at points of continuity (Poisson's integral for the half plane).
\end{problem}

\begin{proof} \text{} \\
  Start by defining a M\"obius transformation $T$ and its inverse $T^{-1}$,
  where $T$ maps the interior of the unit disk to the upper half plane and the
  boundary of the unit disk to the real axis. \begin{align*}
    T(z) &= \frac{1}{2} \cdot \frac{1 - iz}{z - i} \\
    T^{-1}(z) &= \frac{1 + 2iz}{2z + i}
  \end{align*}

  Then using equation (63) with $R = 1$ yields \[
  P_U(z) =
    \frac{1}{2\pi}\int_0^{2\pi}
      \operatorname{Re}\frac{e^{i\theta} + z}{e^{i\theta} - z}\,U(\theta)
    \,d\theta
  \]
  (?)
\end{proof}
% -----------------------------------------------------
% Third problem
% -----------------------------------------------------
\pagebreak

\begin{problem}{3} (page 171) \\
  In Exercise 1, assume that $U$ has a jump at $0$, for instance
  $U(+0) = 0$, $U(-0) = 1$. Show that $\displaystyle P_U(z)-\frac{1}{\pi}\arg z$
  tends to $0$ as $z \rightarrow 0$. Generalize to arbitrary jumps and to the
  case of the circle.
\end{problem}

\begin{proof} \text{} \\
  \[
    P_U(z) - \frac{1}{\pi}\arg z = \frac{1}{\pi}\left(
      \int_{-\infty}^\infty \frac{y}{(x - \xi)^2 + y^2}U(\xi)\,d\xi
      - \arg z
    \right)
  \] tends to $0$ as $z \rightarrow 0$ if and only if \[
    \lim_{z \rightarrow 0}
    \int_{-\infty}^\infty \frac{y}{(x - \xi)^2 + y^2}U(\xi)\,d\xi - \arg z
    = 0.
  \]
\end{proof}
% -----------------------------------------------------
% Fourth problem
% -----------------------------------------------------
\pagebreak

\begin{problem}{5} (page 171) \\
  Show that the mean-value formula (62) remains valid for
  $u=\log|1 + z|$,
  $z_0 = 0$,
  $r = 1$, and use this fact to compute \[
    \int_0^\pi \log\, \sin \theta\, d\theta
  \]
\end{problem}

\begin{proof} \text{} \\
  The mean value formula states that \[
    u(z_0) = \frac{1}{2\pi}\int_0^{2\pi} u(z_0 + re^{i\theta})\,d\theta
  \] so here we want to show that this formula works for the case \[
    \log|1 + 0| = \frac{1}{2\pi}\int_0^{2\pi} \log|e^{i\theta}|\,d\theta.
  \]
\end{proof}

\end{document}
