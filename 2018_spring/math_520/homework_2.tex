\documentclass{article}

\usepackage[margin=1in]{geometry}
\usepackage{amsmath,amsthm,amssymb}
\usepackage{bbm, enumerate}

\newenvironment{problem}[2][Problem]{\begin{trivlist}
\item[\hskip \labelsep {\bfseries #1}\hskip \labelsep {\bfseries #2.}]}{\end{trivlist}}
\newenvironment{note}[1][Note.]{\begin{trivlist}
\item[\hskip \labelsep {\bfseries #1}]}{\end{trivlist}}

\begin{document}

\title{Complex Analysis: Homework 2}
\author{Peter Kagey}

\maketitle

% -----------------------------------------------------
% First problem
% -----------------------------------------------------
\begin{problem}{5} (page 37) \\
  Discuss the uniform convergence of the series \[
    \sum_{n = 1}^{\infty} \frac{x}{n(1 + nx^2)}
  \] for real values of $x$.
\end{problem}

\begin{proof}
\end{proof}

% -----------------------------------------------------
% Second problem
% -----------------------------------------------------
\pagebreak

\begin{problem}{3} (page 41) \\
  Find the radius of convergence of the following power series: \begin{enumerate}[(a)]
    \item $\displaystyle\sum n^p z^n$
    \item $\displaystyle\sum \frac{z^n}{n!}$
    \item $\displaystyle\sum n!z^n$
    \item $\displaystyle\sum q^{n^2}z^n$ where $|q| < 1$
    \item $\displaystyle\sum z^{n!}$
  \end{enumerate}
\end{problem}

\begin{proof}
  \begin{enumerate}[(a)]
    \item By Hadamard's formula, let
      \[
        \frac{1}{R} = \limsup_{n \rightarrow \infty} |n^p|^{1/n}
        = \limsup_{n \rightarrow \infty} |n^p|^{1/n}
        = \limsup_{n \rightarrow \infty} n^{p/n}
        = 1
      \] So the radius of convergence is $R = 1$.

    \item Let N be an arbitrarily large integer, then by Hadamard's formula,\[
      \frac{1}{R} = \limsup_{n \rightarrow \infty} \left(\frac{1}{n!}\right)^{1/n}
      < \limsup_{n \rightarrow \infty} \left(\frac{1}{N^n}\right)^{1/n}
      = \frac{1}{N}.
    \] Because $R > N$ for all $N$, the radius of convergence is $\infty$.

    \item Let N be an arbitrarily large integer, then by Hadamard's formula,\[
      \frac{1}{R} = \limsup_{n \rightarrow \infty} (n!)^{1/n}
      > \limsup_{n \rightarrow \infty} (N^n)^{1/n}
      = N.
    \] Because $R < 1/N$ for all $N$, the radius of convergence is $0$.

    \item By Hadamard's formula, let \[
      \frac{1}{R} = \limsup_{n \rightarrow \infty} (q^{n^2})^{1/n}
      = \limsup_{n \rightarrow \infty} q^{n} = 0 \text { for } |q| < 1.
    \] Thus the radius of convergence is $\infty$.
    \item Notice that $|z^{n!}| \geq |z^n|$ for $|z| \geq 1$,
    and $|z^{n!}| \leq |z^n|$ for $|z| < 1$. \begin{align*}
      \left|\sum z^{n!}\right| \leq \left|\sum z^n\right| < \infty \text { for } |z| < 1 \\
      \left|\sum z^{n!}\right| \geq \left|\sum z^n\right| = \infty \text { for } |z| \geq 1
    \end{align*}
    Thus the radius of convergence is 1.
  \end{enumerate}
\end{proof}

% -----------------------------------------------------
% Third problem
% -----------------------------------------------------
\pagebreak

\begin{problem}{8} (page 41) \\
  For what values of $z$ is \[
    \sum_{n = 0}^\infty \left(
      \frac{z}{1 + z}
    \right)^n
  \] convergent?
\end{problem}

\begin{proof}
  The sum $\sum_{n = 0}^\infty w^n$ is convergent for $|w| < 1$.
  The sum in the problem is convergent when \[
    \left|\frac{z}{1 + z}\right| < 1 \Longrightarrow |z| < |1 + z|.
  \]
  Letting $z = a + bi$, and comparing the squares of the absolute values: \begin{align*}
    a^2 + b^2 &< (1 + a)^2 + b^2 \\
    a^2 &< 1 + 2a + a^2 \\
    -2a &< 1 \\
    a &> -1/2
  \end{align*}
  Thus the sum converges when $\operatorname{Re}(z) > -1/2$.
\end{proof}

% -----------------------------------------------------
% Fourth problem
% -----------------------------------------------------
\pagebreak

\begin{problem}{3} (page 44) \\
  Use the addition formulas to separate
    $\cos(x + iy)$ and $\sin(x + iy)$
    in real and imaginary parts.
\end{problem}

\begin{proof}
  First note the identites of $\sin$ and $\cos$ with purely imaginary inputs: \[
    \cos(iy) = 1 - \frac{i^2y^2}{2!} + \frac{i^4y^4}{4!} - \cdots
      = 1 + \frac{y^2}{2!} + \frac{y^4}{4!} + \cdots
      = \cosh(y)
  \]\[
    \sin(iy) = iy - \frac{i^3y^2}{3!} + \frac{i^5y^5}{5!} - \cdots
      = i(y + \frac{y^3}{3!} + \frac{y^5}{5!} + \cdots)
      = i\sinh(y)
  \]

  Then using the addition formulas
  \begin{align*}
    \cos(a + b) &= \cos(a)\cos(b) - \sin(a)\sin(b) \text{ and }\\
    \sin(a + b) &= \cos(a)\sin(b) + \sin(a)\cos(b),
  \end{align*}
  it is clear that \begin{align*}
    \cos(x + iy) &= \cos(x)\cosh(y) + i\sin(x)\sinh(y) \text { and} \\
    \sin(x + iy) &= \sin(x)\cosh(y) + i\cos(x)\sinh(y).
\end{align*}

\end{proof}

% -----------------------------------------------------
% Fifth
% -----------------------------------------------------
\pagebreak

\begin{problem}{4} (page 44) \\
  Show that \[
    |\cos z|^2
    = \sinh^2 y + \cos^2 x
    = \cosh^2 y - \sin^2 x
    = \frac{1}{2}(\cosh 2y + \cos 2x)
  \] and \[
    |\sin z|^2
    = \sinh^2 y + \sin^2 x
    = \cosh^2 y - \cos^2 x
    = \frac{1}{2}(\cosh 2y - \cos 2x)
  \]
\end{problem}

\begin{proof}
  Starting with the proof of $\cos$. From the above problem \[
    \cos(z) = \cos(x)\cosh(y) + i\sin(x)\sinh(y)
  \] so the square of the absolute value of $\cos(z)$ is \begin{align*}
    |\cos z|^2 &= (\cos(x)\cosh(y))^2 + (\sin(x)\sinh(y))^2 \\
    &= (\cos(x)\cosh(y))^2 + (\sin(x)\sinh(y))^2 + (\cos(x)\sinh(y))^2 - (\cos(x)\sinh(y))^2\\
    &= \sinh^2(y)(\sin^2(x) + \cos^2(x)) + \cos^2(x)(\cosh^2(y)-\sinh^2(y)) \\
    &= \sinh^2(y) + \cos^2(x)\\
    \\
    |\cos z|^2 &= (\cos(x)\cosh(y))^2 + (\sin(x)\sinh(y))^2 \\
    &= (\cos(x)\cosh(y))^2 + (\sin(x)\sinh(y))^2 + (\sin(x)\cosh(y))^2 - (\sin(x)\cosh(y))^2\\
    &= \cosh^2(y)(\sin^2(x) + \cos^2(x)) + \sin^2(x)(\sinh^2(y)-\cosh^2(y)) \\
    &= \cosh^2(y) - \sin^2(x)
  \end{align*}
  Thus adding the two different values together yields \begin{align*}
    2|\cos z|^2 &= \sinh^2(y) + \cos^2(x) + \cosh^2(y) - \sin^2(x) \\
    &= (\sinh^2(y) + \cosh^2(y)) + (\cos^2(x) - \sin^2(x))\\
    &= \cosh(2y) + \cos(2x)
  \end{align*}

  Similarly for $\sin$, from the above problem \[
    \sin(z) = \sin(x)\cosh(y) + i\cos(x)\sinh(y)
  \] so the square of the absolute value of $\cos(z)$ is \begin{align*}
    |\sin z|^2 &= (\sin(x)\cosh(y))^2 + (\cos(x)\sinh(y))^2 \\
    &= (\sin(x)\cosh(y))^2 + (\cos(x)\sinh(y))^2 + (\sin(x)\sinh(y))^2 - (\sin(x)\sinh(y))^2\\
    &= \sinh^2(y)(\sin^2(x) + \cos^2(x)) + \sin^2(x)(\cosh^2(y)-\sinh^2(y)) \\
    &= \sinh^2(y) + \sin^2(x) \\
    \\
    |\sin z|^2 &= (\sin(x)\cosh(y))^2 + (\cos(x)\sinh(y))^2 \\
    &= (\sin(x)\cosh(y))^2 + (\cos(x)\sinh(y))^2 + (\cos(x)\cosh(y))^2 - (\cos(x)\cosh(y))^2\\
    &= \cosh^2(y)(\sin^2(x) + \cos^2(x)) + \cos^2(x)(\sinh^2(y)-\cosh^2(y)) \\
    &= \cosh^2(y) - \cos^2(x)
  \end{align*}

  Thus adding the two different values together yields \begin{align*}
    2|\sin z|^2 &= \sinh^2(y) + \sin^2(x) + \cosh^2(y) - \cos^2(x) \\
    &= (\sinh^2(y) + \cosh^2(y)) - (\cos^2(x) - \sin^2(x)) \\
    &= \cosh(2y) - \cos(2x)
  \end{align*}
\end{proof}
\end{document}
