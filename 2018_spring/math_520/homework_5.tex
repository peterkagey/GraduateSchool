\documentclass{article}

\usepackage[margin=1in]{geometry}
\usepackage{amsmath,amsthm,amssymb}
\usepackage{bbm, enumerate}

\newenvironment{problem}[2][Problem]{\begin{trivlist}
\item[\hskip \labelsep {\bfseries #1}\hskip \labelsep {\bfseries #2.}]}{\end{trivlist}}
\newenvironment{note}[1][Note.]{\begin{trivlist}
\item[\hskip \labelsep {\bfseries #1}]}{\end{trivlist}}

\begin{document}

\title{Complex Analysis: Homework 3}
\author{Peter Kagey}

\maketitle

% -----------------------------------------------------
% First problem
% -----------------------------------------------------
\begin{problem}{6} (page 108) \\
  Assume that $f(z)$ is analytic and satisfies the inequality $|f(z)-1| < 1$ in
  a region $\Omega$. Show that \[
    \int_\gamma \frac{f'(z)}{f(z)}dz = 0
  \] for every closed curve in $\Omega$.
\end{problem}

\begin{proof} \text{} \\
  Because $|f(z) - 1| < 1$, $w = f(z)$ stays strictly in the right half plane
  for $z \in \Omega$.
  Therefore the principal branch of log is analytic for $z \in \Omega$, and so
  $f'(z)/f(z)$ is the derivative of an analytic function in $\Omega$ and thus
  the integral only depends on its endpoints. Since $\gamma$ is a closed curve,
  the integral must vanish.
\end{proof}
% -----------------------------------------------------
% Second problem
% -----------------------------------------------------
\pagebreak

\begin{problem}{7} (page 108) \\
  If $P(z)$ is a polynomial and $C$ denotes the circle $|z-a| = R$,
  what is the value of of $\displaystyle \int_C P(z) d\bar{z}$?\\
  Answer: $-2\pi iR^2P'(a)$.
\end{problem}

\begin{proof}
  Parameterize the circle by $z(t) = Re^{it} + a$ for $\theta \in [0, 2\pi]$,
  then $\bar{z}(t) = \overline{Re^{it} + a} = Re^{-it} + \bar{a}$ and
  $d\bar{z} = -iRe^{-it} dt$.
  \begin{align*}
    \int_C P(z) d\bar{z} &= -iR\int_{0}^{2\pi} P(Re^{it} + a)e^{-it} dt
  \end{align*}
  Notice that the (finite) Taylor series expansion of $P$ around $a$ is \[
    P(z) = P(a) + P'(a)(z-a) + \hdots + \frac{P^{(n)}(a)}{n!}(z - a)^n.
  \]
  So the above integral becomes \begin{align*}
    \int_C P(z) d\bar{z} &= -iR\int_{0}^{2\pi} e^{-it} \left(P(a) + P'(a)(Re^{it}) + \hdots + \frac{P^{(n)}(a)}{n!}(R^{n}e^{nit})\right) dt \\
    &= -iR\int_{0}^{2\pi} P(a)e^{-it} + RP'(a) + \frac{P''(a)}{2}(Re^{it}) \hdots + \frac{P^{(n)}(a)}{n!}(R^{n}e^{(n - 1)it}) dt \\
    &= -iR\int_{0}^{2\pi} RP'(a) dt \\
    &= -2\pi iR^2P'(a)
  \end{align*} because all terms except for the first derivative term vanish due
  to symmetry, (i.e. $c\int_0^{2\pi}e^{nit} dt = 0$ for all
  $n \in \mathbb{Z} \setminus \{0\}$.)
\end{proof}
% -----------------------------------------------------
% Third problem
% -----------------------------------------------------
\pagebreak

\begin{problem}{1} (page 120) \\
  Compute \[
    \int_{|z| = 1} \frac{e^z}{z} dz.
  \]
\end{problem}

\begin{proof}
  Because $e^z$ is analytic in $\mathbb{C}$, and $\gamma = \{|z| = 1\}$ is a
  closed curve in the disk of radius $2$ centered at the origin, Theorem 6 gives
  \[
    \int_\gamma \frac{e^z}{z}\ dz = 2\pi i \cdot n(\gamma, 0) \cdot e^0.
  \]
  The winding number of $\gamma$ around the origin is $1$ and $e^0 = 1$, so
  \[
    \int_\gamma \frac{e^z}{z}\ dz = 2\pi i.
  \].
\end{proof}

% -----------------------------------------------------
% Fourth problem
% -----------------------------------------------------
\pagebreak

\begin{problem}{3} (page 120) \\
  Compute \[
    \int_{|z|=2}\frac{dz}{z^2 + 1}.
  \]
\end{problem}

\begin{proof}
  By partial fraction decomposition and Theorem 6, \begin{align*}
    \int_{|z|=2}\frac{dz}{z^2 + 1} &= \frac{1}{2}\int_{|z|=2}\frac{dz}{z - i} -
    \frac{1}{2}\int_{|z|=2}\frac{dz}{z + i}\\
    &= \frac{1}{2}\cdot n(\gamma, i)\cdot 1 - \frac{1}{2}\cdot n(\gamma, -i)\cdot 1 \\
    &= 0.
  \end{align*}
\end{proof}

% -----------------------------------------------------
% Fifth problem
% -----------------------------------------------------
\pagebreak

\begin{problem}{2} (page 123) \\
  Prove that a function which is analytic in the whole plane and satisfies an
  inequality $|f(z)| < |z|^n$ for some $n$ and all sufficiently large $|z|$
  reduces to a polynomial.
\end{problem}

\begin{proof}
  Because $|f(z)| < |z|^n$, $0 \leq |f(z)/z^n| < 1$ for sufficiently large $|z|$,
  by Theorem 8, \begin{align*}
    f(z) &= f_{n+1}(z)\cdot(z)^{n+1} + \sum_{k = 0}^{n} \frac{f^{(k)}(0)}{k!}(z)^k \\
    1 &> \left|\frac{f(z)}{z^n}\right|\\
    &= \left|f_{n+1}(z)\cdot(z) + f^{(n)}(0) + \sum_{k = 0}^{n-1} \frac{f^{(k)}(0)}{k!}(z)^{k-n}\right| \\
  \end{align*}
  By taking large $|z|$, the absolute value of the sum can be made arbitrarily
  small. Therefore in order to stay bounded, $f_{n+1}(z) = 0$ for all $z$, so \[
    f(z) = \sum_{k = 0}^{n} \frac{f^{(k)}(0)}{k!}(z)^k
  \] which is a polynomial.
\end{proof}

% -----------------------------------------------------
% Sixth problem
% -----------------------------------------------------
\pagebreak

\begin{problem}{3} (page 123) \\
  If $f(z)$ is analytic and $|f(z)| \leq M$ for $|z| \leq R$, find an upper
  bound for $|f^{(n)}(z)|$ in $|z| \leq \rho < R$.
\end{problem}

\begin{proof}
  By Equation (24), \begin{align}
    |f^{(n)}(z)|
    &= \frac{n!}{2\pi}\left|\int_{|\zeta| = R} \frac{f(\zeta)}{(\zeta - z)^{n+1}}d\zeta\right|\\
    &\leq \frac{n!}{2\pi}\int_{|\zeta| = R} \frac{|f(\zeta)|}{|\zeta - z|^{n+1}}|d\zeta|\\
    &\leq \frac{n!M}{2\pi}\int_{|\zeta| = R} \frac{|d\zeta|}{|\zeta - z|^{n+1}}\\
    &\leq \frac{n!M}{2\pi\cdot(R-\rho)^{n+1}}\int_{|\zeta| = R} |d\zeta|\\
    &= \frac{n!2\pi rM}{2\pi\cdot(R-\rho)^{n+1}} \\
    &= \frac{n!RM}{(R-\rho)^{n+1}}.
  \end{align}
  (2) is justified by Equation (9) in Ahlfors;
  (3) because $|f(\zeta)| \leq M$, by hypothesis;
  (4) because $|\zeta - z| \geq (R-\rho)$ for $|z| \leq \rho$;
  (5) by the arc length of the circle of radius $R$; and (6) by simplification.
\end{proof}

% -----------------------------------------------------
% Seventh problem
% -----------------------------------------------------
\pagebreak

\begin{problem}{4} (page 123) \\
  If $f(z)$ is analytic for $|z| < 1$ and $|f(z)| \leq 1/(1-|z|)$, find the best
  estimate of $|f^{(n)}(z)|$ that Cauchy's inequality will yield.
\end{problem}

\begin{proof}
  Again using equation (24) and letting $C$ be the circle of radius $0 < \rho < 1$ about the origin,
  \setcounter{equation}{0}
  \begin{align}
  |f^{(n)}(0)|
  &= \frac{n!}{2\pi}\left|\int_C \frac{f(\zeta)}{\zeta^{n+1}}d\zeta\right|\\
  &= \frac{n!}{2\pi}\left|\int_{|\zeta| = \rho} \frac{f(\zeta)}{\zeta^{n+1}}d\zeta\right|\\
  &\leq \frac{n!}{2\pi}\int_{|\zeta| = \rho} \frac{|f(\zeta)|}{|\zeta|^{n+1}}|d\zeta|\\
  &= \frac{n!}{2\pi\rho^{n+1}}\int_{|\zeta| = \rho} |f(\zeta)||d\zeta|\\
  &\leq \frac{n!}{2\pi\rho^{n+1}}\int_{|\zeta| = \rho} \frac{|d\zeta|}{1-|\zeta|}\\
  &= \frac{n!}{2\pi\rho^{n+1}(1-\rho)}\int_{|\zeta| = \rho} |d\zeta|\\
  &= \frac{n!2\pi\rho}{2\pi\rho^{n+1}(1-\rho)}\\
  &= \frac{n!}{\rho^n(1-\rho)}
  \end{align}
  This last expression is minimized when $\rho^n(1-\rho)$ is maximized \[
    \frac{d}{d\rho}[\rho^n(1-\rho)] = n\rho^{n-1} - (n + 1)\rho^n = 0,
  \] that is, $\rho = n/(n+1)$. In this case \[
    \frac{n!}{\rho^n(1-\rho)} = (n+1)!\left(\frac{n+1}{n}\right)^n.
  \] Therefore the best estimate of $|f^{(n)}(z)|$ is \[
    |f^{(n)}(z)| \leq (n+1)!\left(\frac{n+1}{n}\right)^n.
  \]
\end{proof}

\end{document}
