\documentclass{article}

\usepackage[margin=1in]{geometry}
\usepackage{amsmath,amsthm,amssymb}
\usepackage{bbm, enumerate}

\newenvironment{problem}[2][Problem]{\begin{trivlist}
\item[\hskip \labelsep {\bfseries #1}\hskip \labelsep {\bfseries #2.}]}{\end{trivlist}}
\newenvironment{note}[1][Note.]{\begin{trivlist}
\item[\hskip \labelsep {\bfseries #1}]}{\end{trivlist}}

\begin{document}

\title{Complex Analysis: Homework 11}
\author{Peter Kagey}

\maketitle

% -----------------------------------------------------
% Problem
% -----------------------------------------------------
\begin{problem}{1} (page 251) \\
  If $\Omega$ is the punctured disk $0 < |z| < 1$ and if $f$ is given by
  $f(\zeta) = 0$ for $|\zeta| = 1$, $f(0) = 1$ show that all functions
  $v \in \mathfrak{B}(f)$ are $\leq 0$ in $\Omega$.
\end{problem}
\begin{proof} \text{} \\
\end{proof}
\pagebreak


% -----------------------------------------------------
% Problem
% -----------------------------------------------------
\begin{problem}{2} (page 178) \\
  Show that the series \[
    \zeta(z) = \sum\displaylimits_{n=1}^\infty n^{-z}
  \] converges for $\operatorname{Re} z > 1$ and represent its derivative in
  series form.
\end{problem}
\begin{proof} \text{} \\
  Assume that $z = z + iy$ for $x > 1$. Then \begin{align}
    |\zeta(z)|
      &= \left|\sum_{n=1}^\infty \frac{1}{n^z}\right| \\
      &\leq \sum_{n=1}^\infty \left|\frac{1}{n^z}\right| \\
      &= \sum_{n=1}^\infty \left|\frac{1}{n^{z + iy}}\right| \\
      &= \sum_{n=1}^\infty \frac{1}{|n^{x}||n^{iy}|} \\
      &= \sum_{n=1}^\infty \frac{1}{n^{x}|e^{iy\log(n)}|} \\
      &= \sum_{n=1}^\infty \frac{1}{n^{x}}
  \end{align} which is known to converge by the integral test because $x > 1$. \\
  The derivative of $\zeta(z)$ can be computed term by term \begin{align*}
    \frac{d}{dz}\left[\zeta(z)\right]
    &= \frac{d}{dz}\left[\sum\displaylimits_{n=1}^\infty n^{-z}\right] \\
    &= \sum\displaylimits_{n=1}^\infty\frac{d}{dz}\left[n^{-z}\right] \\
    &= \sum\displaylimits_{n=1}^\infty\frac{d}{dz}\left[e^{-z\log(n)}\right] \\
    &= \sum\displaylimits_{n=1}^\infty -e^{-z\log(n)}\log(n) \\
    &= \sum\displaylimits_{n=1}^\infty -n^{-z}\log(n)
  \end{align*}
\end{proof}
\pagebreak


% -----------------------------------------------------
% Problem
% -----------------------------------------------------
\begin{problem}{4} (page 178) \\
  As a generalization of Theorem 2, prove that if the $f_n(z)$ have at most $m$
  zeros in $\Omega$ then $f(z)$ is either identically zero or has at most $m$
  zeros.
\end{problem}
\begin{proof} \text{} \\
  Suppose for the sake of contradiction that $f(z)$ has $m + 1$ zeros. Then we
  can form a
\end{proof}
\pagebreak


% -----------------------------------------------------
% Problem
% -----------------------------------------------------
\begin{problem}{5} (page 184) \\
  The Fibonacci numbers are defined by $c_0 = 0, c_1 = 1$, \[
    c_{n} = c_{n - 1} + c_{n - 2}.
  \]
  Show that the $c_n$ are Taylor coefficients of a rational function, and
  determine a closed expression for $c_n$.
\end{problem}
\begin{proof} \text{} \\
  Let \begin{align*}
    f(z) &= \sum_{n=0}^\infty c_nz^n \\
    &= \sum_{n=1}^\infty c_nz^n \\
    &= z + \sum_{n=2}^\infty c_nz^n \\
    &= z + \sum_{n=2}^\infty (c_{n-1} + c_{n-2})z^n \\
    &= z + \sum_{n=2}^\infty c_{n-1}z^n + \sum_{n=2}^\infty c_{n-2}z^n \\
    &= z + \sum_{n=1}^\infty c_{n}z^{n+1} + \sum_{n=0}^\infty c_{n}z^{n+2} \\
    &= z + z\sum_{n=1}^\infty c_{n}z^n + z^2\sum_{n=0}^\infty c_{n}z^n \\
    &= z + zf(z) + z^2f(z).
  \end{align*}
  Then solving for $f(z)$ yields \[
    f(z) = \frac{z}{1 - z - z^2},
  \] thus $f$ is a rational function.
  Now we must find a power series representation of $f$, and its coefficients
  will give us the terms of the Fibonacci sequence. Since $f$ has roots
  $\varphi = (1 + \sqrt{5})/2$ and $\psi = (1 - \sqrt{5})/2$, partial fraction
  decomposition yields \begin{align*}
    f(z)
    &= -\frac{z}{(z + \varphi)(z + \psi)}  \\
    &= \frac{A}{z + \varphi} + \frac{B}{z + \psi}
  \end{align*} where $A$ and $B$ satisfy \[
    -z = A(z + \psi) + B(z + \varphi)
  \] and thus the system of equations \begin{align*}
    A + B &= -1 \\
    A\psi + B\varphi &= 0.
  \end{align*} Solving for $A$ and  $B$ gives \begin{align*}
    f(z) &= \frac{1}{\sqrt{5}}\left(\frac{-\varphi}{z + \varphi} + \frac{\psi}{z + \psi}\right) \\
    &= \frac{1}{\sqrt{5}}\left(-\frac{1}{z/\varphi + 1} + \frac{1}{z/\psi + 1}\right).
  \end{align*}
  So the identity \[
    \sum_{n=0}^\infty {z^n} = \frac{1}{1-z}
  \] together with $-1/\psi = \varphi$ and $-1/\varphi = \psi$ gives \begin{align*}
    f(z)
      &= \sum_{n=0}^\infty \frac{-1}{\sqrt{5}}(-z/\varphi)^n + \sum_{n=0}^\infty \frac{1}{\sqrt{5}}(-z/\psi)^n \\
      &= \sum_{n=0}^\infty \frac{1}{\sqrt{5}}\left((-z/\psi)^n - (-z/\varphi)^n \right) \\
      &= \sum_{n=0}^\infty \frac{1}{\sqrt{5}}\left(\varphi^n z^n - \psi^nz^n \right) \\
      &= \sum_{n=0}^\infty \frac{1}{\sqrt{5}}\left(\varphi^n - \psi^n \right)z^n \\
  \end{align*}
  Therefore \[
    c_n = \frac{\phi^n - \psi^n}{\sqrt{5}}.
  \]
\end{proof}
\pagebreak


% -----------------------------------------------------
% Problem
% -----------------------------------------------------
\begin{problem}{4} (page 186) \\
  Show that the Laurent development of $(e^z - 1)^{-1}$ at the origin is of the
  form \[
    \frac{1}{z} - \frac{1}{2} +
    \sum\displaylimits_{k = 1}^\infty (-1)^{k-1}\frac{B_k}{(2k)!}z^{2k-1}.
  \]
  Calculate $B_1$, $B_2$, and $B_3$.
\end{problem}
\begin{proof} \text{} \\
  (An idea from https://math.stackexchange.com/q/1006615/121988.)\\
  We know that $f(z) = (e^z - 1)^{-1}$ has a pole at 0, so consider
  $f(z) = 1/(zg(z))$ where \[
    g(z) = \frac{e^{z} - 1}{z}
         = \frac{1}{z}\left(-1 + \sum_{n=0}^\infty \frac{z^n}{n!}\right)
         = \sum_{n=1}^\infty \frac{z^{n-1}}{n!}
         = \sum_{n=0}^\infty \frac{z^n}{(n + 1)!}
  \] which is analytic everywhere. When $z \not= 0$, $f(z) = h(z)/z$ where
  $h(z) = 1/g(z)$.
  (?)\\
  Another idea: \[
    f(z) = \frac{-1}{1 - e^z} = -\sum_{n=0}^\infty (e^z)^n = -\sum_{n=0}^\infty \left(\sum_{k=0}^\infty\frac{z^k}{k!}\right)^n
  \]
\end{proof}
\pagebreak


% -----------------------------------------------------
% Problem
% -----------------------------------------------------
\begin{problem}{5} (page 186) \\
  Express the Taylor development of $\tan z$ and the Laurent development of
  $\cot z$ in terms of the Bernoulli numbers.
\end{problem}
\begin{proof} \text{} \\
\end{proof}
\pagebreak


% -----------------------------------------------------
% Problem
% -----------------------------------------------------
\begin{problem}{1} (page 190) \\
  Comparing coefficients in the Laurent developments of $\cot\pi z$ and of its
  expression as a sum of partial fractions, find the values of \[
    \sum\displaylimits_{n = 1}^\infty \frac{1}{n^2},
    \hspace{1cm}
    \sum\displaylimits_{n = 1}^\infty \frac{1}{n^4}, \text{ and}
    \hspace{1cm}
    \sum\displaylimits_{n = 1}^\infty \frac{1}{n^6}
  \] with a complete justifaction of the steps that are needed.
\end{problem}
\begin{proof} \text{} \\
  By equation (10), we know that \[
    \pi\cot\pi z = \lim_{m\rightarrow\infty}\sum_{n = -m}^{m} \frac { 1} { z - n } = \frac { 1} { z } + \sum _ { n = 1} ^ { \infty } \frac { 2z } { z ^ { 2} - n ^ { 2} }.
  \]
\end{proof}
\pagebreak


% -----------------------------------------------------
% Problem
% -----------------------------------------------------
\begin{problem}{2} (page 190) \\
  Express \[
    \sum\displaylimits_{-\infty}^\infty \frac{1}{z^3 - n^3}
  \] in closed form.
\end{problem}
\begin{proof} \text{} \\
\end{proof}
\end{document}
