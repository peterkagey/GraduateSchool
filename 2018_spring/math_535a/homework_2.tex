\documentclass{article}

\usepackage[margin=1in]{geometry}
\usepackage{amsmath,amsthm,amssymb}
\usepackage{bbm,enumerate,mathtools}

\newenvironment{problem}[2][Problem]{\begin{trivlist}
\item[\hskip \labelsep {\bfseries #1}\hskip \labelsep {\bfseries #2.}]}{\end{trivlist}}
\newenvironment{note}[1][Note.]{\begin{trivlist}
\item[\hskip \labelsep {\bfseries #1}]}{\end{trivlist}}

\begin{document}

\title{Differential Geometry: Homework 2}
\author{Peter Kagey}

\maketitle

% -----------------------------------------------------
% First problem
% -----------------------------------------------------
\begin{problem}{1}
  Let $V$ be a vector space of a field $k$ and let $W \subset V$ be a subspace. \\
  % (meaning a subset satisfying for any $a, b \in W$, $c\cdot a + d\cdot b \in W$
  % for any scalars $c,d \in k$. Such a subset is automatically a vector space
  % over $k$; with all operations coming from those in $V$).
  (a) Show that $V/W$ is a vector space, with operations induced by those of $V$ in
    the following sense: for $\alpha$ and $\beta$ in $V/W$,
    choose elements $a$ and $b$ with $[a] = \alpha, [b] = \beta$, and define
    $\alpha + \beta = [a + b]$ and $c \cdot \alpha = [c \cdot a]$
\end{problem}

\begin{proof} \text{} \\
  \textbf{Addition is well-defined.}\\
    Let $\alpha = [a_1] = [a_2]$, and let $\beta = [b_1] = [b_2]$.
    Then by definition of the equivalence class
    $\sim$, $a_1 - a_2 \in W$ and $b_1 - b_2 \in W$. In order to show that
    that $\alpha + \beta = [a_1 + b_1] = [a_2 + b_2]$ is well-defined, it is
    sufficient to show that $a_1 + b_1 \sim a_2 + b_2$. By closure of $W$ under
    addition,
    \[
      (a_1 + b_1) - (a_2 + b_2) = (a_1 - a_2) - (b_1 - b_2) \in W
    \]
%
  \textbf{Multiplication is well-defined.}\\
    Let $\alpha = [a_1] = [a_2]$ and $c \in k$.
    Then by definition of the equivalence class
    $\sim$, $a_1 - a_2 \in W$. In order to show that
    that $c \cdot \alpha = [c\cdot a_1] = [c\cdot a_2]$ is well-defined, it is
    sufficient to show that $c\cdot a_1 \sim c\cdot a_2$. By
    distributivity laws and closure of $W$ under scalar multiplication,
    \[
      c\cdot a_1 - c\cdot a_2 = c(a_1 - a_2) \in W
    \]
%
  \textbf{$V/W$ is an abelian group under +.}
  \textit{Associativity.}\\
  This is more-or-less inherited from the associativity of addition in $V$. \[
    (\alpha + \beta) + \gamma
      = [(a + b) + c]
      = [a + (b + c)]
      = \alpha + (\beta + \gamma).
  \]
  \textit{Commutivity.}\\
  This is more-or-less inherited from the commutivity of addition in $V$. \[
    (\alpha + \beta)
      = [a + b]
      = [b + a]
      = \beta + \alpha.
  \]
  \textit{Identity element.}\\
  Let $0 = [\vec{0}]$ where $\vec{0} \in W$. \begin{align*}
    \alpha + 0 = [a + \vec{0}] = [a] = \alpha\\
    0 + \alpha = [\vec{0} + a] = [a] = \alpha
  \end{align*}
  \textit{Inverse element.}\\
  Let $\alpha = [a]$, then $-\alpha = [-a]$. \begin{align*}
    \alpha + -\alpha = [a + -a] = [\vec{0}] = e\\
    -\alpha + \alpha = [-a + a] = [\vec{0}] = e
  \end{align*}
%
  \textbf{Multiplication is well-behaved.}\\
  \begin{enumerate}[(i)]
    \item $1 \cdot \alpha = [1 \cdot a] = [a] = \alpha.$
    \item $
      c_1 \cdot (c_2 \cdot \alpha)
      = c_1 \cdot [c_2\cdot a]
      = [c_1 \cdot (c_2\cdot a)]
      = [(c_1 c_2) \cdot a]
      = (c_1 c_2) \cdot \alpha.
    $
    \item $
      c_1 \cdot (\alpha + \beta)
      = c_1 \cdot [a + b]
      = [c_1 \cdot (a + b)]
      = [c_1 \cdot a + c_1 \cdot b]
      = [c_1 \cdot a] + [c_1 \cdot b]
      = c_1\cdot \alpha + c_1\cdot \beta.
    $
    \item $
      (c_1 + c_2) \cdot \alpha
      = [(c_1 + c_2) \cdot a]
      = [c_1 \cdot a + c_2 \cdot a]
      = [c_1 \cdot a] + [c_2 \cdot a]
      = c_1\cdot \alpha + c_2\cdot \alpha
    $
  \end{enumerate}
\end{proof}

\begin{problem}{1} (b) \\
  The quotient comes equipped with a natural linear map \begin{align*}
    \pi: V &\longrightarrow V/W \\
    v      &\longrightarrow [v] = v + W,
  \end{align*}
  called the \textit{projection}, which has $\ker \pi = W$ (check that $\pi$ is
  linear and has kernel as desired.) Suppose $V$ is finite-dimensional,
  and let $U$ be a subspace complementary to $W$, that is a subspace such that
  $V = W \oplus U$.
  Show that the restriction of projection to $U$ \[
    \pi_U: U \longrightarrow V/W
  \] is an isomorphism.
\end{problem}

\begin{proof} \text{} \\
  To see that $\pi$ is linear, check its additivity and homogeneity.
  The additivity of $\pi$ follows from \[
    \pi(v + w) = [v + w] = [v] + [w] = \pi(v) + \pi(w),
  \]
  and the homogeneity of $\pi$ follows from \[
    \pi(c\cdot v) = [c \cdot v] = c \cdot [v] = c\cdot\pi(v).
  \]\\
  \\
  To check that $\ker \pi = W$, see that \[
    \ker \pi
    = \{ v\ |\ \pi(v) = 0 \}
    = \{ v\ |\ v + W = 0 + W \}
    = \{ v\ |\ v \in W \}
    = W
  \] because $W$ is closed under addition.\\ \\
\end{proof}

\begin{problem}{1} (c) \\
  Let $U$ denote the subspace of $C^\infty(\mathbb{R})$ consisting of functions
  which vanish at 3 and 5 \[
    U = \{ f \in C^\infty(\mathbb{R})\ |\ f(3) = f(5) = 0 \}.
  \] Prove that the quotient vector space $C^\infty(\mathbb{R})/U$ is
  finite-dimensional. What is its dimension?
\end{problem}
\begin{proof} \text{} \\
  Let $\ell_5(x) = (5-x)/2$ and $\ell_3(x) = (x-3)/2$, so that
  $\ell_5(5) = 0 = \ell_3(3)$ and $\ell_5(3) = 1 = \ell_3(5)$. Because
  $\ell_5$ and $\ell_3$ are lines, they are smooth.\\ \\
%
  Then for any function $f \in C^\infty(\mathbb{R})$ \begin{align*}
    f(x) - (f(3)\ell_5(x) + f(5)\ell_3(x)) &\in U \text{ because}\\
    f(5) - (f(3)\ell_5(5) + f(5)\ell_3(5)) &= f(5) - (0 + f(5)) = 0 \text{ and}\\
    f(3) - (f(3)\ell_5(3) + f(5)\ell_3(3)) &= f(3) - (f(3) + 0) = 0.
  \end{align*}
  Since any smooth function $f$ is in a coset with a sum of two smooth functions,
  $[f(x)] = f(3)\cdot[\ell_5(x)] + f(5)\cdot[\ell_3(x)]$,
  $C^\infty(\mathbb{R})/U$ has dimension at most two.\\
  Since $\ell_5 \not\in U$, and $\ell_3 + U \not\in \text{span} \{ 0 + U, \ell_5 + U \}$,
  $C^\infty(\mathbb{R})/U$ has dimension exactly two.
\end{proof}

\begin{problem}{1} (d) \\
  Let $V$ be a vector space and $W \subset V$ be a vector subspace.
  We denote the inclusion map by $i:W \rightarrow V$.
  Denote $V^* = \operatorname{Hom}_k(V, k)$.
  There is a natural induced map $i^*:V^*\rightarrow W^*$ dual to the inclusion
  sending a linear map $\phi \mapsto \phi|_W$.
  The kernel of $i^*$ is called the \textit{annihilator} of $W$ and denoted \[
    \operatorname{Ann}(W) = \{ \phi \in V^*\ |\ \phi|_W = 0 \in W^* \}.
  \]
  It is the set of linear maps from $V$ to $k$ that return $0$ on any element
  in $W$.

  Prove that there is a canonical isomorphism \[
    \operatorname{Ann}(W) \cong (V/W)^*.
  \]
\end{problem}
\begin{proof} \text{} \\
  The ``obvious'' maps to consider are \begin{alignat*}{4}
    \psi&:\operatorname{Ann}(W) &&\rightarrow (V/W)^* &&\text{ via }
      g \mapsto ([\vec{v}] &&\mapsto g(\vec{v}))\\
    \psi^{-1}&: (V/W)^* &&\rightarrow\operatorname{Ann}(W) &&\text{ via }
      f \mapsto (\vec{v} &&\mapsto f([\vec{v}]))
  \end{alignat*}
  It is sufficient to show that these maps
    (i) are well-defined,
    (ii) satisfy $\psi \circ \psi^{-1} = \mathrm{id}_{(V/W)^*}$, and
    (iii) satisfy $\psi^{-1} \circ \psi = \mathrm{id}_{\operatorname{Ann}(W)}$.
  \\
  \\\textbf{Proof of (i).}
  For the well-definedness of $\psi$, it is sufficient to show that if
  $[\vec{v}] = [\vec{u}]$ then $g(\vec{v}) = g(\vec{u})$.
  Since $\vec{v} - \vec{u} \in W$,
  $g(\vec{v} - \vec{u}) = 0 = g(\vec{v}) - g(\vec{u})$, so
  $g(\vec{v}) = g(\vec{u})$\\ \\
  For the well-definedness of $\psi^-1$, it is sufficient to show that
  for all $\varphi \in \operatorname{Ann}(W)$ if
  $[\vec{v}_1] = [\vec{v}_2]$ and the map $\varphi(\vec{v}_1) = f([\vec{v}_1])$
  then $\varphi(\vec{v}_2) = f([\vec{v}_1])$.
  \begin{align*}
    \varphi(\vec{v}_1 - \vec{v}_2) &= 0 \\
    &= \varphi(\vec{v}_1) - \varphi(\vec{v}_2)
  \end{align*} So $\varphi(\vec{v}_1) = \varphi(\vec{v}_2) = f([\vec{v}_1])$.\\
  \\\textbf{Proof of (ii).}
  Let $f: V/W \rightarrow k$ be a linear function.
    \[
      \psi(\psi^{-1}(f))
      = \psi(\vec{v} \mapsto f([\vec{v}]))
      = [\vec{v}] \mapsto f([\vec{v}])
      = f
    \] Therefore $\psi \circ \psi^{-1} = \mathrm{id}_{(V/W)^*}$ \\
  \\\textbf{Proof of (iii).}
  Let $g: \operatorname{Ann}(W)$ be a linear function. \[
    \psi^{-1}(\psi(g))
    = \psi^{-1}([\vec{v}] \mapsto g(\vec{v}))
    = \vec{v} \mapsto g(\vec{v})
    = g
  \] Therefore $\psi^{-1} \circ \psi = \mathrm{id}_{\operatorname{Ann}(W)}$. \\
\end{proof}

% -----------------------------------------------------
% Second problem
% -----------------------------------------------------
\pagebreak

\begin{problem}{2}
  Let \[
    S^n = \{
      (x_1, \hdots, x_{n + 1})\ |\
      x_1^2 + \hdots + x_{n + 1}^2 = 1
    \} \subset \mathbb{R}^{n + 1}.
  \] Prove that $S^n$ has the structure of a smooth manifold,
  using charts associated to the cover $U_N = \{ x_1 \not= 1 \}$, $U_S = \{ x_1 \not= 1 \}$.
\end{problem}

\begin{proof}
  Let $f_p$ be a parameterization of a line that begins at $(1, 0, \hdots, 0)$ and
  equals $p$ at time 1: \[
    f_p(t) = (1 - t)(1, 0, \hdots, 0) + t(x_1, \hdots, x_{n + 1}).
  \]
  This line intersects the subspace
  $T = \{ (x_1, \hdots, x_{n + 1}) \in \mathbb{R}^{n + 1}\ |\ x_1 = 0\} \cong \mathbb{R}^n$
  when \begin{align*}
    0 &= (1 - t) + tx_1 \\
    t &= 1/(1 - x_1).
  \end{align*}
  Thus we can define $\phi_N: U_N \rightarrow T$ by \begin{align*}
    p &\mapsto f_p\left(\frac{1}{1-\pi_1(p)}\right) \\
    (x_1, \hdots, x_{n + 1}) &\mapsto \frac{1}{1-x_1}(0, x_2, \hdots, x_{n + 1}).
  \end{align*}
  Similarly $\phi_S: U_S \rightarrow T$ is defined by \[
    (x_1, \hdots, x_{n + 1}) \mapsto \frac{1}{1+x_1}(0, x_2, \hdots, x_{n + 1}).
  \]

  The functions $\phi_N$ and $\phi_S$ are smooth because the first coordinate is the
  constant map $0$, and the other coordinates are being multiplied by a constant scalar.
\end{proof}

% -----------------------------------------------------
% Third problem
% -----------------------------------------------------
\pagebreak

\begin{problem}{3}
  Prove that the product of two smooth manifolds \begin{align*}
    (M^m,\ \mathcal{A}_M &= \{(U_\alpha,\ \phi_\alpha: U_\alpha \rightarrow \mathbb{R}^m)\}_{\alpha \in I}), \text{ and}\\
    (N^n,\ \mathcal{A}_N &= \{(U_\beta,\ \psi_\beta: U_\beta \rightarrow \mathbb{R}^n)\}_{\beta \in J})
  \end{align*}
  naturally has the structure of a smooth manifold, with atlas given by \[
    \mathcal{A}_{M\times N} = \{
      (U_\alpha \times V_\beta,\
      (\phi_\alpha, \psi_\beta):
        U_\alpha \times V_\beta \rightarrow
        \mathbb{R}^m \times \mathbb{R}^n = \mathbb{R}^{m + n})
    \}_{(\alpha, \beta) \in I \times J}.
  \]

\end{problem}

\begin{proof}
\end{proof}

% -----------------------------------------------------
% Fourth problem
% -----------------------------------------------------
\pagebreak

\begin{problem}{4}
  Prove that the antipodal map $S^n \rightarrow S^n$, $x \mapsto -x$ is a
  diffeomorphism of manifolds.
\end{problem}

\begin{proof}
\end{proof}

% -----------------------------------------------------
% Fifth
% -----------------------------------------------------
\pagebreak

\begin{problem}{5}
  Finish the proof from class that $\mathbb{R}P^n$ is a smooth manifold.
\end{problem}

\begin{proof} \text{} \\
\end{proof}

% -----------------------------------------------------
% Sixth
% -----------------------------------------------------
\pagebreak

\begin{problem}{6}
  Finish the proof from class that $\mathbb{T}^2 = \mathbb{R}^2/\mathbb{Z}^2$
  is a smooth 2-manifold.
\end{problem}

\begin{proof} \text{\\}
\end{proof}

\end{document}
