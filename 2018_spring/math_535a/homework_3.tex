\documentclass{article}

\usepackage[margin=1in]{geometry}
\usepackage{amsmath,amsthm,amssymb}
\usepackage{bbm,enumerate,mathtools}
\usepackage[hidelinks]{hyperref}
\usepackage{tikz}
\newenvironment{problem}[2][Problem]{\begin{trivlist}
\item[\hskip \labelsep {\bfseries #1}\hskip \labelsep {\bfseries #2.}]}{\end{trivlist}}
\newenvironment{note}[1][Note.]{\begin{trivlist}
\item[\hskip \labelsep {\bfseries #1}]}{\end{trivlist}}
\newenvironment{problempart}[1]{\begin{trivlist}\item[\textbf{Part #1.}]}{\end{trivlist}}


\begin{document}

\title{Differential Geometry: Homework 3}
\author{Peter Kagey}

\maketitle

% -----------------------------------------------------
% First problem
% -----------------------------------------------------
\begin{problem}{1}
  Let $x_0 \in \mathbb{R}^n$ be a point and $r_1 < r_2$ positive real numbers.
  Construct (with proof) a $C^\infty$ function
  $h\colon \mathbb{R}^n \rightarrow \mathbb{R}$ which equals 1 inside the ball
  of radius $r_1$ around $x_0$ and which equals $0$ outside the ball of radius
  $r_2$ around $x_0$. Such functions are collectively called smooth bump
  functions.
\end{problem}

\begin{proof} \text{} \\
  I'll construct a bump function from $\mathbb{R}\rightarrow\mathbb{R}$
  following the Wikipedia construction\footnote{
    \url{https://en.wikipedia.org/wiki/Non-analytic_smooth_function\#Smooth_transition_functions}
  }.
  First construct a smooth function $f$ such that $f(0) = 0$ and
  $f^{(k)}(0) = 0$ for all $k \in \mathbb{N}$, but $f(1) \not=0$.
  Let \[
    f(x) = \begin{cases}
      0 & x \leq 0\\
      \exp(-1/x^2) & x > 0
    \end{cases}.
  \]
  Now if there is a differentiable function $p_k(x)$ such that \[
  f^{(k)}(x) = \begin{cases}
    0 & x < 0 \\
    p_k(x)\exp(-1/x^2) & x \geq 0 \\
  \end{cases},
  \]
  then by the product rule, \[
  f^{(k+1)}(x) = \begin{cases}
    0 & x < 0 \\
    (p_k(x)2x^{-3} + p_k'(x))\exp(-1/x^2) & x \geq 0 \\
  \end{cases}.
  \] Therefore $p_0(x) = 1$ and $p_{k + 1}(x) = p_k(x)2x^{-3} + p_k'(x)$.
  And an inductive argument shows that by sufficiently many applications of
  L'H\^{o}pital's rule, \[
    \lim_{x\rightarrow 0}f^{(k)}(x) = 0
  \] for all $k \in \mathbb{N}$ as desired.
  Now, given $\alpha < \beta$, let \[
    b_{\alpha,\beta}(x) = \frac{f(x - \alpha)}{f(x - \alpha) + f(\beta - x)}
  \]
  so that $b_{\alpha,\beta}(x_0) = 0$ for $x_0 \leq \alpha$ and
  $b_{\alpha,\beta}(x_1) = f(x_1-\alpha)/f(x_1-\alpha) = 1$ for $x_1 \geq \beta$.

  The function has strictly positive denominator because $f$ is nonnegative,
  $f(x-\alpha) > 0$ for $x > \alpha$, and
  $f(\beta - x) > 0$ for $x < \beta$. Therefore, as the quotient of a smooth
  function and a smooth positive function, $b_{\alpha,\beta}$ is smooth.
%
  Next, because the difference of smooth functions is smooth, let \[
    g_{r_1,r_2}(x) = b_{-r_2,-r_1}(x) - b_{r_1,r_2}(x).
  \]

  In order to make this work in $\mathbb{R}^n$,
  let $d_{x_0}: \mathbb{R}^n \rightarrow \mathbb{R}$ be the Euclidean distance
  between $x$ and $x_0$.

  Since $d_{x_0}$ is smooth, and the composition of smooth functions is smooth,
  our desired function $h: \mathbb{R}^n \rightarrow \mathbb{R}$ can be defined
  by $h = g_{r_1,r_2} \circ d_{x_0}$.
\end{proof}

% -----------------------------------------------------
% Second problem
% -----------------------------------------------------
\pagebreak

\begin{problem}{2}
  Assume $U \in \mathbb{R}^m$ and $V \in \mathbb{R}^n$ are open sets and
  $f\colon U \rightarrow V$ is an immersion.
  Prove the immersion version of the implicit function theorem, assuming only
  the inverse function theorem: there exists a function $G\colon \tilde{V} \rightarrow Z$
  where $Z$ is an open set in $\mathbb{R}^n$.
\end{problem}

\begin{proof}
  % Denote $f$ by $(x_1,\hdots,x_m) \mapsto (f_1,\hdots,f_n)$. We want to construct
  % $G: \mathbb{R}^n \rightarrow \mathbb{R}^n$ such that $G \circ f = \pi_\text{imm}$
  % and $dG$ is an isomorphism.
  The intuition here is that $G^{-1}:\mathbb{R}^n\rightarrow\mathbb{R}^n$
  (the inverse of the diffeomorphism $G$) looks
  very similar to $f:U\rightarrow\mathbb{R}^n$.

  Let $p\in U$, and $\tilde{U}$ a sufficiently small neighborhood of $p$ so that
  $\tilde{V} = f(\tilde{U})$ is locally Euclidean. Let $\pi_\text{sub}:\mathbb{R}^n \rightarrow \mathbb{R}^m$ be the canonical
  submersion that ``removes'' the last $n-m$ coordinates of a point in
  $\mathbb{R}^n$.
  Denote $f\colon \tilde{U}\rightarrow\tilde{V} \subset \mathbb{R}^n$
  coordinatewise by $f(x) = (f_1(x),\hdots,f_n(x))$.
  Then \[
    G^{-1}(x) = (
      \underbrace{f_1 \circ \pi_\text{sub}(x)}_{1},\hdots,
      \underbrace{f_m \circ \pi_\text{sub}(x)}_{m},
      \underbrace{x_{m+1} + f_{m+1} \circ \pi_\text{sub}(x)}_{m+1},\hdots,
      \underbrace{x_n + f_n \circ \pi_\text{sub}(x)}_n),
  \] that is, apply $f$ to the first $m$ coordinates and leave the other $n-m$
  coordinates the same (perhaps after some permutation of coordinates).

  This results in a derivative matrix that looks like (a permutation of) $df$ in
  the first $m$ columns, $I$ in the $n-m \times n-m$ submatrix in the bottom
  right corner, and $0$ elsewhere.

  Because $f$ is injective, the matrix $df_p$ has full rank, so the rows can be
  permuted in such a way that $e_{m+1}$ through $e_n$ are not in the span of
  $df_p$.

  $G^{-1}$ is constructed in such a way that $dG^{-1}$ has rank $n$. Thus by the
  inverse function theorem, $G = (G^{-1})^{-1}$ is a diffeomorphism with the
  property that $G \circ f = \pi_\text{imm}$.
\end{proof}

% -----------------------------------------------------
% Third problem
% -----------------------------------------------------
\pagebreak

\begin{problem}{3}
\end{problem}

\begin{proof} \text{} \\
  \textbf{($\Longrightarrow$)} Assume that there exists an injective immersion
  $f: M^m \rightarrow Y \subset N^n$ onto $Y$, and let $p_Y$ be a point in $Y$.
  The atlas of $N^n$ contains an open set $U$ around $p_Y$ with a map
  $\phi\colon U \rightarrow \mathbb{R}^n$. Similarly because $f$ is injective,
  the preimage of $p_Y$ is a single point $f^{-1}(p_Y) = p_M \in M^m$,
  and there exists a chart
  $(V, \psi\colon V \rightarrow \mathbb{R}^m) \in \mathcal{A}_{M}$ centered at
  $p_M$.\\

  %
  Because (1) $f$ is an injective immersion and (2) $\psi^{-1}$ and $\phi$ are
  continuous bijections,
  $\phi \circ f \circ \psi^{-1}$ is an injective immersion.
  So by the implicit function theorem (immersion version) there exists a
  diffeomorphism $G$ such that $G \circ \phi \circ f \circ \psi^{-1} = \pi_{imm}$.\\

  Therefore let $\phi_G = G \circ \phi$. Then $(U, \phi_G)$ is a chart in $N$'s maximal atlas $\mathcal{A}_N$, and
  $\phi_G \circ f(M^m) = \phi_G(U \cap Y)$ is the image of the model immersion, so
  $\phi_G(U \cap Y) = \phi_G(U) \cap (\mathbb{R}^m \times \{ 0 \})$.
  \\~\\
%
  \textbf{($\Longleftarrow$)} Assume that there exists a subset $Y \subset N^n$
  such that for each point $p \in Y$ there exists a chart $(U_p, \phi_p)$ with
  $\phi_p(U_p\cap Y)$ = $\phi_p(U_p) \cap (\mathbb{R}^m \times \{ 0\})$. Composing
  with the model submersion $\pi_{\text{sub}}: \mathbb{R}^n \rightarrow \mathbb{R}^m$
  yields $\pi_{\text{sub}} \circ \phi_p: Y \rightarrow \mathbb{R}^m$.

  Therefore $Y$ is a manifold with atlas
  $\mathcal{A}_Y = \{\, (U_p \cap Y, \pi_{\text{sub}} \circ \phi_p) \,\}_{p \in Y}$,
  and naturally has embedding via the inclusion map $i\colon Y \rightarrow \mathbb{R}^n$,
  which is (as shown in lecture) a injective immersion.
\end{proof}

% -----------------------------------------------------
% Fourth problem
% -----------------------------------------------------
\pagebreak

\begin{problem}{4}
  Prove the following result: if $f\colon M^m \rightarrow N^n$ is a submersion
  between two smooth manifolds, or more generally if $f$ is simply a smooth map
  and $y \in N$ is a regular value of $f$, then $S = f^{-1}(y)$ has the
  structure of a smooth submanifold of $M$ of dimension $m - n$.
\end{problem}

\begin{proof} \text{}\\
  For each point $p \in S$, there exists a chart that contains $p$:
    $(U_p, \phi_p: M^m \rightarrow \mathbb{R}^m)$.
  Similarly, the maximal atlas of $N$ contains many charts centered at $y$, namely
    $(f(U_p), \psi: N^n \rightarrow \mathbb{R}^n)$.
  Because $f$ is a submersion by hypothesis, the implicit value theorem
  (submersion version) guarantees the existence of a diffeomorphism
  $F_p: \phi_p(U_p) \rightarrow \mathbb{R}^m$ such that
  $\psi \circ f \circ \phi_p^{-1} \circ F_p^{-1}$ is the model submersion
  $\pi_\text{sub}: F_p(\phi_p(U_p)) \rightarrow \mathbb{R}^n$.

  Because $\psi$ is centered at $y$, \[
    F_p\circ\phi_p\circ f^{-1}(y) = \pi_\text{sub}^{-1} \circ \psi(y)
      = \{ 0 \in \mathbb{R}^n\} \times \mathbb{R}^{m - n}
  \]
  Therefore by permuting coordinates and applying ``Definition 2'' of a submanifold,
  $S$ is a submanifold of $M^m$ with atlas \[
    \mathcal{A} = \{\, (
      U_p \cap S,
      (\pi \circ F_p \circ \phi_p)\colon U_p \cap S \rightarrow \mathbb{R}^{m-n}
    ) \,\}
  \] where $\pi$ is the projection onto the last $m - n$ coordinates.
\end{proof}

% -----------------------------------------------------
% Fifth
% -----------------------------------------------------
\pagebreak

\begin{problem}{5}
  Prove that $S^n = \{\,x_1^2+\hdots+x_{n+1}^2=1\,\} \subset \mathbb{R}^{n+1}$
  can be given the structure of an $n$-dimensional manifold by exhibiting it as
  the regular value of some smooth map between manifolds.
\end{problem}

\begin{proof} \text{} \\
  Let $f(x) = x_1^2+\hdots+x_n^2$, which is a smooth map from $\mathbb{R}^{n+1}$
  to $\mathbb{R}$.
  Then \begin{align*}
    df(p) &= \left[
      \frac{\partial f}{\partial x_1}(p) \frac{\partial f}{\partial x_2}(p) \hdots
      \frac{\partial f}{\partial x_{n+1}}(p)
    \right] \\
    &= [2p_1 2p_2 \hdots 2p_{n+1}].
  \end{align*}
  So $df$ has rank $1$ for all $x \in \mathbb{R}^{n + 1} \setminus \{0\}$, so
  $f$ is submersive for all $x \not= 0$. Therefore the preimage $f^{-1}(1)$ has
  the structure of a manifold of dimension $n + 1 - 1 = n$.
\end{proof}

% -----------------------------------------------------
% Sixth
% -----------------------------------------------------
\pagebreak

\begin{problem}{6} $ $\\
  \begin{enumerate}[(a)]
    \item Show that $\operatorname{Sym}(n)$ is a submanifold of
    $M_n(\mathbb{R})$ (and in particular a manifold), and compute its dimension.
    \item Prove that $I \in \operatorname{Sym}(n)$ is a regular value of $\phi$.
    \item Prove that $O(n)$ is a submanifold of $M_n(\mathbb{R})$. What is its dimension?
    \item Prove that $O(n)$ is compact.
  \end{enumerate}
\end{problem}

\begin{proof} \text{\\}
  \begin{problempart}{(a)}
    Let $A_{ij}$ be an $n\times n$ matrix where the $ij$ and $ji$ entries are $1$
    and all other entries are $0$. Then $\operatorname{Sym}(n)$ has basis
    $\{A_{ij} : i \leq j\}$, and has a (smooth) linear isomorphsim
    $\varphi:\operatorname{Sym}(n) \rightarrow \mathbb{R}^{n(n+1)/2}$. Similarly
    let $\psi: M_n(\mathbb{R}) \rightarrow \mathbb{R}^{n^2}$ be the analogous
    linear isomorphism between $ M_n(\mathbb{R})$ and $\mathbb{R}^{n^2}$.

    Then $\psi^{-1} \circ \pi_\text{imm} \circ \varphi: \operatorname{Sym}(n) \hookrightarrow (\mathbb{R})$
    is an embedding which demonstrates that $\operatorname{Sym}(n)$ is a submanifold of $(\mathbb{R})$.
  \end{problempart}
  \begin{problempart}{(b)}\footnote{\url{https://math.stackexchange.com/a/383458/121988}}
    It is sufficient to show that $\phi$ is a submersion for all points
    $p\in\phi^{-1}(I)$. So letting $B$ be an arbitrary invertible matrix in
    $M_n(\mathbb{R})$, and taking the derivative \[
      \phi'_p(A) = \lim_{h \rightarrow 0} \frac{\phi(A + hB) - \phi(A)}{h}B^{-1}.
    \]
    It now must be shown that $\phi'(I)$ is a matrix of rank $m$.
  \end{problempart}
  \begin{problempart}{(c)}
    This follows from the corollary to the implicit function theorem. As shown in
    part (b), $I$ is a regular value of $\phi$, so $\phi^{-1}(I) = O(n)$ is a
    submanifold of $M_n(\mathbb{R})$ of dimension $n^2 - n(n+1)/2 = n(n-1)/2$.
  \end{problempart}
  \begin{problempart}{(d)}
    Because $\phi$ is continuous and the singleton set $\{ I \}$ is closed,
    $\phi^{-1}(I) = O(n)$ is closed as well.
    Since $\phi(A) = 1$ for each $j \in \{ 1,\hdots,n\}$, \[
      \sum_{i=1}^{n} a_{ij}^2 = 1.
    \]
    Since $a_{ij}^2$ is positive, each entry must be strictly less than $1$, and
    therefore $O(n)$ is closed and bounded.
  \end{problempart}
\end{proof}

% -----------------------------------------------------
% Seventh
% -----------------------------------------------------
\pagebreak

\begin{problem}{7} $ $\\
  \begin{problempart}{(a)}
    It is sufficient to show that
    (i) there exists an identity morphism for each object in $\text{Alg}_k$,
    (ii) the composition of two (composable) $k$-algebra homomorphisms is a
      $k$-algebra homomorphism, and
    (iii) $k$-algebra homomorphisms are associative.
    \begin{enumerate}[(i)]
      \item For each object $x\in\operatorname{ob}(\text{Alg}_k)$, let
        $1_X \in \operatorname{hom}_{\text{Alg}_k}(X,X)$ be the
        identity map that sends each element $x \in X$ to itself.
        Clearly $1_X$ is a $k$-algebra homomorphism because $1_X$ is a linear map
        of vector spaces which is compatible with the multiplication maps \[
          1_X(\alpha \cdot \beta) = \alpha \cdot \beta
          = 1_X(\alpha) \cdot 1_X(\beta)
        \] and preserves the identity elements ($1_X(1) = 1.$)\\
        Also if $f \in \operatorname{hom}_{\text{Alg}_k}(Z,X)$ is a $k$-algebra
        homomorphism, \[
          1_X \circ f(\alpha) = 1_X(f(\alpha)) = f(\alpha),
        \] and if $g \in \operatorname{hom}_{\text{Alg}_k}(X,Y)$ is a $k$-algebra
        homomorphism \[
          g\circ 1_X(\alpha) = g(1_X(\alpha)) = g(\alpha).
        \] So indeed $1_X \circ f = f$ and $g \circ 1_X = g$, and therefore $1_X$
        is an identity morphism.
      \item Let
        $f \in \operatorname{hom}_{\text{Alg}_k}(Z,X)$ and
        $g \in \operatorname{hom}_{\text{Alg}_k}(X,Y)$.\\
        Then $g \circ f$ is compatible with the multiplication maps \[
          g \circ f(\alpha \cdot \beta) = g(f(\alpha) \cdot f(\beta))
          = g(f(\alpha)) \cdot g(f(\beta))
          = g\circ f(\alpha) \cdot g\circ f(\beta),
        \] and $g \circ f$ preserves the identity elements \[
          g\circ f(1) = g(1) = 1.
        \] Therefore $g \circ f \in \operatorname{hom}_{\text{Alg}_k}(Z,Y)$.
      \item For each composable triple $f, g$, and $h$ \[
      h \circ (g \circ f) = (h \circ g) \circ f
      \] because associativity is inherited from ordinary composition of functions.
    \end{enumerate}
  \end{problempart}
  \begin{problempart}{(b)}
    It is sufficient to show that
    (i) $C^0(X)$ is a vector space over $\mathbb{R}$,
    (ii) multiplication is bilinear,
    (iii) multiplication is associative, and
    (iv) there is a multiplicative identity.
    \begin{enumerate}[(i)]
      \item $C^0(X)$ is a vector space with pointwise addition and ordinary scalar
        multiplication. In general continuous functions are closed under addition.
        Multiplying (or dividing) every element in an open set $U$ by a scalar $a$
        yields an open set $aU$,
        so if $f^{-1}(U)$ is open for every open set $U$,
        then $(af)^{-1}(aU)$ is also open.
        Therefore $C^0(X)$ is closed under scalar multiplication.\\
        $C^0(X)$ inherits structure from $\mathbb{R}$ so that.
        \begin{itemize}
          \item Associativity and commutativity of addition follow from $\mathbb{R}$.
          \item The zero function (which is proven to be in $C^0(X)$ below) satisfies $f + 0 = f$ for all $f$.
          \item All elements are invertible with respect to addition: $f(x) + (-1)\cdot f(x) = 0$.
          \item The scalar $1 \in \mathbb{R}$ behaves as an identity element for scalar multiplication: $1f = f$.
          \item Everything distributes nicely: $a(b \cdot f) = (ab) \cdot f$, $a(f + g) = af + ag$, and $(a + b)f = af + bf$.
        \end{itemize}
        Lastly, it is important to check that continuous functions remain
        continuous after addition and scalar multiplication.
      \item Bilinearity follows from well-behaved distributivity on $\mathbb{R}.$
        Let $f,g,h\in C^0(X)$ and $\alpha \in \mathbb{R}$, then
      \begin{alignat*}{3}
        &(f + g) \times h &&= (f \cdot h) + (g \cdot h) &&= (f \times h) + (g \times h)\\
        &f \times (g + h) &&= (f \cdot g) + (f \cdot h) &&= (f \times g) + (f \times h)\\
        &(\alpha \cdot f) \times g &&= \alpha \cdot (f \cdot g) &&= \alpha \cdot (f \times g) \\
        &f \times (\alpha \cdot g) &&= \alpha \cdot (f \cdot g) &&= \alpha \cdot (f \times g).
      \end{alignat*}
      \item Associativity follows from associativity on $\mathbb{R}$. \[
        (f \times g) \times h = (f \cdot g) \cdot h = f \cdot (g \cdot h) = f \times (g \times h).
      \]
      \item The multiplicative identity is the constant function $1$. Constant
      functions are in $C^0(X)$ because any open set that contains the constant
      has a preimage of $X$ (which is an open set) and any set that does not
      contain the constant has a preimage of $\emptyset$ (which is also an open set.)
      For each function $f\in C^0(X)$ and each point $x \in X$ \[
        1(x) \times f(x) = 1 \cdot f(x) = f(x) = f(x) \cdot 1 = f(x) \times 1(x),
      \] therefore $1 \times f = f = f \times 1$.
    \end{enumerate}
  \end{problempart}
  \begin{problempart}{(c)}
    Let $f$ be a continuous map $f \in \operatorname{hom}_{\textbf{Top}}(X, Y)$.\\
    In order to prove that $F$ is a contravariant functor, it is sufficient to show that
      (i) $F(f)$ is an $\mathbb{R}$-algebra homomorphism,
      (ii) $F\colon\operatorname{hom}_{\textbf{Top}}(X,Y) \rightarrow \operatorname{hom}_{\textbf{Alg}_\mathbb{R}}(F(Y), F(X))$
        sends identity morphisms to identity morphisms, and
      (iii) $F(f) \circ F(g) = F(g\circ f)$ for composable morphisms.
    \begin{enumerate}[(i)]
      \item{
        Let $g,h \in C^0(Y)$.
        Then $F(f)\colon C^0(Y) \rightarrow C^0(X)$ is an $\mathbb{R}$-algebra
        homomorphism because it is compatible with multiplication maps \[
          F(f)(g \cdot h) = (g \cdot h)\circ f
          = (g \circ f) \cdot (h\circ f)
          = F(f)(g) \cdot F(f)(h)
        \] and because it preserves the (multiplicative) identity element
        (the constant map $1$) \[
          F(f)(y\mapsto 1) = (y \mapsto 1) \circ f = (x \mapsto 1)
        \]
      }
      \item{
        Let $\operatorname{id}_X \in \operatorname{hom}_{\textbf{Top}}(X, X)$.
        Then for all $g \in F(X) = C^0(X)$, \[
          F(\operatorname{id}_X)(g) = g \circ \operatorname{id}_X
          = g
          = \operatorname{id}_{C^0(X)}(g).
        \] Therefore $F(\operatorname{id}_X) = \operatorname{id}_{C^0(X)}$.
      }
      \item{
        Let $g \in \operatorname{hom}_{\textbf{Top}}(Y, Z)$ and $h \in C^0(Z)$.
        Then \[
          (F(f) \circ F(g))(h) = F(f)(F(g)(h))
          = F(f)(h\circ g)
          = (h\circ g\circ f)
          = F(g\circ f)(h),
        \] so $F(f) \circ F(g) = F(g\circ f)$.
        }
    \end{enumerate}
  \end{problempart}
  \begin{problempart}{(d)}
    It is sufficient to show that (i) there exists a functor $\textit{Forget}$ from
    $\textbf{Alg}_\mathbb{R}$ to $\textbf{Set}$ (that maps algebras to their
    underlying sets and algebra homomorphisms to the corresponding map of sets),
    (ii) this functor is faithful, and (iii) this functor is not full.
    \begin{enumerate}
      \item{
        $\mathit{Forget}$ naturally maps identity morphisms to identity morphisms
        because the identity morphism on an $\mathbb{R}$-algebra is the same as
        the identity morphism on the underlying set, namely $x \mapsto x$.
        Composition is compatible because it is the same as the set theoretic
        function composition.
      }
      \item{
        By definition of $k$-algebra homomorphism equality, if two $k$-algebra
        homomorphisms $\phi\colon A \rightarrow B$ and $\psi\colon A \rightarrow B$
        have the same underlying map of sets, then they are equal. Therefore
        $\textit{Forget}$ is faithful.
      }
      \item{
        Let $\phi\colon C^0(X) \rightarrow C^0(X)$ be the function
        $\phi(f) = (x\mapsto x + 1) \circ f$. Then the unity element (the constant
        function $x \mapsto 1$)
        is not preserved under $\phi$, so $\phi$ is not an $\mathbb{R}$-algebra
        homomorphism. Therefore $\phi$ is not in the image of $\textit{Forget}$, and so $\textit{Forget}$ is
        not full.
      }
    \end{enumerate}
  \end{problempart}
\end{problem}

\end{document}
