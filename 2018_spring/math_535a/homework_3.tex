\documentclass{article}

\usepackage[margin=1in]{geometry}
\usepackage{amsmath,amsthm,amssymb}
\usepackage{bbm,enumerate,mathtools}

\newenvironment{problem}[2][Problem]{\begin{trivlist}
\item[\hskip \labelsep {\bfseries #1}\hskip \labelsep {\bfseries #2.}]}{\end{trivlist}}
\newenvironment{note}[1][Note.]{\begin{trivlist}
\item[\hskip \labelsep {\bfseries #1}]}{\end{trivlist}}
\newenvironment{problempart}[1]{\begin{trivlist}\item[\textbf{Part #1.}]}{\end{trivlist}}


\begin{document}

\title{Differential Geometry: Homework 2}
\author{Peter Kagey}

\maketitle

% -----------------------------------------------------
% First problem
% -----------------------------------------------------
\begin{problem}{1}
\end{problem}

\begin{proof} \text{} \\
\end{proof}

% -----------------------------------------------------
% Second problem
% -----------------------------------------------------
\pagebreak

\begin{problem}{2}
\end{problem}

\begin{proof}
\end{proof}

% -----------------------------------------------------
% Third problem
% -----------------------------------------------------
\pagebreak

\begin{problem}{3}
\end{problem}

\begin{proof} \text{} \\
\end{proof}

% -----------------------------------------------------
% Fourth problem
% -----------------------------------------------------
\pagebreak

\begin{problem}{4}
\end{problem}

\begin{proof} \text{}\\
\end{proof}

% -----------------------------------------------------
% Fifth
% -----------------------------------------------------
\pagebreak

\begin{problem}{5}
\end{problem}

\begin{proof} \text{} \\
\end{proof}

% -----------------------------------------------------
% Sixth
% -----------------------------------------------------
\pagebreak

\begin{problem}{6}
\end{problem}

\begin{proof} \text{\\}
\end{proof}

% -----------------------------------------------------
% Seventh
% -----------------------------------------------------
\pagebreak

\begin{problem}{7}
\end{problem}

\begin{proof} \text{}\\
\begin{problempart}{(a)}
  It is sufficient to show that
  (i) there exists an identity morphism for each object in $\text{Alg}_k$,
  (ii) the composition of two (composable) $k$-algebra homomorphisms is a
    $k$-algebra homomorphism, and
  (iii) $k$-algebra homomorphisms are associative.
  \begin{enumerate}[(i)]
    \item For each object $x\in\operatorname{ob}(\text{Alg}_k)$, let
      $1_X \in \operatorname{hom}_{\text{Alg}_k}(X,X)$ be the
      identity map that sends each element $x \in X$ to itself.
      Clearly $1_X$ is a $k$-algebra homomorphism because $1_X$ is a linear map
      of vector spaces which is compatible with the multiplication maps \[
        1_X(\alpha \cdot \beta) = \alpha \cdot \beta
        = 1_X(\alpha) \cdot 1_X(\beta)
      \] and preserves the identity elements ($1_X(1) = 1.$)\\
      Also if $f \in \operatorname{hom}_{\text{Alg}_k}(Z,X)$ is a $k$-algebra
      homomorphism, \[
        1_X \circ f(\alpha) = 1_X(f(\alpha)) = f(\alpha),
      \] and if $g \in \operatorname{hom}_{\text{Alg}_k}(X,Y)$ is a $k$-algebra
      homomorphism \[
        g\circ 1_X(\alpha) = g(1_X(\alpha)) = g(\alpha).
      \] So indeed $1_X \circ f = f$ and $g \circ 1_X = g$, and therefore $1_X$
      is an identity morphism.
    \item Let
      $f \in \operatorname{hom}_{\text{Alg}_k}(Z,X)$ and
      $g \in \operatorname{hom}_{\text{Alg}_k}(X,Y)$.\\
      Then $g \circ f$ is compatible with the multiplication maps \[
        g \circ f(\alpha \cdot \beta) = g(f(\alpha) \cdot f(\beta))
        = g(f(\alpha)) \cdot g(f(\beta))
        = g\circ f(\alpha) \cdot g\circ f(\beta),
      \] and $g \circ f$ preserves the identity elements \[
        g\circ f(1) = g(1) = 1.
      \] Therefore $g \circ f \in \operatorname{hom}_{\text{Alg}_k}(Z,Y)$.
    \item For each composable triple $f, g$, and $h$ \[
    h \circ (g \circ f) = (h \circ g) \circ f
    \] because associativity is inherited from ordinary composition of functions.
  \end{enumerate}
\end{problempart}

\begin{problempart}{(b)}
  It is sufficient to show that
  (i) $C^0(X)$ is a vector space over $\mathbb{R}$,
  (ii) multiplication is bilinear,
  (iii) multiplication is associative, and
  (iv) there is a multiplicative identity.
  \begin{enumerate}[(i)]
    \item $C^0(X)$ is a vector space with pointwise addition and ordinary scalar
      multiplication. In general continuous functions are closed under addition.
      Multiplying (or dividing) every element in an open set $U$ by a scalar $a$
      yields an open set $aU$,
      so if $f^{-1}(U)$ is open for every open set $U$,
      then $(af)^{-1}(aU)$ is also open.
      Therefore $C^0(X)$ is closed under scalar multiplication.\\
      $C^0(X)$ inherits structure from $\mathbb{R}$ so that.
      \begin{itemize}
        \item Associativity and commutivity of addition follow from $\mathbb{R}$.
        \item The zero function (which is proven to be in $C^0(X)$ below) satisfies $f + 0 = f$ for all $f$.
        \item All elements are invertible with respect to addition: $f(x) + (-1)\cdot f(x) = 0$.
        \item The scalar $1 \in \mathbb{R}$ behaves as an identity element for scalar multiplication: $1f = f$.
        \item Everything distributes nicely: $a(b \cdot f) = (ab) \cdot f$, $a(f + g) = af + ag$, and $(a + b)f = af + bf$.
      \end{itemize}
      Lastly, it is important to check that continuous functions remain
      continuous after addition and scalar multiplication.
    \item Bilinearity follows from well-behaved distributivity on $\mathbb{R}.$
      Let $f,g,h\in C^0(X)$ and $\alpha \in \mathbb{R}$, then
    \begin{alignat*}{3}
      &(f + g) \times h &&= (f \cdot h) + (g \cdot h) &&= (f \times h) + (g \times h)\\
      &f \times (g + h) &&= (f \cdot g) + (f \cdot h) &&= (f \times g) + (f \times h)\\
      &(\alpha \cdot f) \times g &&= \alpha \cdot (f \cdot g) &&= \alpha \cdot (f \times g) \\
      &f \times (\alpha \cdot g) &&= \alpha \cdot (f \cdot g) &&= \alpha \cdot (f \times g).
    \end{alignat*}
    \item Associativity follows from associativity on $\mathbb{R}$. \[
      (f \times g) \times h = (f \cdot g) \cdot h = f \cdot (g \cdot h) = f \times (g \times h).
    \]
    \item The multiplicative identity is the constant function $1$. Constant
    functions are in $C^0(X)$ because any open set that contains the constant
    has a preimage of $X$ (which is an open set) and any set that does not
    contain the constant has a preimage of $\emptyset$ (which is also an open set.)
    For each function $f\in C^0(X)$ and each point $x \in X$ \[
      1(x) \times f(x) = 1 \cdot f(x) = f(x) = f(x) \cdot 1 = f(x) \times 1(x),
    \] therefore $1 \times f = f = f \times 1$.
  \end{enumerate}
\end{problempart}
\begin{problempart}{(c)}
  Let $f$ be a continuous map $f \in \operatorname{hom}_{\textbf{Top}}(X, Y)$.\\
  In order to prove that $F$ is a contravariant functor, it is sufficient to show that
    (i) $F(f)$ is an $\mathbb{R}$-algebra homomorphism,
    (ii) $F\colon\operatorname{hom}_{\textbf{Top}}(X,Y) \rightarrow \operatorname{hom}_{\textbf{Alg}_\mathbb{R}}(F(Y), F(X))$
      sends identity morphisms to identity morphisms, and
    (iii) $F(f) \circ F(g) = F(g\circ f)$ for composable morphisms.
\end{problempart}
\begin{enumerate}[(i)]
  \item{
    Let $g,h \in C^0(Y)$.
    Then $F(f)\colon C^0(Y) \rightarrow C^0(X)$ is an $\mathbb{R}$-algebra
    homomorphism because it is compatible with multiplication maps \[
      F(f)(g \cdot h) = (g \cdot h)\circ f
      = (g \circ f) \cdot (h\circ f)
      = F(f)(g) \cdot F(f)(h)
    \] and because it preserves the (multiplicative) identity element
    (the constant map $1$) \[
      F(f)(y\mapsto 1) = (y \mapsto 1) \circ f = (x \mapsto 1)
    \]
  }
  \item{
    Let $\operatorname{id}_X \in \operatorname{hom}_{\textbf{Top}}(X, X)$.
    Then for all $g \in F(X) = C^0(X)$, \[
      F(\operatorname{id}_X)(g) = g \circ \operatorname{id}_X
      = g
      = \operatorname{id}_{C^0(X)}(g).
    \] Therefore $F(\operatorname{id}_X) = \operatorname{id}_{C^0(X)}$.
  }
  \item{
    Let $g \in \operatorname{hom}_{\textbf{Top}}(Y, Z)$ and $h \in C^0(Z)$.
    Then \[
      (F(f) \circ F(g))(h) = F(f)(F(g)(h))
      = F(f)(h\circ g)
      = (h\circ g\circ f)
      = F(g\circ f)(h),
    \] so $F(f) \circ F(g) = F(g\circ f)$.
  }
\end{enumerate}
\begin{problempart}{(d)}
  It is sufficent to show that (i) there exists a functor $\textit{Forget}$ from
  $\textbf{Alg}_\mathbb{R}$ to $\textbf{Set}$ (that maps algebras to their
  underlying sets and algebra homomorphisms to the corresponding map of sets),
  (ii) this functor is faithful, and (iii) this functor is not full.
  \begin{enumerate}
    \item{
      $\mathit{Forget}$ naturally maps identity morphisms to identity morphisms
      because the identity morphism on an $\mathbb{R}$-algebra is the same as
      the identity morphism on the underlying set, namely $x \mapsto x$.
      Composition is compatible because it is the same as the set theoretic
      function composition.
    }
    \item{(?)
      Let $A = \mathbb{R}$ with the ordinary multiplication $a \times b = ab$, and
      let $B = \mathbb{R}$ with the multiplication $a \times b = ab/2$. Then
      $A \not= B$ as $\mathbb{R}$-algebras, but $A = B$ as sets.
    }
    \item{
      Let $\phi\colon C^0(X) \rightarrow C^0(X)$ be the function
      $\phi(f) = (x\mapsto x + 1) \circ f$. Then the unity element (the constant
      function $x \mapsto 1$)
      is not preserved under $\phi$, so $\phi$ is not an $\mathbb{R}$-algebra
      homomorphism. Therefore $\phi$ is not in the image of $\textit{Forget}$, and so $\textit{Forget}$ is
      not full.
    }
  \end{enumerate}
\end{problempart}
\end{proof}

\end{document}
