\documentclass{article}

\usepackage[margin=1in]{geometry} 
\usepackage{amsmath,amsthm,amssymb}
\usepackage{bbm, enumerate}

\newenvironment{problem}[2][Problem]{\begin{trivlist}
\item[\hskip \labelsep {\bfseries #1}\hskip \labelsep {\bfseries #2.}]}{\end{trivlist}}
\newenvironment{note}[1][Note.]{\begin{trivlist}
\item[\hskip \labelsep {\bfseries #1}]}{\end{trivlist}}

\begin{document}

\title{Spring 2015: Real Analysis Graduate Exam}
\author{Peter Kagey}

\maketitle

% -----------------------------------------------------
% First problem
% -----------------------------------------------------
\begin{problem}{1}
Consider the sequence \[
	f_n(x)=
    	\left(1 + \frac{x}{n}\right)^{-n} 
        cos\left(\frac{x}{n}\right), 
	\hspace{10pt}
	n = 1,2,...\ .
\] Evaluate \[
	\lim_{n}\int_{0}^{\infty}f_n(x)\ dx,
\] being careful to justify your answer.
\end{problem}

\begin{proof}
% Note: prove that (1+x/n)^-n < (1 + x/2)^-2.
The boundedness of cosine and non-negativity of $1 + \frac{x}{n}$ imply that for $n > 2$ and $x > 0$ \[
	\left|f_n(x)\right| \leq 
    \left(1 + \frac{x}{n}\right)^{-n} \leq
    \left(1 + \frac{x}{2}\right)^{-2} =
    g(x).
\]

The function $g$ is integrable on $[0, \infty)$ because by the $p$-test \[
	\int_{0}^{\infty} g(x)\ dx = 
    \int_{0}^{\infty} \frac{dx}{1 + x + \frac{x^2}{4}} \leq
    4\int_{0}^{\infty} \frac{dx}{x^2} <
    \infty
\]

Thus the Dominated Convergence Theorem with $g$ allows the limit to be moved inside the integral:
	\[
	    \lim_{n} \int_{0}^{\infty} f_n(x)\ dx =
        \int_{0}^{\infty} \lim_{n} f_n(x)\ dx =
        \int_{0}^{\infty} \frac{
        	\lim_{n} \cos\left(x/n\right)
        }{
	        \lim_{n} \left(1 + x/n\right)^{n}
        } dx =
        \int_{0}^{\infty} \frac{\cos(0)}{e^x} dx =
        \int_{0}^{\infty} e^{-x} dx =
        1
    \]
\end{proof}

% -----------------------------------------------------
% Second problem
% -----------------------------------------------------
\pagebreak

\begin{problem}{2} Suppose that $f: [0, \infty) \rightarrow \mathbb{R}$ is Lebesgue integrable.
	\begin{enumerate}[(i)]
    	\item Show that there exists a sequence $x_n \rightarrow \infty$ such that $f(x_n) \rightarrow 0$.
        \item Is it true that $f(x)$ must converge to $0$ as $x \rightarrow \infty$? Give a proof or counterexample.
        \item Suppose additionally that $f$ is differentiable and $f'(x) \rightarrow 0$ as $x \rightarrow \infty$. Is it true that $f(x)$ must converge to $0$ as $x \rightarrow \infty$? Give a proof or counterexample.
    \end{enumerate}
\end{problem}

\begin{proof}
	\begin{enumerate}[(i)]
    	\item Suppose that there was no point $x_\varepsilon$ such that $|f(x_\varepsilon)| < \varepsilon$. Then $|f| \geq \varepsilon$ for all $\varepsilon > 0$. This means that \[
        	\int_{0}^{\infty} |f|\ dm \geq \int_{0}^{\infty} \varepsilon\ dm = \epsilon m((0, \infty)) = \infty,
        \] which is a contradiction, because $f$ is integrable by hypothesis.
        Therefore for each $\varepsilon > 0$, there exists some $x_\varepsilon$ such that $f(x_\varepsilon) < \varepsilon$. Take a sequence such that $|f(x_n)| < 1/n$ for all $n$. This sequence converges to 0.
        \item Let $f = \mathbbm{1}_{\mathbb{Q}}$ be the indicator function for the rational numbers. Then $f$ does not converge to $0$, but it is integrable: \[
        	\int_0^\infty f\ dm = 
            m(\mathbb{Q}) =
            0
        \]
        \item The idea is that if $f \not\rightarrow 0$, then there exists some $\varepsilon$ such that for any $M$, there exists $x_0 > M$ such that $f(x_0) > \varepsilon$. Choose $M$ large enough that $|f'(x)| < 2$, then the area from $x_0$ to the next $x$-intercept must be greater than $\varepsilon$. But this means that the integral $\int_0^\infty |f|\ dm = \sum_{i = 1}^{\infty} \varepsilon = \infty$.	
    \end{enumerate}
\end{proof}

% -----------------------------------------------------
% Third problem
% -----------------------------------------------------
\pagebreak

\begin{problem}{3} Define $f_n(x) = a\exp(-nax)-b\exp(-nbx)$ where $0 < a < b$.
	\begin{enumerate}[(i)]
    	\item Show that \[
        	\sum_{n = 1}^{\infty} \int_{0}^{\infty} f_n(x) dx = 0
        \] and \[
        	\int_{0}^{\infty} \sum_{n = 1}^{\infty} f_n(x) dx = \log(b/a).
        \]
        \item What can you deduce about the value of \[
	        \int_{0}^{\infty} \sum_{n = 1}^{\infty} \left|f_n(x)\right| dx?
        \]
    \end{enumerate}
\end{problem}

\begin{proof}
	\begin{enumerate}[(i)]
    	\item Notice that \[
        	\int_0^\infty f_n(x)\ dx = 
            \frac{\exp(-nbx)}{n} - \frac{\exp(-nax)}{n} \Big|_{0}^{\infty} = 
            0 - \left(\frac{-1}{n} + \frac{1}{n}\right) = 
            0
        \] because $exp(-kx) \rightarrow 0$ as $x \rightarrow \infty$ for $k > 0$, and $\exp(0) = 1$.
        (from Daniel Douglas)
        \item
    \end{enumerate}
\end{proof}

% -----------------------------------------------------
% Fourth problem
% -----------------------------------------------------
\pagebreak

\begin{problem}{4} Assume that $f$ is integrable on $[0, 1]$ with respect to the Lebesgue measure $m$, and let $F(x) = \int_{0}^{x}f(t) dt$. Assume that $\phi: \mathbb{R} \rightarrow \mathbb{R}$ is Lipschitz, i.e., there exists a constant $C \geq 0$ such that \[
	\phi(x_1) - \phi(x_2) \leq C(x_1 - x_2),
    \hspace{10pt} x_1, x_2 \in \mathbb{R}.
\]
Prove that there exists a function $g$ which is integrable on $[0, 1]$ such that $\phi(F(x)) = \int_{0}^{x} g(t) \ dt$ for $x \in [0, 1]$.
\end{problem}

\begin{note} This is similar to problem \# from the Spring 20\#\# Real Analysis Exam.
\end{note}

\begin{proof}
Replace this text with the details of your proof or solution.
\end{proof}

\end{document}
