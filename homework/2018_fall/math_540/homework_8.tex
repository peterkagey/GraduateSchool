\documentclass{article}

\usepackage[margin=1in]{geometry}
\usepackage{amsmath,amsthm,amssymb}
\usepackage{bbm,enumerate,mathtools,multicol}
\usepackage[shortlabels]{enumitem}
\usepackage[hidelinks]{hyperref}
\usepackage{tikz}
\usetikzlibrary{matrix, arrows}

\newenvironment{problem}[2][Problem]{\begin{trivlist}
\item[\hskip \labelsep {\bfseries #1}\hskip \labelsep {\bfseries #2.}]}{\end{trivlist}}
\newenvironment{solution}[1][Solution.]{\begin{trivlist}
\item[\hskip \labelsep {\bfseries #1}]}{\end{trivlist}}
\newenvironment{problempart}[1]{\begin{trivlist}\item[\textbf{Part #1.}]}{\end{trivlist}}

\newcommand{\fn}[3]{#1 \colon #2 \rightarrow #3}
\newcommand{\inv}[1]{#1^{-1}}
\newcommand{\set}[1]{\left\{ #1 \right\}}
\newcommand{\id}[1]{\operatorname{Id}_{#1}}
\DeclareMathOperator{\im}{Im}

\begin{document}

\title{Topology: Homework 8}
\author{Peter Kagey}

\maketitle

% -----------------------------------------------------
% First problem
% -----------------------------------------------------
\begin{problem}{1} \text{} \\
  \begin{enumerate}[a.]
    \item Express the map
    $\fn{\delta_i \circ F_j}{\Delta_n}{\Delta_n \times [0, 1]}$
    in terms of
    $(F_{j'} \times \id{[0,1]}) \circ \delta_{i'}$,
    $\delta_{i\pm1} \circ F_{j'}$, $i_0 = \id{\Delta_n} \times ~ 0$, or $i_1 = \id{\Delta_n} \times ~ 1$.
    \item Let $\fn{f_0, f_1}{X}{Y}$ be homotopic by a homotopy
    $\fn{H}{X \times [0, 1]}{Y}$. Define a linear map $\fn{K_n}{C_n(X)}{C_{n+1}(Y)}$ by \[
      K_n(\sigma) = \sum_{i=0}^{n} (-1)^{i} H\circ(\sigma \times \id{[0,1]}) \circ \delta_i
    \] for every simplex $\sigma \in C_n(X)$.\\
    Show that \[
      \partial_{n+1} \circ K_n + K_{n - 1} \circ \partial_n
      = C_n(f_1) - C_n(f_0).
    \]
  \end{enumerate}
\end{problem}

\begin{proof} \text{} \\
  \begin{enumerate}[a.]
    \item There are six cases to consider:
    \begin{enumerate}[(i)]
      \item
      When $i = j = 0$ \begin{alignat*}{2}
        (t_0, t_1, \hdots, t_n)
        &\xmapsto{F_0} (0, t_0, t_1, \hdots, t_n)
        &&\xmapsto{\delta_0} ((t_0, t_1, \hdots, t_n), \underbrace{t_0 + t_1 + \hdots + t_n}_1) \\
        (t_0, t_1, \hdots, t_n) & &&\xmapsto{i_1} ((t_0, t_1, \hdots, t_n), 1)
      \end{alignat*}
      so $\delta_i \circ F_j = i_1$.
      \item
      When $i = j > 0$ \begin{alignat*}{2}
        (t_0, t_1, \hdots, t_n)
        &\xmapsto{F_i} (t_0, t_1, \hdots, t_{i-1}, 0, t_i, \hdots, t_n)
        \xmapsto{\delta_i} &&((t_0, t_1, \hdots, t_{i-1}, 0 + t_i, \hdots t_n), t_i + \hdots + t_n) \\
        (t_0, t_1, \hdots, t_n)
        &\xmapsto{F_i} (t_0, t_1, \hdots, t_{i-1}, 0, t_i, \hdots, t_n)
        \xmapsto{\delta_{i-1}} &&((t_0, t_1, \hdots, t_{i-1} + 0, t_i, \hdots t_n), 0 + t_i + \hdots + t_n)
      \end{alignat*}
      so $\delta_i \circ F_i = \delta_{i-1} \circ F_i$.
      \item
      When $j - 1 = i < n$ \begin{alignat*}{2}
        (t_0, t_1, \hdots, t_n)
        &\xmapsto{F_j} (t_0, t_1, \hdots, t_{j-1}, 0, t_j, \hdots, t_n)
        \xmapsto{\delta_{j-1}} &&((t_0, t_1, \hdots, t_{j-1} + 0, t_j, \hdots t_n), 0 + t_j + \hdots + t_n) \\
        (t_0, t_1, \hdots, t_n)
        &\xmapsto{F_j} (t_0, t_1, \hdots, t_{j-1}, 0, t_{j}, \hdots, t_n)
        \xmapsto{\delta_j} &&((t_0, t_1, \hdots, t_{j-1}, 0 + t_j, \hdots t_n), t_j + \hdots + t_n)
      \end{alignat*}
      so $\delta_i \circ F_j = \delta_{i+1} \circ F_j$.
      \item
      When $j - 1 = i = n$ \begin{alignat*}{2}
        (t_0, t_1, \hdots, t_n)
        &\xmapsto{F_{n+1}} (t_0, t_1, \hdots, t_n, 0)
        &&\xmapsto{\delta_n} ((t_0, t_1, \hdots t_n), 0) \\
        (t_0, t_1, \hdots, t_n) &
        &&\xmapsto{i_0} ((t_0, t_1, \hdots, t_n), 0)
      \end{alignat*}
      so $\delta_n \circ F_{n+1} = i_0$.
      \item
      When $j - 1 > i$ \begin{alignat*}{2}
        (t_0, t_1, \hdots, t_n)
        &\xmapsto{F_j} &&(t_0, \hdots, t_{j-1}, 0, t_j, \hdots, t_n) \\
        &\xmapsto{\delta_i} && (
          (t_0, \hdots, t_i + t_{i + 1}, \hdots, t_{j-1}, 0, t_j, \hdots, t_n),
          t_{i+1} + \hdots + t_n
        ) \\
        (t_0, t_1, \hdots, t_n)
        &\xmapsto{\delta_i} &&
        ((t_0, \hdots, t_i + t_{i + 1}, \hdots, t_n), t_{i + 1} \hdots t_n)\\
        &\xmapsto{F_{j-1} \times \id{[0,1]}} &&(
          (t_0, \hdots, t_i + t_{i + 1}, \hdots, t_{j-1}, 0, t_j, \hdots, t_n),
          t_{i+1} + \hdots + t_n
        )
      \end{alignat*}
      so $\delta_i \circ F_j = (F_{j-1} \times \id{[0,1]}) \circ \delta_i$.
      \item
      When $i > j$ \begin{alignat*}{2}
        (t_0, t_1, \hdots, t_n)
        &\xmapsto{F_j} &&(t_0, \hdots, t_{j-1}, 0, t_j, \hdots, t_n) \\
        &\xmapsto{\delta_i} && (
          (t_0, \hdots, t_{j-1}, 0, t_j, \hdots, t_{i-1} + t_i, \hdots, t_n),
          t_i + \hdots + t_n
        ) \\
        (t_0, t_1, \hdots, t_n)
        &\xmapsto{\delta_{i-1}} &&
        ((t_0, \hdots, t_{i-1} + t_i, \hdots, t_n), t_i \hdots t_n)\\
        &\xmapsto{F_j \times \id{[0,1]}} &&(
          (t_0, \hdots, t_{j-1}, 0, t_j, \hdots, t_{i-1} + t_i, \hdots, t_n),
          t_i + \hdots + t_n
        )
      \end{alignat*}
      so $\delta_i \circ F_j = (F_j \times \id{[0,1]}) \circ ~ \delta_{i-1}$.
    \end{enumerate}
    \item The two terms of the sum can be written as \[
      \partial_{n+1}(K_n(\sigma))
      = \partial_{n+1} \left(\sum_{i=0}^{n} (-1)^{i} H\circ(\sigma \times \id{[0,1]}) \circ \delta_i\right)
      = \sum_{j=0}^{n+1}\sum_{i=0}^{n} (-1)^{i + j} H\circ(\sigma \times \id{[0,1]}) \circ \delta_i \circ F_j
    \] and \begin{align}
      K_{n-1}(\partial_{n}(\sigma))
      &= \sum_{i=0}^{n-1} (-1)^{i} H\circ(\partial_n(\sigma) \times \id{[0,1]}) \circ \delta_i \\
      &= \sum_{i=0}^{n-1} (-1)^{i} H\circ\left(\sum_{j=0}^n(-1)^j\sigma \circ F_j \times \id{[0,1]}\right) \circ \delta_i \\
      &= \sum_{i=0}^{n-1}\sum_{j=0}^n (-1)^{i+j} H\circ\left(\sigma \circ F_j \times \id{[0,1]}\right) \circ \delta_i \\
      &= \sum_{i=0}^{n-1}\sum_{j=0}^n (-1)^{i+j} H\circ(\sigma \times \id{[0,1]}) \circ (F_j \times \id{[0,1]}) \circ \delta_i
    \end{align}
    This final sum (4) can be split based on cases.
    \begin{align}
      K_{n-1}(\partial_{n}(\sigma))
      &=\sum_{i=0}^{n-1}\sum_{j=0}^n (-1)^{i+j} H\circ(\sigma \times \id{[0,1]}) \circ (F_j \times \id{[0,1]}) \circ \delta_i \\
      % i + 1 > j
      &= \sum_{i=0}^{n-1}\sum_{j=0}^i (-1)^{i+j} H\circ(\sigma \times \id{[0,1]}) \circ (F_j \times \id{[0,1]}) \circ \delta_i \\
      &\hspace{1cm}+ \sum_{i=0}^{n-1}\sum_{j=i+1}^n (-1)^{i+j} H\circ(\sigma \times \id{[0,1]}) \circ (F_j \times \id{[0,1]}) \circ \delta_i
    \end{align}
    Then these sums can be reindexed based on the above identities \begin{align}
      K_{n-1}(\partial_{n}(\sigma))
      &= \sum_{i=1}^{n}\sum_{j=0}^{i-1} (-1)^{i-1+j} H\circ(\sigma \times \id{[0,1]}) \circ (F_j \times \id{[0,1]}) \circ \delta_{i-1}\\
      &\hspace{1cm}+ \sum_{i=0}^{n-1}\sum_{j=i+2}^{n+1} (-1)^{i+j-1} H\circ(\sigma \times \id{[0,1]}) \circ (F_{j-1} \times \id{[0,1]}) \circ \delta_i\\
      &= \sum_{i=1}^{n}\sum_{j=0}^{i-1} (-1)^{i-1+j} H\circ(\sigma \times \id{[0,1]}) \circ \delta_i \circ F_j\\
      &\hspace{1cm}+ \sum_{i=0}^{n-1}\sum_{j=i+2}^{n+1} (-1)^{i+j-1} H\circ(\sigma \times \id{[0,1]}) \circ \delta_i \circ F_j
    \end{align}
    Then adding this to $\partial_{n+1}(K_n(\sigma))$ yields \begin{align}
      \partial_{n+1}(K_n(\sigma)) + K_{n-1}(\partial_{n}(\sigma))
      &= \sum_{j=0}^{n+1}\sum_{i=0}^{n} (-1)^{i + j} H\circ(\sigma \times \id{[0,1]}) \circ \delta_i \circ F_j \\
      &\hspace{1cm}- \sum_{i=1}^{n}\sum_{j=0}^{i-1} (-1)^{i+j} H\circ(\sigma \times \id{[0,1]}) \circ \delta_i \circ F_j \\
      &\hspace{1cm}- \sum_{i=0}^{n-1}\sum_{j=i+2}^{n+1} (-1)^{i+j} H\circ(\sigma \times \id{[0,1]}) \circ \delta_i \circ F_j.
    \end{align}
    By swapping the sums on the first term (12), and splitting the second (13) and third (14)
    sums to their overlap $(1 \leq i \leq n-1)$ gives \begin{align}
      \partial_{n+1}(K_n(\sigma)) + K_{n-1}(\partial_{n}(\sigma))
      &= \sum_{i=0}^{n}\sum_{j=0}^{n+1} (-1)^{i + j} H\circ(\sigma \times \id{[0,1]}) \circ \delta_i \circ F_j \\
      &\hspace{1cm}- \sum_{i=1}^{n-1}\sum_{j=0}^{i-1} (-1)^{i+j} H\circ(\sigma \times \id{[0,1]}) \circ \delta_i \circ F_j \\
      &\hspace{1cm}- \sum_{j=0}^{n-1} (-1)^{n+j} H\circ(\sigma \times \id{[0,1]}) \circ \delta_n \circ F_j \\
      &\hspace{1cm}- \sum_{j=2}^{n+1} (-1)^{j} H\circ(\sigma \times \id{[0,1]}) \circ \delta_0 \circ F_j\\
      &\hspace{1cm}- \sum_{i=1}^{n-1}\sum_{j=i+2}^{n+1} (-1)^{i+j} H\circ(\sigma \times \id{[0,1]}) \circ \delta_i \circ F_j.
    \end{align}
    Then the second (16)and fifth (19) terms sum over $1 \leq i \leq n - 1$, but miss
    values where $j = 1$ or $j = i + 1$, so we'll add these back in \begin{align}
      \partial_{n+1}(K_n(\sigma)) + K_{n-1}(\partial_{n}(\sigma))
      &= \sum_{i=0}^{n}\sum_{j=0}^{n+1} (-1)^{i + j} H\circ(\sigma \times \id{[0,1]}) \circ \delta_i \circ F_j \\
      &\hspace{1cm}- \sum_{i=1}^{n-1}\sum_{j=0}^{n+1} (-1)^{i+j} H\circ(\sigma \times \id{[0,1]}) \circ \delta_i \circ F_j \\
      &\hspace{1cm}+ \sum_{i=1}^{n-1}\sum_{j=i}^{i+1} (-1)^{i+j} H\circ(\sigma \times \id{[0,1]}) \circ \delta_i \circ F_j \\
      &\hspace{1cm}- \sum_{j=0}^{n-1} (-1)^{n+j} H\circ(\sigma \times \id{[0,1]}) \circ \delta_n \circ F_j \\
      &\hspace{1cm}- \sum_{j=2}^{n+1} (-1)^{j} H\circ(\sigma \times \id{[0,1]}) \circ \delta_0 \circ F_j.
    \end{align}
    Subtracting (21) from (20), and
    splitting (22) into two sums gives \begin{align}
      \partial_{n+1}(K_n(\sigma)) + K_{n-1}(\partial_{n}(\sigma))
      &= \sum_{j=0}^{n+1} (-1)^{j} H\circ(\sigma \times \id{[0,1]}) \circ \delta_0 \circ F_j \\
      &\hspace{1cm}+ \sum_{j=0}^{n+1} (-1)^{n + j} H\circ(\sigma \times \id{[0,1]}) \circ \delta_n \circ F_j \\
      &\hspace{1cm}+ \sum_{i=1}^{n-1} H\circ(\sigma \times \id{[0,1]}) \circ \delta_i \circ F_i \\
      &\hspace{1cm}- \sum_{i=1}^{n-1} H\circ(\sigma \times \id{[0,1]}) \circ \delta_i \circ F_{i+1} \\
      &\hspace{1cm}- \sum_{j=0}^{n-1} (-1)^{n+j} H\circ(\sigma \times \id{[0,1]}) \circ \delta_n \circ F_j \\
      &\hspace{1cm}- \sum_{j=2}^{n+1} (-1)^{j} H\circ(\sigma \times \id{[0,1]}) \circ \delta_0 \circ F_j.
    \end{align} Then subtracting (30) from (25) and (29) from (26) gives \begin{align}
      \partial_{n+1}(K_n(\sigma)) + K_{n-1}(\partial_{n}(\sigma))
      &= \sum_{j=0}^{1} (-1)^{j} H\circ(\sigma \times \id{[0,1]}) \circ \delta_0 \circ F_j \\
      &\hspace{1cm}+ \sum_{j=n}^{n+1} (-1)^{n + j} H\circ(\sigma \times \id{[0,1]}) \circ \delta_n \circ F_j \\
      &\hspace{1cm}+ \sum_{i=1}^{n-1} H\circ(\sigma \times \id{[0,1]}) \circ \delta_i \circ F_i \\
      &\hspace{1cm}- \sum_{i=1}^{n-1} H\circ(\sigma \times \id{[0,1]}) \circ \delta_i \circ F_{i+1}.
    \end{align}
    Then using the substitution
    $\delta_i \circ F_{i+1} = \delta_{i+1} \circ F_{i+1}$
    when $i < n$, subtracting $(34)$ from $(33)$, and writing everything
    termwise gives \begin{align}
      \partial_{n+1}(K_n(\sigma)) + K_{n-1}(\partial_{n}(\sigma))
      &= H\circ(\sigma \times \id{[0,1]}) \circ \underbrace{\delta_0 \circ F_0}_{i_1} \\
      &\hspace{1cm}- H\circ(\sigma \times \id{[0,1]}) \circ \delta_0 \circ F_1 \\
      &\hspace{1cm}+ H\circ(\sigma \times \id{[0,1]}) \circ \delta_n \circ F_n \\
      &\hspace{1cm}- H\circ(\sigma \times \id{[0,1]}) \circ \underbrace{\delta_n \circ F_{n+1}}_{i_0} \\
      &\hspace{1cm}+ H\circ(\sigma \times \id{[0,1]}) \circ \underbrace{\delta_1 \circ F_1}_{\delta_0 \circ F_1}\\
      &\hspace{1cm}- H\circ(\sigma \times \id{[0,1]}) \circ \delta_{n} \circ F_{n}.
    \end{align}
    So $(36)$ cancels with $(39)$ and $(37)$ cancels with $(40)$ leaving \begin{align*}
      \partial_{n+1}(K_n(\sigma)) + K_{n-1}(\partial_{n}(\sigma))
      &= H\circ(\sigma \times \id{[0,1]}) \circ i_1
      - H\circ(\sigma \times \id{[0,1]}) \circ i_0\\
      &= H\circ(\sigma \times 1)
      - H\circ(\sigma \times 0) \\
      &= C_n(f_1)(\sigma) - C_n(f_0)(\sigma).
    \end{align*}
  \end{enumerate}

\end{proof}
\pagebreak
% -----------------------------------------------------
% Second problem
% -----------------------------------------------------
\begin{problem}{2} \text{} \\
  Let $X$ be a topological space. For all $n$, let $C_n(X)$ be the usual
  $R$-module of singular $n$-chains in $X$ with coefficients in the ring $R$.
  In particular, $C_0(X) = \set{\sum_{i=1}^k a_ix_i : a_i \in R, x_i \in X}$
  consists of all linear combinations of points in $X$. Consider the
  homomorphism $\fn{\widetilde \partial_0}{C_0(X)}{R}$ defined by the property
  that $
    \widetilde \partial_0\left(\sum_{i=1}^k a_ix_i\right) = \sum_{i=1}^k a_i
  $
  \\
  For $n \in \mathbb Z$, define \[
    \widetilde C_n(X) = \begin{cases}
      C_n(X) & n \geq 0 \\
      R      & n    = -1 \\
      0      & n \leq -2
    \end{cases}
  \] and define $\fn{\widetilde\partial_n}{\widetilde C_n(X)}{\widetilde C_{n-1}(X)}$ by
  the property that \[
    \widetilde \partial_n = \begin{cases}
      \partial_n        & n > 0 \\
      \widetilde \partial_0 & n = 0 \\
      0                 & n < 0.
    \end{cases}
  \]
  Finally, let $\widetilde H_n(X) = \ker(\widetilde\partial_n)/\im(\widetilde\partial_{n+1})$
  \begin{enumerate}[a.]
    \item Show that $\widetilde H_n(X) = H_n(X)$ when $n \neq 0$.
    \item Show that $\widetilde H_0(X) = 0$ if $X$ is path connected.
    \item Show that $\widetilde H_0(X) \cong R^{n-1}$ if $X$ has $n$
    path-connected components.
  \end{enumerate}
\end{problem}

\begin{proof} \text{} \\
  \begin{enumerate}[a.]
    \item For $n > 0$, $\widetilde\partial_n = \partial_n$ and
    $\widetilde C_n(X) = C_n(X)$, so in particular,
    $\ker(\widetilde\partial_n) = \ker(\partial_n)$ and
    $\im(\widetilde\partial_n) = \im(\partial_n)$. Therefore \[
      \widetilde H_n(X)
      = \ker(\widetilde\partial_n)/\im(\widetilde\partial_{n+1})
      = \ker(\partial_n)/\im(\partial_{n+1})
      = H_n(X)
    \] for $n > 0$.
    \\
    When $n < 0$, $\widetilde\partial_n = 0$, so
    $\ker(\widetilde\partial_n) = \ker(0) = 0$. In particular,
    $\widetilde H_n(X) = \ker(0)/\im(\widetilde\partial_{n+1}) = 0 = H_n(X)$
    with the last equality by convention.
    \item %$\widetilde H_0(X) = \ker(\widetilde \partial_0)/\im(\partial_1)$.
    First note that \[
      \partial_1(\sigma_i) = \sum_{j=0}^1 (-1)^j \sigma_i \circ F_j = \sigma_i \circ F_0 - \sigma_i \circ F_1
    \] so if $\sigma_i(0, 1) = x_0$ and $\sigma_i(1, 0) = x_1$, then
    $\partial_1(\sigma_i)$ is the constant map from the $0$-simplex to the
    difference of the end points of $\sigma_i$, namely $1 \mapsto x_0 - x_1$.
    \\
    Let $c = \sum_{i} c_i \sigma_i$ be an element of $C_1(X)$. Then \[
      \partial_1\!\!\left(\sum_{i} c_i \sigma_i\right)
      = \sum_{i} c_i \partial_1(\sigma_i)
      = \sum_{i} c_i (x_{i,0} - x_{i,1}).
    \]
    Then any element in $\im(\partial_1)$ maps to $0$ under
    $\widetilde\partial_0$ \begin{align*}
      \widetilde\partial_0\!\!\left(\sum_{i} c_i (x_{i,0} - x_{i,1})\right)
      &= \widetilde\partial_0\!\!\left(\sum_{i} c_i x_{i,0} - \sum_{i}c_i x_{i,1}\right) \\
      &= \widetilde\partial_0\!\!\left(\sum_{i} c_i x_{i,0}\right) - \widetilde\partial_0\!\!\left(\sum_{i}c_i x_{i,1}\right) \\
      &= \sum_{i} c_i - \sum_{i} c_i \\
      &= 0,
    \end{align*}
    which shows that $\im(\partial_1) \subset \ker(\widetilde \partial_0)$.
    \\
    Next I will show that $\im(\partial_1) \subset \ker(\widetilde \partial_0)$:
    \\
    Let $c \in \ker(\widetilde\partial_0) \subset C_0(X)$ be
    written as $c = \sum_i c_i x_i$. Then, since $X$ is path-connected,
    for each $x_i$, there exists a path path $\fn{\sigma_i}{\Delta_1}{X}$ from
    $x_i$ to some designated basepoint $x_0$.
    Then let $c_1 \in C_1(X)$ be defined as $\sum_i c_i \sigma_i$. Then \begin{align*}
      \partial_1(c_1)
      &= \partial_1\!\!\left(\sum_i c_i \sigma_i\right) \\
      &= \sum_i c_i (x_i - x_0) \\
      &= \sum_i c_i x_i - \sum_i c_i x_0 \\
      &= \underbrace{\sum_i c_i x_i}_c - \underbrace{\left(\sum_i c_i\right)}_0 x_0 \\
      &= c.
    \end{align*}
    Since each set contains the other,
    $\im(\partial_1) = \ker(\widetilde \partial_0)$ and
    $\widetilde H_0(X) = \ker(\widetilde \partial_0)/\im(\partial_1) = 0$.
    \item Suppose $X$ has $n$ path-connected components $X_1, X_2, \hdots, X_n$.
    Then $C_n(X)$ is freely generated \[
      C_n(X) = \bigoplus_{i=1}^n C_n(X_i).
    \] The above result that
    $\im(\partial_1) \subset \ker(\widetilde\partial_0)$ did not depend on
    path-connectedness, so it still holds.
    Also the image of $\partial_1$ can be written as the disjoint union of
    boundaries of components \[
      \im(\partial_1) = \coprod_{i=1}^n \im(\partial_1|_{C_1(X_i)}).
    \]
    I'm not sure what to do next with this: \[
      \bigoplus_{i=1}^n C_n(X_i) \xrightarrow{\partial_1}
      \bigoplus_{i=1}^n C_n(X_i) \xrightarrow{\widetilde\partial_0}
      R.
    \]
    % Let $x_0^{(1)}, x_0^{(2)}, \hdots, x_0^{(n)}$ be basepoints in the components
    % $X_1, X_2, \hdots, X_n$. And suppose by induction (with base case of $n=1$
    % shown in the above part) that if $X'$ has $n-1$ components, that $H_n$ can
    % be written with a basis $\sigma_1, \sigma_2, \hdots, \sigma_{n-1}$. Then
    % for some $c = \sum_i c_ix_i$ in $\ker(\widetilde\partial_0)$ we can write
    % \begin{align*}
    %   \partial_1(c' - c_n)
    %   &= \partial_1(c') - \partial_1(c_n) \\
    %   &= \partial_1(c') - \partial_1(c_n)
    % \end{align*}
  \end{enumerate}
\end{proof}
\end{document}
