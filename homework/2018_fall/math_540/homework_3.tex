\documentclass{article}

\usepackage[margin=1in]{geometry}
\usepackage{amsmath,amsthm,amssymb}
\usepackage{bbm,enumerate,mathtools,multicol}
\usepackage[hidelinks]{hyperref}
\usepackage{tikz}
\usetikzlibrary{matrix, arrows}

\newenvironment{problem}[2][Problem]{\begin{trivlist}
\item[\hskip \labelsep {\bfseries #1}\hskip \labelsep {\bfseries #2.}]}{\end{trivlist}}
\newenvironment{solution}[1][Solution.]{\begin{trivlist}
\item[\hskip \labelsep {\bfseries #1}]}{\end{trivlist}}
\newenvironment{problempart}[1]{\begin{trivlist}\item[\textbf{Part #1.}]}{\end{trivlist}}

\begin{document}

\title{Topology: Homework 3}
\author{Peter Kagey}

\maketitle

% -----------------------------------------------------
% First problem
% -----------------------------------------------------
\begin{problem}{1} \text{} \\
  In the free group $F_2 = \langle a, b; \rangle$, consider the elements \begin{align*}
      A &= b^2 a^{-1} b^{-3} a b^{-1} a b^{-2} \\
      B &= b^2 a^{-1} b a^{-1} b a b^{2} \\
      C &= b^{-2} a^{-1} b^2 a^3 b \\
      D &= b^{-2} a^{-1} b^{2} a b^{-2} a^3
  \end{align*}
  \begin{enumerate}[(a)]
    \item Compute $(AB)C$ and $A(BC)$ in the given order, and verify that
      $(AB)C = A(BC)$.
    \item Compute $(AB)D$ and $A(BD)$ in the given order, and verify that
      $(AB)D = A(BD)$.
  \end{enumerate}
\end{problem}

\begin{proof} \text{} \\
  \begin{enumerate}[(a)]
    \item First computing $AB$ gives \begin{align*}
      AB &= b^2 a^{-1} b^{-3} a b^{-1} a \underbrace{b^{-2} \cdot b^2}_{\operatorname{id}_B} a^{-1} b a^{-1} b a b^{2} \\
         &= b^2 a^{-1} b^{-3} a b^{-1} \underbrace{a a^{-1}}_{\operatorname{id}_A} b a^{-1} b a b^{2} \\
         &= b^2 a^{-1} b^{-3} a \underbrace{b^{-1} b}_{\operatorname{id}_B} a^{-1} b a b^{2} \\
         &= b^2 a^{-1} b^{-3} \underbrace{a a^{-1}}_{\operatorname{id}_A} b a b^{2} \\
         &= b^2 a^{-1} \underbrace{b^{-3} b}_{b^{-2}} a b^{2} \\
         &= b^2 a^{-1} b^{-2} a b^{2}.
    \end{align*}
    Then computing $(AB)C$ gives \begin{align*}
      (AB)C &= b^2 a^{-1} b^{-2} a \underbrace{b^{2} \cdot b^{-2}}_{\operatorname{id}_B} a^{-1} b^2 a^3 b \\
         &= b^2 a^{-1} b^{-2} \underbrace{a a^{-1}}_{\operatorname{id}_A} b^2 a^3 b \\
         &= b^2 a^{-1} \underbrace{b^{-2} b^2}_{\operatorname{id}_B} a^3 b \\
         &= b^2 \underbrace{a^{-1} a^3}_{a^2} b \\
         &= b^2 a^2 b.
    \end{align*}
    Similarly, computing $BC$ gives \begin{align*}
      BC &= b^2 a^{-1} b a^{-1} b a \underbrace{b^{2} \cdot b^{-2}}_{\operatorname{id}_B} a^{-1} b^2 a^3 b \\
         &= b^2 a^{-1} b a^{-1} b \underbrace{a a^{-1}}_{\operatorname{id}_A} b^2 a^3 b \\
         &= b^2 a^{-1} b a^{-1} \underbrace{b b^{2}}_{b^3} a^3 b \\
         &= b^2 a^{-1} b a^{-1} b^3 a^3 b.
    \end{align*}
    Then computing $A(BC)$ gives \begin{align*}
      A(BC) &= b^2 a^{-1} b^{-3} a b^{-1} a \underbrace{b^{-2} \cdot b^{2}}_{\operatorname{id}_B} a^{-1} b a^{-1} b^3 a^3 b \\
         &= b^2 a^{-1} b^{-3} a b^{-1} \underbrace{a \cdot a^{-1}}_{\operatorname{id}_A} b a^{-1} b^3 a^3 b \\
         &= b^2 a^{-1} b^{-3} a \underbrace{b^{-1} \cdot b}_{\operatorname{id}_B} a^{-1} b^3 a^3 b \\
         &= b^2 a^{-1} b^{-3} \underbrace{a \cdot a^{-1}}_{\operatorname{id}_A} b^3 a^3 b \\
         &= b^2 a^{-1} \underbrace{b^{-3} \cdot b^{3}}_{\operatorname{id}_B} a^3 b \\
         &= b^2 \underbrace{a^{-1} \cdot a^3}_{a^2} b \\
         &= b^2 a^2 b.
    \end{align*}
    Therefore \[
      (AB)C = b^2a^2b = A(BC).
    \]
    \item We already computed $AB = b^2 a^{-1} b^{-2} a b^{2}$, so computing
    $(AB)D$ \begin{align*}
      (AB)D &= b^2 a^{-1} b^{-2} a \underbrace{b^{2} \cdot b^{-2}}_{\operatorname{id}_B} a^{-1} b^2 a b^{-2} a^3 \\
         &= b^2 a^{-1} b^{-2} \underbrace{a a^{-1}}_{\operatorname{id}_A} b^2 a b^{-2} a^3 \\
         &= b^2 a^{-1} \underbrace{b^{-2} b^2}_{\operatorname{id}_B} a b^{-2} a^3 \\
         &= b^2 \underbrace{a^{-1} a}_{\operatorname{id}_A} b^{-2} a^3 \\
         &= \underbrace{b^2 b^{-2}}_{\operatorname{id}_B} a^3 \\
         &= a^3.
    \end{align*}
    Similarly, computing $BD$ gives \begin{align*}
      BD &= b^2 a^{-1} b a^{-1} b a \underbrace{b^{2} \cdot b^{-2}}_{\operatorname{id}_B} a^{-1} b^2 a b^{-2} a^3 \\
         &= b^2 a^{-1} b a^{-1} b \underbrace{a a^{-1}}_{\operatorname{id}_A} b^2 a b^{-2} a^3 \\
         &= b^2 a^{-1} b a^{-1} \underbrace{b b^{2}}_{b^3} a b^{-2} a^3 \\
         &= b^2 a^{-1} b a^{-1} b^3 a b^{-2} a^3.
    \end{align*}
    Then computing $A(BC)$ gives \begin{align*}
      A(BC) &= b^2 a^{-1} b^{-3} a b^{-1} a \underbrace{b^{-2} \cdot b^{2}}_{\operatorname{id}_B} a^{-1} b a^{-1} b^3 a b^{-2} a^3 \\
         &= b^2 a^{-1} b^{-3} a b^{-1} \underbrace{a \cdot a^{-1}}_{\operatorname{id}_A} b a^{-1} b^3 a b^{-2} a^3 \\
         &= b^2 a^{-1} b^{-3} a \underbrace{b^{-1} \cdot b}_{\operatorname{id}_B} a^{-1} b^3 a b^{-2} a^3 \\
         &= b^2 a^{-1} b^{-3} \underbrace{a \cdot a^{-1}}_{\operatorname{id}_A} b^3 a b^{-2} a^3 \\
         &= b^2 a^{-1} \underbrace{b^{-3} \cdot b^{3}}_{\operatorname{id}_B} a b^{-2} a^3 \\
         &= b^2 \underbrace{a^{-1} \cdot a}_{\operatorname{id}_A} b^{-2} a^3 \\
         &= \underbrace{b^2 \cdot b^{-2}}_{\operatorname{id}_B} a^3 \\
         &= a^3.
    \end{align*}
    Therefore \[
      (AB)D = a^3 = A(BD).
    \]
  \end{enumerate}
\end{proof}
\pagebreak
% -----------------------------------------------------
% Second problem
% -----------------------------------------------------
\begin{problem}{2} \text{} \\
  Consider the group $G = \langle a, b; a^2b^{-3} = 1 \rangle$ = $F(a, b)/\langle a^2b^{-3} \rangle$. Let
  $\tau \in \mathfrak S_3$ be the transposition $\tau = (1\ 2)$ and let $\rho$
  be the cyclic permutation $\rho = (1\ 2\ 3)$.
  \begin{enumerate}[a.]
    \item Show that there is a unique group homomorphism
    $\phi\colon G \rightarrow \mathfrak S_3$ sending $a$ to $\tau$ and $b$ to
    $\rho$. Conclude that $G$ is not abelian.
    \item Show that there is a surjective homomorphism
    $\psi\colon G \rightarrow \mathbb Z$. Conclude that $G$ is infinite.
  \end{enumerate}
\end{problem}

\begin{proof} \text{} \\
  \begin{enumerate}[a.]
    \item First, $\langle a^2b^{-3} \rangle \subset \ker(\phi)$ because \[
      \phi(a^2b^{-3})
      = \phi(a^2)\phi(b^{-3})
      = \underbrace{(1\ 2)(1\ 2)}_{\operatorname{id}_{\mathfrak S_3}}
      \underbrace{(1\ 2\ 3)(1\ 2\ 3)(1\ 2\ 3)}_{\operatorname{id}_{\mathfrak S_3}}
      = \operatorname{id}_{\mathfrak S_3}
    \]
    This map is the unique homomorphism that sends $\tau \mapsto (1\ 2)$ and
    $\rho \mapsto (1\ 2\ 3)$, because it prescribes where to send all elements
    of $\langle a, b \rangle$, and so the quotient map inherits this uniqueness.
    \item We will use the map defined by, $\psi(a) = 3$ and $\psi(b) = 2$.
    Since \[
      \psi(a^2b^{-3}) = \psi(a^2)\psi(b^{-3}) = 2(3) + -3(2) = 0
    \] we have that $\langle a^2b^{-3} \rangle \subset \ker(\psi)$, and so
    $\psi$ defines a homomorphism.
    Thus it only remains to check that $\psi$ is surjective:
    any even number $2n \in \mathbb Z$ can be written as $\psi([a^n])$, and
    similarly any odd number $2n + 1 \in \mathbb Z$ can be written as
    $\psi([a^{n-1}b])$.
    Thus $G$ is infinite.
  \end{enumerate}
\end{proof}
\pagebreak
% -----------------------------------------------------
% Third problem
% -----------------------------------------------------
\begin{problem}{3} \text{} \\
  Let $X$ be a metric space with metric $d_0$, and pick a base point
  $x_0 \in X$. Let $\Omega_{x_0}X$ denote the space of paths
  $\alpha\colon [0, 1] \rightarrow X$ with $\alpha(0) = \alpha(1) = x_0$.
  \begin{enumerate}[a.]
    \item Define $x_2(X; x_0) = \pi_1(\Omega_{x_0}X; c_{x_0})$.
    Interpret $\pi_2(X; x_0)$ as a set of maps. What is the geometric
    interpretation of the group law in this context.
    $[0, 1] \times [0, 1] \rightarrow X$ modulo a certain equivalence relation.
    \item Show that $\pi_2(X; x_0)$ is an abelian group.
  \end{enumerate}
\end{problem}

\begin{solution} \text{}
  \begin{enumerate}[a.]
    \item Let elements of $\pi_2(X, x_0)$ be equivalence classes of
    (continuous) maps $\alpha\colon [0,1] \times [0, 1] \rightarrow X$ such
    that \[
      \alpha(0, t) = \alpha(1, t) = x_0 = \alpha(s, 0) = \alpha(s, 1)
    \] where two maps $\alpha_0$ and $\alpha_1$ are equivalent if there exists a
    continuous map $H\colon ([0, 1] \times [0, 1]) \times [0, 1] \rightarrow X$
    between them such that \begin{align*}
      H(s, t, 0) &= \alpha_0(s, t), \\
      H(s, t, 1) &= \alpha_1(s, t), \\
      H(0, t, r) &= H(1, t, r) = x_0, \text{ and} \\
      H(s, 0, r) &= H(s, 1, r) = x_0.
    \end{align*}
    The geometric interpretation of the group law is a path that starts and ends
    as the constant path at $x_0$, and is a continuous collection of loops
    in between.
    \item We'll define the homotopy between $\alpha * \beta$ and
    $\beta * \alpha$ by
      $H\colon ([0, 1] \times [0, 1]) \times [0, 1] \rightarrow X$
    as three maps $H_1$, $H_2$, and $H_3$: \[
      H(s, t, r) = \begin{cases}
        H_1(s, t, 3r)     & r \in [0, 1/3] \\
        H_2(s, t, 3r - 1) & r \in [1/3, 2/3] \\
        H_3(s, t, 3r - 2) & r \in [2/3, 1] \\
      \end{cases}.
    \]
    Define \[
      H_1(s, t, r) = \begin{cases}
        x_0 & s \in [0, 1/2], t \in [0, r/2] \\
        \alpha(2s, ?) & s \in [0, 1/2], t \in [r/2, 1] \\
        \beta(2s - 1, ?) & s \in [1/2, 1], t \in [0, r/2] \\
        x_0 & s \in [1/2, 1], t \in [r/2, 1]
      \end{cases}.
    \]

    (I ran out of time.)
  \end{enumerate}
\end{solution}
\end{document}
