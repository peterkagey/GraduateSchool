\documentclass{article}

\usepackage[margin=1in]{geometry}
\usepackage{amsmath,amsthm,amssymb}
\usepackage{bbm,enumerate,enumitem,mathtools,multicol}
\usepackage[hidelinks]{hyperref}
\usepackage{tikz}
\usetikzlibrary{matrix, arrows}

\newenvironment{problem}[2][Problem]{\begin{trivlist}
\item[\hskip \labelsep {\bfseries #1}\hskip \labelsep {\bfseries #2.}]}{\end{trivlist}}
\newenvironment{solution}[1][Solution.]{\begin{trivlist}
\item[\hskip \labelsep {\bfseries #1}]}{\end{trivlist}}
\newenvironment{problempart}[1]{\begin{trivlist}\item[\textbf{Part #1.}]}{\end{trivlist}}

\begin{document}

\title{Topology: Homework 2}
\author{Peter Kagey}

\maketitle

% -----------------------------------------------------
% First problem
% -----------------------------------------------------
\begin{problem}{1} \text{} \\
  Let \begin{alignat*}{4}
    p      \colon&&\mathbb R &\rightarrow S^1 \text{ by }& x      &\mapsto (\cos x, \sin x) \\
    \alpha \colon&& [0, 1]   &\rightarrow S^1 \text{ by }& t      &\mapsto p(2\pi t),
  \end{alignat*}
  let $f$ be some map with the property that $f(-x, -y) = -f(x, y)$
      % f      \colon&& S^1      &\rightarrow S^1 \text{ by }& (x, y) &\mapsto (-x, -y) \\
  And let $\widetilde\beta\colon [0, 1] \rightarrow \mathbb R$ be a lift of the
  path $\beta = f \circ \alpha$.
  \begin{enumerate}[label=\textbf{\alph*.}]
    \item Show that there exists an integer $n_0 \in \mathbb Z$ such that
    $\widetilde\beta(\frac{1}{2}) = \widetilde\beta(0) + 2n_0\pi + \pi$.
      \begin{proof} \text{} \\
        We know that $p \circ \widetilde\beta = f \circ \alpha$, which is to
        say, \[
          (\cos(\widetilde\beta(t)), \sin(\widetilde\beta(t))) = f(\cos(2\pi t), \sin(2\pi t)).
        \]
        And let's further decompose $f$ as $f(x, y) = (f_x(x,y), f_y(x,y))$.
        \\
        Thus, when $t = 0$, $\cos(\widetilde\beta(0)) = f_x(1, 0)$.
        % , so
        % $\widetilde\beta(0) = 2n\pi + \pi$ for some $n \in \mathbb Z$.\\
        Similarly, when $t = \frac{1}{2}$,
        $\cos(\widetilde\beta(\frac{1}{2})) = f_x(-1, 0) = -f_x(1, 0)$.
        Therefore \[
          \cos(\widetilde\beta(0)) +
          \cos(\widetilde\beta(\textstyle\frac{1}{2})) = 0
        \] so $\widetilde\beta(0)$ and $\widetilde\beta(\frac{1}{2})$ differ by
        $\pi$ up to some multiple of $2\pi$:
        \[
          \widetilde\beta(\textstyle{\frac{1}{2}})
          = \widetilde\beta(0) + 2n_0\pi + \pi.
        \]
      \end{proof}
    \item Show that, for the integer $n_0$ of part \textbf{a},
    $\widetilde\beta(t) = \widetilde\beta(t - \frac{1}{2}) + 2n_0\pi + \pi$ for
    every $t \in \left[\frac{1}{2}, 1\right]$.
      \begin{proof} \text{} \\
        Writing things explicitly gives \begin{align*}
          (\cos(\widetilde\beta(t)), \sin(\widetilde\beta(t)))
          &= f(\cos(2\pi t), \sin(2\pi t)) \\
          \textstyle(\cos(\widetilde\beta(t - \frac{1}{2})), \sin(\widetilde\beta(t - \frac{1}{2})))
          &= f(\textstyle(\cos(2\pi (t - \frac{1}{2})), \sin(2\pi (t - \frac{1}{2})))) \\
          &= f(\textstyle(\cos(2\pi t - \pi), \sin(2\pi t - \pi))) \\
          &= f(\textstyle(-\cos(2\pi t), -\sin(2\pi t))) \\
          &= -f(\textstyle(\cos(2\pi t), \sin(2\pi t)))
        \end{align*}
        So similarly to part \textbf{a}, \[
          \cos(\widetilde\beta(t)) + \textstyle(\cos(\widetilde\beta(t - \frac{1}{2}))
          = f_x(\cos(2\pi t), \sin(2\pi t)) - f_x(\cos(2\pi t), \sin(2\pi t))
          = 0
        \]
        the arguments $\widetilde\beta(t)$ and $\widetilde\beta(t - \frac{1}{2})$
        must differ by $\pi$ up to a multiple of $2\pi$. By continuity, this
        multiple of $2\pi$ must be the same as in part \textbf{a}.
        Thus
        \[
          \widetilde\beta(t) = \widetilde\beta(t - \textstyle\frac{1}{2}) + 2n_0\pi + \pi.
        \]
        Thus \begin{align*}
          \widetilde\beta(1)                 &= \textstyle\widetilde\beta(\frac12) + 2n_0\pi + \pi,\\
          \textstyle\widetilde\beta(\frac12) &= \widetilde\beta(0) + 2n_0\pi + \pi, \text{ and so} \\
          \textstyle\widetilde\beta(1)       &= \widetilde\beta(0) + 4n_0\pi + 2\pi,
        \end{align*}
        and because $4n_0 + 2 \neq 0$ for all $n_0 \in \mathbb Z$,
        $\widetilde\beta(1) \neq \widetilde\beta(0)$.
      \end{proof}
    \item Show that, if $x_0 = (1, 0)$ and $y_0 = f(1, 0)$, the induced
    homomorphism $f_*\colon\pi_1(S^1;x_0)\rightarrow\pi_1(S^1;y_0)$ is
    non-trivial.
      \begin{proof} \text{} \\
        Consider the generating element $[\alpha] \in \pi(S^1; x_0)$, which maps
        to $[f \circ \alpha] = [p \circ \widetilde\beta]$ under $f_*$.
        Since part \textbf{b} shows that
        $\widetilde\beta(1) \neq \widetilde\beta(0)$, we know that
        $p \circ \widetilde\beta$ must describe a non-trivial loop around $S^1$,
        and thus the homomorphism is non-trivial.
      \end{proof}
    \item Consider the $2$-dimensional sphere $S^2$ and identify the unit circle
    $S^1$ to its equator. \begin{enumerate}[label=(\roman*)]
      \item Show that for every map $F\colon S^2 \rightarrow S^1$, the
      restriction $f\colon S^1 \rightarrow S^1$ which sends
      $(x, y) \mapsto F(x, y, 0)$ is homotopic to a constant map.
        \begin{proof} \text{} \\
          This is similar to ...
        \end{proof}
      \item Show that there is no map $F \colon S^2 \rightarrow S^1$ such that
      $F(-x,-y,-z) = -F(x,y,z)$ for every $(x, y, z) \in S^2$.
        \begin{proof}
          This follows from parts \textbf{c} and \textbf{d} (i).
          Suppose that there were such a map. Then its restriction to the
          equator would satisfy the conditions for $f$ above, and thus by
          part \textbf{c}, it would not be nullhomotopic. However, part
          \textbf{d} (i) showed that such a map \textit{must} be nullhomotopic,
          a contradiction. Thus no map may exist.
        \end{proof}
    \end{enumerate}
    \item Let $f\colon S^2 \rightarrow \mathbb R^2$ be continuous. Show that
    there exists at least one pair of antipodal points that have the same image
    under $g$.
      \begin{proof}
        As per the hint, consider the map $F\colon S^2 \rightarrow S^1$ by \[
          F(x, y, z) = \frac{g(x, y, z) - g(-x, -y, -z)}{||g(x, y, z) - g(-x, -y, -z)||}.
        \] This function meets the criteria in part \textbf{d} (ii), namely \[
          -F(x,y,z)
          = \frac{g(-x, -y, -z) - g(x, y, z)}{||g(x, y, z) - g(-x, -y, -z)||}
          = \frac{g(-x, -y, -z) - g(x, y, z)}{||g(-x, -y, -z) - g(x, y, z)||}
          = F(-x, -y, -z).
        \]
        Therefore no such continuous function exists on all of $S^2$, and
        so there exists some $(x, y, z)$ such that
        $||g(-x, -y, -z) - g(x, y, z)|| = 0$, and thus there exists some
        $(x, y, z)$ such that $g(-x, -y, -z) = g(x, y, z)$.
      \end{proof}
    \item Let $A$ and $B$ be two bounded domains in the $xy$-plane
    $\mathbb R^2 \times \{ 0 \} \subset \mathbb R^3$.
    \begin{enumerate}[label=(\roman*)]
      \item For each unit vector $\vec u \in S^2 \subset \mathbb R^3$, let
      $P_{\vec{u}}$ be the plane in $\mathbb R^3$ passing through the point
      $(0, 0, 1)$ and orthogonal to $\vec u$, and let $H_{\vec{u}}$ be the
      half-space delimited by $P_{\vec{u}}$ such that $\vec u$ points toward
      the interior of $H_{\vec{u}}$.
      \\
      Show that there exists $\vec u \in S^2$ such that
      $\operatorname{area}(A \cap H_{\vec u}) = \frac{1}{2}\operatorname{area}(A)$
      and
      $\operatorname{area}(B \cap H_{\vec u}) = \frac{1}{2}\operatorname{area}(B)$.
        \begin{proof}
          As suggested by the hint, let $g\colon S^2 \rightarrow \mathbb R^2$ be
          defined by \[
            g(\vec u) = (
              \operatorname{area}(A \cap H_{\vec u}),
              \operatorname{area}(B \cap H_{\vec u})
            ).
          \]
          Notice that $P_{- \vec u} = P^c_{\vec u}$, so \[
            g(-\vec u) = (
              \operatorname{area}(A) - \operatorname{area}(A \cap H_{\vec u}),
              \operatorname{area}(B) - \operatorname{area}(B \cap H_{\vec u})
            )
          \] and moreover, $g(\vec u) = g(-\vec u)$ precisely when \begin{align*}
            \operatorname{area}(A \cap H_{\vec u}) &= \frac{1}{2} \operatorname{area}(A) \\
            \operatorname{area}(B \cap H_{\vec u}) &= \frac{1}{2} \operatorname{area}(B).
          \end{align*}
          It takes some measure-theoretic argument to show that $g$ is
          continuous, but taking that for granted, part \textbf{e} proves that
          there exists a pair of antipodal points that have
          the same image under $g$. Thus the desired vector $\vec u$ is the
          one that satisfies the antipodal equality property.
        \end{proof}
      \item Show that there exists a line in $\mathbb R^2$ that divides each of
      $A$ and $B$ into halves of equal area.
        \begin{proof}
          Simply choose any vector $\vec u$ that satisfies part \textbf{f} (i),
          and take the line which is the intersection of $P_{\vec u}$ and the
          $xy$-plane.
        \end{proof}
    \end{enumerate}
  \end{enumerate}
\end{problem}

\end{document}
