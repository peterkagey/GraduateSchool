\documentclass{article}

\usepackage[margin=1in]{geometry}
\usepackage{amsmath,amsthm,amssymb}
\usepackage{bbm,enumerate,mathtools,multicol}
\usepackage[shortlabels]{enumitem}
\usepackage[hidelinks]{hyperref}
\usepackage{tikz}
\usetikzlibrary{matrix, arrows}

\newenvironment{problem}[2][Problem]{\begin{trivlist}
\item[\hskip \labelsep {\bfseries #1}\hskip \labelsep {\bfseries #2.}]}{\end{trivlist}}
\newenvironment{solution}[1][Solution.]{\begin{trivlist}
\item[\hskip \labelsep {\bfseries #1}]}{\end{trivlist}}
\newenvironment{problempart}[1]{\begin{trivlist}\item[\textbf{Part #1.}]}{\end{trivlist}}

\newcommand{\fn}[3]{#1 \colon #2 \rightarrow #3}
\newcommand{\inv}[1]{#1^{-1}}
\newcommand{\set}[1]{\left\{ #1 \right\}}
\newcommand{\id}[1]{\operatorname{Id}_{#1}}
\newcommand{\ra}{\rightarrow}
\DeclareMathOperator{\im}{Im}

\begin{document}

\title{Topology: Homework 8}
\author{Peter Kagey}

\maketitle

% -----------------------------------------------------
% First problem
% -----------------------------------------------------
\begin{problem}{1} \text{} \\
  Suppose that $A$ is homotopoy equivalent to a point. Show that $H_n(X,A)$
  is isomorphic to $H_n(X)$ for every $n \geq 1$.
\end{problem}

\begin{proof} \text{} \\
  By the snake lemma we can turn the short exact sequences \[
    0 \ra C(A) \ra C(X) \ra C(X,A) \ra 0
  \] into the long exact sequence \[
    \hdots \ra H_n(A) \ra H_n(X) \ra H_n(X,A) \xrightarrow{\delta_n} H_{n-1}(A) \ra \hdots.
  \]
  Since $A$ is homotopy equivalent to a point by hypothesis, for all $n > 0$,
  $H_n(A) = 0$.
  Therefore for $n > 1$ \[
    \underbrace{H_n(A)}_0 \ra H_n(X) \ra H_n(X,A) \xrightarrow{\delta_n} \underbrace{H_{n-1}(A)}_0
  \] and thus map $H_n(X) \ra H_n(X,A)$ has a kernel of $0$ so is injective, and it
  has an image of $H_n(X,A)$, so it's surjective, and thus $H_n(X) \cong H_n(X,A)$.
  \\~\\
  In the case of $n = 1$, the long exact sequence is \[
    \hdots \ra
    \underbrace{H_1(A)}_0 \ra H_1(X) \ra H_1(X,A) \xrightarrow{\delta_1} \underbrace{H_0(A)}_R \ra \hdots,
  \]
  so the map $H_1(X) \rightarrow H_1(X, A)$ is injective. Thus it is enough to
  show that the map is surjective. However, take the equivalence class
  $[c] \in H_1(X, A) = H_1(X)/H_1(A)$, with representative $c \in H_1(X)$.
  So the quotient map $c \mapsto [c]$ is surjective.
\end{proof}
\pagebreak
% -----------------------------------------------------
% Second problem
% -----------------------------------------------------
\begin{problem}{2} \text{} \\
  Suppose that $X$ is homotopy equivalent to a point. Show that $H_n(X,A)$ is
  isomorphic to $H_{n-1}(A)$ for every $n \geq 2$. Show that this is in general
  false if $n = 1$.
\end{problem}

\begin{proof} \text{} \\
  By the same construction above, we have the long exact sequence \[
    \hdots \ra H_n(X) \ra H_n(X,A) \xrightarrow{\delta_n} H_{n-1}(A) \ra H_{n-1}(X) \ra \hdots.
  \] If $n > 1$, then we have the short exact sequence \[
    \underbrace{H_n(X)}_0 \ra H_n(X,A) \xrightarrow{\delta_n} H_{n-1}(A) \ra \underbrace{H_{n-1}(X)}_0
  \] so $\delta_n$ is an isomorphism.
  \\~\\
  In the case of $n=1$, $H_1(X,A) = C_1(X)/B_1(X,A)$
  \[
    \underbrace{H_1(X)}_0 \ra H_1(X,A) \xrightarrow{\delta_1} H_0(A) \ra \underbrace{H_0(X)}_R.
  \]
  The map $\fn{\delta_1}{H_1(X, A)}{H_0(A)}$ is injective, so it is
  enough to show that $\delta_1$ is not surjective.
  Let $X = {0}$ and $X = A$. Then $X_1(X, A) = 0$ and $X_0(A) = R$, so this map
  cannot be surjective for all pairs $(X, A)$ as shown by this counterexample.
\end{proof}
\pagebreak
% -----------------------------------------------------
% Third problem
% -----------------------------------------------------
\begin{problem}{3} \text{} \\
  For $A \subset X$, suppose that the inclusion map $\fn iAX$ is a homotopy
  equivalence. Show that $H_n(X, A) = 0$ for every $n$.
\end{problem}

\begin{proof} \text{} \\
  Firstly, we have the exact sequence \[
    \hdots \ra H_n(X,A) \xrightarrow{\delta_n} H_{n-1}(A) \xrightarrow{i_*} H_{n-1}(X) \xrightarrow{j_*} H_{n-1}(X,A) \hdots.
  \] where $\ker(i_*) = 0 = \im(\delta_n)$, where the kernel is trivial because
  $i_*$ is an isomorphism. Thus $\delta_n$ must be the zero map with kernel \[
    \ker(\delta_n) = H_n(X,A).
  \]
  Since $\fn{j_* \circ i_*}{H_{n-1}(A)}{H_{n-1}(X,A)}$ maps everything to the
  zero element in the quotient, $j_*$ must be the zero map, so
  \[
    0 = \im(j_*) = \ker(\delta_n) = H_n(X,A).
  \]
  % Since the sequence is exact, $\ker(\delta_n) = H_n(X,A) = \im(H_{n}(X) \ra H_{n}(X,A))$,
  % so the map $H_{n}(X) \ra H_{n}(X,A)$ is surjective, and because of isomorphism
  % Since $i_*$ is an isomorphism, $\im(i_*) = H_n(X) = \ker(j_*)$
\end{proof}
\pagebreak
% -----------------------------------------------------
% Fourth problem
% -----------------------------------------------------
\begin{problem}{4} \text{} \\
  Suppose that $X = X_1 \cup X_2$ for two subspaces $X_1, X_2 \subset X$. Let
  $C_n^{X_1X_2}(X) = C_n(X_1) + C_n(X_2) \subset C_n(X)$ consist of chains
  $c \in C_n(X)$ that can be written as a linear combination of simplices that
  are either completely contained in $X_1$ or completely contained in $X_2$.
  Let $H_n^{X_1X_2}$ denote the homology modules of the corresponding chain
  \\~\\
  Prove that $\fn{H_n(i)}{H_n^{X_1X_2}(X)}{H_n(X)}$ is an isomorphism for every $n$
  if and only if $X_1 - X_2$ can be excised from the pair $(X, X_1)$.
  \begin{enumerate}[a.]
    \item Suppose that $\fn{H_n(i)}{H_n^{X_1X_2}(X)}{H_n(X)}$ is an isomorphism
    for every $n$. We want to show that
    $\fn{H_n(j)}{H_n(X_2,X_1 \cap X_2)}{H_n(X, X_1)}$ is surjective.
    For this, consider $[c] \in H_n(X,X_1)$ represented by $c \in C_n(X)$ with
    $\partial c \in C_{n-1}(X_1)$.
    \begin{enumerate}[(i)]
      \item Let $c' = \partial c$. Considering the classes $[c'] \in H_n^{X_1X_2}(X)$ and
      $[c'] \in H_n(X)$, show that there exists $c_1 \in C_n(X_1)$ and
      $C_2 \in C_n(X_2)$ such that $c' = \partial c_1 + \partial c_2$.
      \item Show that there exists $c'_1  \in C_n(X_1)$, $c'_2 \in C_n(X_2)$ and
      $c' \in C_{n+1}(X)$ such that \[
        c-c_1-c_2 = c'_1 + c'_2 + \partial c'.
      \]
      \item Show that $\partial(c_2 + c_2') \in C_{n-1}(X_1 \cap X_2)$, so that
      $c_2 + c_2'$ defines a class $[c_2 + c_2'] \in H_n(X_2, X_1 \cap X_2)$.
      \item Show that $H_n(j)([c_2 + c_2']) = [c] \in H_n(X, X_1)$, which
      concludes the proof that $H_n(j)$ is surjective.
    \end{enumerate}
    \item Suppose that $\fn{H_n(i)}{H_n^{X_1X_2}(X)}{H_n(X)}$ is an isomorpishm
    for every $n$. We want to show that
    $\fn{H_n(j)}{H_n(X_2,X_1 \cap X_2)}{H_n(X, X_1)}$ is injective. For this,
    consider \[
      [c_2] \in \ker H_n(j) \subset H_n(X_2, X_1 \cap X_2)
    \] represented
    by $c_2 \in C_n(X_2)$ with $\partial c_2 \in C_{n-1}(X_1 \cap X_2)$.
    \begin{enumerate}[(i)]
      \item Show that there exists $c_1 \in C_n(X_1), c_1' \in C_{n+1}(X_1)$ and
      $c_2' \in C_{n+1}(X_2)$ such that $c_2 = c_1 + \partial c'_1 + \partial c'_2$. (Hint: use part a (i).)
    \end{enumerate}
  \end{enumerate}
\end{problem}

\begin{proof} \text{} \\
  \begin{enumerate}[a.]
    \item This part will assume that $\fn{H_n(i)}{H_n^{X_1X_2}(X)}{H_n(X)}$ is
    an isomorphism for every $n$, and prove that
    $\fn{H_n(j)}{H_n(X_2,X_1 \cap X_2)}{H_n(X, X_1)}$ is surjective.
    \begin{enumerate}[(i)]
      \item Since $H_n(i)$ is a bijection, $\inv{H_n(i)}([c]) = [c_1 + c_2] \in H_n^{X_1X_2}(X)$. Therefore $c = c_1 + c_2 + \partial \tilde c$ where $\partial\tilde c \in \im(\partial_{n+1})$.
      Moreover, \[
        c'
        = \partial c
        = \partial(c_1 + c_2 + \partial\tilde c)
        = \partial c_1 + \partial c_2 + \underbrace{\partial\partial\tilde c}_0.
      \]
      \item Similarly, \[
        \inv{H_n(i)}(\underbrace{[c - c_1 - c_2]}_{\in H_n(X)}) = [c_1' + c_2'] \in H_n^{X_1X_2}(X)
      \] which means that \[
        c - c_1 - c_2 = c_1' + c_2' + \underbrace{\partial c'}_{\in \im(\partial)},
      \] by definition of the quotient.
      \item By the above \begin{align*}
        c_2' + c_2 &= c - c_1 - c_1' - \partial c' \\
        \partial(c_2' + c_2)
        &= \partial(c - c_1 - c_1' - \partial c') \\
        &= \partial c - \partial(c_1 + c_1')
      \end{align*} Since $\partial(c_1' + c_1) \in C_n(X_1)$,
      $\partial(c_2' + c_2) \in C_n(X_2)$, and
      $\partial c \in C_n(X_1)$ by hypothesis, the right hand side is in
      $C_n(X_1)$ and the left hand side is in $C_n(X_2)$, so both must be in
      $C_n(X_1 \cap X_2)$.
      \item Using the above, \[
        H_n(j)([c_2 + c_2'])
        = H_n(j)([c - c_1 - c_1' - \partial c'])
        = [j(c - c_1 - c_1' - \partial c')].
      \]
      Since
      \[
        \partial(c - c_1 - c_1' - \partial c')
        = \partial c - \underbrace{\partial(c_1 + c_1' + \partial c')}_{C_n(X_1)}
      \] this equivalence class is \[
        [j(c - c_1 - c_1' - \partial c')] = [c] \in H_n(X, X_1)
      \] so $H_n(j)$ is surjective.
    \end{enumerate}
    \item This part will assume that $\fn{H_n(i)}{H_n^{X_1X_2}(X)}{H_n(X)}$ is
    an isomorphism for every $n$, and prove that
    $\fn{H_n(j)}{H_n(X_2,X_1 \cap X_2)}{H_n(X, X_1)}$ is injective.
    \begin{enumerate}[(i)]
      \item Since $[c_2] \in \ker H_n(j)$,
      $[c_2] = 0 \in X_n(X, X_1) = Z(X, X_1)/B(X, X_1)$, so
      $c_2 \in B(X, X_1)$. By definition of relative boundary, this means that
      there exists some $c \in C_{n+1}(X)$ such that
      $c_2 - \partial c \in C_n(X_1)$. Let $c_1 = c_2 - \partial c \in C_n(X_1)$
      so that $c_2 = c_1 + \partial c$. By part \textbf{a (i)},
      $\partial c = \partial c_1' + \partial c_2'$ for some
      $c_1' \in C_{n+1}(X_1)$ and $c_2' \in C_{n+1}(X_2)$, so \begin{align*}
        c_2 &= c_1 + \partial c \\
        &= c_1 + \partial c_1' + \partial c_2'.
      \end{align*}
      \item Solving for $c_1 + \partial c_1'$ yields \[
        \underbrace{c_1 + \partial c_1'}_{\in C_n(X_1)} = \underbrace{c_2 - \partial c_2'}_{\in C_n(X_2)},
      \] so $c_1 + \partial c_1' \in C_n(X_1 \cap X_2)$.
      \item Therefore \[
        c_2 - \partial c_2' = \underbrace{
          c_1 + \partial c_1'
        }_{C_n(X_1 \cap X_2)} \in C_n(X_1 \cap X_2)
      \] so $c_2 \in B_n(X_2, X_1 \cap X_2)$, so
      $[c_2] = 0 \in H_n(X_2, X_1 \cap X_2) = Z_n(X_2, X_1 \cap X_2)/B_n(X_2, X_1 \cap X_2)$.
      Therefore $H_n(j)$ is injective.
    \end{enumerate}
    \item This part will assume that $X_1 - X_2$ can be excised from the pair
    $(X, X_1)$ and prove that $\fn{H_n(i)}{H_n^{X_1X_2}(X)}{H_n(X)}$ is injective.
    \begin{enumerate}[(i)]
      \item Since $[c_1 + c_2] \in \ker H_n(i)$ (and thus there exists
      $c \in C_n(X)$ such that $c_1 + c_2 = \partial c$) we can check
      \begin{align*}
        \partial(c_1 + c_2) &= \underbrace{\partial \partial c}_0 \\
        \partial c_1 &= -\partial c_2 \in C_{n-1}(X_2),
      \end{align*}
      So $\partial c_1 \in C_{n-1}(X_1) \cap C_{n-1}(X_2) = C_{n-1}(X_1 \cap X_2)$,
      and therefore $c_2$ defines a class $[c_2] \in H_n(X_2, X_1 \cap X_2)$.
      \item By rearranging $c_1 + c_2 = \partial c$, it can be seent that
      $c_2 - \partial c = c_1 \in C_n(X_1)$, and so $c_2 \in B_n(X, X_1)$.
      Therefore $[c_2] = 0 \in Z_n(X, X_1)/B_n(X, X_1) = H_n(X, X_1).$ Since
      $H_n(j)$ is injective, \[
        [c_2] = 0 \in H_n(X_2, X_1 \cap X_2) = Z_n(X_2, X_1 \cap X_2)/B_n(X_2, X_1 \cap X_2)
      \] so $c_2 \in B_n(X_2, X_1 \cap X_2)$ and so there exists $c'_2$ such that
      $c_2 - \partial c'_2 \in C_n(X_1 \cap X_2)$.
      Name this element $c_{12} = c_2 - \partial c'_2$. Then rearranging, \[
        c_2 = c_{12} - \partial c'_2.
      \]
      \item By hypothesis $(X_2, X_1 \cap X_2) \rightarrow (X, X_1)$ is an
      excision so $H_n(j)$ is an isomorphism.
      We know by the first two parts that \begin{align*}
        \partial c &= c_1 + c_2, \text{ and}\\
        c_2 &= c_{12} - \partial{c'_2}
      \end{align*} so it follows that $\partial(c + c'_2) = c_1 + c_{12}$, and
      we can consider $[c - c'_2] \in H_{n+1}(X, X_1)$.
      \\
      Because $H_n(j)$ is an isomorphism, by taking the inverse map, there
      exists $H_n(j)^{-1}([c-c'_2]) = [c' + c''] \in H_{n+1}(X_2, X_1 \cap X_2)$,
      meaning there exists some $c'' \in C_{n+2}$ such that \[
        c-c'2 = c''_2 + c_1' + \partial c'',
      \] as desired.
      \item From above, we can write \begin{align*}
        \partial(c - c'_2) &= \partial(c''_2 + c'_1 + \partial c'') \\
        c_1 + c_2 &= \partial c'_2 + \partial c''_2 + \partial c'_1 \\
        c_1 + c_2 &= \partial(c'_2 + c''_2 + c'_1)
      \end{align*} so $c_1 + c_2 \in \im(\partial)$ and thus
      $[c_1 + c_2] \in H^{X_1X_2}_n(X)$, and so $H_n(i)$ is injective.
    \end{enumerate}
    \item This part will assume that $X_1 - X_2$ can be excised from the pair
    $(X, X_1)$ and prove that $\fn{H_n(i)}{H_n^{X_1X_2}(X)}{H_n(X)}$ is surjective.
    \\~\\
    Let $[c] \in H_n(X)$, which meaning that the representative $c \in C_n$ is
    in the kernel $\ker(\partial_n)$. We will construct $c_1 \in H_n(X_1)$ and
    $c_2 \in H_n(X_2)$ such that $c = c_1 + c_2$.
    \\
    By the isomorpishm $H_n(j)$ there must exist $[c] \cong [c_2]$, namely
    $c = c_2 + \partial c_{12}$.
    Therefore $c = c_1 + c_2$ and \[
      H_n(i)([c_1 + c_2]) = [c] \in H_n(X),
    \] so $H_n(i)$ is surjective.
    \item Given that $(X_2, X_1 \cap X_2) \ra (X, X_1)$ is an excision implies
    that $H_n(j_1)$ is an isomorphism. Thus the previous two parts showed that
    $H_n(i)$ is also an isomorphism, so $c = c_1 + c_2$. Thus reversing the
    roles in the first two parts shows that $H_n(j_2)$ is also an isomorpism,
    meaning that $(X_1, X_1 \cap X_2) \ra (X, X_2)$ is an excision.
  \end{enumerate}
\end{proof}
\end{document}
