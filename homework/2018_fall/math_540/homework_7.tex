\documentclass{article}

\usepackage[margin=1in]{geometry}
\usepackage{amsmath,amsthm,amssymb}
\usepackage{bbm,enumerate,mathtools,multicol}
\usepackage[shortlabels]{enumitem}
\usepackage[hidelinks]{hyperref}
\usepackage{tikz}
\usetikzlibrary{matrix, arrows}

\newenvironment{problem}[2][Problem]{\begin{trivlist}
\item[\hskip \labelsep {\bfseries #1}\hskip \labelsep {\bfseries #2.}]}{\end{trivlist}}
\newenvironment{solution}[1][Solution.]{\begin{trivlist}
\item[\hskip \labelsep {\bfseries #1}]}{\end{trivlist}}
\newenvironment{problempart}[1]{\begin{trivlist}\item[\textbf{Part #1.}]}{\end{trivlist}}

\newcommand{\fn}[3]{#1 \colon #2 \rightarrow #3}
\newcommand{\inv}[1]{#1^{-1}}
\newcommand{\set}[1]{\{ #1 \}}

\begin{document}

\title{Topology: Homework 7}
\author{Peter Kagey}

\maketitle

% -----------------------------------------------------
% First problem
% -----------------------------------------------------
\begin{problem}{1} \text{} \\
  Let $X$ be path connected and locally path connected, and choose a base point
  $x_0 \in X$. Let $\fn{p}{\widetilde X}{X}$ and $\fn{p'}{\widetilde X'}{X}$ be
  two coverings with $\inv p(x_0) = \inv{p'}(x_0) = F$ (for some set $F$).
  Let $\fn{\rho, \rho'}{\pi_1(X; x_0)}{\operatorname{Bij}(F)}$ be the respective
  monodromy antihomomorphisms of $p$ and $p'$. Show that the two coverings are
  isomorphic (by an isomorphism that is not necessarily the identity on $F$)
  if and only if there exists $\theta \in \operatorname{Bij}(F)$ such that \[
    \rho'([\alpha]) = \theta \circ \rho([\alpha]) \circ \inv\theta
  \] for every $[\alpha] \in \pi(X; x_0)$.
\end{problem}

\begin{proof} \text{} \\
  $(\Longrightarrow)$ Suppose that the coverings are isomorphic, with
  isomorphism $\fn{\varphi}{\widetilde X}{\widetilde X'}$.\\
  Since $p \circ \inv\varphi = p'$, \begin{align*}
    F = \inv{p'}(x_0) &= \inv{(p \circ \inv\varphi)}(x_0) \\
    &= \varphi(\inv{p'}(x_0)) \\
    &= \varphi(F)
  \end{align*} and so the homeomorpism restricted to the fiber
  $\varphi|_F\colon F \rightarrow F$ is a bijection of $F$.
  \\
  Setting $\theta = \varphi|_F\colon F \rightarrow F$ and letting
  $\widetilde y \in F$ be an arbitrary element in the fiber yields \begin{align*}
    \rho'([\alpha])(y) &= \widetilde\alpha'_{y}(1) \\
    &= \varphi(\widetilde\alpha_{\inv\varphi(y)}(1)) \\
    &= \varphi \circ \rho([\alpha]) \circ \inv\varphi(y) \\
  \end{align*} where $\widetilde\alpha'_y$ is a path from $y$ to $\widetilde x'$,
  and $\widetilde\alpha_{\inv\varphi(y)}$ is a path from
  $\inv\varphi(y) \in \widetilde X$ to $\inv\varphi(\widetilde x')$.
  \\~\\
  $(\Longleftarrow)$ Suppose that there exists
  $\theta \in \operatorname{Bij}(F)$ such that \[
    \rho'([\alpha]) = \theta \circ \rho([\alpha]) \circ \inv\theta
  \] for every $[\alpha] \in \pi(X; x_0)$.
  We'll define the homeomorpism from $\widetilde X \rightarrow \widetilde X'$
  as follows:
  \begin{enumerate}
    \item Let $\fn{\widetilde\beta}{[0,1]}{\widetilde X}$ be any path from
    $\widetilde x_0$ to $\widetilde x$.
    \item Project this path down to $X$ resulting in a map
    $\fn{p \circ \widetilde \beta}{[0, 1]}{X}$, which describes a path from
    $x_0$ to $p(\widetilde x)$.
    \item Lift this path by to a path in $X'$ based at $\theta(\widetilde x_0)$.
  \end{enumerate}
  This map is continuous because it is the composition of lifts and continuous
  functions, and it is invertible by performing the procedure with the roles of
  $\widetilde X$ and $\widetilde X'$ switched. Thus it is a homeomorphism,
  $\fn{\varphi}{\widetilde X}{\widetilde X'}$, which respects composition by
  construction, $p' \circ \varphi = p$.
\end{proof}
\pagebreak
% -----------------------------------------------------
% Second problem
% -----------------------------------------------------
\begin{problem}{2} \text{} \\
  Consider the torus $X = S^1 \times S^1$ and consider all coverings
  $\fn{p}{\widetilde X}{X}$ with fiber $F = \set{1, 2, 3}$. Up to covering
  isomorphism fixing $F$, how many such coverings are there? And how many up to
  covering isomorphsim not fixing $F$?
\end{problem}

\begin{proof} \text{} \\
  Note that $D_3$ is isomorphic to the dihedral group
  $D_3 = \langle \sigma, \tau | \sigma^3 = \tau^2 = 1, \tau\sigma\tau = \inv\sigma \rangle$.
  The fundamental group of the torus is
  $\pi_1(X; x_0) = \langle a, b \rangle$. So we will consider all monodromy
  group antihomomorphisms of the covering $\widetilde X \rightarrow X$ with
  fiber $F = \set{1,2,3}$, namely antihomomorpishms of the form \[
    \fn{\rho}{\langle a, b; aba^{-1}b^{-1} = 1 \rangle}{D_3}
  \] where $D_3$ is the symmetric group.
  In particular, by the universal property of free products, the antihomomorphism
  must satisfy \[
  \rho(ab\inv a\inv b)
    = \rho(\inv a) \circ \rho(\inv b) \circ \rho(a) \circ \rho (b))
    = 1.
  \]
  Thus there are eighteen maps:
  \begin{alignat}{2}
    a &\mapsto \operatorname{id}_{D_3} &&\hspace{1cm} b \mapsto \operatorname{id}_{D_3} \\
    a &\mapsto \operatorname{id}_{D_3} &&\hspace{1cm} b \mapsto \tau \\
    a &\mapsto \operatorname{id}_{D_3} &&\hspace{1cm} b \mapsto \sigma \\
    a &\mapsto \tau &&\hspace{1cm} b \mapsto \operatorname{id}_{D_3} \\
    a &\mapsto \sigma &&\hspace{1cm} b \mapsto \operatorname{id}_{D_3} \\
    a &\mapsto \tau &&\hspace{1cm} b \mapsto \tau \\
    a &\mapsto \sigma &&\hspace{1cm} b \mapsto \sigma \\
    a &\mapsto \sigma &&\hspace{1cm} b \mapsto \inv\sigma
  \end{alignat}
  where
  (1) gives one maps,
  (2) gives two maps,
  (3) gives three maps,
  (4) gives two maps,
  (5) gives three maps,
  (6) gives three maps,
  (7) gives two maps, and
  (8) gives two maps.
  Thus there are $1 + 2 + 3 + 2 + 3 + 3 + 2 + 2 = 18$ different
  antihomomorphisms and eight conjugacy classes. So there are $18$ covering
  spaces up to isomorphism fixing $F$.
  \\~\\
  By Problem 1, the number of conjugacy classes counts the number of
  isomorphisms not necessarily fixing $F$, so there are eight isomorphisms.
\end{proof}
\pagebreak
% -----------------------------------------------------
% Third problem
% -----------------------------------------------------
\begin{problem}{3} \text{} \\
  Consider the Klein bottle $K$
  (with $\pi_1(K; x_0) = \langle a, b; ab\inv{a} = \inv{b}\rangle$)
  and consider all coverings
  $\fn{p}{\widetilde X}{X}$ with fiber $F = \set{1, 2, 3}$. Up to covering
  isomorphism fixing $F$, how many such coverings are there? And how many up to
  covering isomorphsim not fixing $F$?
\end{problem}

\begin{proof} \text{} \\
  Note that $D_3$ is isomorphic to the dihedral group
  $D_3 = \langle \sigma, \tau | \sigma^3 = \tau^2 = 1, \tau\sigma\tau = \inv\sigma \rangle$.
  \setcounter{equation}{0}
  There are eighteen antihomomorphisms \begin{alignat}{2}
    % (id,id)
    a &\mapsto \operatorname{id}_{D_3} &&\hspace{1cm} b \mapsto \operatorname{id}_{D_3} \\
    % (id,12)
    a &\mapsto \operatorname{id}_{D_3} &&\hspace{1cm} b \mapsto \tau \\
    % (12,id)
    a &\mapsto \tau &&\hspace{1cm} b \mapsto \operatorname{id}_{D_3} \\
    % (12,12)
    a &\mapsto \tau &&\hspace{1cm} b \mapsto \tau \\
    % (12,123)
    a &\mapsto \tau &&\hspace{1cm} b \mapsto \sigma \\
    % (123,id)
    a &\mapsto \sigma &&\hspace{1cm} b \mapsto \operatorname{id}_{D_3}
  \end{alignat} where
  (1) gives one map,
  (2) gives three maps,
  (3) gives three maps,
  (4) gives three maps,
  (5) gives six maps, and
  (6) gives two maps,
  totaling $1 + 3 + 3 + 3 + 6 + 2 = 18$ maps with six conjugacy classes.
  Thus there are eighteen covering isomorphisms fixing $F$.
  \\~\\
  By Problem 1, the number of conjugacy classes counts the number of
  isomorphisms not necessarily fixing $F$, so there are six isomorphisms.
\end{proof}
\pagebreak
% -----------------------------------------------------
% Fourth problem
% -----------------------------------------------------
\begin{problem}{4} \text{} \\
  Let $\fn{\rho}{\pi_1(X; x_0)}{\operatorname{Bij}(\mathbb Z)}$ be the monodromy
  antihomomorphism of the parking structure covering. Compute
  $\rho([\alpha]), \rho([\beta]), \rho([\gamma]) \in \operatorname{Bij}(\mathbb Z)$
  for the paths $\alpha, \beta, \gamma$ shown on the picture.
\end{problem}

\begin{proof} \text{} \\
  When you traverse $\alpha$, you go up one floor. When you traverse $\beta$
  you go up two floors, and when you traverse $\gamma$ you go down one floor.
  Thus the map sends \begin{align*}
    \rho([\alpha]) &= (n \mapsto n + 1) \\
    \rho([\beta])  &= (n \mapsto n + 2) \\
    \rho([\gamma]) &= (n \mapsto n - 1).
  \end{align*}
\end{proof}

\end{document}
