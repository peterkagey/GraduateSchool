\documentclass{article}

\usepackage[margin=1in]{geometry}
\usepackage{amsmath,amsthm,amssymb}
\usepackage{bbm,enumerate,mathtools}
\usepackage[hidelinks]{hyperref}
\usepackage{tikz}
\usetikzlibrary{matrix, arrows}

\newenvironment{problem}[2][Problem]{\begin{trivlist}
\item[\hskip \labelsep {\bfseries #1}\hskip \labelsep {\bfseries #2.}]}{\end{trivlist}}
\newenvironment{solution}[1][Solution.]{\begin{trivlist}
\item[\hskip \labelsep {\bfseries #1}]}{\end{trivlist}}
\newenvironment{problempart}[1]{\begin{trivlist}\item[\textbf{Part #1.}]}{\end{trivlist}}

\begin{document}

\title{Combinatorics: Homework 0}
\author{Peter Kagey}

\maketitle

% -----------------------------------------------------
% First problem
% -----------------------------------------------------
\begin{problem}{2 (a)} \text{} \\
  How many subsets of the set $[10]$ contain at least one odd integer?
\end{problem}

\begin{solution} \text{} \\
  There are $2^5$ subsets of $[10]$ that do \textit{not} contain any odd
  integers, namely all subsets of $\{2,4,6,8,10\}$.
  So all of the other $2^10 - 2^5$ subsets of $[10]$ contain at least one odd
  integer.
\end{solution}

% --- 1 (b) --------
\begin{problem}{2 (b)} \text{} \\
  In how many ways can seven people be seated in a circle if two arrangements
  are considered the same whenever each person has the same neighbors
  (not necessarily on the same side)?
\end{problem}

\begin{solution} \text{} \\
  Fix the first person as the ``head of the table''. Now we can arrange all of
  the other people in $6!$ ways. However, two seatings are equivalent if they
  are mirror images of the other, so there are $6!/2$ possible seatings.
\end{solution}

% --- 1 (c) --------
\begin{problem}{2 (c)} \text{} \\
  How many permutations of $w\colon [6] \rightarrow [6]$ satisfy $w(1) \neq 2$?
\end{problem}

\begin{solution} \text{} \\
  There are $5!$ permutations where $w(1) = 2$, so subtracting these from the
  full number of permutations yields $6! - 5!$.
\end{solution}

% --- 1 (d) --------
\begin{problem}{2 (d)} \text{} \\
  How many permutations of $[6]$ have exactly two cycles?
\end{problem}

\begin{solution} \text{} \\
  If we write the permutations in cycle notation, the partitions must have one
  of three cycle structures: $(a_1)(a_2\ a_3\ a_4\ a_5\ a_6)$,
  $(a_1\ a_2)(a_3\ a_4\ a_5\ a_6)$, or $(a_1\ a_2\ a_3)(a_4\ a_5\ a_6)$.

  There are $\binom{6}{1}$, $\binom{6}{2}$, and $\frac{1}{2}\binom{6}{3}$ ways
  of choosing the cycle strutures respectively. (The last is halved because
  choosing $\{a_1, a_2, a_3\}$ gives the same permutation as choosing $\{a_4, a_5, a_6\}$.)

  Then within each cycle structure, there is $0!4!$, $1!3!$ and $2!^2$ ways of
  choosing the order of each cycle, by an argument similar to part (b).

  Thus the number of such permutations is \[
    4!\binom{6}{1} + 3!\binom{6}{2} + \frac{2!^2}{2}\binom{6}{3}.
  \]
\end{solution}

% --- 1 (e) --------
\begin{problem}{2 (e)} \text{} \\
  How many partitions of $[6]$ have exactly three blocks?
\end{problem}

\begin{solution} \text{} \\
  We have three increasing compositions of $6$: $1 + 1 + 4$, $1 + 2 + 3$, and
  $2 + 2 + 2$.
  For the first, we just need to choose four elements, and leave the remaining
  two in singleton sets. There are $\binom{6}{4}$ ways of doing this.

  For the second, we can first choose the elements that go in the $3$-element set,
  and from the remaining $3$ elements choose the element that goes in the $2$-element set.
  There are $\binom{6}{3}\binom{3}{2}$ ways of doing this.

  For the third composition, we can use a similar strategy as the second composition,
  but because the order of choosing the sets can overcount, we must divide by $3!$,
  which counts the number of ways to order the three $2$-element sets.

  Thus the number of partitions with three blocks is \[
    \binom{6}{4}
    + \binom{6}{3}\binom{3}{2}
    + \frac{1}{3!}\binom{6}{2}\binom{4}{2}.
  \]
\end{solution}

% --- 1 (f) --------
\begin{problem}{2 (f)} \text{} \\
  There are four men and six women. Each man marries one of the women.
  In how many ways can this be done?
\end{problem}

\begin{solution} \text{} \\
  The first man could (in principle) marry any of the six women. And since
  we're assuming that these couples are not polygomous, the next many could
  marry any of the remaining five, and so on with the final two.\\
  Thus there are \[
    6\cdot5\cdot4\cdot3
  \] such potential pairings.
\end{solution}

% --- 1 (g) --------
\begin{problem}{2 (g)} \text{} \\
  Ten people split up into five groups of two each. In how many ways can this
  be done?
\end{problem}

\begin{solution} \text{} \\
  We can choose the people from the first group in $10 \cdot 9$ ways. But choosing
  person A and then person B is the same as choosing person B and then person A,
  so we divide by two.
  If the groups are labeled $1, 2, 3, 4, 5$, then there are $10!/2^5$ ways to
  choose the groups. But since the groups aren't labeled, we divide by the $5!$
  different orderings of the groups, resulting in a total of \[
    \frac{10!}{5!2^5}
  \] different pairings.
\end{solution}

% --- 1 (h) --------
\begin{problem}{2 (h)} \text{} \\
  How many compositions of $19$ use only the parts $2$ and $3$?
\end{problem}

\begin{solution} \text{} \\
  We must have an odd number of $3$s appear in the composition, in particular,
  there can be one $3$, three $3$s or five $3$s, which means that there must be
  eight, five, and two $2$s respectively. Therefore the number of compositions
  is \[
    \binom{1 + 8}{1} + \binom{3 + 5}{3} + \binom{5 + 2}{5}.
  \]
\end{solution}

% --- 1 (i) --------
\begin{problem}{2 (i)} \text{} \\
  In how many different ways can the letters of the word \textbf{MISSISSIPPI}
  be arranged if the four \textbf{S}s cannot appear consecutively?
\end{problem}

\begin{solution} \text{} \\
  The 11-letter word \textbf{MISSISSIPPI} contains 1 \textbf{M}, 2 \textbf{P},
  4 \textbf{I}s, and 4 \textbf{S}s. Thus (if we ignore the constraint), there
  are $\binom{11}{1}\binom{10}{2}\binom{8}{4}$ distinct anagrams.
  \\
  If all of the \textbf{S}s are to be placed together, we can treat
  \textbf{SSSS} as its own individual letter, meaning that there are
  $\binom{8}{1}\binom{7}{2}\binom{5}{4}$ anagrams with the \textbf{S}s together.
  \\
  This means that the number of anagrams where the \textbf{S}s cannot appear
  consecutively is \[
    \binom{11}{1}\binom{10}{2}\binom{8}{4} - \binom{8}{1}\binom{7}{2}\binom{5}{4}
  \]
\end{solution}

% --- 1 (j) --------
\begin{problem}{2 (j)} \text{} \\
  How many sequences $(a_1, a_2, \hdots, a_{12})$ are there consisting of four
  $0$s and eight $1$s if no consecutive terms are both $0$s?
\end{problem}

\begin{solution} \text{} \\
  Since the $0$s cannot be placed together, we can place the $0$s in any of the
  nine positions illustrated: \[
    \underbrace{} 1 \underbrace{} 1 \underbrace{} 1 \underbrace{} 1
    \underbrace{} 1 \underbrace{} 1 \underbrace{} 1 \underbrace{} 1
    \underbrace{}.
  \]
  Thus there are $\binom{9}{4}$ possible sequences.
\end{solution}

% --- 1 (k) --------
\begin{problem}{2 (k)} \text{} \\
  A box is filled with three blue socks, three red socks, and four green
  socks. Eight socks are pulled out, one at a time. In how many ways can this
  be done? (Socks of the same color are indistinguishable.)
\end{problem}

\begin{solution} \text{} \\
  We must draw either two, three, or four green socks.
  If we draw just two, then there are \[
    \underbrace{\binom{8}{2}}_\text{green}
    \binom{6}{3}
  \] ways to draw.
  If we draw three, then there are \[
    \underbrace{\binom{8}{3}}_\text{green}
    \underbrace{\binom{5}{3}}_\text{red} +
    \underbrace{\binom{8}{3}}_\text{green}
    \underbrace{\binom{5}{3}}_\text{blue}
  \] ways to choose.
  If we draw all four, then there are \[
    \underbrace{\binom{8}{4}}_\text{green}
    \underbrace{\binom{4}{3}}_\text{red} +
    \underbrace{\binom{8}{4}}_\text{green}
    \underbrace{\binom{4}{3}}_\text{blue} +
    \underbrace{\binom{8}{4}}_\text{green}
    \binom{4}{2}
  \]
\end{solution}

% --- 1 (l) --------
\begin{problem}{2 (l)} \text{} \\
  How many functions $f\colon [5]\rightarrow[5]$ are at most two-to-one?
\end{problem}

\begin{solution} \text{} \\
  Again, we must do some case work. We must either have a one-to-one function;
  a function that misses exactly one element, and hits one element twice;
  or a function that misses exactly two elements, and hits two elements twice.
  \begin{enumerate}
    \item In the first case, this is just a permutation: there are $5!$ such functions.
    \item In the second case, there are $5$ ways to choose the missed element and $4$
      ways to choose the doubled element in the codomain. Then there are
      $\binom{5}{2}$ ways choose the two elements in the domain that will be mapped
      to the doubled element. This leaves $3!$ ways to map the three elements in the
      domain onto the remaining three elements of the codomain. \[
        5\cdot4\cdot\binom{5}{2}\cdot3!
      \]
    \item First choose two elements in the domain and the element in the
    codomain that they map to. There are $\binom{5}{2}\cdot5$ ways of doing this.
    Next repeat the process; there are $\binom{3}{2}\cdot4$ ways of doing this
    again. Switching the orders that we choose the groups does not affect the
    function, so this double counts. Lastly, we have one element left in the
    domain, and three choices for where it goes. \[
      \frac{1}{2}\cdot\binom{5}{2}\cdot5\cdot\binom{3}{2}\cdot4\cdot3
    \]
  \end{enumerate}
  Thus the number of total maps is \[
    5! + 5!\binom52 + \frac{5!}{4}\binom52\binom32.
  \]
\end{solution}
\end{document}
