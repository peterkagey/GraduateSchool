\documentclass{article}

\usepackage[margin=1in]{geometry}
\usepackage{amsmath,amsthm,amssymb}
\usepackage{bbm,enumerate,mathtools}
\usepackage[hidelinks]{hyperref}
\usepackage{tikz}
\usetikzlibrary{matrix, arrows}

\newenvironment{problem}[2][Problem]{\begin{trivlist}
\item[\hskip \labelsep {\bfseries #1}\hskip \labelsep {\bfseries #2.}]}{\end{trivlist}}
\newenvironment{solution}[1][Solution.]{\begin{trivlist}
\item[\hskip \labelsep {\bfseries #1}]}{\end{trivlist}}
\newenvironment{problempart}[1]{\begin{trivlist}\item[\textbf{Part #1.}]}{\end{trivlist}}
\newcommand{\set}[1]{\{ #1 \}}
\newcommand{\ang}[1]{\langle #1 \rangle}
\begin{document}

\title{Combinatorics: Homework 11}
\author{Peter Kagey}

\maketitle

% -----------------------------------------------------
% First problem
% -----------------------------------------------------
\begin{problem}{1}
  Prove that every group of $2n$ children in which every child is friends with
  at least $n$ other children can be partitioned into pairs of friends in at
  least two different ways.
\end{problem}

\begin{proof} $ $ \\
  If there is a Hamiltonian cycle through the friendship network, then the
  children can be placed in a circle based on this cycle and labeled
  $v_1, v_2, \hdots, v_{2n}$.
  Then the children can be paired up
  $\set{v_1v_2, v_3v_4, \hdots, v_{2n-1}v_{2n}}$ or
  $\set{v_2v_3, v_4v_5, \hdots, v_{2n-2}v_{2n-1}, v_{2n}v_1}$.
  \\~\\
  Firstly, this graph must be connected; if not, its smallest connected component has at
  most $n$ vertices, which means that the degree of every vertex in the smallest
  connected component is at most $n-1$.
  \\~\\
  Suppose $v_1v_2\hdots v_k$ is the longest path in $G$, which is not a cycle.
  This means that all of $v_k$'s and $v_0$'s adjacent vertices are in
  $\set{v_1, \hdots, v_{k-1}}$. Since the $\deg(v_0) > n$ and $\deg(v_k) > n$,
  by the pigeonhole principle (since $k \leq 2n$), there exists some vertex
  $v_i \in \set{v_1, \hdots, v_{k-1}}$ such that $v_0v_{i+1} \in E$ and
  $v_iv_k \in E$.
  Then the path from $v_1$ to $v_{i+1}$ to $v_k$ to $v_{i}$ back to $v_1$ must
  be a Hamiltonian cycle---if it's not, then it misses some vertex $v_j$, which
  can be used to create a contradiction: a path from $v_j$ to the
  cycle exists because the graph is connected, and this path followed by
  a walk around the cycle creates a longer path than the assumed longest path.
\end{proof}
\pagebreak
% -----------------------------------------------------
% Second problem
% -----------------------------------------------------
\begin{problem}{2}
  Find the chromatic polynomial for the graph with \begin{align*}
    V &= \set{v_1, \hdots, v_n} \\
    E &= \set{v_1 v_2, v_2v_3, \hdots v_{n-1}v_{n}, v_1v_n}
      \cup \set{v_1v_3, v_1v_4, \hdots, v_1v_{n-1}},
  \end{align*} using the facts that the chromatic polynomial for the cyclic
  graph $C_n$ is \[
    P_{C_n}(k) = (k-1)^n + (-1)^n(k-1)
  \] and that the chromatic polynomial for any graph $G = G_1 \cup G_2$ where
  $G_1 \cap G_2 = \set{v}$ is \[
    P_G(k) = \frac{1}{k}p_{G_1}(k)p_{G_2}(k).
  \]
\end{problem}

\begin{solution} \text{} \\
  There are $k$ ways
  to pick the color for $v_1$, $(k-1)$ ways to pick the color for $v_2$, since
  $v_1$ is adjacent to $v_2$, and then $(k-2)$ ways to pick the color for $v_i$
  as $i$ goes from $3$ to $n$, since $v_i$ is adjacent to $v_1$ and $v_{i-1}$.
  Therefore if $G_n$ is the graph with $n$ vertices \[
    P_{G_n}(k) = k(k-1)(k-2)^{n-2}.
  \]
  \\~\\
  In order to use the hint, contraction/deletion gives \[
    P_{G_n}(k) = P_{G_n'}(k) - P_{G_n''}(k)
  \] where $G_n' = G_n - v_{n-1}v_n$ and $G_n'' = G_n/v_{n-1}v_n$.
  \\
  With induction hypothesis given above, using the information that $G_3 = C_3$,
  \begin{align*}
    P_{G_3}(k)
    &= P_{C_3}(k) \\
    &= (k-1)^3 - (k-1) \\
    &= (k-1)((k-1)^2 - 1) \\
    &= k(k-1)(k-2)
  \end{align*} so the base case is satisfied for $n=3$.
  \\
  The first graph in the contraction/deletion argument satisfies \[
    G_n' = G_{n-1} \cup \underbrace{(\set{v_1, v_n}, \set{v_1v_n})}_{C_2}
  \] where $G_{n-1} \cap C_2 = \set{v_{n-1}}$. Therefore, by the facts given
  along with the induction hypothesis, the chromatic
  polynomial of $G_n'$ is given by \begin{align*}
    P_{G_n'}(k)
    &= \frac 1k
      \cdot \underbrace{k(k-1)(k-2)^{n-2}}_{\displaystyle P_{G_{n-1}}(k)}
      \cdot \underbrace{k(k-1)}_{\displaystyle P_{C_2}(k)} \\
    &= k(k-1)^2(k-2)^{n-2}.
  \end{align*}
  Similarly, $G_n'' = G_{n-1}$, so by the induction hypothesis together with
  contraction deletion, \begin{align*}
    P_{G_n}(k) &= P_{G_n'}(k) - P_{G_n''}(k) \\
      &= k(k-1)^2(k-2)^{n-2} - k(k-1)(k-2)^{n-2} \\
      &= k(k-1)(k-2)^{n-2} ((k-1) - 1) \\
      &= k(k-1)(k-2)^{n-1},
  \end{align*} as desired.
\end{solution}
\pagebreak
% -----------------------------------------------------
% Third problem
% -----------------------------------------------------
\begin{problem}{3}
  Using the deletion/retraction recurrence on a graph, prove that the number of
  acyclic orientations of $G$ is equal to $(-1)^{|V|}p_G(-1)$ where an acyclic
  orientation is an assignment of a direction to each edge such that there are
  no directed cycles.
\end{problem}

\begin{proof} \text{} \\
  Let $A(G)$ be the number of acyclic orientations on $G$.
  By induction on $|V| + |E|$ with base case of the singleton graph $\mathbf 1$, which has
  chromatic polynomial $P_{\mathbf 1}(k) = k$.
  The only assignment of direction to each edge is the empty assignment, and
  \[
    (-1)^{|\mathbf 1|}P_1(-1) = (-1)^1(-1) = 1 = A(\mathbf 1)
  \] as desired.
  \\~\\
  Recall the usual contraction/deletion recurrence \[
    P_{G}(k) = P_{G'}(k) - P_{G''}(k)
  \] with $G' = G - uv$ and $G'' = G/uv$.
  \\
  Furthermore, the relation $A(G) = A(G') + A(G'')$ holds.
  \\~\\
  Applying the induction hypothesis together with the recurrence gives
  \begin{align*}
    A(G) &= A(G') + A(G'')\\
    &= (-1)^{|V|}P_{G'}(-1) + (-1)^{|V|-1}P_{G'}(-1) \\
    &=(-1)^{|V|}(P_{G'}(-1) - P_{G''}(-1)) \\
    &= (-1)^{|V|}P_G(-1)
  \end{align*} as desired.
\end{proof}
\pagebreak
% -----------------------------------------------------
% Fourth problem
% -----------------------------------------------------
\begin{problem}{4}
  Let $G$ be a planar connected bipartite graph $V = V_1 \coprod V_2$ and
  $E \subset V_1 \times V_2$ such that there is no $4$-cycle and no vertex of
  degree $1$. Show that $3(|V|-2) \geq 2|E|$.
\end{problem}

\begin{solution} \text{} \\
  Since $G$ is planar, the Euler characteristic states that \[
    v - 2 = e - f
  \]
  Since $G$ is bipartite, any cycle must have even length, so $G$ does not
  contain any $3$-cycles or $5$-cycles. Since $G$ is
  assumed to be simple (and therefore does not have multiple edges) $G$ does
  not contain any $2$-cycles. By assumption, $G$ does not contain any $4$-cycles.
  Thus any face of $G$ must be adjacent to $6$ or more edges, whenever $G$ has
  at least $6$ edges, and $G$ must have at least $6$ edges by the ``no vertex of
  degree $1$ criterion'' coupled with having no cycles smaller than $6$.
  \\~\\
  Since every edge is adjacent to at most two faces, and since $e \geq 6$, \[
    \frac f2 \leq \frac e6
  \] and so in particular \begin{alignat*}{2}
    v - 2 &= e - f &&\geq e - \frac 13e \\
    3(|V| - 2)& &&\geq 2|E|.
  \end{alignat*}
\end{solution}
\end{document}
