\documentclass{article}

\usepackage[margin=1in]{geometry}
\usepackage{amsmath,amsthm,amssymb}
\usepackage{bbm,enumerate,mathtools}
\usepackage[hidelinks]{hyperref}
\usepackage{tikz}
\usetikzlibrary{matrix, arrows}

\newenvironment{problem}[2][Problem]{\begin{trivlist}
\item[\hskip \labelsep {\bfseries #1}\hskip \labelsep {\bfseries #2.}]}{\end{trivlist}}
\newenvironment{solution}[1][Solution.]{\begin{trivlist}
\item[\hskip \labelsep {\bfseries #1}]}{\end{trivlist}}
\newenvironment{problempart}[1]{\begin{trivlist}\item[\textbf{Part #1.}]}{\end{trivlist}}

\begin{document}

\title{Combinatorics: Homework 8}
\author{Peter Kagey}

\maketitle

% -----------------------------------------------------
% First problem
% -----------------------------------------------------
\begin{problem}{34} $[2]$ \\
  Find all nonisomorphic posets P such that \[
    F(J(P), x) = (1 + x)(1 + x^2)(1 + x + x^2)
  \]
\end{problem}

\begin{solution} \text{}
\end{solution}
\pagebreak
% -----------------------------------------------------
% Second problem
% -----------------------------------------------------
\begin{problem}{46 a} $[2]$ \\
  Let $f(n)$ be the number of sublattices of rank $n$ of the boolean algebra
  $B_n$. Show that $f(n)$ is also the number of partial orders $P$ on $[n]$.
\end{problem}

\begin{solution} \text{} \\
  Let $\phi$ be a map which sends a poset $P$ (with underlying set $[n]$) to
  $J(P) \subset B_n$.
  Since $B_n$ is a finite distributive lattice, and every sublattice of a
  finite distributive lattice is a finite distributive lattice, we can
  take the set of join irreducibles of any rank $n$ sublattice, and this will
  be isomorphic to a poset on $[n]$.
\end{solution}
\pagebreak
% -----------------------------------------------------
% Third problem
% -----------------------------------------------------
\begin{problem}{53} $[2]$ \\
  Let $P$ be a finite $n$-element poset. Simplify the two sums \[
    f(P) = \sum_{I \subset J(P)} e(I)e(\overline I),
  \] \[
    g(P) = \sum_{I \subset J(P)} \binom{n}{\#I} e(I)e(\overline I),
  \] where $\overline I$ denotes the complement $P - I$ of the order ideal $I$.
\end{problem}

\begin{proof}
\end{proof}
\pagebreak
% -----------------------------------------------------
% Fourth problem
% -----------------------------------------------------
\begin{problem}{57} ~
  \begin{enumerate}[a.]
    \item $[2]$ Let $P$ be an $n$-element poset. If $t \in P$, then set
    $\lambda_t = \#\{s \in P : s \leq t\}$.
    Show that \[
      e(P) \geq \frac{n!}{ \prod_{t \in P} \lambda_t}.
    \]
    \item $[2+]$ Show that the equality holds if and only if every component of $P$ is
    a rooted tree.
  \end{enumerate}
\end{problem}

\begin{proof}
  \begin{enumerate}[a.]
    \item
    \item By induction, the base case is clear. When $n=1$, there is only
    one poset (which is a rooted tree) with one linear extension. \[
      e([1]) = \frac{1!}{\lambda_{1}} = 1
    \]
  \end{enumerate}
  Thus given some rooted tree $P$, we can take the subposet $P - \hat 1$, which
  is a disjoint union of rooted trees
  $P_1 + P_2 + \hdots + P_k$ with $n_1, n_2, \hdots, n_k$ elements respectively.
  Since $P$ has a unique maximum, it must be labeled with $n$, then we can
  then choose which letters go in each sub-tree (using a multinomial
  coefficient), and then there are $e(P_i)$ ways to order the $n_i$ labels
  for each $P_i$. Therefore \begin{align*}
      e(P) &= \binom{n - 1}{n_1, n_2, \hdots, n_k}e(P_1)e(P_2)\hdots e(P_k)
      \\
      &= \binom{n - 1}{n_1, n_2, \hdots, n_k}
      \frac{n_1!}{\prod_{t \in P_1} \lambda_t}
      \frac{n_2!}{\prod_{t \in P_2} \lambda_t}
      \hdots
      \frac{n_k!}{\prod_{t \in P_k} \lambda_t}
      \\
      &= \left(\frac{(n - 1)!}{n_1! n_2! \hdots n_k!}\right)
      \frac{n_1!n_2!\hdots n_k!}{\prod_{t \in P - \hat 1} \lambda_t}
      \\
      &= \frac{(n-1)!}{\prod_{t \in P - \hat 1} \lambda_t}
  \end{align*}
  Since all $n$ elements of $P$ are less than or equal to $\hat 1$,
  $\lambda_{\hat 1} = n$, \[
    n\prod_{t \in P - \hat 1} \lambda_t = \prod_{t \in P} \lambda_t
  \] and thus \[
    e(P) = \frac{n(n-1)!}{n\prod_{t \in P - \hat 1} \lambda_t} = \frac{n!}{\prod_{t \in P} \lambda_t}
  \] as desired.
\end{proof}
\end{document}
