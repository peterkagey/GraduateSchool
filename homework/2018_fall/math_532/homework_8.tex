\documentclass{article}

\usepackage[margin=1in]{geometry}
\usepackage{amsmath,amsthm,amssymb}
\usepackage{bbm,enumerate,mathtools}
\usepackage[hidelinks]{hyperref}
\usepackage{tikz}
\usetikzlibrary{matrix, arrows}

\newenvironment{problem}[2][Problem]{\begin{trivlist}
\item[\hskip \labelsep {\bfseries #1}\hskip \labelsep {\bfseries #2.}]}{\end{trivlist}}
\newenvironment{solution}[1][Solution.]{\begin{trivlist}
\item[\hskip \labelsep {\bfseries #1}]}{\end{trivlist}}
\newenvironment{problempart}[1]{\begin{trivlist}\item[\textbf{Part #1.}]}{\end{trivlist}}

\begin{document}

\title{Combinatorics: Homework 8}
\author{Peter Kagey}

\maketitle

% -----------------------------------------------------
% First problem
% -----------------------------------------------------
\begin{problem}{1: 58 (a)} $[2]$ \\
  For $u \in \mathfrak S_k$, let $s_u(n) = \#S_u(n)$ the number of permutations
  $w \in \mathfrak S_n$ avoiding $u$. If also $v \in \mathfrak S_k$, then write
  $u \sim v$ if $s_u(n) = s_v(n)$ for all $n \geq 0$.
  \\~\\
  Let $u, v \in \mathfrak S_k$. Suppose that the permutation matrix $P_v$ can
  be obtained from $P_u$ by one of the eight dihedral symmnetries of the square.
  Show that $u \sim v$.
  \\~\\
  We then say that $u$ and $v$ are equivalent by symmetry, denoted
  $u \approx v$. What are the $\approx$ equivalence classes for $\mathfrak S_3$?
\end{problem}

\begin{solution} \text{}
  Suppose that $\sigma$ is an element of the dihedral group of the square
  and $P_v$ and $P_u$ are permutation matrices such that
  $P_v = \sigma P_u$ under the group action.
  \\
  Then there is an ``obvious'' bijection between $S_u(n)$ and $S_v(n)$, namely
  $f\colon S_u(n) \rightarrow S_v(n)$ maps $P_u \mapsto \sigma P_u$. To go back,
  simply do the group action of $\sigma^{-1}$.
  \\~\\
  There are only two equivalence classes for $\mathfrak S_3$: \begin{align*}
    \begin{bmatrix} 1 & 0 & 0 \\ 0 & 1 & 0 \\ 0 & 0 & 1 \end{bmatrix}
    &\approx
    \begin{bmatrix} 0 & 0 & 1 \\ 0 & 1 & 0 \\ 1 & 0 & 0 \end{bmatrix}
    \text{ and}
    \\
    \begin{bmatrix} 1 & 0 & 0 \\ 0 & 0 & 1 \\ 0 & 1 & 0 \end{bmatrix}
    &\approx
    \begin{bmatrix} 0 & 1 & 0 \\ 1 & 0 & 0 \\ 0 & 0 & 1 \end{bmatrix}
    \approx
    \begin{bmatrix} 0 & 1 & 0 \\ 0 & 0 & 1 \\ 1 & 0 & 0 \end{bmatrix}
    \approx
    \begin{bmatrix} 0 & 0 & 1 \\ 1 & 0 & 0 \\ 0 & 1 & 0 \end{bmatrix}.
  \end{align*}
\end{solution}
\pagebreak
% -----------------------------------------------------
% Second problem
% -----------------------------------------------------
\begin{problem}{2}
  Find the recurrence and generating function formula for the ``corner colored''
  paths $(0, 0) \rightarrow (n, n)$ with steps $(1, 0)$ or $(0, 1)$, on or above
  the main diagonal such that every inner corner of the path of the kind
  $(a, b) \rightarrow (a + 1, b) \rightarrow (a + 1, b + 1)$ can be colored in
  one of two possible colors.
\end{problem}

\begin{solution} \text{} \\
  We'll use the same technique that was used to compute the Catalan numbers in
  class; namely, we'll sum over all positions where the Dyck paths first hit the
  diagonal (stictly above $(0, 0)$) and take the products of the smaller Dyck
  paths below and this point.
  \begin{align*}
    f(0) &= 1 \\
    f(n) &= f(n - 1) + \sum_{k=1}^{n - 1}2f(k-1)f(n-k)
  \end{align*}
  \\~\\
  We also do the same sort of generating function argument. \begin{align*}
    F(x) &= \sum_{n=0}^\infty f(n)x^n \\
         &= 1 + \sum_{n=1}^\infty f(n)x^n \\
         &= 1 + \sum_{n=1}^\infty \left(f(n - 1) + \sum_{k=1}^{n - 1}2f(k-1)f(n-k)\right)x^n \\
         &= 1 + \underbrace{\sum_{n=1}^\infty f(n - 1)x^n}_{xF(x)} + \sum_{n=1}^\infty\left(\sum_{k=1}^{n - 1}2f(k-1)f(n-k)\right)x^n.
  \end{align*}
  By reindexing the sums on the right with the standard trick, letting
  $n = k + j$ where $j$ now runs from $0$ to infinity, we have
  \begin{align*}
    F(x) &= 1 + xF(x) + \sum_{j=0}^\infty\left(\sum_{k=1}^{\infty}2f(k-1)f(j)\right)x^{k + j} \\
    &= 1 + xF(x) + 2x\sum_{j=0}^\infty\sum_{k=0}^{\infty}f(k)f(j)x^k x^j \\
    &= 1 + xF(x) + 2xF(x)^2
  \end{align*}
  which means that solving $0 = 2xF(x)^2 + (x - 1)F(x) + 1$ for $F(x)$ by the
  quadratic formula yields \[
    \frac{-(x - 1) \pm \sqrt{(x - 1)^2 - 4\cdot2x}}{2\cdot 2x}.
  \] When $x$ is near zero, $F(x)$ should be close to $f(0)$, so the root we
  care about is \[
    F(x) = \frac{1 - x - \sqrt{x^2 - 10x + 1}}{4x}
  \]
\end{solution}
\pagebreak
% -----------------------------------------------------
% Third problem
% -----------------------------------------------------
\begin{problem}{3}
  Find the number of $132$-avoiding alternating permutations of length $2n$.
\end{problem}

\begin{proof}
  I will construct a bijection
  $\varphi\colon \mathfrak S_{n}^{(132)} \rightarrow \mathfrak S_{2n,\text{alt}}^{(132)}$
  between $132$-avoiding permutations of length $n$ and
  $132$-avoiding alternating permutations of length $n$.

  The map $\varphi$ is simple, but its inverse is more complicated. In
  particular, $\varphi$ takes in a word in $\mathfrak S_{2n,\text{alt}}^{(132)}$
  and outputs the relative order of the odd letters.
  \[
    w_1 w_2 w_3 \hdots w_{2n-1} w_{2n} \xmapsto{\varphi}
    \operatorname{order}(w_1 w_3 \hdots w_{2n-1})
  \]
  For example, if $w = 65748231$, then
  $\varphi(w) = \operatorname{order}(6783) = 2341$.
  \\~\\
  Going back is a bit trickier.
  \begin{enumerate}
    \item Start with the permutation $w' \in \mathfrak S_{n}^{(132)}$.
    \item For $i$ increasing from $1$ to $n-1$, recursively insert
      $a = \min(w_1, w_2, \hdots w_{2i})$ into the $2i$th position, then
      increment all letters that are greater than or equal to $a$, except
      for the newly inserted letter.
    \item Increment everything and append 1.
  \end{enumerate}
  For example, starting with $w' = 2341$ this algorithm recovers the original
  alternating permutation. \begin{align*}
    \begin{tabular}{ c c c c c c c c }
    $2$ & $\underbrace{\hspace{1.5cm}}_{\min(2, 3)}$  & $3$ &  & $4$ & & $1$ & \\
    $\downarrow$ & & $\downarrow$ &  & $\downarrow$ & & $\downarrow$ & \\
    $3$ & $2$ & $4$ & $\underbrace{\hspace{1.5cm}}_{\min(2, 3, 4, 5)}$ & $5$ & & $1$ & \\
    $\downarrow$ & $\downarrow$ & $\downarrow$ & & $\downarrow$ & & $\downarrow$ & \\
    $4$ & $3$ & $5$ & $2$ & $6$ & $\underbrace{\hspace{1.5cm}}_{\min(2, 3, 4, 5, 6, 1)}$ & $1$ & \\
    $\downarrow$ & $\downarrow$ & $\downarrow$ & $\downarrow$ & $\downarrow$ & & $\downarrow$ & \\
    $5$ & $4$ & $6$ & $3$ & $7$ & $1$ & $2$ & $\underbrace{\hspace{1.5cm}}_{1}$ \\
    $\downarrow$ & $\downarrow$ & $\downarrow$ & $\downarrow$ & $\downarrow$ & $\downarrow$ & $\downarrow$ & \\
    $6$ & $5$ & $7$ & $4$ & $8$ & $2$ & $3$ & 1
    \end{tabular}
  \end{align*}

  In practice, this is a sequence of maps \[
    \mathfrak S_n^{(132)} \xrightarrow{\psi_1}
    \mathfrak S_{n+1}^{(132)} \xrightarrow{\psi_2}
    \mathfrak S_{n+3}^{(132)} \xrightarrow{\psi_3}
    \hdots \xrightarrow{\psi_{n-1}}
    \mathfrak S_{2n-1} \xrightarrow{\psi_n}
    \mathfrak S_{2n}
  \]
  Where each $\psi_i$ has the property that for all $i \neq n$, if the preimage
  is alternating for letters $w_1 < w_2 > w_3 < \hdots > w_{2i-1}$, then the
  image will be alternating for letters $w_1 < w_2 > w_3 < \hdots > w_{2i+1}$.
  (In the case of $\psi_n$,  there is no $w_{2n + 1}$ letter, but this holds up
  to $w_{2n}$.)
  Also $\psi_i$ preserves the relative order of all of the letters
  away from position $i$.
  \\
  There is only one map $\psi_i$ that satisfies this: \begin{enumerate}
    \item If the inserted letter is greater than either of the neighboring
    letters, then it fails to satify the
    alternating condition.
    \item If the inserted letter is less than its left neighbor $w_{2i-1}$, but
    greater than some letter $w_j$ in the prefix, then the subsequence
    $w_j, w_{2i-1}, a$ is not $132$-avoiding.
    \item If the inserted letter, $a$, is less than
    $\min(w_1, w_2, \hdots w_{2i})$, then there exists some letter $w_k$ with
    $k > 2i$ such that the subsequence $a, w_{2i}, w_k$ is not $132$ avoiding.
  \end{enumerate}
  Therefore the map
  $\phi^{-1} = \psi_n \circ \psi_{n - 1} \circ \hdots \circ \psi_1$ recovers the
  orginal sequence, and so
  $\phi \circ \phi^{-1} = \phi^{-1} \circ \phi = \operatorname{id}$, and
  $\phi$ is a bijection.
  Thus $\#\mathfrak S_{n}^{(132)} = \#\mathfrak S_{2n,\text{alt}}^{(132)} = C_n$.
\end{proof}
\end{document}
