\documentclass{article}

\usepackage[margin=1in]{geometry}
\usepackage{amsmath,amsthm,amssymb}
\usepackage{bbm,enumerate,mathtools}
\usepackage[hidelinks]{hyperref}
\usepackage{tikz}
\usetikzlibrary{matrix, arrows}

\newenvironment{problem}[2][Problem]{\begin{trivlist}
\item[\hskip \labelsep {\bfseries #1}\hskip \labelsep {\bfseries #2.}]}{\end{trivlist}}
\newenvironment{solution}[1][Solution.]{\begin{trivlist}
\item[\hskip \labelsep {\bfseries #1}]}{\end{trivlist}}
\newenvironment{problempart}[1]{\begin{trivlist}\item[\textbf{Part #1.}]}{\end{trivlist}}

\begin{document}

\title{Combinatorics: Homework 6}
\author{Peter Kagey}

\maketitle

% -----------------------------------------------------
% First problem
% -----------------------------------------------------
\begin{problem}{14 (a)} $[2+]$ \\
  Let $A_k(n)$ denote the number of $k$-element antichains in the boolean algebra
  $B_n$. Show that \begin{enumerate}[(i)]
    \item $A_1(n) = 2^n$
    \item $A_2(n) = \displaystyle \frac{1}{2}(4^n - 2\cdot3^n + 2^n)$
    \item $A_3(n) = \displaystyle \frac{1}{6}(8^n - 6\cdot6^n + 6\cdot5^n + 3\cdot4^n - 6\cdot3^n + 2\cdot2^n)$
  \end{enumerate}
\end{problem}

\begin{solution} \text{}
  \begin{enumerate}[(i)]
    \item Clearly any singleton set $\{ A \}$ where $A \in 2^{[n]}$ is a $1$-element antichain, so \[
      A_1(n) = \#(2^{[n]}) = 2^n.
    \]
    \item It is easiest to count ordered pairs $(A, B)$ such that
    $A \not\subset B$ and $B \not\subset A$, and then divide by two
    to count the $2$-element sets. By the Principle of Inclusion-Exclusion,
    we can start by counting all ordered pairs without restriction.
    For each $i \in [n]$, we can have \begin{enumerate}[(1)]
      \item $i \in A$ and $i \in B$,
      \item $i \in A$ and $i \not\in B$,
      \item $i \not\in A$ and $i \in B$, or
      \item $i \not\in A$ and $i \not\in B$.
    \end{enumerate}
    Thus there are $4^n$ choices of sets without restriction.
    \\~\\
    However, we must remove the case where $A \subset B$ (respectively
    $B \subset A$) which means that for each $i \in [n]$ we can have
    \begin{enumerate}[(1)]
      \item $i \in A \cap B$
      \item $i \in A^c \cap B$ (respectively $i \in A \cap B^c$), or
      \item $i \in A^c \cap B^c$.
    \end{enumerate}
    Thus there are $2\cdot3^n$ possibilities.
    However, this double-counts the case where $A \subset B$ and $B \subset A$, namely $A = B$, so we need to add these $2^n$ pairs back.
    Thus \[
      A_2(n) = \frac12(4^n - 2\cdot3^n + 2^n).
    \]
    \item Again by the Principle of Inclusion-Exclusion, we will count ordered
    triples $(A_1, A_2, A_3)$, that meet the criteria and then divide them by
    the number of permutations.
    There are nine cases, which correspond to subsets of ordered pairs.
    (All indices are distict):
    % \begin{enumerate}[(1)]
    %   \item Any triple,  $\emptyset$
    %   \item $A_i \subset A_j$, $\{(i, j)\}$
    %   \item $A_i \subset A_j$ and $A_i \subset A_k$, $\{(i, j), (i, k)\}$
    %   \item $A_i \subset A_j$ and $A_k \subset A_j$, $\{(i, j), (k, j)\}$
    %   \item $A_i = A_j$ $\{(i, j), (j, i)\}$
    %   \item $A_i \subset A_j \subset A_k$ $\{(i, j), (j, k), (i, k)\}$
    %   \item $A_i = A_j \subset A_k$ $\{(i, j), (j, i), (i, k), (j, k)\}$
    %   \item $A_i \subset A_j = A_k$ $\{ (i, j), (i, k), (j, k), (k, j)\}$
    %   \item $A_i = A_j = A_k$ $\{(i, j), (i, k), (j, i), (j, k), (k, i), (k, j)\}$
    % \end{enumerate}
    % Next we will count how many of each case
    \begin{alignat}{2}
      &\#\{(A_1, A_2, A_3) : A_i \in B_n\}                                   &&= 8^n \\
      &\#\{(A_1, A_2, A_3) : A_i \in B_n, A_i \subset A_j\}                  &&= 6\cdot6^n \\
      &\#\{(A_1, A_2, A_3) : A_i \in B_n, A_i \subset A_j, A_i \subset A_k\} &&= 3\cdot5^n \\
      &\#\{(A_1, A_2, A_3) : A_i \in B_n, A_i \subset A_j, A_k \subset A_j\} &&= 3\cdot5^n \\
      &\#\{(A_1, A_2, A_3) : A_i \in B_n, A_i = A_j\}                        &&= 3\cdot4^n \\
      &\#\{(A_1, A_2, A_3) : A_i \in B_n, A_i \subset A_j \subset A_k\}      &&= 6\cdot5^n \\
      &\#\{(A_1, A_2, A_3) : A_i \in B_n, A_i = A_j \subset A_k\}            &&= 3\cdot3^n \\
      &\#\{(A_1, A_2, A_3) : A_i \in B_n, A_i \subset A_j = A_k\}            &&= 3\cdot3^n \\
      &\#\{(A_1, A_2, A_3) : A_i \in B_n, A_i = A_j = A_k\}                  &&= 2^n
  \end{alignat}
  \begin{enumerate}[(1)]
    \item Start with one copy of each of the $8^n$ triples. \[
      1\cdot8^n + ...
    \]
    \item For each of the six pairs $(i, j)$
    such that $A_i \subset A_j$,
    (1) adds it once,
    so subtract these off.
    \[
      8^n + -1\cdot6\cdot6^n + ...
    \]
    \item For each of the three pairs $(i, \{j, k\})$
    such that $A_i \subset A_j$ and $A_i \subset A_k$,
    (1) adds it once and
    (2) removes it twice,
    so we need to add it back once. \[
      8^n - 6\cdot6^n + 1\cdot3\cdot5^n + ...
    \]
    \item For each of the three pairs $(\{i, k\}, j)$
    such that $A_i \subset A_j$ and $A_k \subset A_j$,
    (1) adds it once and
    (2) removes it twice,
    so we need to add it back once. \[
      8^n - 6\cdot6^n + 3\cdot5^n + 1\cdot3\cdot5^n + ...
    \]
    \item For each of the three unordered pairs $\{i, j\}$
    such that $A_i = A_j$,
    (1) adds it once,
    (2) removes it twice, and the relation is not a subset of (3) or (4),
    so we need to add it back once \[
      8^n - 6\cdot6^n + 6\cdot5^n + 1\cdot3\cdot4^n + ...
    \]
    \item For each of the six ordered triples $(i, j, k)$
    such that $A_i \subset A_j \subset A_k$,
    (1) adds it once,
    (2) removes it three times, and
    (3) and (4) both add it back once,
    so it doesn't affect the sum: \[
      8^n - 6\cdot6^n + 6\cdot5^n + 3\cdot4^n + 0\cdot6\cdot5^n + ...
    \]
    \item For each of the three pairs $(\{i, j\}, k)$
    such that $A_i = A_j \subset A_k$,
    (1) adds it once,
    (2) removes it four times,
    (3) adds it twice,
    (4) adds it once, and
    (5) adds it once,
    so we need to subtract if off once: \[
      8^n - 6\cdot6^n + 6\cdot5^n + 3\cdot4^n + -1\cdot3\cdot3^n + ...
    \]
    \item For each of the three pairs $(i, \{j, k\})$
    such that $A_i \subset A_j = A_k$,
    (1) adds it once,
    (2) removes it four times,
    (3) adds it once,
    (4) adds it twice, and
    (5) adds it once,
    so we need to subtract it off once: \[
      8^n - 6\cdot6^n + 6\cdot5^n + 3\cdot4^n -3\cdot3^n + -1\cdot3\cdot3^n...
    \]
    \item Lastly, for the single unordered triple $\{ i, j, k \}$
    such that $A_i = A_j = A_k$,
    (1) adds it once,
    (2) removes it six times,
    (3) and (4) each add it three times,
    (5) adds it three times,
    (6) does not affect the sum,
    (7) and (8) each subtract it once,
    resulting in it needing to be added back in twice: \[
      8^n - 6\cdot6^n + 6\cdot5^n + 3\cdot4^n - 6\cdot3^n + 2\cdot2^n
    \]
    \end{enumerate}
    This gives us our desired sum.
  \end{enumerate}
\end{solution}
\pagebreak
% -----------------------------------------------------
% Second problem
% -----------------------------------------------------
\begin{problem}{14 (b)} $[2+]$ \\
  Show that for fixed $k \in \mathbb P$ there exist integers
  $a_{k,2},a_{k,3},\hdots,a_{k,2^k}$ such that \[
    A_k(n) = \frac{1}{k!} \sum_{i = 2}^{2^k} a_{k,i}i^n.
  \]
  Show in particular that \begin{enumerate}[(i)]
    \item $a_{k,2^k} = 1$,
    \item $a_{k,i} = 0$ if $3\cdot2^{k-2} < i < 2^k$, and
    \item $a_{k,3\cdot2^{k-2}} = k(k-1)$.
  \end{enumerate}
\end{problem}

\begin{solution} \text{} \\
  This can be proven by a similar argument to above: counting ordered
  $k$-tuples by the Principle of Inclusion-Exclusion, where we add and subtract
  all possible sets of set inclusions. \[
    \{(A_1, A_2, \hdots, A_k) : A_i \subset A_j \text{ for all } (i, j) \in S \subset [k] \times [k] \} \subset B_n^k
  \]
  For some given set of set inclusions, this constrains the element-wise truth
  table of possible values. That is, for all $i \in [n]$, we can write a list
  of possible ways to put $i$ in each set, as illustrated in part
  \textbf{a (ii)}. Since each list cooresponds to a subset of $[k]$, there are
  at most $2^k$ ways to place $i$ in each subset. Since there are $n$ $i$s, this
  means that there are at most $2^{kn}$ such sets. Since we can choose to put
  $i$ in every set or in no set, there are at least $2^n$ such sets.
  \\
  \begin{enumerate}[(i)]
    \item We begin with all possible sets, so there are $2^k$ possible choices for
    each of the $n$ positions, thus the ``no relation'' $k$-tuple that we start from
    has $1\cdot 2^{nk}$ elements.
    \item Since the above part takes care of the ``no relation'' tuples, we must
    have at least one restriction, namely $A_i \subset A_j$. This means that
    for each $m \in [n]$ we can have
    \begin{itemize}
      \item $m \in A_i$ and $m \in A_j$,
      \item $m \not\in A_i$ and $m \in A_j$, or
      \item $m \not\in A_i$ and $m \not\in A_j$.
    \end{itemize}
    And once these choices are made, there are no restrictions on choosing
    whether $m$ is in the other $k - 2$ sets. Thus, with the minimal positive number
    of relations (one) there are at least $3 \cdot 2^{k - 2}$ choices, so
    there is no relation corresponding to $\ell$ choices where
    $3 \cdot 2^{k - 2} < \ell < 2^k$.
    \item Similarly, the coefficient is given by the number of (ordered) ways to
    choose two positions in the $k$-tuple for the relation $A_i \subset A_j$.
    Namely, \[
      \#\{(i, j) | i \neq j, i \in [k], j \in [k] \} = k(k-1).
    \]
    Since we subtract these off, the coefficient is $-k(k-1)$, as desired.
  \end{enumerate}
\end{solution}
\pagebreak
% -----------------------------------------------------
% Third problem
% -----------------------------------------------------
\begin{problem}{25 (a)} $[2]$ \\
  Let $f_i(m, n)$ be the number of $m \times n$ matrices $0$s and $1$s with
  at least one $1$ in every row and column, and with a total of $i$ $1$s. Use
  the Principle of Inclusion-Exclusion to show that \[
    \sum_i f_i(m, n) t^i
    = \sum_{k=0}^n (-1)^k \binom nk ((1 + t)^{n - k} - 1)^m.
  \]
\end{problem}

\begin{proof} \text{} \\
  First, I will count a related function: let $g_i(m, n)$ be the number of
  $m \times n$ matrices $0$s and $1$s with at least one $1$ in every row, and
  with a total of $i$ $1$s.
  \\
  Then by the Principle of Inclusion-Exclusion, we can start with all matrices
  that have $i$ ones, and remove those that have no $1$s in the $k$th row,
  add back those that have no $1$s in the $k_1$th and $k_2$th row, etc \[
    g_i(m, n) = \sum_{j=0}^m (-1)^j\binom mj
    \underbrace{\binom {n(m-j)}{i}}_{
      \substack{
        \text{Ways to place $i$ $1$s in} \\
        n \times m-j \text{ submatrix}
      }
    }
  \]
  From here, we can do an analougous process for $f_i(m, n)$, only with columns
  instead of rows \begin{align*}
    f_i(m, n)
    &= \sum_{k=0}^n (-1)^k\binom nk g_i(m, n - k) \\
    &= \sum_{k=0}^n (-1)^k\binom nk \left(
      \sum_{j=0}^m (-1)^j\binom mj \binom {(n - k)(m-j)}{i}
    \right)
  \end{align*} so the generating function \[
    \sum_i f_i(m, n) t^i = \sum_i \left(\sum_{k=0}^n (-1)^k\binom nk \left(
      \sum_{j=0}^m (-1)^j\binom mj \binom {(n - k)(m-j)}{i}
    \right)\right) t^i.
  \]

  By rearranging the sums, it is enough to show that for some arbitrary
  $n, k \in \mathbb P$, \begin{align*}
  \sum_i \sum_{j=0}^m (-1)^j\binom mj \binom {(n - k)(m-j)}{i} t^i
    &= ((1 + t)^{n - k} - 1)^m \\
    &= \sum_{j=0}^m \binom mj ((1 + t)^{n - k})^{m-j} (-1)^j \\
    &= \sum_{j=0}^m (-1)^j\binom mj \left(\sum_{i}\binom{(n - k)(m - j)}{i} t^i\right) \\
    &= \sum_i\sum_{j=0}^m (-1)^j\binom mj \binom{(n - k)(m - j)}{i} t^i.
  \end{align*}
  Therefore the desired identitity holds.
\end{proof}
\pagebreak
% -----------------------------------------------------
% Second problem
% -----------------------------------------------------
\begin{problem}{25 (b)} $[2]$ \\
  Show that \[
    \sum_{m, n \geq 0} \sum_{i \geq 0} f_i(m,n)t^i\frac{x^m y^n}{m!n!}
    = e^{-x-y}\sum_{i \geq 0} \sum_{j \geq 0}(1 + t)^{ij}\frac{x^i y^j}{i!j!}.
  \]
\end{problem}

\begin{proof} \text{} \\
  I'll use Apoorva Shah's technique and substitute the sum from part
  \textbf{a.}:\[
    \sum_{m, n \geq 0} \sum_{i \geq 0} f_i(m,n)t^i\frac{x^m y^n}{m!n!}
    = \sum_{m, n \geq 0} \left(
      \sum_{0\leq k \leq n} (-1)^k \binom nk ((1 + t)^{n - k} - 1)^m
    \right)\frac{x^m y^n}{m!n!}.
  \]
  First, expanding the binomial $((1 + t)^{n - k} - 1)^m$ yields \[
  \sum_{m, n \geq 0} \left(
    \sum_{0\leq k \leq n} (-1)^k \binom nk
    \left(
    \sum_{0\leq j\leq m} \binom{m}{j}(-1)^j((1 + t)^{n - k})^{m - j}
    \right)
  \right)\frac{x^m y^n}{m!n!}.
  \]
  Letting $n = p + k$, and $m = q + j$ and summing over $p$ and $q$ yields \[
  \sum_{p, q \geq 0} \left(
    \sum_{0\leq k} (-1)^k \binom{p + k}{k}
    \left(
    \sum_{0\leq j} \binom{q + j}{j}(-1)^j((1 + t)^{p + k - k})^{q + j - j}
    \right)
  \right)\frac{x^{q + j} y^{p + k}}{(q + j)!(p + k)!},
  \] which by simplifying, rearranging, and expanding the binomial coefficients gives \[
  \sum_{p, q, k, j \geq 0} \left(
    (-1)^k(-1)^j \frac{(p + k)!}{p!k!}
    \frac{(q + j)!}{q!j!}(1 + t)^{pq})
  \right)\frac{x^{q + j} y^{p + k}}{(q + j)!(p + k)!}.
  \] Since the indicies are independent, this can be split as \[
    \left(
      \sum_k (-1)^k\frac{y^k}{k!}
    \right)
    \left(
      \sum_j (-1)^j\frac{x^j}{j!}
    \right)
    \left(
      \sum_{p, q \geq 0} (1 + t)^{pq}\frac{x^qy^p}{q!p!}
    \right).
  \] Since the first two factors evaluate to $e^{-y}$ and $e^{-x}$ respectively,
  by renaming the indices $p$ and $q$ to $i$ and $j$ gives the desired identity.
\end{proof}
\end{document}
