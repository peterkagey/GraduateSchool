\documentclass{article}

\usepackage[margin=1in]{geometry}
\usepackage{amsmath,amsthm,amssymb}
\usepackage{bbm,enumerate,mathtools}
\usepackage[hidelinks]{hyperref}
\usepackage{tikz}
\usetikzlibrary{matrix, arrows}

\newenvironment{problem}[2][Problem]{\begin{trivlist}
\item[\hskip \labelsep {\bfseries #1}\hskip \labelsep {\bfseries #2.}]}{\end{trivlist}}
\newenvironment{solution}[1][Solution.]{\begin{trivlist}
\item[\hskip \labelsep {\bfseries #1}]}{\end{trivlist}}
\newenvironment{problempart}[1]{\begin{trivlist}\item[\textbf{Part #1.}]}{\end{trivlist}}
\newcommand{\set}[1]{\{ #1 \}}
\newcommand{\ang}[1]{\langle #1 \rangle}
\begin{document}

\title{Combinatorics: Homework 10}
\author{Peter Kagey}

\maketitle

% -----------------------------------------------------
% First problem
% -----------------------------------------------------
\begin{problem}{70} $ $
  \begin{enumerate}[a.]
    \item $[2-]$ Let $E_n$ denote the poset of all subsets of $[n]$ whose
    elements have even sum ordered by inclusion. Find $\#E_n$.
    \item $[2+]$ Compute $\mu(S, T)$ for all $S \leq T$ in $E_n$ as a recursion \[
      a_m = -\sum_{k<m}\binom{2m}{2k}a_k.
    \]
  \end{enumerate}
\end{problem}

\begin{solution} $ $
  \begin{enumerate}[a.]
    \item Consider all $2^{n-1}$ subsets of $\set{2,3,\hdots,n}$.
    There are $a$ sets with an even sum, and $b$ with an odd sum, where
    $a + b = 2^{n-1}$. This means that if we add $1$ to each set, there are
    $b$ new sets with an even sum and $a$ odd sets with an odd sum.

    Therefore there are
    $a$ subsets of $[n]$ with an even sum which do not contain $1$, and
    $b$ subsets of $[n]$ with an even sum which contain $1$, totaling
    $a+b = 2^{n-1}$ subsets of $[n]$ with an even sum.
    \item Using the identity \[
      \mu(S, T) = \begin{cases}
        1 & S = T \\
        \displaystyle\sum_{S \leq U < T} -\mu(S,U) & S < T
      \end{cases}
    \]
  \end{enumerate}
\end{solution}
\pagebreak
% -----------------------------------------------------
% Second problem
% -----------------------------------------------------
\begin{problem}{89} $[2]$ \\
  For a finite lattice $L$, let $f_L(m)$ be the number of $m$-tuples
  $(t_1, \hdots, t_m) \in L^m$ such that
  $t_1 \wedge t_2 \wedge \hdots \wedge t_m = \hat 0$. Prove via M\"obius
  inversion that \[
    f_L(m) = \sum_{t \in L} \mu(\hat 0, t)(\#V_t)^m
  \] where $V_t = \set{s \in L : s \geq t}$.
\end{problem}

\begin{solution} \text{} \\
  Generalize $f_L(m)$ by defining \[
    g^m_L(s) = \#\set{t_1, \hdots, t_m : t_1 \wedge t_2 \wedge \hdots \wedge t_m = s}.
  \] In particular $g^m_L(\hat 0) = f_L(m)$.
  \\~\\
  Notice that $t_1 \wedge t_2 \wedge \hdots \wedge t_m \in V_t$ if and only if
  $t_1, \hdots, t_m \in V_t$. Therefore if the number of tuples
  $(t_1, \hdots, t_m)$ with entries in $V_s$ is exactly the number of tuples
  such that $t_1 \wedge t_2 \wedge \hdots \wedge t_m \in V_t$, that is:
  \[
    (\#V_s)^m = \sum_{t \geq s} g_L^m(t).
  \]
  Therefore by the dual form of the M\"obius inversion formula, \[
    g_L^m(s) = \sum_{t \geq s} \mu(\hat 0, t)(\#V_t)^m,
  \] so in particular when $s = \hat 0$, \[
    f_L(m) = g_L^m(\hat 0) = \sum_{t \geq 0} \mu(\hat 0, t)(\#V_t)^m
    = \sum_{t \in L} \mu(\hat 0, t)(\#V_t)^m.
  \]
  % Want to show that
\end{solution}
\end{document}
