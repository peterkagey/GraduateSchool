\documentclass{article}

\usepackage[margin=1in]{geometry}
\usepackage{amsmath,amsthm,amssymb}
\usepackage{bbm,enumerate,mathtools}
\usepackage[hidelinks]{hyperref}
\usepackage{tikz}
\usetikzlibrary{matrix, arrows}

\newenvironment{problem}[2][Problem]{\begin{trivlist}
\item[\hskip \labelsep {\bfseries #1}\hskip \labelsep {\bfseries #2.}]}{\end{trivlist}}
\newenvironment{solution}[1][Solution.]{\begin{trivlist}
\item[\hskip \labelsep {\bfseries #1}]}{\end{trivlist}}
\newenvironment{problempart}[1]{\begin{trivlist}\item[\textbf{Part #1.}]}{\end{trivlist}}

\begin{document}

\title{Combinatorics: Homework 1}
\author{Peter Kagey}

\maketitle

% -----------------------------------------------------
% First problem
% -----------------------------------------------------
\begin{problem}{5} \text{} \\
  Show that \[
    \sum_{n_1,\hdots,n_k\geq0} \min(n_1,\hdots,n_k)x_1^{n_1}\cdots x_k^{n_k}
    = \frac{x_1\cdots x_k}{(1 - x_1)\cdots(1-x_k)(1 - x_1 \cdots x_k)}
  \]
\end{problem}

\begin{solution} \text{} \\
  First, we can rewrite the summand as \[
  \min(n_1,\hdots,n_k)x_1^{n_1}\cdots x_k^{n_k}
  = \sum_{j = 1}^{\min(n_1,\hdots,n_k)} (x_1^{n_1-j}\cdots x_k^{n_k-j})(x_1 \cdots x_k)^j.
  \] And we can rewrite the original sum as \begin{align*}
    \sum_{n_1,\hdots,n_k\geq0} \min(n_1,\hdots,n_k)x_1^{n_1}\cdots x_k^{n_k}
    &= \sum_{M=0}^\infty\sum_{\min(n_i)=M}\sum_{j = 1}^{\min(n_i)} (x_1^{n_1-j}\cdots x_k^{n_k-j})(x_1 \cdots x_k)^j.
  \end{align*}
  Now renaming $m_i := n_i - j$ gives \begin{align*}
    \sum_{\substack{\min(m_i) \geq 0 \\ j \geq 1}} (x_1^{m_1}\cdots x_k^{m_k})(x_1\cdots x_k)^j
    &=
    \left(\sum_{\min(m_i) \geq 0}x_1^{m_1}\cdots x_k^{m_k}\right)
    \left(\sum_{j \geq 1}(x_1\cdots x_k)^j\right) \\
    &= \left(
      \left(\sum_{m_1 \geq 0}x_1^{m_1}\right)
      \cdots
      \left(\sum_{m_k \geq 0}x_k^{m_k}\right)
    \right)
    \left(\sum_{j \geq 1}(x_1\cdots x_k)^j\right) \\
    &= \left(
      \left(\frac{1}{1 - x_1}\right)
      \cdots
      \left(\frac{1}{1 - x_k}\right)
    \right)
    \left(\frac{x_1\cdots x_k}{1 - x_1\cdots x_k}\right)
  \end{align*}
  as desired.
\end{solution}
\pagebreak
% -----------------------------------------------------
% Second problem
% -----------------------------------------------------
\begin{problem}{9 (a)} \text{} \\
  Let $f(m,n)$ be the number of paths from $(0,0)$ to
  $(m,n) \in \mathbb N \times \mathbb N$, where each step is of the form
  $(1, 0)$, $(0, 1)$, or $(1, 1)$.
  Show that \[
    \sum_{m \geq 0}\sum_{n \geq 0} f(m,n)x^m y^n = (1 - x - y - xy)^{-1}
  \]
\end{problem}

\begin{solution} \text{} \\
  The stupid way to do this is to enumerate all walks by number of steps.
  Each step can be a step in the $(1, 0)$ direction (call this $x$), a step
  in the $(0, 1)$ direction (call this $y$), or a step in the $(1, 1)$
  direction (call this $xy$).
  \\~\\
  Then the number of $k$-step walks is given by $(x + y + xy)^k$. Summing
  over any number of steps yields \[
    \sum_{k=0}^\infty (x + y + xy)^k = \frac{1}{1 - (x + y + xy)}.
  \]
\end{solution}
\pagebreak
% -----------------------------------------------------
% Third problem
% -----------------------------------------------------
\begin{problem}{12} \text{} \\
  Choose $n$ points on the circumference of a circle in ``general position''.
  Draw all $\binom n 2$ chords connecting two of the points. How many regions
  does this divide the interior of the circle into?
\end{problem}

\begin{solution} \text{} \\
  I'll use Grant Sanderson's very elegant, but perhaps-too-clever solution
  (which can be found in his 3Blue1Brown video), in which he uses Euler's
  formula, $F - E + V = 2$, to count the number of faces.
  \\~\\
  First, the number of vertices can be partitioned into the number of vertices
  on the boundary of the circle, plus the number of vertices strictly in the
  interior. By hypothesis, there are $n$ points on the boundary. To count the
  interior vertices, notice that every interior vertex is defined by the two
  chords that intersect to form it, and given any four distinct points on the
  boundary, there is exactly one way to choose two chords that intersect.
  Thus there are $n + \binom n 4$ vertices.
  \\~\\
  Next there are $\binom n 2$ chords, but each vertex takes in two pairs of
  intersecting lines and splits them into four lines. Thus, subtracting the
  interior vertices from the chords yields $\binom n 2 + 2\binom n 4$ edges.
  Plus, the arcs around the circle form an additional $n$ edges.
  \\~\\
  Solving for $F$ yields \[
    F = 2 + E - V
    = 2 + \underbrace{\binom n 2 + 2\binom n 4 + n}_E \underbrace{- n - \binom n 4}_{-V}
    = 2 + \binom n 2 + \binom n 4
  \]
  But this also counts the region \textit{outside} of the circle, so the
  interior of the circle has one fewer face, just \[
    1 + \binom n 2 + \binom n 4.
  \]
\end{solution}
\pagebreak
% -----------------------------------------------------
% Fourth problem
% -----------------------------------------------------
\begin{problem}{26} \text{} \\
  Let $\bar{c}(m,n)$ denote the number of compositions of $n$ with largest part
  at most $m$. Show that \[
    \sum_{n \geq 0}\bar c(m,n)x^n = \frac{1-x}{1-2x+x^{m+1}}
  \]
\end{problem}

\begin{solution} \text{} \\
  The idea here is that we will label our $x$s. In particular, \[
    (x_1 + x_2^2 + \hdots + x_m^m)^k
  \] encodes all sums with $k$ summands and terms in $[m]$.
  Thus we can sum up all sums with any number of summands \[
    \sum_{k=0}^\infty (x_1 + x_2^2 + \hdots + x_m^m)^k =
    \frac{1}{1 - (x + x^2 + \hdots + x^m)}.
  \]
  By dropping our labels (i.e. setting $x_i = x$ for all $x_i$) and multiplying
  the numerator and denominator by $1-x$, we get our result.
  In particular, when you multiply the denominator, there is cancellation in
  all but three terms
  \begin{alignat*}{2}
    (1 - x^1 - x^2 - \hdots - x^m)(1 - x) &&= 1 - &x^1 - x^2 - \hdots - x^m \\
                                          &&    - &x^1 + x^2 + \hdots + x^m + x^{m+1} \\
                                          &&= 1 - &2x + x^{m+1}.
  \end{alignat*}
  Thus \[
    \sum_{n \geq 0}\bar c(m,n)x^n = \frac{1-x}{1-2x+x^{m+1}}.
  \]
\end{solution}

\end{document}
