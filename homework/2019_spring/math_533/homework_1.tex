\documentclass{article}

\usepackage[margin=1in]{geometry}
\usepackage{amsmath,amsthm,amssymb}
\usepackage{bbm,enumerate,mathtools}
\usepackage[hidelinks]{hyperref}
\usepackage{tikz}
\usetikzlibrary{matrix, arrows}

\newenvironment{problem}[2][Problem]{\begin{trivlist}
\item[\hskip \labelsep {\bfseries #1}\hskip \labelsep {\bfseries #2.}]}{\end{trivlist}}
\newenvironment{solution}[1][Solution.]{\begin{trivlist}
\item[\hskip \labelsep {\bfseries #1}]}{\end{trivlist}}
\newenvironment{problempart}[1]{\begin{trivlist}\item[\textbf{Part #1.}]}{\end{trivlist}}

\begin{document}

\title{Algebraic Combinatorics: Homework 1}
\author{Peter Kagey}

\maketitle

% -----------------------------------------------------
% First problem
% -----------------------------------------------------
\begin{problem}{1}
  For $n \geq 0$, define a sequence $f(n)$ by the recurrence \[
    f(n) = f(n-1) + f(n-2)
  \] for $n \geq 2$ with initial conditions $f(0) = f(1) = 1$.
  \begin{enumerate}[(a)]
    \item Define $F(x) = \sum_{n\geq0} f(n) x^n$ to be the ordinary generating
    function of $f(n)$. Use the recurrence relation for $f(n)$ to prove
    $F(x) = 1 + xF(x) + x^2F(x)$
    \item Solve (a) for $F(x)$ and use partial fraction decomposition to write
    $F(x)$ as \[
      F(x) = \frac{A}{1 - \alpha x} + \frac B{1 - \beta x}.
    \]
    \item Extract the coefficients of $F(x)$ from $b$ to give an explicit
    formula for $f(n)$.
  \end{enumerate}
\end{problem}

\begin{solution} $ $
  \begin{enumerate}[(a)]
    \item \begin{align*}
      F(x) &= \sum_{n\geq0} f(n) x^n \\
      &= 1 + x + \sum_{n\geq2} f(n) x^n \\
      &= 1 + x + \sum_{n\geq2} f(n-1) + f(n-2) x^n \\
      &= 1 + x + \sum_{n\geq2} f(n-1)x^n + \sum_{n\geq2} f(n-2) x^n \\
      &= 1 + x + \sum_{n\geq1} f(n)x^{n+1} + \sum_{n\geq0} f(n) x^{n+2} \\
      &= 1 + \underbrace{x + x\sum_{n\geq1} f(n)x^n}_{xF(x)} + x^2\underbrace{\sum_{n\geq0} f(n) x^n}_{F(x)} \\
      &= 1 + xF(x) + x^2F(x)
    \end{align*}
    \item Using the identity $F(x) = 1 + xF(x) + x^2F(x)$ and solving for $F(x)$
    yields $F(x) = 1/(1 - x - x^2)$. Then using the quadratic formula to
    find the roots of $1 - x - x^2$ (equivalently $x^2 + x - 1 = 0$) gives \[
      % \varphi =
      \underbrace{\frac{-1 - \sqrt{5}}{2}}_{-\varphi}
      \hspace{1cm}\text{}\hspace{1cm}
      % \overline\varphi =
      \underbrace{\frac{-1 + \sqrt{5}}{2}}_{-\overline\varphi}.
    \] where $-(x + \varphi)(x + \overline\varphi) = 1 - x - x^2$.
    Then partial fraction decomposition on \[
      \frac{1}{1 - x - x^2}
      = \frac{A}{x + \varphi} + \frac{B}{x + \overline\varphi}
    \] where $A$ and $B$ satisfy the system of equations \begin{align*}
        A + B &= 0 \\
        \overline\varphi A + \varphi B &= 1
    \end{align*} which gives $A = -B = 1/\sqrt 5$ resulting in \begin{align*}
      F(x)
      &= \frac{1}{1 - x - x^2} \\
      &= \frac{1}{(x + \varphi)\sqrt{5}} - \frac{1}{(x+\overline\varphi)\sqrt{5}} \\
      &= \frac{1}{\varphi\sqrt{5}}\cdot\frac{1}{(x/\varphi + 1)}
      - \frac{1}{\overline\varphi\sqrt{5}}\cdot\frac{1}{(x/\overline\varphi + 1)}
    \end{align*}
    \item We can write \begin{align*}
      F(x) &= \frac{1}{\varphi\sqrt{5}}\sum_{i\geq0}(x/\varphi)^n
      - \frac{1}{\overline\varphi\sqrt{5}}\sum_{i\geq0}(x/\overline\varphi)^n \\
      &= \frac{1}{\sqrt{5}}\left(
        \sum_{i\geq0}\frac{x^n}{\varphi^{n+1}}
        - \sum_{i\geq0}\frac{x^n}{\overline\varphi^{n+1}}
      \right) \\
      &= \frac{1}{\sqrt{5}}
        \sum_{i\geq0}\left(\frac{1}{\varphi^{n+1}} - \frac{1}{\overline\varphi^{n+1}}
      \right)x^n \\
      &= \frac{1}{\sqrt{5}}
        \sum_{i\geq0}\left(\frac{\overline\varphi^{n+1} - \varphi^{n+1}}{(\overline\varphi\varphi)^{n+1}}
      \right)x^n \\
      &= \frac{1}{\sqrt{5}}
        \sum_{i\geq0}\left(\varphi^{n+1} - \overline\varphi^{n+1}
      \right)x^n.
    \end{align*} Therefore the coefficients are given by \begin{align*}
      f(n)
      = \frac{1}{\sqrt{5}}(\varphi^{n+1} - \overline\varphi^{n+1})
      = \frac{1}{\sqrt{5}}\left(\frac{1 + \sqrt 5}{2}\right)^{n+1}
      - \frac{1}{\sqrt{5}}\left(\frac{1 - \sqrt 5}{2}\right)^{n+1}.
    \end{align*}
  \end{enumerate}
\end{solution}
\pagebreak
\begin{problem}{2} Prove the product formula for exponential generating
  functions: \[
    \left(\sum_{n\geq0} f_n\frac{x^n}{n!}\right)
    \left(\sum_{n\geq0} g_n\frac{x^n}{n!}\right)
    = \sum_{n\geq0} \left(
      \sum_{k=0}^n \binom nk f_kg_{n-k}
    \right)\frac{x^n}{n!}
  \]
\end{problem}
\begin{solution} \text{} \\
  Using the definition of multiplication of (ordinary) generating functions
  gives \begin{align*}
    \left(\sum_{n\geq0} f_n\frac{x^n}{n!}\right)
    \left(\sum_{n\geq0} g_n\frac{x^n}{n!}\right)
    &= \sum_{n\geq0} \left(
      \sum_{k=0}^n \frac{f_k}{k!}\frac{g_{n-k}}{(n-k)!}
    \right)\!x^n \\
    &= \sum_{n\geq0} \left(
      \sum_{k=0}^n \frac{f_k}{k!}\frac{g_{n-k}}{(n-k)!}\frac{n!}{n!}
    \right)\!x^n \\
    &= \sum_{n\geq0} \left(
      \sum_{k=0}^n \binom nk f_k g_{n-k}\frac{1}{n!}
    \right)\!x^n \\
    &= \sum_{n\geq0} \left(
      \sum_{k=0}^n \binom nk f_kg_{n-k}
    \right)\frac{x^n}{n!}
  \end{align*} as desired.
\end{solution}
\pagebreak
%%%%%%%%%%%%
%
% Problem 3
%
%%%%%%%%%%%%
\begin{problem}{3}
  Assume that $\displaystyle 1 + \sum_{n\geq1}f(n)x^n = \exp\sum_{n\geq1}h(n)\frac{x^n}{n!}$,
  and show that the following are equivalent for $N \geq 1$ fixed.
  \begin{enumerate}
    \item $f(n) \in \mathbb Z$ for all $n \in [N]$.
    \item $h(n) \in \mathbb Z$ and
      $\sum_{d|n} h(d)\mu(n/d) \cong 0\ (\mathrm{mod}\ n)$ for all $n \in [N]$.
    \item $h(n) \in \mathbb Z$ and $h(n) \cong  h(n/p)\ (\mathrm{mod}\ p^r)$
      whenever $n \in [N]$ and $p$ is a prime that divides $n$ at least once
      and at most $r$ times.
    \item there exists a polynomial
      $P(t) = \prod_{i=1}^N (t-\alpha_i) \in \mathbb Z[t]$ with complex roots
      such that $h(n) = \sum_{i=1}^N \alpha_i^n$ for all $n \in [N]$.
  \end{enumerate}
\end{problem}

\begin{solution} \text{} \\
  First note that $ $
\end{solution}
\pagebreak
%%%%%%%%%%%%
%
% Problem 4
%
%%%%%%%%%%%%
\begin{problem}{4}
  Let $f(n) = 1 \cdot 3 \cdot 5 \cdots (2n-1)$ and $g(n) = 2^nn!$.
  \begin{enumerate}[(a)]
    \item Show that $G(x) = F(x)^2$.
    \item Give a combinatorial proof.
  \end{enumerate}
\end{problem}

\begin{solution} \text{} \\
  \begin{enumerate}[(a)]
    \item
      Firstly, the generating function $G$ is given by \begin{align*}
        G(x)
        = \sum_{n\geq0}g(n)\frac{x^n}{n!}
        = \sum_{n\geq0} 2^nn!\frac{x^n}{n!}
        = \sum_{n\geq0} (2x)^n
        = \frac{1}{1-2x}.
      \end{align*}
      Next the generating function $F$ is given by \begin{align*}
        F(x) &= \sum_{n\geq0} f(n)\frac{x^n}{n!} \\
        &= \sum_{n\geq0} 1 \cdot 3 \cdot 5 \cdots (2n-1)\frac{x^n}{n!} \\
        &= \sum_{n\geq0}
          \left(\frac1{-2}\right)
          \left(\frac{3}{-2}\right)
          \left(\frac{5}{-2}\right)
          \cdots
          \left(\frac{2n-1}{-2}\right)
          (-2)^n\frac{x^n}{n!} \\
        &= \sum_{n\geq0}\frac{(-\frac12)(-\frac12-1)(-\frac12-2)\hdots(-\frac12 - n + 1)}{n!}(-2)^nx^n \\
        &= \sum_{n\geq0}\binom{-1/2}{n}(-2x)^n \\
        &= (1 + (-2x))^{-1/2} \\
        &= \frac{1}{\sqrt{1 - 2x}}
      \end{align*}
      So it is clear that $F(x)^2 = G(x)$.
    \item
      Using the product formula for exponential generating functions gives \[
        F(x) \cdot F(x) = \sum_{n \geq 0}\left(\sum_{k=0}^n\binom{n}{k}f(k)f(n-k)\right)\frac{x^n}{n!},
      \] so it is sufficient to show that \[
        g(n) = 2^nn! = \sum_{k=0}^n\binom{n}{k}f(k)f(n-k).
      \]
      $g(n)$ gives the number of two-colorings of all permutations of $[n]$.
  \end{enumerate}
\end{solution}
\end{document}
