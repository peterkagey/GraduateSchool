\documentclass{article}

\usepackage[margin=1in]{geometry}
\usepackage{amsmath,amsthm,amssymb}
\usepackage{bbm,enumerate,mathtools,mathrsfs}
\usepackage[hidelinks]{hyperref}
\usepackage{tikz}
\usetikzlibrary{matrix, arrows, patterns}

\newenvironment{problem}[2][Problem]{\begin{trivlist}
\item[\hskip \labelsep {\bfseries #1}\hskip \labelsep {\bfseries #2.}]}{\end{trivlist}}
\newenvironment{solution}[1][Solution.]{\begin{trivlist}
\item[\hskip \labelsep {\bfseries #1}]}{\end{trivlist}}
\newenvironment{problempart}[1]{\begin{trivlist}\item[\textbf{Part #1.}]}{\end{trivlist}}

\newenvironment{definition}[1][Definition.]{
  \begin{trivlist} \item[\hskip \labelsep {\bfseries #1}]
}{\end{trivlist}}

\newenvironment{example}[1][Example.]{
  \begin{trivlist} \item[\hskip \labelsep {\bfseries #1}]
}{\end{trivlist}}

\newenvironment{note}[1][Note.]{
  \begin{trivlist} \item[\hskip \labelsep {\bfseries #1}]
}{\end{trivlist}}

\newenvironment{theorem}[1][Theorem.]{
  \begin{trivlist} \item[\hskip \labelsep {\bfseries #1}]
}{\end{trivlist}}

\newenvironment{exercise}[1][Exercise.]{
  \begin{trivlist} \item[\hskip \labelsep {\bfseries #1}]
}{\end{trivlist}}

\newcommand{\set}[1]{\{ #1 \}}
\newcommand{\inv}[1]{#1^{-1}}
\newcommand{\ang}[1]{\langle #1 \rangle}
\newcommand{\paren}[1]{\left( #1 \right)}
\newcommand{\fn}[3]{#1 \colon #2 \rightarrow #3}
\newcommand{\wt}{\operatorname{wt}}
\newcommand{\id}{\operatorname{id}}

\begin{document}

\title{Math 533: Homework 7}
\author{Peter Kagey}
\date{April 24, 2019}

\maketitle

% -----------------------------------------------------
% First problem
% -----------------------------------------------------
\begin{problem}{1}
  Write the character table for $S_4$.
\end{problem}

\begin{proof}
  First we have cycle types corresponding to the partitions \begin{alignat*}{2}
    1 &\text{ permutation with cycle type }  && (1, 1, 1, 1) \\
    6 &\text{ permutations with cycle type } && (2, 1, 1) \\
    3 &\text{ permutations with cycle type } && (1, 1, 1, 1) \\
    8 &\text{ permutations with cycle type } && (3, 1) \\
    6 &\text{ permutations with cycle type } && (4).
  \end{alignat*}
  We begin the character table with the trivial and the signed representations.
  Next, since there is only one way to write $4! = 24$ as a sum of five squares,
  namely $1 + 1 + 2^2 + 3^2 + 3^2$, this is the first column.
  \[
  \begin{array}{ l|rrrrr }
                    & (1,1,1,1) & (2,1,1) & (2,2) & (3,1) & (4) \\
   \hline
   \chi^\text{triv} & 1 &  1 & 1 &  1 &  1 \\
   \chi^\text{sgn}  & 1 & -1 & 1 &  1 & -1 \\
   \chi'            & 2 & ? &  ? &  ? &  ? \\
   \chi''           & 3 & ? &  ? &  ? &  ? \\
   \chi'''          & 3 & ? &  ? &  ? &  ?
 \end{array}
  \]
  We know that the defining/standard representation has \begin{alignat*}{2}
      &\chi^\text{std}((1)(2)(3)(4)) &&= 4 \\
      &\chi^\text{std}((12)(3)(4)) &&= 2 \\
      &\chi^\text{std}((12)(34)) &&= 0 \\
      &\chi^\text{std}((123)(4)) &&= 1 \\
      &\chi^\text{std}((1234)) &&= 0 \\
  \end{alignat*}
  And by taking the inner products with the two known characters gives \begin{alignat*}{2}
    &\frac{1}{4!}\ang{\chi^\text{def}, \chi^{(1,1,1,1)}} &&= \frac{
      1 \cdot 4 +
      6 \cdot 2 +
      3 \cdot 0 +
      8 \cdot 1 +
      6 \cdot 0
    }{4!} = 1 \\
    &\frac{1}{4!}\ang{\chi^\text{def}, \chi^{(4)}} &&= \frac{
      1 \cdot 4
      - 6 \cdot 2
      + 3 \cdot 0
      + 8 \cdot 1
      - 6 \cdot 0
    }{4!} = 0
  \end{alignat*}
  Subtracting off the trivial representation gives
  \[
    \begin{array}{ l|rrrrr }
            & (1,1,1,1) & (2,1,1) & (2,2) & (3,1) & (4) \\ \hline
     \chi^3 & 3 & 1 & -1 &  0 & -1
   \end{array}
  \]
  which is indeed irreducible
  \[
    \ang{\chi^3,\chi^3} = \frac{
      1 \cdot 3^2 +
      6 \cdot 1^2 +
      3 \cdot (-1)^2 +
      8 \cdot 0 +
      6 \cdot (-1)^2
    }{4!} = 1.
  \]
  Then the tensor product of $\chi^\text{sgn}$ and $\chi^3$ gives $\chi^4$
  \[
    \begin{array}{ l|rrrrr }
            & (1,1,1,1) & (2,1,1) & (2,2) & (3,1) & (4) \\ \hline
     \chi^4 & 3 & -1 & -1 &  0 & 1
   \end{array}
  \]
  which is also irreducible
  \[
    \ang{\chi^4,\chi^4} = \frac{
      1 \cdot 3^2 +
      6 \cdot (-1)^2 +
      3 \cdot (-1)^2 +
      8 \cdot 0 +
      6 \cdot 1^2
    }{4!} = 1.
  \]
  This leaves just one more row which can be deduced from orthogonality.
  \begin{align}
    \ang{\chi^\text{triv}, \chi^5} &=
    1 \cdot 2 +
    6a +
    3b +
    8c +
    6d = 0 \\
    \ang{\chi^\text{sgn}, \chi^5} &=
    1 \cdot 2 -
    6a +
    3b +
    8c -
    6d = 0 \\
    \ang{\chi^3, \chi^5} &=
    1 \cdot 6 +
    6a -
    3b +
    8 \cdot 0 -
    6d = 0 \\
    \ang{\chi^4, \chi^5} &=
    1 \cdot 6 -
    6a -
    3b +
    8 \cdot 0 +
    6d = 0
  \end{align}
  Adding equations (1) and (2) gives $4 + 6b + 16c = 0$, and
  adding equations (3) and (4) gives $12 - 6b = 0$, so $b = 2$ and $c = -1$.

  This gives \[
    \begin{array}{ l|rrrrr }
            & (1,1,1,1) & (2,1,1) & (2,2) & (3,1) & (4) \\ \hline
     \chi^5 & 2 & ? & 2 &  -1 & ?
   \end{array}
  \] but \[
    \ang{\chi^4,\chi^4} = \frac{
      1 \cdot 2^2 +
      6a^2 +
      3 \cdot 2^2 +
      8 \cdot (-1)^2 +
      6d^2
    }{4!}
    = \frac{
      4 +
      6a^2 +
      12 +
      8 +
      6d^2
    }{4!}
    = 1
  \] so $a = d = 0$, and the full table is \[
    \begin{array}{ l|rrrrr }
                        & (1,1,1,1) & (2,1,1) & (2,2) & (3,1) & (4) \\
       \hline
       \chi^\text{triv} & 1 &  1 & 1 &  1 &  1 \\
       \chi^\text{sgn}  & 1 & -1 & 1 &  1 & -1 \\
       \chi^3           & 3 & 1 & -1 &  0 & -1 \\
       \chi^4           & 3 & -1 & -1 &  0 & 1 \\
       \chi^5           & 2 & 0 & 2 &  -1 & 0
     \end{array}.
   \]
\end{proof}
\end{document}
