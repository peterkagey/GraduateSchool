\documentclass{article}

\usepackage[margin=1in]{geometry}
\usepackage{amsmath,amsthm,amssymb}
\usepackage{bbm,enumerate,mathtools,mathrsfs}
\usepackage[hidelinks]{hyperref}
\usepackage{tikz}
\usetikzlibrary{matrix, arrows, patterns}

\newenvironment{problem}[2][Problem]{\begin{trivlist}
\item[\hskip \labelsep {\bfseries #1}\hskip \labelsep {\bfseries #2.}]}{\end{trivlist}}
\newenvironment{solution}[1][Solution.]{\begin{trivlist}
\item[\hskip \labelsep {\bfseries #1}]}{\end{trivlist}}
\newenvironment{problempart}[1]{\begin{trivlist}\item[\textbf{Part #1.}]}{\end{trivlist}}

\newenvironment{definition}[1][Definition.]{
  \begin{trivlist} \item[\hskip \labelsep {\bfseries #1}]
}{\end{trivlist}}

\newenvironment{example}[1][Example.]{
  \begin{trivlist} \item[\hskip \labelsep {\bfseries #1}]
}{\end{trivlist}}

\newenvironment{note}[1][Note.]{
  \begin{trivlist} \item[\hskip \labelsep {\bfseries #1}]
}{\end{trivlist}}

\newenvironment{theorem}[1][Theorem.]{
  \begin{trivlist} \item[\hskip \labelsep {\bfseries #1}]
}{\end{trivlist}}

\newenvironment{exercise}[1][Exercise.]{
  \begin{trivlist} \item[\hskip \labelsep {\bfseries #1}]
}{\end{trivlist}}

\newcommand{\set}[1]{\{ #1 \}}
\newcommand{\inv}[1]{#1^{-1}}
\newcommand{\ang}[1]{\langle #1 \rangle}
\newcommand{\paren}[1]{\left( #1 \right)}
\newcommand{\fn}[3]{#1 \colon #2 \rightarrow #3}
\newcommand{\wt}{\operatorname{wt}}
\newcommand{\id}{\operatorname{id}}

\begin{document}

\title{Math 533: Homework 7}
\author{Peter Kagey}
\date{April 24, 2019}

\maketitle

% -----------------------------------------------------
% First problem
% -----------------------------------------------------
\begin{problem}{1}
  Write the character table for $S_4$.
\end{problem}

\begin{proof}
  First, we know that the trivial and the signed representation have to be in
  there. Next, we know that there is only one way to write $4! = 24$ as a
  sum of five squares, namely $1 + 1 + 4 + 9 + 9$.
  \[
  \begin{array}{ l|rrrrr }
        & (1,1,1,1) & (2,1,1) & (2,2) & (3,1) & (4) \\
   \hline
   \chi^{(1,1,1,1)} & 1 &  1 & 1 &  1 &  1 \\
   \chi^{(2,1,1)}   & 3 & ? &  ? &  ? &  ? \\
   \chi^{(2,2)}     & 2 & ? &  ? &  ? &  ? \\
   \chi^{(3,1)}     & 3 & ? &  ? &  ? &  ? \\
   \chi^{(4)}       & 1 & -1 & 1 &  1 & -1
 \end{array}
  \]
  We know that the defining representation has \begin{alignat*}{2}
      &\chi^\text{def}((1)(2)(3)(4)) &&= 4 \\
      &\chi^\text{def}((12)(3)(4)) &&= 2 \\
      &\chi^\text{def}((12)(34)) &&= 0 \\
      &\chi^\text{def}((123)(4)) &&= 1 \\
      &\chi^\text{def}((1234)) &&= 0 \\
  \end{alignat*}
  And by taking the inner products with the two known characters gives \begin{alignat*}{2}
    &\frac{1}{4!}\ang{\chi^\text{def}, \chi^{(1,1,1,1)}} &&= \frac{
      1 \cdot 4 +
      6 \cdot 2 +
      3 \cdot 0 +
      8 \cdot 1 +
      6 \cdot 0
    }{4!} = 1 \\
    &\frac{1}{4!}\ang{\chi^\text{def}, \chi^{(4)}} &&= \frac{
      1 \cdot 4
      - 6 \cdot 2
      + 3 \cdot 0
      + 8 \cdot 1
      - 6 \cdot 0
    }{4!} = 0
  \end{alignat*}
\end{proof}
\pagebreak

% -----------------------------------------------------
% Second problem
% -----------------------------------------------------
\begin{problem}{3}
\end{problem}

\begin{proof} ~
  \begin{enumerate}[(a)]
    \item
  \end{enumerate}
\end{proof}

\end{document}
