\documentclass{article}

\usepackage[margin=1in]{geometry}
\usepackage{amsmath,amsthm,amssymb}
\usepackage{bbm,enumerate,mathtools,mathrsfs}
\usepackage[hidelinks]{hyperref}
\usepackage{tikz}
\usetikzlibrary{matrix, arrows}

\newenvironment{problem}[2][Problem]{\begin{trivlist}
\item[\hskip \labelsep {\bfseries #1}\hskip \labelsep {\bfseries #2.}]}{\end{trivlist}}
\newenvironment{solution}[1][Solution.]{\begin{trivlist}
\item[\hskip \labelsep {\bfseries #1}]}{\end{trivlist}}
\newenvironment{problempart}[1]{\begin{trivlist}\item[\textbf{Part #1.}]}{\end{trivlist}}

\newenvironment{definition}[1][Definition.]{
  \begin{trivlist} \item[\hskip \labelsep {\bfseries #1}]
}{\end{trivlist}}

\newenvironment{example}[1][Example.]{
  \begin{trivlist} \item[\hskip \labelsep {\bfseries #1}]
}{\end{trivlist}}

\newenvironment{note}[1][Note.]{
  \begin{trivlist} \item[\hskip \labelsep {\bfseries #1}]
}{\end{trivlist}}

\newenvironment{theorem}[1][Theorem.]{
  \begin{trivlist} \item[\hskip \labelsep {\bfseries #1}]
}{\end{trivlist}}

\newenvironment{exercise}[1][Exercise.]{
  \begin{trivlist} \item[\hskip \labelsep {\bfseries #1}]
}{\end{trivlist}}

\newcommand{\set}[1]{\{ #1 \}}
\newcommand{\ang}[1]{\langle #1 \rangle}
\newcommand{\paren}[1]{\left( #1 \right)}
\newcommand{\fn}[3]{#1 \colon #2 \rightarrow #3}

\begin{document}

\title{Math 533: Homework 3}
\author{Peter Kagey}
\date{Wednesday, February 13, 2019}

\maketitle

% -----------------------------------------------------
% First problem
% -----------------------------------------------------
\begin{problem}{1}
\end{problem}

\begin{proof} ~
  \begin{enumerate}
    \item ~\\\vspace{10cm}
    \item A binary tree can be decomposed as a root and the left side and right side (which are themselves trees)
    \begin{align*}
      B(x) &= \underbrace{1}_\text{root} + \underbrace{B(x)}_\text{left}\underbrace{B(x)}_\text{right} \\
      &= 1 + B(x)^2.
    \end{align*}
    \item Solving the quadratic $xB(x)^2 - B(x) + 1 = 0$ yields \[
      B(x) = \frac{1 \pm \sqrt{1 - 4x}}{2x}
    \] where we take minus, so that $\displaystyle\lim_{x\rightarrow0}B(x) = 1$.
  \item We can write \begin{align*}
    1 - \sqrt{1 - 4x}
      &= 1 - \sum_{n = 0}^\infty \binom{\frac 12}{n}(-4)^nx^n \\
      &= - \sum_{n=1}^\infty \frac{(\frac 12)(\frac 12 - 1) \hdots (\frac 12 - n + 1)}{n!}(-4)^nx^n \\
      &= \sum_{n=1}^\infty 1 \cdot 3 \cdots (2(n-1) - 1)\frac{(2x)^n}{n!} \\
      &= \sum_{n=1}^\infty \frac{(2n-2)!}{2^{n-1}(n-1)!}\frac{(2x)^n}{n!} \\
    \frac{1 - \sqrt{1 - 4x}}{2x}
    &= \sum_{n=1}^\infty \frac{(2n-2)!}{2^{n-1}(n-1)!}2^{n-1}\frac{x^{n-1}}{n!} \\
    &= \sum_{n=1}^\infty \frac{1}{n!}\binom{2n-2}{n-1}x^{n-1} \\
    &= \sum_{n=0}^\infty \frac{1}{(n+1)!}\binom{2n}{n}x^n.
  \end{align*}
  Thus \[
    b_n = \frac{1}{(n+1)!}\binom{2n}{n}.
  \]
  \end{enumerate}
\end{proof}
% -----------------------------------------------------
% Second problem
% -----------------------------------------------------
\begin{problem}{2}
\end{problem}

\begin{proof} ~
  \begin{enumerate}[(a)]
    \item It is sufficient to evaluate both sides at $G(x)$: \begin{align*}
      \underbrace{F(G(x))}_{x} &= G(x) \cdot E(\underbrace{F(G(x))}_x) \\
      x &= G(x) \cdot E(x) \\
      &= \frac{x}{E(x)}\cdot E(x)
    \end{align*}
    \item
  \end{enumerate}
\end{proof}
\pagebreak
% -----------------------------------------------------
% Third problem
% -----------------------------------------------------
\begin{problem}{3}
\end{problem}

\begin{proof}
  This falls nicely to a generating function approach. Let $r(n, j)$ be the
  number of rooted, ordered trees on $n$ vertices with $j$ leaves, and define
  the generating function \[
    T(x,y) = \sum_{n, j \geq 0} r(n, j) x^n y^j.
  \]
  Then $T(x,y)$ satisfies the recurrence \begin{align*}
    T(x, y) &= xy + xT(x,y) + xT(x,y)^2 + xT(x,y)^3 + \hdots \\
            &= xy + \frac{xT(x,y)}{1-T(x,y)} \\
    T(x, y) - T(x, y)^2 &=   xy(1-T(x,y)) + xT(x,y),
  \end{align*} and so it is sufficient to find the roots of the quadratic
  $T(x, y)^2 + (x - xy - 1)T(x, y) + xy$: \[
    T(x,y) = \frac{1 + xy - x \pm \sqrt{(x - xy - 1)^2 - 4xy}}{2}.
  \]
  Then taking the root which subtracts the radical gives \begin{align*}
    T(x,y)
    &= \frac 12\paren{1 + xy - x - \sqrt{1 - 2 x + x^2 - 2 x y - 2 x^2 y + x^2 y^2}} \\
    &= \frac 12\paren{
      xy - x - \sum_{n=1}^\infty\binom{\frac12}{n}(-2x + x^2 - 2 x y - 2 x^2 y + x^2 y^2)^n
    } \\
    &= \frac 12\paren{
      xy - x - \sum_{n=1}^\infty\binom{\frac12}{n}(-2 + x - 2y - 2 x y + x y^2)^n x^n
    }\\
    &= \frac 12\paren{
      xy - x - \sum_{n=1}^\infty\binom{\frac12}{n}\sum_{m=0}^n\binom{n}{m}(-2 - 2y)^m(- 2 x y + x y^2 + x)^{n-m} x^n
    } \\
    &= \frac 12\paren{
      xy - x - \sum_{n=1}^\infty\binom{\frac12}{n}\sum_{m=0}^n\binom{n}{m}(-2)^m(y + 1)^m(y - 1)^{2n-2m} x^{2n-m}
    }.
  \end{align*}
  Then letting $n = m + j$, \[
  \frac 12\paren{
    1 + xy - x - \sum_{m,j\geq0}^\infty\binom{\frac12}{m+j}\binom{m+j}{m}(-2)^m(y + 1)^m(y - 1)^{2j} x^{m + 2j}
  }
  \]
\end{proof}
% -----------------------------------------------------
% Fourth problem
% -----------------------------------------------------
\begin{problem}{4}
\end{problem}

\begin{proof} ~
\end{proof}
\pagebreak
% -----------------------------------------------------
% Fifth problem
% -----------------------------------------------------
\begin{problem}{5}
  Let $B_{n+1}$ be the $n + 1 \times n + 1$ matrix with entries \[
    b_{ij} = \begin{cases}
      2 & i = j \\
      -1 & |i - j| = 1 \\
      -1 & i = n + 1 \text{ and } j \in \set{1, n} \\
      -1 & j = n + 1 \text{ and } i \in \set{1, n} \\
      0 & \text{otherwise}
    \end{cases}.
  \] This is the Laplacian matrix for the cyclic graph $C_{n+1}$.
  Then by the matrix-tree theorem, deleting the $n+1$ row and column gives the
  number of spanning trees of the cyclic graph on $n+1$ vertices, $C_{n+1}$.
  The only way to get a spanning tree of $C_{n+1}$ is to delete a vertex,
  so the number of spanning trees is equal to the number of vertices. Thus
  $\det(A_n) = n + 1$.
\end{problem}

\begin{proof} ~
\end{proof}
\end{document}
