\documentclass{article}

\usepackage[margin=1in]{geometry}
\usepackage{amsmath,amsthm,amssymb}
\usepackage{bbm,enumerate,mathtools,mathrsfs}
\usepackage[hidelinks]{hyperref}
\usepackage{tikz}
\usetikzlibrary{matrix, arrows}

\newenvironment{problem}[2][Problem]{\begin{trivlist}
\item[\hskip \labelsep {\bfseries #1}\hskip \labelsep {\bfseries #2.}]}{\end{trivlist}}
\newenvironment{solution}[1][Solution.]{\begin{trivlist}
\item[\hskip \labelsep {\bfseries #1}]}{\end{trivlist}}
\newenvironment{problempart}[1]{\begin{trivlist}\item[\textbf{Part #1.}]}{\end{trivlist}}

\newenvironment{definition}[1][Definition.]{
  \begin{trivlist} \item[\hskip \labelsep {\bfseries #1}]
}{\end{trivlist}}

\newenvironment{example}[1][Example.]{
  \begin{trivlist} \item[\hskip \labelsep {\bfseries #1}]
}{\end{trivlist}}

\newenvironment{note}[1][Note.]{
  \begin{trivlist} \item[\hskip \labelsep {\bfseries #1}]
}{\end{trivlist}}

\newenvironment{theorem}[1][Theorem.]{
  \begin{trivlist} \item[\hskip \labelsep {\bfseries #1}]
}{\end{trivlist}}

\newenvironment{exercise}[1][Exercise.]{
  \begin{trivlist} \item[\hskip \labelsep {\bfseries #1}]
}{\end{trivlist}}

\newcommand{\set}[1]{\{ #1 \}}
\newcommand{\ang}[1]{\langle #1 \rangle}
\newcommand{\paren}[1]{\left( #1 \right)}
\newcommand{\fn}[3]{#1 \colon #2 \rightarrow #3}

\begin{document}

\title{Math 510b: Homework 1}
\author{Peter Kagey}

\maketitle

% -----------------------------------------------------
% First problem
% -----------------------------------------------------
\begin{problem}{5.3} ~
  \begin{enumerate}[(i)]
    \item Given an example of a commutative ring containing two prime ideals $P$ and $Q$
      for which $P \cap Q$ is not a prime ideal.
    \item If $P_1 \supseteq P_2 \supseteq \hdots \supseteq P_n \supseteq \hdots$
      is a decreasing sequence of prime ideals in a commutative ring $R$, prove
      that $\bigcap_{n \geq 1} P_n$ is a prime ideal.
  \end{enumerate}
\end{problem}

\begin{proof} ~
  \begin{enumerate}[(i)]
    \item Let $R = \mathbb Z$ and let $P = \paren 2$ and $Q = \paren 3$.
    Then $P \cap Q = \paren 6$, which is not a prime ideal because
    $2 \cdot 3 \in \paren 6$, but $2,3 \not\in \paren 6$.

    \item Let $P = \bigcap_{n \geq 1} P_n$, and note that for all
    $n \in \mathbb N, P \subseteq P_n$. This implies that $P$ is a proper ideal,
    because $P \subseteq P_1 \subsetneq R$.
    Next let $ab \in P$. Since $P \subseteq P_n$ for all $n$, this means that
    for all $n \in \mathbb N$, $ab \in P_n$ and $a \in P_n$ or $b \in P_n$. Thus
    since $a \in P_n$ for all $n$ or $b \in P_n$ for all $n$, either $a$ or $b$
    is in the intersection $P$.
  \end{enumerate}
\end{proof}
\vspace{1cm}
% -----------------------------------------------------
% Second problem
% -----------------------------------------------------
\begin{problem}{5.6}
  Prove that the ideal $I = \paren{x^2 - 2, y^2 + 1, z} \subseteq \mathbb Q[x,y,z]$ is a
  proper ideal.
\end{problem}

\begin{proof} It is sufficient to show that
  $1 \not\in I = \set{(x^2-2)q_1 + (y^2 + 1)q_2 + z q_3 : q_1, q_2, q_3 \in \mathbb Q}$.
  If $f(x, y, z) = (x^2-2)q_1 + (y^2 + 1)q_2 + z q_3$ with $q_1, q_2,$ or $q_3$
  not equal to zero, then $\deg(f) \geq 1$ because $\mathbb Q$ is a field of
  characteristic zero. If $q_1 = q_2 = q_3 = 0$, then $f(x,y,z) = 0$. Thus there
  are no elements in $f(x) \in I$ with $\deg(f) = \deg(1) = 0$, and so $I$ is a
  proper ideal.
\end{proof}
\pagebreak
% -----------------------------------------------------
% Third problem
% -----------------------------------------------------
\begin{problem}{5.13}
  A commutative ring $R$ is a local ring if it has a unique maximal ideal.
  \begin{enumerate}[(i)]
    \item If $p$ is a prime, prove that the ring of $p$-adic fractions \[
      \mathbb Z_p = \set{a/b \in \mathbb Q : p \nmid b},
    \] is a local ring.
    \item If $k$ is a field, prove that the ring $k[[x]]$ of all power series
      is a local ring.
    \item If $R$ is a local ring with unique maximal ideal $\mathfrak m$ prove
    that $a \in R$ is a unit if and only if $a \not\in \mathfrak m$.
  \end{enumerate}
\end{problem}

\begin{proof} ~
  \begin{enumerate}[(i)]
    \item Let $M = \paren p$. Then $\mathbb Z_p/\paren p$ is a field, and so $M$
    is maximal:
    \\
    Let $\overline{a/b} \neq \overline 0$ be the equivalence class
    of $a/b$ in $\mathbb Z_p/\paren p$. Then $p \nmid a$ and $p \nmid b$.
    Thus $(\overline{a/b})^{-1} = \overline{b/a}$, so $\mathbb Z_p/\paren p$ is
    a field.
    \\~\\
    This $M$ is unique because
    \item Let $M = \paren x$. Then $k[[x]] / \paren x$ with quotient map which
    sends \[
      \sum_{n=0}^\infty a_n x^n \mapsto a_0.
    \] is clearly isomorphic to $k$, a field.
    \item By the hint, assume that every non-unit in a commutative ring lies in
    some maximal ideal. \\
    $(\Longrightarrow)$ Assume that $a \in R$ is a unit.
    \\
    $(\Longleftarrow)$. Assume that $a \not\in \mathfrak m$.
  \end{enumerate}
\end{proof}
\pagebreak
% -----------------------------------------------------
% Fourth problem
% -----------------------------------------------------
\begin{problem}{5.17}
  Prove that a UFD $R$ is a PID if and only if every nonzero prime ideal is a
  maximal ideal.
\end{problem}

\begin{proof} ~ \\
  $(\Longrightarrow)$
  Assume that $R$ is a PID, let $\ang p \subset R$ be a
  nonzero prime ideal, and let $\ang m$ be another ideal such that
  $\ang p \subseteq \ang m \subsetneq R$. Thus $m \mid p$, but since $p$ is
  prime. Note that $m$ is not a unit, since $\ang m \neq R$, so since $p$ is
  prime (and thus irreducible), $p = um$ with $u$ a unit, so $\ang p = \ang m$,
  and thus $\ang p$ is maximal.
%
  \\~\\
  $(\Longleftarrow)$.
  % Assume that every nonzero prime ideal in $R$ is a maximal
  % ideal, and let $I \subset R$ be an ideal.
  % We want to show that every ideal can be written as $\ang x$ for some $x \in R$
  Let $\mathcal S = \set{J \triangleleft R : I \subseteq J \text{ is not principal}}$.
  It is enough to show that $\mathcal S$ is empty.\\
  Assume that $\set {S_n}$ is a chain of proper ideals in $\mathcal S$ such that
  $S_i \subseteq S_{i+1}$.
  Now the union $S = \bigcup_n S_n$ cannot be principal
  because if $S = (r)$, then there exists some $i$ such that $r \in S_i$ and
  thus $S_i = (r)$. A contradiction because $S_i$ is not principal due to its
  inclusion in $\mathcal S$.
  \\~\\
  Therefore $S$ is maximal and non-principal, so by Zorn's Lemma, $\mathcal S$
  has a maximal element, $M$. Note that $M$ is not a prime ideal of $R$ because
  all prime ideals are principal. Thus there exists some $ab \in M$ such that
  $a, b \not\in M$, so $M \subsetneq M + (a) \subsetneq R$ and
  $M \subsetneq M + (b) \subsetneq R$, so these must be principal ideals.
  However, if they are, then $M = (M + (a))(M + (b)) = (a)(b) = (ab)$, a
  contradiction to the claim that $M$ is not principal.
  \\~\\
  Thus $\mathcal S$ is empty, so all ideals are principal, meaning $R$ is a PID.
\end{proof}
\pagebreak
% -----------------------------------------------------
% Fifth problem
% -----------------------------------------------------
\begin{problem}{5.23}
  Prove that $f(x, y) = xy^3 + x^2y^2 - x^5y + x^2 + 1$ is an irreducible
  polynomial in $\mathbb R[x, y]$.
\end{problem}

\begin{proof} ~
  Consider $f$ as a polynomial in $y$ over $\mathbb R[x]$. Then $x^2 + 1$ is
  irreducible in $\mathbb R[x]$ \begin{align*}
    f(x, y) &= xy^3 + x^2y^2 - x^5y + (x^2 + 1) \\
    &= y(xy^2 + x^2y - x^5) + (x^2 + 1)
  \end{align*}
\end{proof}
\pagebreak
% -----------------------------------------------------
% Sixth problem
% -----------------------------------------------------
\begin{problem}{5.24}
  Let \[
    D = \det\!\paren{\begin{bmatrix}x & y \\ z & w\end{bmatrix}} = xw - yz \in \mathbb Z[x,y,z,w].
  \]
  \begin{enumerate}
    \item Prove that $(D)$ is a prime ideal in $\mathbb Z[x,y,z,w]$.
    \item Prove that $\mathbb Z[x,y,z,w]/(D)$ is not a UFD.
  \end{enumerate}
\end{problem}

\begin{proof} ~
  \begin{enumerate}
    \item Since $\mathbb Z[x,y,z,w]$ is a UFD (by induction with base case
    $\mathbb Z$) it is enough to show that $D$ is an irreducible element.
    Since $z$ is prime view $D$ as a polynomial in $w$ over $\mathbb[x,y,z]$.
    Then $z \mid -yz$, $z^2 \nmid -yz$, and $z \nmid xw$, so by Eisenstein's
    criteria, $D$ is irreducible. Therefore $D$ is prime, and thus $(D)$ is a
    prime ideal.
    % TODO
    \item Notice that in this ring $\bar x \bar w = \bar y \bar z$, and
    $\bar x, \bar y, \bar z, \bar w$ are prime, and are all distinct up to unit.
  \end{enumerate}
\end{proof}
\pagebreak
% -----------------------------------------------------
% Seventh problem
% -----------------------------------------------------
\begin{problem}{5.40}
  Prove that every non-unit in a commutative ring lies in some maximal ideal.
\end{problem}

\begin{proof}
  On the first day of class we saw the corollary of Zorn's lemma which states
  \begin{quote}
    If $1 \in R$ and $I \neq R$ is any proper ideal of $R$
    (left, right, or two-sided), then there exists a maximal ideal $M$ such that
    $I \subseteq M \subset R$.
  \end{quote}
  We know that if $a \in R$ is a non-unit, then $\ang a$ is a proper ideal of
  $R$, and in particular, $1 \not\in \ang a$ and thus $\ang a$ fulfills the
  hypotheses of the corollary.
\end{proof}
\pagebreak
% -----------------------------------------------------
% Eighth problem
% -----------------------------------------------------
\begin{problem}{8}
  Let $R$ be the ring of integers in $F = \mathbb Q[\sqrt m]$.
  Show that \begin{enumerate}
    \item if $m \cong 2, 3 \bmod 4$ then $R = \mathbb Z[\sqrt m]$, and
    \item if $m \cong 1 \bmod 4$ then $R = \mathbb Z[a]$, where
      $a = (1 + \sqrt m)/2$.
  \end{enumerate}
\end{problem}

\begin{proof}
  \begin{enumerate}
    \item The elements of $\mathbb Q[\sqrt m]$ look like $a + b\sqrt m$ with
      $a, b \in \mathbb Q$, and so have monic (and hence minimal) polynomial \[
        (x - a)^2 - b^2m = x^2 - 2ax + (a^2 - b^2m)
      \] which by construction this has $a + b\sqrt m$ as a root.
      In order for $\alpha = a + b\sqrt m$ to be in $R$, $-2a$
      and $a^2 - b^2m$ must be integers. (So surely $\mathbb Z[\sqrt m] \in R$)
      Thus $a = c/2$.
      \\~\\
      If $c$ is even (i.e. $a \in \mathbb Z$), then it is sufficient that $b^2m$
      is an integer---but this occurs precisely when the denominator of $b$ divides
      $m$ at least twice. Thus if $c$ is even, $\alpha = a + b\sqrt m$ with
      $a, b \in \mathbb Z$.
      \\~\\
      If $c$ is odd, that is, it can be written as $c = 2k + 1$, then \[
        a^2 - b^2m
        = \frac{(2k + 1)^2 - 4b^2m}{4}
        = \frac{4k^2 + 4k  +  1 - 4b^2m}{4}
      \]  which is not an integer if $b \in \mathbb Z$, because the numerator is
      not divisible by $4$. (In particular, it is congruent to $1 \bmod 4$.)
      Thus $b/2 = d$ where $d$ is odd. In other words, $2b = 2j + 1$, and \[
        a^2 - b^2m
        = \frac{4k^2 + 4k  +  1 - (2j + 1)^2m}{4}
        = \frac{4k^2 + 4k  +  1 - (4j^2 + 4j + 1)m}{4}.
      \]
      So this case only occurs when $m \equiv 1 \bmod 4$, therefore
      $R = \mathbb Z[\sqrt m]$ for $m \not\equiv 1 \bmod 4$.
    \item
      When $m \equiv 1 \bmod 4$, $R$ has elements of the form $a + b\sqrt m$
      and $k + \frac 12 + (j + \frac 12)\sqrt m$, which have minimal polynomials
      $x^2 - 2ax + a^2 - b^2m$ and \[
        x^2 - (2k + 1)x +\paren{k + \frac {1}{2}}^2 - \paren{j +  \frac{1}{2}}^2m
      \] which has roots \[
        \frac{(2k + 1) \pm (2j + 1)\sqrt m}{2}
        = \frac{(2k + 1) \pm (2j + 1)\sqrt m}{2}
        % = \frac{(2k + 1) \pm (2j + 1)\sqrt m}{2}
      \] and thus $F = \mathbb Z \left[\frac{1 + \sqrt m}{2} \right]$.
  \end{enumerate}


  % If $c$ is odd, then we can rewrite the constant coefficient as \[
  % \]
  % Thus $a = c/2$ for $c$ an integer.
  % This means that $a = c/2$ for $c \in \mathbb Z$, and thus in order for
  % $c^2/4 - b^2m$ to be an integer, $c = d/2$ for some $d \in \mathbb Z$.
  % Therefore, we need an integer solution to \[
  %   \frac{c^2}{4} + \frac{d^2}{4}m
  % \]
\end{proof}
\end{document}
