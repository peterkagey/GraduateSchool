\documentclass{article}

\usepackage[margin=1in]{geometry}
\usepackage{amsmath,amsthm,amssymb}
\usepackage{bbm,enumerate,mathtools,mathrsfs}
\usepackage[hidelinks]{hyperref}
\usepackage{tikz}
\usetikzlibrary{matrix, arrows}

\newenvironment{problem}[2][Problem]{\begin{trivlist}
\item[\hskip \labelsep {\bfseries #1}\hskip \labelsep {\bfseries #2.}]}{\end{trivlist}}
\newenvironment{solution}[1][Solution.]{\begin{trivlist}
\item[\hskip \labelsep {\bfseries #1}]}{\end{trivlist}}
\newenvironment{problempart}[1]{\begin{trivlist}\item[\textbf{Part #1.}]}{\end{trivlist}}

\newenvironment{definition}[1][Definition.]{
  \begin{trivlist} \item[\hskip \labelsep {\bfseries #1}]
}{\end{trivlist}}

\newenvironment{example}[1][Example.]{
  \begin{trivlist} \item[\hskip \labelsep {\bfseries #1}]
}{\end{trivlist}}

\newenvironment{note}[1][Note.]{
  \begin{trivlist} \item[\hskip \labelsep {\bfseries #1}]
}{\end{trivlist}}

\newenvironment{theorem}[1][Theorem.]{
  \begin{trivlist} \item[\hskip \labelsep {\bfseries #1}]
}{\end{trivlist}}

\newenvironment{exercise}[1][Exercise.]{
  \begin{trivlist} \item[\hskip \labelsep {\bfseries #1}]
}{\end{trivlist}}

\newcommand{\set}[1]{\{ #1 \}}
\newcommand{\ang}[1]{\langle #1 \rangle}
\newcommand{\paren}[1]{\left( #1 \right)}
\newcommand{\fn}[3]{#1 \colon #2 \rightarrow #3}

\begin{document}

\title{Math 510b: Homework 2}
\author{Peter Kagey}
\date{Monday, February 11, 2019}

\maketitle

% -----------------------------------------------------
% First problem
% -----------------------------------------------------
\begin{problem}{1 (Artin)}
  Prove that the ideal $(x + y^2, y + x^2 + 2xy^2 + y^4)$ in $\mathbb C[x, y]$is a maximal ideal.
\end{problem}

\begin{proof} ~
\end{proof}
\vspace{1cm}
% -----------------------------------------------------
% Second problem
% -----------------------------------------------------
\begin{problem}{2 (Artin)}
  Let $I$ be the principal ideal of $\mathbb C[x, y]$ generated by the
  polynomial $y^2 + x^3 - 17$. Which of the following sets generate maximal
  ideals in the quotient ring $C[x, y]/I$?
  \begin{enumerate}[(a)]
    \item $(x - 1, y - 4)$
    \item $(x + 1, y + 4)$
    \item $(x^3 - 17, y^2)$
  \end{enumerate}
\end{problem}

\begin{proof}
\end{proof}
\pagebreak
% -----------------------------------------------------
% Third problem
% -----------------------------------------------------
\begin{problem}{6 (Artin)}
  Prove that the kernel of the homomorphism $\mathbb Z \rightarrow \mathbb R$
  sending $x \mapsto 1 + \sqrt 2$ is a principal ideal, and find a generator
  for this ideal.
\end{problem}

\begin{proof} ~
\end{proof}
\pagebreak
% -----------------------------------------------------
% Fourth problem
% -----------------------------------------------------
\begin{problem}{7 (Artin)}
  Let $f$ be an irreducible polynomial in $\mathbb C[x, y]$, and let $g$ be
  another polynomial. Prove that if the variety of zeros of $g$ in $\mathbb C^2$
  contains the variety of zeros of $f$, then $f$ divides $g$.
\end{problem}

\begin{proof} ~ \\
\end{proof}
\pagebreak
% -----------------------------------------------------
% Fifth problem
% -----------------------------------------------------
\begin{problem}{8 (Artin)}
  Determine the points of intersection of the two complex plane curves in each
  of the following \begin{enumerate}[(a)]
    \item $y^2 - x^3 + x^2 = 1, x + y = 1$
    \item $x^2 + xy + y^2 = 1, x^2 + 2y^2 = 1$
    \item $y^2 = x^3, xy = 1$
    \item $x + y + y^2 = 0, x - y + y^2 = 0$
  \end{enumerate}
\end{problem}

\begin{proof} ~
\end{proof}
\pagebreak
% -----------------------------------------------------
% Sixth problem
% -----------------------------------------------------
\begin{problem}{9 (Artin)}
  Prove that two quadratic polynomials $f, g$ in two variables have at most four
  common zeros unless they have a non-constant factor in common.
\end{problem}

\begin{proof} ~
\end{proof}
\pagebreak
% -----------------------------------------------------
% Seventh problem
% -----------------------------------------------------
\begin{problem}{10 (Artin)}
  An algebraic curve $\mathcal C$ in $\mathbb C^2$ is called irreducible if it
  is the locus of zeros of an irreducible polynomial $f(x, y)$---one which
  cannot be factored as a product of nonconstant polynomials. A point
  $p \in \mathcal C$ is called a singular point of the curve if
  $\partial f/\partial x = \partial f/\partial y = 0$ at $p$. Otherwise $p$ is
  a nonsingular point. Prove that an irreducible curve has only finitely many
  singular points.
\end{problem}

\begin{proof}
\end{proof}
\pagebreak
% -----------------------------------------------------
% Eighth problem
% -----------------------------------------------------
\begin{problem}[Extra]{Problem}
  Let $R = \mathbb Z(\sqrt{-5}) = \set{a + b\sqrt{-5}:a, b \in \mathbb Z} \subset \mathbb C$.
  Define
  $\fn NR{\mathbb Z_{\geq 0}}$ by sending
  $a + b\sqrt{-5} \mapsto a^2 + b^2$.
  \\~\\
  Show:
  \begin{enumerate}[(a)]
    \item $N(xy) = N(x)N(y)$ for all $x,y \in R$.
    \item If $x$ is a unit in $R$ then $N(x) = 1$. Thus the only units in $R$
      are $\pm 1$.
    \item There does not exist $x \in R$ with $N(x) = 3$.
    \item If $N(x) = 9$ then $x$ is irreducible in $R$.
    \item Note that $9 = 3 \cdot 3 = (2 + \sqrt{-5})(2 - \sqrt{-5})$, and
      conclude that $3$ is irreducible in $R$ but not prime.
    \item Factorization into irreducible elements in $R$ is not unique.
    \item Comparing this example to $\mathbb Z[i]$, what goes wrong here that
      works for $\mathbb Z[i]$?
    \item Find an ideal in $R$ which is not principal.
  \end{enumerate}
\end{problem}

\begin{proof}
\end{proof}
\end{document}
