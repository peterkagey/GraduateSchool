\documentclass{article}

\usepackage[margin=1in]{geometry}
\usepackage{amsmath,amsthm,amssymb}
\usepackage{bbm,enumerate,mathtools,mathrsfs}
\usepackage[hidelinks]{hyperref}
\usepackage{tikz}
\usetikzlibrary{matrix, arrows}

\newenvironment{problem}[2][Problem]{\begin{trivlist}
\item[\hskip \labelsep {\bfseries #1}\hskip \labelsep {\bfseries #2.}]}{\end{trivlist}}
\newenvironment{solution}[1][Solution.]{\begin{trivlist}
\item[\hskip \labelsep {\bfseries #1}]}{\end{trivlist}}
\newenvironment{problempart}[1]{\begin{trivlist}\item[\textbf{Part #1.}]}{\end{trivlist}}

\newenvironment{definition}[1][Definition.]{
  \begin{trivlist} \item[\hskip \labelsep {\bfseries #1}]
}{\end{trivlist}}

\newenvironment{example}[1][Example.]{
  \begin{trivlist} \item[\hskip \labelsep {\bfseries #1}]
}{\end{trivlist}}

\newenvironment{note}[1][Note.]{
  \begin{trivlist} \item[\hskip \labelsep {\bfseries #1}]
}{\end{trivlist}}

\newenvironment{theorem}[1][Theorem.]{
  \begin{trivlist} \item[\hskip \labelsep {\bfseries #1}]
}{\end{trivlist}}

\newenvironment{exercise}[1][Exercise.]{
  \begin{trivlist} \item[\hskip \labelsep {\bfseries #1}]
}{\end{trivlist}}

\newcommand{\set}[1]{\{ #1 \}}
\newcommand{\ang}[1]{\langle #1 \rangle}
\newcommand{\paren}[1]{\left( #1 \right)}
\newcommand{\fn}[3]{#1 \colon #2 \rightarrow #3}

\begin{document}

\title{Math 510b: Homework 2}
\author{Peter Kagey}
\date{Monday, February 11, 2019}

\maketitle

% -----------------------------------------------------
% First problem
% -----------------------------------------------------
\begin{problem}{1 (Artin)}
  Prove that the ideal $I = (x + y^2, y + x^2 + 2xy^2 + y^4)$ in
  $\mathbb C[x, y]$ is a maximal ideal.
\end{problem}

\begin{proof}
  It is sufficient to show that $R = \mathbb C[x, y]/I$ is a field.
  Notice that in $R$, $x + y^2 = 0$, so $x = -y^2$. Substituting into the second
  polynomial gives \[
    y + (-y^2)^2 + 2(-y^2)y^2 + y^4
    = y + y^4 + -2y^4 + y^4
    = y
    = 0,
  \] so $y=0$ in $R$. Thus $x = y = 0$ and $\mathbb C[x, y]/I \cong \mathbb C$,
  a field, so $I$ is maximal.
\end{proof}
% -----------------------------------------------------
% Second problem
% -----------------------------------------------------
\begin{problem}{2 (Artin)}
  Let $I$ be the principal ideal of $\mathbb C[x, y]$ generated by the
  polynomial $y^2 + x^3 - 17$. Which of the following sets generate maximal
  ideals in the quotient ring $R = \mathbb C[x, y]/I$?
  \begin{enumerate}[(a)]
    \item $(x - 1, y - 4)$
    \item $(x + 1, y + 4)$
    \item $(x^3 - 17, y^2)$
  \end{enumerate}
\end{problem}

\begin{proof} ~
  \begin{enumerate}
    \item In this ideal, we have that $\bar x = \bar 1$ and $\bar y = \bar 4$.
      In $I$ we have that $\bar y^2 + \bar x^3 = 17$.
      So $R/(x - 1, y - 4) \cong \mathbb C$, a field. Thus $(x - 1, y - 4)$ is a
      maximal ideal.
    \item In this ideal, we have that $\bar x = -\bar 1$ and $\bar y = -\bar 4$.
      In $I$ we have that $\bar y^2 + \bar x^3 = 17$, which is not satisfied by
      the above substitution. Thus $R/(x + 1, y + 4)$ is not a field, and
      $(x - 1, y - 4)$ is not a maximal ideal.
    \item Notice that $I \subset \mathbb (x^3 - 17, y^2)$ since
    $y^2 + x^3 - 17 = (x^3 - 17) + (y^2)$, and so
    $\mathbb C[x, y]/I \cong \mathbb C[x, y]/(x^3 - 17, y^2)$. This is not a
    field because $y^2 = 0$, but $y \neq 0$, and fields do not have zero
    divisors.
  \end{enumerate}
\end{proof}
\pagebreak
% -----------------------------------------------------
% Third problem
% -----------------------------------------------------
\begin{problem}{6 (Artin)}
  Prove that the kernel of the homomorphism $\fn f {\mathbb Z[x]}{\mathbb R}$
  sending $x \mapsto 1 + \sqrt 2$ is a principal ideal, and find a generator
  for this ideal.
\end{problem}

\begin{proof} ~
  Notice that the polynomial $x^2 - 2x - 1 \in \mathbb Z[x]$ has roots
  $1 \pm \sqrt 2$, so is irreducible in $\mathbb Z[x]$.
  Notice that $x - (1 + \sqrt2)$ is a factor of every polynomial
  $f \in \ker f \subset \mathbb Z[x] \subset \mathbb R[x]$ via the natural
  inclusion map, so $x^2 - 2x - 1$ must be a factor of every polynomial in the
  kernel, when looking at coefficients in $\mathbb Z$.
  Thus $\ker f = (x^2 - 2x - 1)$.
\end{proof}
% -----------------------------------------------------
% Fourth problem
% -----------------------------------------------------
\begin{problem}{7 (Artin)}
  Let $f$ be an irreducible polynomial in $\mathbb C[x, y]$, and let $g$ be
  another polynomial. Prove that if the variety of zeros of $g$ in $\mathbb C^2$
  contains the variety of zeros of $f$, then $f$ divides $g$.
\end{problem}

\begin{proof} ~ \\
  We know that for any two nonzero polynomials $f, g \in \mathbb C[x,y]$,
  $V(f, g)$ only differs from $V(\gcd(f, g))$ by a finite set. Therefore the
  variety \[
    V\!\paren{\frac f{\gcd(f, g)}, \frac g{\gcd(f, g)}}
  \]
  is empty. Thus $\frac f{\gcd(f, g)}$ is constant and $\gcd(f, g) = f$ (up to
  unit) and $f \mid g$.
  % $V(f) \subset V(g)$ then $(g) \subset (f)$, so  $g \mid f$.
\end{proof}
\pagebreak
% -----------------------------------------------------
% Fifth problem
% -----------------------------------------------------
\begin{problem}{8 (Artin)}
  Determine the points of intersection of the two complex plane curves in each
  of the following \begin{enumerate}[(a)]
    \item $y^2 - x^3 + x^2 = 1, x + y = 1$
    \item $x^2 + xy + y^2 = 1, x^2 + 2y^2 = 1$
    \item $y^2 = x^3, xy = 1$
    \item $x + y + y^2 = 0, x - y + y^2 = 0$
  \end{enumerate}
\end{problem}

\begin{proof} ~
  \begin{enumerate}[(a)]
    \item Notice that $y = 1 - x$ by the second equation, so substituting this
    into the first yields \begin{align*}
      -x^3 + x^2 - 2x + 1 &= 1 \\
      x(-x^2 + x - 2) &= 0
    \end{align*}
    So $x = 0$ or $x = \frac 12 (1 \pm i\sqrt 7)$, and so the solutions are \[
      (0, 1),
      \paren{\frac{1 + i\sqrt 7}2, \frac{1 - i\sqrt 7}2}, \text{ and }
      \paren{\frac{1 - i\sqrt 7}2, \frac{1 + i\sqrt 7}2}.
    \]
    \item Subtracting the second equation from the first gives \[
      y(y - x) = 0
    \]  so either $y = 0$ or $x = y$.
    If $y = 0$, $x = \pm 1$.
    If $x = y$, $3x^2 = 1$ so $x = y = \pm\sqrt{1/3}$.
    \item By the second equation, $x = 1/y$, so the first yields $y^2 =  1/y^3$.
    Multiplying both sides by $y^3$ gives $y^5 = 1$. Thus $y = \exp(\frac25\pi k i)$
    for some $k \in \set{1,2,3,4}$ (i.e. $y$ is a fifth root of unity).
    To satisfy the second equation, $x = \exp(-\frac25\pi k i)$, the complex conjugate.
    Thus any $k$ such that there exists $j, n \in \mathbb N$ satisfying \[
      \frac 45\pi k i = -\frac 65 j \pi i + 2\pi n.
    \] Taking $k = j = n$ always works, so all fifth roots of unity are points of intersection.
    \item Subtracting the second equation from the first yields $2y = 0$, so $y = 0$.
    Then $x + 0 + 0^2 = 0$, so $x = 0$. The point $(0,0)$ satisfies both
    equations, and furthermore is the only point of intersection.
  \end{enumerate}
\end{proof}
\pagebreak
% -----------------------------------------------------
% Sixth problem
% -----------------------------------------------------
\begin{problem}{9 (Artin)}
  Prove that two quadratic polynomials $f, g$ in two variables have at most four
  common zeros unless they have a non-constant factor in common.
\end{problem}

\begin{proof} ~
  We know that \begin{enumerate}
    \item $V(f, g)$ is a finite set since $f, g$ are relatively prime, and
    \item specializing $y$ results in polynomials with at most two roots by the
    fundamental theorem of algebra.
  \end{enumerate}
  If $f, g$ have fewer than four common zeros, we're done, so assume that they
  have at least four common zeros. Since no three can fall on the same line
  (otherwise we could subtract off by the line, which could contradict 2.) we
  know that the points must noncolinear. Thus there can be at most four common
  zeros.
\end{proof}
\pagebreak
% -----------------------------------------------------
% Seventh problem
% -----------------------------------------------------
\begin{problem}{10 (Artin)}
  An algebraic curve $\mathcal C$ in $\mathbb C^2$ is called irreducible if it
  is the locus of zeros of an irreducible polynomial $f(x, y)$---one which
  cannot be factored as a product of nonconstant polynomials. A point
  $p \in \mathcal C$ is called a singular point of the curve if
  $\partial f/\partial x = \partial f/\partial y = 0$ at $p$. Otherwise $p$ is
  a nonsingular point. Prove that an irreducible curve has only finitely many
  singular points.
\end{problem}

\begin{proof}
  It is sufficient to show that the variety
  $V(\partial f/\partial x, \partial f/\partial y)$
  is finite, which by the Nullstellensatz means that it's sufficient to show that
  $\mathbb C[x, y]/(\partial f/\partial x, \partial f/\partial y)$ is finite.
\end{proof}
\pagebreak
% -----------------------------------------------------
% Eighth problem
% -----------------------------------------------------
\begin{problem}[Extra]{Problem}
  Let $R = \mathbb Z(\sqrt{-5}) = \set{a + b\sqrt{-5}:a, b \in \mathbb Z} \subset \mathbb C$.
  Define
  $\fn NR{\mathbb Z_{\geq 0}}$ by sending
  $a + b\sqrt{-5} \mapsto a^2 + 5 b^2$.
  \\~\\
  Show:
  \begin{enumerate}[(a)]
    \item $N(xy) = N(x)N(y)$ for all $x,y \in R$.
    \item If $x$ is a unit in $R$ then $N(x) = 1$. Thus the only units in $R$
      are $\pm 1$.
    \item There does not exist $x \in R$ with $N(x) = 3$.
    \item If $N(x) = 9$ then $x$ is irreducible in $R$.
    \item Note that $9 = 3 \cdot 3 = (2 + \sqrt{-5})(2 - \sqrt{-5})$, and
      conclude that $3$ is irreducible in $R$ but not prime.
    \item Factorization into irreducible elements in $R$ is not unique.
    \item Comparing this example to $\mathbb Z[i]$, what goes wrong here that
      works for $\mathbb Z[i]$?
    \item Find an ideal in $R$ which is not principal.
  \end{enumerate}
\end{problem}

\begin{proof}
  \begin{enumerate}[(a)]
    \item On the right hand side: \begin{align*}
      N((a_1 + a_2\sqrt{-5})(b_1 + b_2\sqrt{-5}))
      &= N(a_1b_1 + a_1b_2\sqrt{-5} + a_2b_1\sqrt{-5} + a_2b_2\sqrt{-5}^2) \\
      &= N(a_1b_1 + 5a_2b_2 + (a_1b_2 + a_2b_1)\sqrt{-5}) \\
      &= (a_1b_1 - 5a_2b_2)^2 + 5(a_1b_2 + a_2b_1)^2 \\
      &= a_1^2b_1^2 -10a_1a_2b_1b_2 + 25a_2^2b_2^2 + 5a_1^2b_2^2 + 10a_1a_2b_1b_2 + 5a_2^2b_1^2 \\
      &= a_1^2b_1^2 + 5a_1^2b_2^2 + 5a_2^2b_1^2 + 25a_2^2b_2^2 \\
    \end{align*}
    On the left hand side:
    \begin{align*}
      N(a_1 + a_2\sqrt{-5})N(b_1 + b_2\sqrt{-5})
      &= (a_1^2 + 5a_2^2)(b_1^2 + 5b_2^2) \\
      &= a_1^2b_1^2 + 5a_1^2b_2^2 + 5a_2^2b_1^2 + 25a_2^2b_2^2
    \end{align*}
    \item $N(r) \in \mathbb Z_0^+$ for $r \in R$, so if there exists some
    $r^{-1} \in R$ such that $r(r^{-1}) = 1$, then $N(r)N(r^{-1}) = N(1) = 1$,
    so $N(r) = N(r^{-1}) = 1$.
    \item If $b = 0$, then $N(a) = a^2 = 3$ does not have a solution in the
      integers. If $b > 0$, then $N(a + b\sqrt{-5}) \geq 5$.
      Thus there is no $r \in R$ with $N(r) = 3$.
    \item Suppose for the sake of contradiction that $x = rs$, with $r, s$
    non-unit. Since $9$ only has factors of $1, 3, 9$, and $N(x) = N(r)N(s)$
    this implies one of the following contradictions \begin{enumerate}[(i)]
      \item $N(r) = 1$, in which case $r$ is a unit,
      \item $N(r) = 3$, a contradiction by (c), or
      \item $N(r) = 9$, in which case $s$ is a unit.
    \end{enumerate}
    Thus if $N(x) = 9$, $x$ is irreducible.
    \item Since $N(3) = 9$, $3$ is irreducible by part (d). However, $3$ is not
    prime because $3 \nmid (2 \pm \sqrt{-5})$.
    \item $N(2 + \sqrt{-5}) = N(2 - \sqrt{-5}) = 2^2 + 5(1^2) = 9$, so $9$
    factors into irreducibles in two different ways.
    \item In the case of $\mathbb Z[i]$, all irreducible elements are prime, so
      (e) fails.
    \item Consider the ideal $I = (3, 2 + \sqrt{-5})$. Notice that if $I$ is
    principal with generator $g$, then $N(3) = 9 = N(rg) = N(r)N(g)$.
    \begin{enumerate}[(i)]
      \item If $N(g) = 9$, then $r$ must be a unit, and so $r = \pm 1$ and
      $g = \pm 3$. However, this contradicts $2 + \sqrt{-5} \in I$.
      \item If $N(g) = 3$, this contradicts $(c)$.
      \item If $N(g) = 1$, then $g = \pm 1$, so $I = R$, and $1$ can
      be expressed as \begin{align*}
        1 &= 3(a_1 + a_2\sqrt{-5}) + (2 + \sqrt{-5})(b_1 + b_2\sqrt{-5}) \\
        &= 3a_1 + 2b_1 - 5b_2 + (3a_2 + b_1 + 2b_2) \sqrt{-5}
      \end{align*}
      Then looking modulo $3$, this means that
      $2b_1 + b_2 \cong 1 \bmod 3$ and $b_1 + 2b_2 \cong 1 \bmod 3$. Adding this
      together yields $0 \cong 1 \bmod 3$, a contradiction.
    \end{enumerate}
  \end{enumerate}
  Therefore $I$ cannot be principal.
\end{proof}
\end{document}
