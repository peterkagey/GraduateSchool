\documentclass{article}

\usepackage[margin=1in]{geometry}
\usepackage{amsmath,amsthm,amssymb}
\usepackage{bbm,enumerate,mathtools,mathrsfs}
\usepackage[hidelinks]{hyperref}
\usepackage{tikz}
\usetikzlibrary{matrix, arrows}

\newenvironment{problem}[2][Problem]{\begin{trivlist}
\item[\hskip \labelsep {\bfseries #1}\hskip \labelsep {\bfseries #2.}]}{\end{trivlist}}
\newenvironment{solution}[1][Solution.]{\begin{trivlist}
\item[\hskip \labelsep {\bfseries #1}]}{\end{trivlist}}
\newenvironment{problempart}[1]{\begin{trivlist}\item[\textbf{Part #1.}]}{\end{trivlist}}

\newenvironment{definition}[1][Definition.]{
  \begin{trivlist} \item[\hskip \labelsep {\bfseries #1}]
}{\end{trivlist}}

\newenvironment{example}[1][Example.]{
  \begin{trivlist} \item[\hskip \labelsep {\bfseries #1}]
}{\end{trivlist}}

\newenvironment{note}[1][Note.]{
  \begin{trivlist} \item[\hskip \labelsep {\bfseries #1}]
}{\end{trivlist}}

\newenvironment{theorem}[1][Theorem.]{
  \begin{trivlist} \item[\hskip \labelsep {\bfseries #1}]
}{\end{trivlist}}

\newenvironment{exercise}[1][Exercise.]{
  \begin{trivlist} \item[\hskip \labelsep {\bfseries #1}]
}{\end{trivlist}}

\newcommand{\set}[1]{\{ #1 \}}
\newcommand{\ang}[1]{\langle #1 \rangle}
\newcommand{\paren}[1]{\left( #1 \right)}
\newcommand{\fn}[3]{#1 \colon #2 \rightarrow #3}

\begin{document}

\title{Math 510b: Homework 3}
\author{Peter Kagey}
\date{Wednesday, February 27, 2019}

\maketitle

% -----------------------------------------------------
% First problem
% -----------------------------------------------------
\begin{problem}{5.57 (Rotman)} % https://artofproblemsolving.com/wiki/index.php/Nilradical
  % https://math.stackexchange.com/a/859408/121988
  If $R$ is a commutative ring, then its \textbf{nilradical}
  $\operatorname{nil}(R)$ is defined to be the intersection of all of the prime
  ideals in $R$. Prove that $\operatorname{nil}(R)$ is the set of all the
  nilpotent elements in $R$: \[
    \operatorname{nil}(R) = \set{r \in R : r^m = 0 \text{ for some } m \geq 1}
  \]
\end{problem}

\begin{proof} ~ \\
  $(\Longleftarrow)$ Let $\mathfrak p$ be an arbitrary prime ideal, and assume
  that $x \in \sqrt{(0)}$. Then $x^n \in \mathfrak p$ because $x^m = 0$ for some
  $m > 1$. Next, if $x^n \in \mathfrak p$, then $x \cdot x^{n-1} \in \mathfrak p$,
  meaning that either $x \in \mathfrak p$ (in which case we're done) or
  $x^{n-1} \in \mathfrak p$, which can be continued by induction to show that
  $x \in \mathfrak p$. Since $\mathfrak p$ was an arbitrary prime ideal, $x$ is
  in the intersection of all prime ideals.
  \\~\\ % Exercise 5.7
  $(\Longrightarrow)$ Now to prove the converse, let $x \not\in \sqrt{(0)}$.
  Next, let $\mathcal S = \set{I \text{ ideal of } R : I \text{ has no element of the form } x^n}$.
  This set is nonempty because it contains $(0)$.
  Thus by Zorn's Lemma $\mathcal S$ has a maximal element $\mathfrak m$ because
  the union of all ideals in a chain is in $\mathcal S$ and serves as an upper
  bound.
\end{proof}
\pagebreak
% -----------------------------------------------------
% Second problem
% -----------------------------------------------------
\begin{problem}{10.27 (Rotman)} %
  If $R$ is an integrally closed domain and $S \subseteq R$ is multiplicative,
  prove that $S^{-1}R$ is also integrally closed.
\end{problem}

\begin{proof} ~
  \begin{definition}
    A domain $R$ is called an \textit{integrally closed domain} if every element
    $\alpha \in \operatorname{Frac}(R)$ is the root of a monic polynomial in
    $R[x]$.
  \end{definition}
  Let $f_1/f_2 \in \operatorname{Frac}(S^{-1}R)$. We want to show that $f_1/f_2$
  can be written as the root of a monic polynomial with coefficients in $S^{-1}R$: \[
    \paren{\frac{f_1}{f_2}}^n
    + \frac{r_{n-1}}{s_{n-1}}\cdot\paren{\frac{f_1}{f_2}}^{n-1}
    + \hdots
    + \frac{r_0}{s_0}
    = 0.
  \] Let $s = s_{n-1} \hdots s_0$, since we know that since $R$ is integrally closed and \[
    \paren{\frac{sf_1}{f_2}}^n
    + \frac{s r_{n-1}}{s_{n-1}}\cdot\paren{\frac{sf_1}{f_2}}^{n-1}
    + \frac{s^2 r_{n-2}}{s_{n-2}}\cdot\paren{\frac{sf_1}{f_2}}^{n-2}
    + \hdots
    + \frac{s^n r_0}{s_0}
    = 0
  \]  has coefficients in $R$, that $\frac{sf_1}{f_2}$ is a root of a polynomial
  with coefficients in $R$, and thus $\frac{f_1}{f_2}$ is a root of a polynomial
  with coefficients in $S^{-1}R$. Therefore $S^{-1}R$ is integrally closed.
\end{proof}
\pagebreak
% -----------------------------------------------------
% Third problem
% -----------------------------------------------------
\begin{problem}{10.39 (Rotman)} % p. 915
  Let $k$ be a field and let $\mathfrak m$ be a maximal ideal in
  $k[x_1, \hdots, x_n]$. Prove that \[
    \mathfrak m = (f_1(x_1), f_2(x_1, x_2), \hdots, f_n(x_1, \hdots, x_n)).
  \]
\end{problem}

\begin{proof}
  Consider $\mathfrak m' = \mathfrak m \cap k[x_1, \hdots, x_{n-1}]$, which is
  maximal in $k[x_1, \hdots, x_{n-1}]$. By Corollary 10.71,
  $\mathfrak m = (\mathfrak m', g_n(x_n))$ where $f_n$ has coefficients in
  $k[x_1, \hdots, x_{n-1}]$, so that $g_n(x_n) = f_n(x_1, \hdots, x_n)$.
  \\~\\
  Then continuing by induction,
  $\mathfrak m^{(k)} = \mathfrak m^{(k-1)} \cap k[x_1, \hdots, x_{n-k}]$, and so
  \begin{align*}
    \mathfrak m^{(k)}
    &= (\mathfrak m^{(k+1)}, g_{n-k}(x_{n-k})) \\
    &= (\mathfrak m^{(k+1)}, f_{n-k}(x_1, \hdots, x_{n-k})) \\
    \mathfrak m
    &= (\mathfrak m', f_n(x_1, \hdots, x_n)) \\
    &= (\mathfrak m'', f_{n-1}(x_1, \hdots, x_{n-1}), f_n(x_1, \hdots, x_n)) \\
    &= \hdots \\
    &= (\mathfrak m^{(n)}, f_{1}(x_1), \hdots, f_{n-1}(x_1, \hdots, x_{n-1}), f_n(x_1, \hdots, x_n))
  \end{align*}
  Where $\mathfrak m^{(n)}$ is a maximal ideal in $k$, a field, so it can be omitted. Thus \[
    \mathfrak m = (
      \mathfrak m^{(n)}, f_{1}(x_1),
      \hdots,
      f_{n-1}(x_1, \hdots, x_{n-1}),
      f_n(x_1, \hdots, x_n)
    ),
  \] as desired.
\end{proof}
\pagebreak
% -----------------------------------------------------
% Fourth problem
% -----------------------------------------------------
\begin{problem}{10.40 (Rotman)} % p. 915
  Prove that if $R$ is Noetherian then $\operatorname{nil}(R)$ is a nilpotent ideal.
\end{problem}

\begin{proof}
  We'll do this combinatorially. Notice that $R$ is Noetherian so the nilpotent
  ideal is finitely generated, $\operatorname{nil}(R) = (a_1, \hdots, a_n)$,
  and thus every element can be written as \[
    \operatorname{nil}(R) = \sum_{i=1}^n a_iR.
  \] Therefore \[
    \operatorname{nil}(R)^k = \sum_{i=1}^N \paren{\prod_{j=1}^k a_{i_j}R}.
  \]
  Then let $p = \max\set{p_i : a_i^{p_i} = 0}$ and let $k = np$. Then by the
  pigeonhole principal, some $a_i$ must show up at least $p$ times in the
  product, so $\prod_{j=1}^k a_{i_j}R = 0R = (0)$.
\end{proof}
\pagebreak
% -----------------------------------------------------
% Fifth problem
% -----------------------------------------------------
\begin{problem}{4 (Artin)} ~
  \begin{enumerate}[(a)]
    \item Determine the prime ideals of the polynomial ring $\mathbb C[x, y]$ in two variables.
    \item Show that unique factorization of ideals does not hold in the ring $\mathbb C[x, y]$.
  \end{enumerate}
\end{problem}

\begin{proof}

\end{proof}
\pagebreak
% -----------------------------------------------------
% Sixth problem
% -----------------------------------------------------
\begin{problem}{15 (Artin)}
  Determine the singular points of $x^3 + y^3  - 3xy = 0$.
\end{problem}

\begin{proof}
  Singular points occur when \begin{alignat*}{2}
    \frac{\partial}{\partial x} &= x^3 + y^3  - 3xy = 3x^2 - 3y &&= 0 \\
    \frac{\partial}{\partial y} &= x^3 + y^3  - 3xy = 3y^2 - 3x &&= 0
  \end{alignat*}
  Then solving for $y$ in the first equation yields $y = x^2$, and substituting it
  into the second equation gives $3x^4 - 3x = 0 = 3x(x^3 - 1)$ which has roots at
  $x = 0$ and $x = 1$. Using the relation $y = x^2$ gives roots $(0, 0)$ and
  $(1, 1)$.
\end{proof}
\pagebreak
% -----------------------------------------------------
% Seventh problem
% -----------------------------------------------------
\begin{problem}{16 (Artin)} ~
  \begin{enumerate}[(a)]
    \item Consider the map $\fn \psi {\mathbb C[x, y]}{\mathbb C[t]}$ which sends
    $f(x, y) \mapsto f(t^2, t^3)$. Prove that its kernel is a principal ideal
    and that its image is the set of polynomials $p(t)$ such that $p'(0) = 0$.
    \item Consider the map $\fn \phi {\mathbb C[x, y]}{\mathbb C[t]}$ which sends
    $f(x, y) \mapsto f(t^2 - t, t^3 - t^2)$. Prove that its kernel is a principal ideal
    and that its image is the set of polynomials $p(t)$ such that $p(0) = p(1)$.
    Give an intuitive explanation in terms of the geometry of the variety
    $\set{f=0}$ in $\mathbb C^2$.
  \end{enumerate}
\end{problem}

\begin{proof} ~
  \begin{enumerate}[(a)]
    \item The $\ker(\psi)$ consists of polynomials of the form \[
      \psi\paren{\sum_{n=0}^N \sum_{i=0}^n c_{n,i} x^i y^{n-i}}
      = \sum_{n=0}^N \sum_{i = 0}^n c_{n,i}(t^2)^i(t^3)^{n-i} = 0,
    \] and thus $\ker(\psi)$ is principal.
    \\
    Now polynomials with $p'(0) = 0$ are exactly of the form \[
      p(t) = c^nt^n + \hdots + c_3t^3 + c_2t^2 + c_0,
    \] that is, those with vanishing linear term. Notice that all non-negative
    integers besides $1$ can be written as $2k + 3\ell$ for some $k$ and $\ell$,
    so all polynomials with $p'(0) = 0$ are in $\operatorname{Im}(\psi)$.
    Similarly, $1 \neq 2k + 3\ell$ for any $k, \ell \in \mathbb N_{\geq 0}$,
    so $p(t)$ has no linear terms and $p'(0) = 0$.
    \item The $\ker(\phi)$ consists of polynomials of the form \begin{align*}
      \phi\paren{\sum_{n=0}^N \sum_{i=0}^n c_{n,i} x^i y^{n-i}}
      &= \sum_{n=0}^N \sum_{i = 0}^n c_{n,i}(t^2-t)^i(t^3-t^2)^{n-i} = 0 \\
      &= \sum_{n=0}^N \sum_{i = 0}^n c_{n,i}(t-1)^n t^{2n-i}.
    \end{align*}
  \end{enumerate}
\end{proof}
\end{document}
