\documentclass{article}

\usepackage[margin=1in]{geometry}
\usepackage{amsmath,amsthm,amssymb}
\usepackage{bbm, enumerate}

\newenvironment{problem}[2][Problem]{\begin{trivlist}
\item[\hskip \labelsep {\bfseries #1}\hskip \labelsep {\bfseries #2.}]}{\end{trivlist}}
\newenvironment{note}[1][Note.]{\begin{trivlist}
\item[\hskip \labelsep {\bfseries #1}]}{\end{trivlist}}

\begin{document}

\title{Complex Analysis: Homework 6}
\author{Peter Kagey}

\maketitle

% -----------------------------------------------------
% First problem
% -----------------------------------------------------
\begin{problem}{2} (page 129f) \\
  Show that a function which is analytic in the whole plane and has a
  nonessential singularity at $\infty$ reduces to a polynomial.
\end{problem}
\begin{proof} \text{} \\
  Let $f$ be a function which is analytic in the whole plane and has a
  nonessential singularity at $\infty$.
  % In particular, $f(1/z)$ also has a
  % nonessential singularity at $0$. Because the singularity is nonessential,
  % there exists some $m$ such that $z^m \cdot f(1/z)$ is analytic at 0.
  \\~\\
  \textbf{Case 1 (removable singularity)} Assume that $f$ has a removable
    singularity at $\infty$. Then $\displaystyle\lim_{z\rightarrow\infty} f(z) \in \mathbb{C}$,
    and since $f$ is analytic on the whole plane by Liouville's Theorem, $f$ is
    constant.
  \\~\\
  \textbf{Case 2 (pole)} Assume $f$ has a pole at $\infty$, and let
    $g(z) = f(1/z)$. In this case $\lim_{z\rightarrow\infty} f(z) = \infty$,
    so $\lim_{z\rightarrow0} g(z) = \infty$. Then $h(z) = z^m\cdot g(z)$ is analytic
    in the whole plane, so by Taylor's Theorem expanded about $0$, \[
      z^m\cdot g(z) = h(0) + \frac{h'(0)}{1!}z + \hdots
      + \frac{h^{(m-1)}(0)}{(m-1)!}z^{m-1}
      + h_m(z)z^m.
    \] Dividing by $z^m$ yields \[
      g(z) = \frac{h(0)}{z^m} + \frac{h'(0)}{1!z^{m-1}} + \hdots
      + \frac{h^{(m-1)}(0)}{(m-1)!z}
      + h_m(z),
    \] and so \[
      f(z) = g(1/z) = h(0)z^m + \frac{h'(0)}{1!}z^{m-1} + \hdots
      + \frac{h^{(m-1)}(0)}{(m-1)!}z
      + h_m(1/z).
    \]
    So it is sufficient to show that $h_m$ reduces to a constant. But this
    follows because \[
      h_m(z) = g(z) - \left(\frac{h(0)}{z^m} + \frac{h'(0)}{1!z^{m-1}} + \hdots
        + \frac{h^{(m-1)}(0)}{(m-1)!z}\right)
    \] is bounded and analytic on the whole plane, so must be a constant by
    Liouville's Theorem.
\end{proof}
% -----------------------------------------------------
% Second problem
% -----------------------------------------------------
\pagebreak

\begin{problem}{4} (page 129f) \\
  Show that any function which is meromorphic in the extended plane is rational.
\end{problem}

\begin{proof} \text{} \\
  Let $f$ be a function which is meromorphic in the extended plane with poles
    $\{ p_1, p_2, \hdots, p_n \} \subset \mathbb{C}$.
  Then if $k_i$ is the order of $p_i$, let \[
    g(z) = f(z) \cdot \prod_{i=1}^{n} (z - p_i)^{k_i}
  \] which is analytic on $\mathbb{C}$. From the above question (Problem 2), we
  know $g(z)$ is rational, because any singularities at $\infty$ are
  nonessential. Therefore \[
    f(z) = \frac{g(z)}{\prod_{i=1}^{n} (z - p_i)^{k_i}}
  \] is a rational function.
\end{proof}
% -----------------------------------------------------
% Third problem
% -----------------------------------------------------
\pagebreak

\begin{problem}{5} (page 129f) \\
  Prove that an isolated singularity of $f(z)$ is removable as soon as either
  $\operatorname{Re} f(z)$ or $\operatorname{Im} f(z)$ is bounded above or
  below.\\
  (Hint: Apply a fractional linear transformation.)
\end{problem}

\begin{proof} \text{} \\
  In any of these four cases, the half-plane can be mapped into the unit disk.
  For example, if $\operatorname{Re}(f(z)) > M$, then the transformation
  $z \mapsto z - M$ followed by $z \mapsto (z-1)/(z+1)$ yields a bounded
  function \[
    g(z) = \frac{f(z) - M - 1}{f(z) - M + 1},
  \] where $|g(z)| \leq 1$. Therefore $g$ has a removable singularity.
  Extending $g$ and applying the inverse linear transformation results in an
  analytic extension of $f$, showing that the singularity is removable.
\end{proof}
% -----------------------------------------------------
% Fourth problem
% -----------------------------------------------------
\pagebreak

\begin{problem}{6} (page 129f) \\
  Show that an isolated singularity of $f(z)$ cannot be a pole of $\exp f(z)$.\\
  (Hint: $f$ and $e^f$ cannot have a common pole (why?). Now apply Theorem 9.)
\end{problem}

\begin{proof} \text{} \\
  Firstly $f$ and $e^f$ cannot have a common pole because if $f$ has a pole at
  $z_0$, then $g(z) = f(1/z)$ has a zero at $z_0$, and so $e^{g(z_0)} = 1$, and
  thus $e^f = 1$ at every pole of $f$.

  In every neighborhood of a isolated singularity of $f$, $f(z)$ can be made
  arbitrarily close to $\infty$ (on the Riemann Sphere), and so $\exp(f(z))$ can
  be made arbitrarily close to any complex value $w = \alpha + \beta i$,
  by choosing $z$ such that $f(z) = a + bi$ has $a$ near $\alpha$ and $b$ near
  $\beta + 2\pi k$ for some (large) $k$.

  Therefore by Theorem 9, the pole at $f(z)$ is an essential singularity of
  $\exp f(z)$.
\end{proof}
% -----------------------------------------------------
% Fifth problem
% -----------------------------------------------------
\pagebreak

\begin{problem}{1} (page 133) \\
  Determine explicitly the largest disk about the origin whose image under the
  mapping $f(z) = z^2 + z$ is one to one.
\end{problem}

\begin{proof} \text{} \\
  By corollary 2, because $f'(z) = 2z + 1$ vanishes at
  $z = 1/2$, the disk must not contain the point $z = 1/2$. Thus, the disk can
  have radius at most $1/2$.

  Suppose that $w \not= z$ but $w^2 + w = z^2 + z$. Then $w^2 - z^2 = z - w$, so
  $(w + z) = -1$. If $w$ and $z$ are in the disk of radius $1/2$, then
  $w + z \not= -1$, so $f$ is indeed injective on the disk of radius $1/2$.
\end{proof}
% -----------------------------------------------------
% Sixth problem
% -----------------------------------------------------
\pagebreak

\begin{problem}{3} (page 133) \\
  Apply the representation $f(z) = w_0 + \zeta(z)^n$ to $\cos z$ with $z_0 = 0$.
  Determine $\zeta(z)$ explicitly.
\end{problem}

\begin{proof} \text{} \\
  Firstly $\cos(0) = 1$, so we can write \[
    \cos(z) - 1 = z^2\frac{\cos(z) - 1}{z^2}
  \] so $g(z) = (\cos(z) - 1)/z^2$ and so by the double-angle formula \begin{align*}
    h(z) &= \sqrt{g(z)}\\
    &= \sqrt{\frac{\cos(z) - 1}{z^2}}\\
    &= \sqrt{\frac{-2\sin^2(z/2)}{z^2}}\\
    &= \frac{i\sqrt{2}\sin(z/2)}{z}.
  \end{align*}
  And therefore $\zeta(z) = zh(z) = i\sqrt{2}\sin(z/2)$.
\end{proof}

\end{document}
