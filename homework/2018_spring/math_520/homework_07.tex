\documentclass{article}

\usepackage[margin=1in]{geometry}
\usepackage{amsmath,amsthm,amssymb}
\usepackage{bbm, enumerate}

\newenvironment{problem}[2][Problem]{\begin{trivlist}
\item[\hskip \labelsep {\bfseries #1}\hskip \labelsep {\bfseries #2.}]}{\end{trivlist}}
\newenvironment{note}[1][Note.]{\begin{trivlist}
\item[\hskip \labelsep {\bfseries #1}]}{\end{trivlist}}

\begin{document}

\title{Complex Analysis: Homework 7}
\author{Peter Kagey}

\maketitle

% -----------------------------------------------------
% First problem
% -----------------------------------------------------
\begin{problem}{1} (page 136) \\
  Show by use of (36), or directly, that $|f(z)| \leq 1$ for $|z| \leq 1$
  implies \[
    \frac{|f'(z)|}{1 - |f(z)|^2} \leq \frac{1}{1 - |z|^2}.
  \]
\end{problem}
\begin{proof} \text{} \\
  Starting from (36) with $M = 1$ and $R = 1$, we have \[
    \left| \frac{f(z) - f(z_0)}{1 - \overline{f(z_0)}f(z)} \right| \leq
    \left| \frac{z - z_0}{1 - \overline{z_0}z} \right|
  \] so multiplying by the denominator of the left hand side and dividing by the
  numerator of the right hand side yields \[
  \left| \frac{f(z) - f(z_0)}{z - z_0} \right| \leq
  \left| \frac{1 - \overline{f(z_0)}f(z)}{1 - \overline{z_0}z} \right|.
  \]
  In particular, this holds in the limit: \begin{align*}
    \lim_{z \rightarrow z_0} \left| \frac{f(z) - f(z_0)}{z - z_0} \right| &\leq
    \lim_{z \rightarrow z_0} \left| \frac{1 - \overline{f(z_0)}f(z)}{1 - \overline{z_0}z} \right| \\
    |f'(z)| &\leq \left|\frac{1 - |f(z)|^2}{1 - |z|^2}\right|\\
    &= \frac{1 - |f(z)|^2}{1 - |z|^2}
  \end{align*}
  where the final equality holds because $1 - |f(z)|^2 \geq 0$ and
  $1 - |z|^2 \geq 0$.\\~\\
  Thus dividing both sides by $1 - |f(z)|^2 \geq 0$ gives the desired inequality
  \[
    \frac{|f'(z)|}{1 - |f(z)|^2} \leq \frac{1}{1 - |z|^2}.
  \]
\end{proof}
% -----------------------------------------------------
% Second problem
% -----------------------------------------------------
\pagebreak

\begin{problem}{2} (page 136) \\
  If $f(z)$ is analytic and $\operatorname{Im} f(z) \geq 0$ for
  $\operatorname{Im} z > 0$, show that \[
    \left|\frac{f(z)-f(z_0)}{f(z)-\overline{f(z_0)}}\right| \leq
    \left|\frac{z-z_0}{z - \overline{z_0}}\right|
  \] and \[
    \frac{|f'(z)|}{\operatorname{Im} f(z)} \leq \frac{1}{y}
  \]
\end{problem}

\begin{proof} \text{} \\
  For the first inequality, let \[
    \hat{f}(z) = \frac{f(z)-f(z_0)}{f(z)-\overline{f(z_0)}}
    \hspace{0.3cm}\text{and}\hspace{0.3cm}
    g(z) = \frac{z-z_0}{z - \overline{z_0}}
  \] then $\hat{f}$ and $g$ are both $0$ at $z_0$, and are both analytic on the
  upper half plane punctured at $\overline{f(z_0)}$ and $\overline{z_0}$
  respectively. By the hypothesis, we know that $|\hat{f}(z)| \leq 1$ and
  $|g(z)| \leq 1$.
  For $z_0$ on the boundary of $\operatorname{Im} z > 0$
  (which is the real axis), $g(z) = 1$, so by Theorem $12'$ the function \[
    \left|\frac{\hat{f}(z)}{g(z)}\right| = |\hat{f}(z)| \leq 1
  \] on the boundary, so this must be the maximum in the upper half plane. Therefore \[
    |\hat{f}(z)| \leq |g(z)|
  \].

  The second inequality follows from the first. Multiplying
  by the denominator on the left hand side and dividing by the numerator on the
  right hand side \[
    \left|\frac{f(z) - f(z_0)}{z - z_0}\right| \leq
    \left|\frac{f(z)-\overline{f(z_0)}}{z - \overline{z_0}}\right|
  \] and so in the limit (Noting that for
  $w \in \mathbb{C},\ w-\overline{w} = 2\operatorname{Im} w$) \begin{align*}
    \lim_{z \rightarrow z_0}\left|\frac{f(z) - f(z_0)}{z - z_0}\right| &\leq
    \lim_{z \rightarrow z_0}\left|\frac{f(z)-\overline{f(z_0)}}{z - \overline{z_0}}\right|\\
    |f'(z)| &\leq
    \left|\frac{2\operatorname{Im}f(z)}{2\operatorname{Im}z}\right|.
  \end{align*}
  Thus dividing both sides by $|\operatorname{Im} f(z)|$ yields \begin{align*}
    \frac{|f'(z)|}{|\operatorname{Im} f(z)|} &\leq \frac{1}{|\operatorname{Im} z|}\\
    \frac{|f'(z)|}{\operatorname{Im} f(z)} &\leq \frac{1}{\operatorname{Im} z},
  \end{align*} where the final inequality holds by the hypothesis that
  $\operatorname{Im} f(z) \geq 0$ for $\operatorname{Im} z > 0$.
\end{proof}
% -----------------------------------------------------
% Third problem
% -----------------------------------------------------
\pagebreak

\begin{problem}{3} (page 136) \\
  In Ex. 1 and 2, prove that equality implies that $f(z)$ is a linear transformation.
\end{problem}

\begin{proof} \text{} \\
  (?)
\end{proof}
% -----------------------------------------------------
% Fourth problem
% -----------------------------------------------------
\pagebreak

\begin{problem}{1} (page 154) \\
  How many roots does the equation $z^7 - 2z^5 + 6z^3 - z + 1 = 0$ have in the
  disk $|z| < 1$?
\end{problem}

\begin{proof} \text{} \\
  Let $f(z) = 6z^3$, and let $g(z) = z^7 - 2z^5 + 6z^3 - z + 1$ be the function
  described above. Then by the triangle inequality, \[
    |f-g| = |z^7 - 2z^5 - z + 1| \leq |z|^7 + 2|z|^5 + |z| + 1,
  \] so $|f-g| \leq 5$ on the boundary of the unit circle. Similarly, \[
    |f| = 6|z|^3 = 6
  \] on the boundary of the unit circle.
  Since $|f-g| < |f|$ on this curve, by Rouch\'e's Theorem $g(z)$ has three roots
  because $f(z) = 6z^3$ has three roots.

\end{proof}
% -----------------------------------------------------
% Fifth problem
% -----------------------------------------------------
\pagebreak

\begin{problem}{2} (page 154) \\
  How many roots of the equation $z^4 - 6z + 3 = 0$ have their modulus between
  $1$ and $2$?
\end{problem}

\begin{proof} \text{} \\
  It is sufficient to count the number of roots in the disk of radius 2 and
  subtract the number of roots in the unit disk.\\
  By Rouch\'e's Theorem with $f(z) = z^4$ and $g(z) = z^4 - 6z + 3$, on the circle
  of radius 2, $|z| = 2$, \[
    |z^4 - (z^4 - 6z + 3)| = |6z - 3| \leq 15 < 16 = 2^4 = |z|^4 = |z^4|
  \] $|z^4 - (z^4 - 6z + 3)|$ has all four of its roots inside the circle of
  radius 2.\\
  By another application of Rouch\'e's Theorem, this time with
  $f(z) = -6z+3$ and $g(z) = z^4 - 6z + 3$ on the unit circle, $|z| = 1$, \begin{align*}
    |f-g| &= |-6z + 3 - (z^4 - 6z + 3)| = |-z^4| = |z|^4 = 1, \text{ and}\\
    |f| &= |-6z + 3| \geq 6|z| - 3 = 3, \text{ so}\\
    |f-g| &< |f| \text{ for } |z| = 1.
  \end{align*} Since $f$ has one root inside the unit circle (at $z = 1/2$), $g$
  also has one root inside the unit circle.
  \\ Therefore $z^4 - 6z + 3$ has three roots with modulus between 1 and 2.
\end{proof}
% -----------------------------------------------------
% Sixth problem
% -----------------------------------------------------
\pagebreak

\begin{problem}{3} (page 154) \\
  How many roots of the equation $z^4 + 8z^3 + 3z^2 + 8z + 3 = 0$ lie in the
  right half plane? (Hint: Sketch the image of the imaginary axis and apply the
  argument principle to a large half disk.)
\end{problem}

\begin{proof} \text{} \\
  The plan is to apply the argument theorem to the curve with one segment along
  the imaginary axis from $Ri$ to $-Ri$ and the other segment along $Re^{it}$
  for $t \in [-\pi/2, \pi/2]$, and then making $R$ big enough to catch all
  roots.\\
  First consider the image of the imaginary axis: \[
    f(it) = t^4 + -8it^3 - 3t^2 + 8it + 3 = (t^4 - 3t^2 + 3) + i(-8t^3 + 8t)
  \] which has a strictly positive real part, because $t^4 - 3t^2 + 3$
  has discriminant $-3$. Thus the image of the imaginary axis is entirely in the
  right half plane, and in particular does not wind around the origin.\\
  Thus we're concerned with how many times the semicircle winds around the
  origin. The image of the semicircle \begin{align*}
    f(Re^{it}) &=
      R^4 e^{4it}
      + 8R^3 e^{3it}
      - 3R^2 e^{2it}
      + 8R e^{it}
      + 3 \\
    &= R^4 e^{4it}\left(
      1 + \frac{8}{R}e^{-it}
      - \frac{3}{R^2}e^{-2it}
      + \frac{8}{R^3}e^{-3it}
      + \frac{3}{R^4}e^{-4it}
    \right)
  \end{align*} can be approximated by $R^4 e^{4it}$, which winds around the
  origin twice on the interval $t \in [-\pi/2, \pi/2]$.\\
  Therefore $f$ has two roots in the right half plane.
\end{proof}

\end{document}
