\documentclass{article}

\usepackage[margin=1in]{geometry}
\usepackage{amsmath,amsthm,amssymb}
\usepackage{bbm, enumerate, tikz}
\usepackage{multicol}

\newenvironment{problem}[2][Problem]{\begin{trivlist}
\item[\hskip \labelsep {\bfseries #1}\hskip \labelsep {\bfseries #2.}]}{\end{trivlist}}
\newenvironment{note}[1][Note.]{\begin{trivlist}
\item[\hskip \labelsep {\bfseries #1}]}{\end{trivlist}}

\begin{document}

\title{Complex Analysis: Homework 14}
\author{Peter Kagey}

\maketitle

% -----------------------------------------------------
% Problem
% -----------------------------------------------------
\begin{problem}{2} (page 227) \\
  Show that the functions $z^n$, $n$ a nonnegative integer, form a normal family
  in $|z| < 1$, also in $|z| > 1$, but not in any region that contains a point
  on the unit circle.
\end{problem}
\begin{proof} \text{} \\
  By Arzela's theorem, it is enough to show that for $\Omega_< = \{z : |z| < 1\}$
  and $\Omega_> = \{ z : |z| > 1 \}$,
  (i) $\mathfrak{F}$ is
  equicontinuous on every compact set $E \subset \Omega$, and (ii) for any
  $z \in \Omega$ and $f \in \mathfrak{F}$, $f(z)$ lies in some compact subset
  of $\mathbb{C}$; and that at least one of these hypothesis fails when $\Omega$
  contains a point $z$ such that $|z| = 1$.
  \\~\\
  \textbf{Equicontinuity of $\mathfrak{F}$ on $\Omega_<$.}\\
  Suppose that $E$ is a compact subset of $\Omega_<$.
  There exists some closed ball $\overline{B_r}(0)$ of radius $r < 1$ centered at
  zero around $E$. Then given some $\varepsilon > 0$, we can construct a
  $\delta$ such that $|z^n - z_0^n| < \epsilon$ whenever $|z - z_0| < \delta$.
  Notice that \begin{align*}
    |z^n - z_0^n| &= \left|(z - z_0)\sum_{k=0}^{n-1}z^k z_0^{n-1-k}\right| \\
    &\leq |z - z_0|\left|\sum_{k=0}^{n-1}r^{n-1}\right| \\
    &= |z - z_0|\cdot \underbrace{nr^{n-1}}_{\rightarrow 0}\\
    &\leq |z - z_0|\cdot \max_n \left(nr^{n-1}\right)
  \end{align*} where $\max_n \left(nr^{n-1}\right) < \infty$. Thus taking \[
    \delta < \frac{\varepsilon}{\displaystyle \max_n \left(nr^{n-1}\right)}
  \] is sufficient.
  \\~\\
  \textbf{Boundedness of $\mathfrak{F}$ on $\Omega_<$.}\\
  It is easy enough to see that $f(z) \in \overline{B_r}(0)$ for all
  $f \in \mathfrak{F}$: \[
    |f(z)| = |z^n| = |z|^n \leq r^n \leq r.
  \]\\
  \textbf{Theorem 17 on $\Omega_>$.}\\
  Theorem 17 says that $\mathfrak{F}$ is normal in the classical sense if and
  only if \[
    \rho(f_n) = \frac{2n|z|^{n-1}}{1 + |z|^{2n}}
  \] is locally bounded.
  Suppose that $z \in E$, a compact subset of $\Omega_>$, which is to say that
  there exists some $r$ such that $|z| > r$ for all $z \in E$. Then we have
  the bound \begin{align*}
    \rho(f_n) &= \frac{2n|z|^{n-1}}{1 + |z|^{2n}} \\
    &< \frac{2n|z|^{n-1}}{|z|^{2n}} \\
    &= \frac{2n}{|z|^{n+1}} \\
    &\leq \frac{2n}{r^{n+1}} \\
    &\leq \max_n\frac{2n}{r^{n+1}} < \infty
  \end{align*} which is bounded since $r > 1$.
  \\~\\
  \textbf{Theorem 17 on $|z| = 1$.}\\
  This follows from the above argument. When $|z| = 1$, we have \[
    \rho(f_n) = \frac{2n}{1 + 1} = n,
  \] which is unbounded. Thus by the ``only if'' of Theorem 17, $\mathfrak{F}$
  is not normal on on any compact set that intersects the boundary of the unit
  disk.
\end{proof}
\pagebreak

% -----------------------------------------------------
% Problem
% -----------------------------------------------------
% https://matthewhr.files.wordpress.com/2012/09/selected-solutions-to-ahlfors.pdf
\begin{problem}{3} (page 227) \\
  If $f(z)$ is analytic in the whole plane, show that the family $\mathfrak{F}$
  formed by all
  functions $f(kz)$ with constant $k \in \mathbb{R}$ is normal in the annulus $r_1 < |z| < r_2$
  if and only if $f$ is a polynomial.
\end{problem}
\begin{proof} \text{} \\
  Because $f$ is entire, we can write \[
    f(z) = \sum_{j=0}^\infty a_j z^j \text{ and }
    f(kz) = \sum_{j=0}^\infty a_j k^j z^j.
  \]
  $(\mathbf{\Longrightarrow})$\\
  % Assume that $\mathfrak{F}$ is a normal family.
  By contrapositive, assume that $f$ is \textit{not} a polynomial, which is to
  say that $f(z) = a_0 + a_1z + \hdots$ with infinitely many nonzero coefficients.
  By Theorem 17 it is enough to check that \[
    \rho(f_k)
      = \frac{2|f_k'(z)|}{1 + |f(kz)|^2}
      = \frac{
          \displaystyle2\left|\sum_{j=1}^\infty ja_jk^jz^{j-1}\right|
        }{
          \displaystyle1 + \left|\sum_{j=0}^\infty a_j(kz)^j\right|^2
        }
  \] is \textit{not} locally bounded, which is to say, it can be made
  arbitrarily large on some compact set.
  \\~\\
  $(\mathbf{\Longleftarrow})$\\
  Assume that $f$ is a polynomial, $f(z) = a_0 + a_1z + \hdots + a_nz^n$.
  By Theorem 17 it is enough to check that \[
  \rho(f_k)
    = \frac{2|f_k'(z)|}{1 + |f(kz)|^2}
  \] is locally bounded---which means that it is sufficient to check that
  $\rho(f)$ is \textit{totally} bounded.
  In particular, look at the function $g(z) = f(1/z)$ \[
    \rho(g)
    = \frac{
      \displaystyle2\left|\sum_{j=1}^n \frac{ja_j}{z^{j-1}}\right|
    }{
      \displaystyle1 + \left|\sum_{j=0}^n \frac{a_j}{z^j}\right|^2
    }
  \]
  Then multiplying the numerator and denominator by $|z|^{2n}$ yields \[
    \rho(g)
    = \frac{
      \displaystyle2\left|\sum_{j=1}^n ja_jz^{n-j+1}\right|
    }{
      \displaystyle |z|^{2n} + \left|\sum_{j=0}^n a_j z^{n-j}\right|^2
    }
  \] which is continuous in a delta ball around $0$. So $|g(z)| \leq M$ for
  $z < \delta$, and $|f(z)| \leq M$ for $z > 1/\delta$. Since $f$ is bounded
  for bounded $z$ (in particular, $|z| \leq 1/\delta$), $f$ is totally bounded,
  and so if $f$ is a polynomial, then $\mathfrak{F} = \{f(kz)\}_{k \in \mathbb{C}}$ is a normal
  family.

\end{proof}
\pagebreak

% -----------------------------------------------------
% Problem
% -----------------------------------------------------
% http://www-users.math.umn.edu/~phamx352/docs/Cpx2Hw2.pdf
% https://matthewhr.files.wordpress.com/2012/09/selected-solutions-to-ahlfors.pdf
\begin{problem}{1} (page 232) \\
  If $z_0$ is real and $\Omega$ is symmetric with respect to the real axis,
  prove that $f$ satisfies with the symmetry relation
  $f(\bar{z}) = \overline{f(z)}$ using the uniqueness condition in Theorem 1.
\end{problem}
\begin{proof} \text{} \\
  First notice that the map $g(z) = \overline{f(\bar{z})}$ is holomorphic:
  write $f(z) = u(x, y) + iv(x, y)$ so that $g(z) = u(x, -y) - iv(x, -y)$. Since
  $g$ is continuous (being the sum/composition of continuous functions), it only
  remains to check that the Cauchy-Riemann Equations are satisfied:
  \begin{align}
    \frac{\partial}{\partial x}\left[u(x, -y)\right] &= \frac{\partial u}{\partial x}(x, -y) \\
    \frac{\partial}{\partial y}\left[u(x, -y)\right] &= -\frac{\partial u}{\partial y}(x, -y)\\
    \frac{\partial}{\partial x}\left[-v(x, -y)\right] &= -\frac{\partial v}{\partial x}(x, -y)\\
    \frac{\partial}{\partial y}\left[-v(x, -y)\right] &= \frac{\partial v}{\partial y}(x, -y),
  \end{align} where the equality of (1) and (4) along with (2) and (3) follows by
  the Cauchy-Riemann Equations on $f$. Thus $g$ is holomorphic.
  \\~\\
  Now it just needs to be shown that $g$ is conformal. Notice that the above
  equations together with the knowledge that $f$ is conformal show that the
  derivative of $g$ never vanishes. Since $g$ is the composition of a
  bijection on $\Omega$, followed by $f$, followed by a bijection on
  $\mathbb{D}$, $g$ is also a one-to-one surjection onto the disk.
  \\~\\
  Next, notice that $g$ maps $z_0$ to zero:
    $g(z_0) = \overline{f(\bar{z_0})} = \overline{f(z_0)} = \overline{0} = 0$.
  Also the derivative at $g'(z_0)$ is positive because $z_0$ has imaginary part
  of zero: \[
    g'(z_0) = g'(x_0 + 0i) = \frac{\partial u}{\partial x}(x_0, 0) = f'(z_0) > 0
  \]
  Therefore Theorem 17 guarantees that $g$ is identically $f$, and the symmetry
  relation follows. \begin{align*}
    f(z) &= \overline{f(\bar{z})}\\
    \overline{f(z)} &= f(\bar{z}).
  \end{align*}
\end{proof}
\pagebreak

% -----------------------------------------------------
% Problem
% -----------------------------------------------------
\begin{problem}{2} (page 232) \\
  What is the corresponding conclusion if $\Omega$ is symmetric with respect to
  the point $z_0$?
\end{problem}
\begin{proof} \text{} \\
  Suppose $\Omega$ is symmetric with respect to the point $z_0$, that is
    (i) $f(z_0) = 0$ and
    (ii) if $z_1 \in \Omega$, then $z_0 - (z_1 - z_0) = 2z_0 - z_1 \in \Omega$.
  Denote this point by $\widetilde{z_1}$.
  \\~\\
  Define $g\colon \Omega \rightarrow \mathbb{D}$ by $g(z) = -f(\tilde{z})$.
  Notice now that \begin{enumerate}
    \item $g(z_0) = -f(\widetilde{z_0}) = -f(z_0 - (z_0 - z_0)) = -f(z_0) = -0 = 0$
    \item The map $g$ is conformal because it is the composition of conformal maps.
    \item The derivative $g'$ is positive at $z_0$ \[
      g'(z_0)
        = \frac{d}{dz}\left[-f(\hat{z})\right]_{z = z_0}
        = -f'(\hat{z_0})\frac{d}{dz}\left[2z_0 - z\right]_{z = z_0}
        = f'(\hat{z_0})
        = f'(z_0)
        > 0.
    \]
    Because $f$ was the unique function with these properties, $g(z) = f(z)$, so
    \[
      f(\widetilde{z_1}) = -f(z_1).
    \]
  \end{enumerate}
\end{proof}

\end{document}
