\documentclass{article}

\usepackage[margin=1in]{geometry}
\usepackage{amsmath,amsthm,amssymb}
\usepackage{bbm, enumerate, tikz}
\usepackage{multicol}

\newenvironment{problem}[2][Problem]{\begin{trivlist}
\item[\hskip \labelsep {\bfseries #1}\hskip \labelsep {\bfseries #2.}]}{\end{trivlist}}
\newenvironment{note}[1][Note.]{\begin{trivlist}
\item[\hskip \labelsep {\bfseries #1}]}{\end{trivlist}}

\begin{document}

\title{Complex Analysis: Homework 14}
\author{Peter Kagey}

\maketitle

% -----------------------------------------------------
% Problem
% -----------------------------------------------------
\begin{problem}{2} (page 227) \\
  Show that the functions $z^n$, $n$ a nonnegative integer, form a normal family
  in $|z| < 1$, also in $|z| > 1$, but not in any region that contains a point
  on the unit circle.
\end{problem}
\begin{proof} \text{} \\
\end{proof}
\pagebreak

% -----------------------------------------------------
% Problem
% -----------------------------------------------------
% https://matthewhr.files.wordpress.com/2012/09/selected-solutions-to-ahlfors.pdf
\begin{problem}{3} (page 227) \\
  If $f(z)$ is analytic in the whole plane, show that the family formed by all
  functions $f(kz)$ with constant $k$ is normal in the annulus $r_1 < |z| < r_2$
  if and only if $f$ is a polynomial.
\end{problem}
\begin{proof} \text{} \\
\end{proof}
\pagebreak

% -----------------------------------------------------
% Problem
% -----------------------------------------------------
% http://www-users.math.umn.edu/~phamx352/docs/Cpx2Hw2.pdf
% https://matthewhr.files.wordpress.com/2012/09/selected-solutions-to-ahlfors.pdf
\begin{problem}{1} (page 232) \\
  If $z_0$ is real and $\Omega$ is symmetric with respect to the real axis,
  prove by the uniqueness that $f$ satisfies with the symmetry relation
  $f(\bar{z}) = \overline{f(z)}$
\end{problem}
\begin{proof} \text{} \\
\end{proof}
\pagebreak

% -----------------------------------------------------
% Problem
% -----------------------------------------------------
\begin{problem}{2} (page 232) \\
  What is the corresponding conclusion if $\Omega$ is symmetric with respect to
  the point $z_0$?
\end{problem}
\begin{proof} \text{} \\
\end{proof}

\end{document}
