\documentclass{article}

\usepackage[margin=1in]{geometry}
\usepackage{amsmath,amsthm,amssymb}
\usepackage{bbm,enumerate,mathtools}

\newenvironment{problem}[2][Problem]{\begin{trivlist}
\item[\hskip \labelsep {\bfseries #1}\hskip \labelsep {\bfseries #2.}]}{\end{trivlist}}
\newenvironment{note}[1][Note.]{\begin{trivlist}
\item[\hskip \labelsep {\bfseries #1}]}{\end{trivlist}}

\begin{document}

\title{Differential Geometry: Homework 1}
\author{Peter Kagey}

\maketitle

% -----------------------------------------------------
% First problem
% -----------------------------------------------------
\begin{problem}{1}
  Show that the induced topology
  (for a subset $X \subset Y$ of a topological space $Y$)
  and the quotient topology (for a surjection $X \twoheadrightarrow Y$ from a
  topological space $X$ onto a set $Y$)
  satisfy the axioms of a topological space.
\end{problem}

\begin{proof} \text{} \\
  \textbf{Induced topology.}
  Let $\mathcal{T}_X = \{i^{-1}(U_y) | U_y \in \mathcal{T}_Y\}$.
  \begin{itemize}
    \item $\emptyset, X \in \mathcal{T_X}$:
    $i^{-1}(Y) = X$ and $i^{-1}(\emptyset) = \emptyset$.
    ($Y, \emptyset \in \mathcal{T}_Y$.)
    \item Closed under arbitrary union:\\
    Let $\{U_{X,j}\}_{j \in I}$ be a collection of open sets in $\mathcal{T}_X$.
    Then \[
      \bigcup\limits_{j\in I} U_{X, j}
      = \bigcup\limits_{j\in I} i^{-1}(U_{Y,j})
      = i^{-1}(\bigcup_{j\in I} U_{Y, j})
    \] where $\bigcup_{j\in I} U_{Y, j} \in \mathcal{T}_Y$
    because $Y$ is a topological space.
    \item Closed under finite intersection.\\
    Similarly \[
      \bigcap_{j = 1}^N U_{X, j}
      = \bigcap_{j = 1}^N i^{-1}(U_{Y,j})
      = i^{-1}(\bigcap_{j = 1}^N U_{Y, j})
    \] where $\bigcap_{j = 1}^N U_{Y, j} \in \mathcal{T}_Y$
    because $Y$ is a topological space.
  \end{itemize}

  \textbf{Quotient topology.}
  Let $\mathcal{T}_Y = \{V \subset Y\ |\ p^{-1}(V) \in \mathcal{T}_X\}$. \\
  \begin{itemize}
    \item $\emptyset, Y \in \mathcal{T_Y}$:\\
    Because $p$ is surjective, $p^{-1}(Y) = X \in \mathcal{T}_X$. Also
    $p^{-1}(\emptyset) = \emptyset \in \mathcal{T}_X$.

    \item Closed under arbitrary union:\\
    Let $\{U_{Y,j}\}_{j \in I}$ be a collection of open sets in $\mathcal{T}_Y$.
    Then $p^{-1}(U_{Y, j}) \in \mathcal{T}_X$ for all $j \in I$, so \[
      \bigcup\limits_{j\in I} p^{-1}(U_{Y,j})
      = p^{-1}(\bigcup\limits_{j\in I} U_{Y,j})
      \in \mathcal{T}_X
    \] because $\mathcal{T}_X$ is closed under union.
    Therefore $\bigcup_{j\in I} U_{Y, j} \in \mathcal{T}_Y$.
    \item Closed under finite intersection:\\
    Similarly \[
      \bigcap_{j = 1}^N p^{-1}(U_{Y, j})
      = p^{-1}\left(\bigcap_{j = 1}^N U_{Y, j}\right)
      \in \mathcal{T}_X
    \] because $\mathcal{T}_X$ is closed under finite intersection.
    Therefore $\bigcap_{j = 1}^N U_{Y, j} \in \mathcal{T}_Y$.
  \end{itemize}
\end{proof}

% -----------------------------------------------------
% Second problem
% -----------------------------------------------------
\pagebreak

\begin{problem}{2}
  Show that the topological spaces $S^1 \subset \mathbb{R}^2$
  (with topology induced by the inclusion into $\mathbb{R}^2$)
  and $[0, 1]/\{0, 1\}$
  (with the quotient topology from the topology on $[0, 1] \subset \mathbb{R}$)
  are homeomorphic.
\end{problem}

\begin{proof}
  Let $f: [0, 1] \rightarrow S^1$ be the map $t \xmapsto{f} (\cos(2\pi t), \sin(2\pi t))$.
  Because $\sin$ and $\cos$ are continuous functions
  with respect to the standard topology on $\mathbb{R}^2$,
  $f$ is continuous, so it is sufficient to show that $f$
  has a two-sided inverse.

  Let \[
    f^{-1}(x, y) = \begin{cases}
      0 \sim 1 & (x, y) = (0, 0) \\
      1/2 & (x, y) = (-1, 0) \\
      \frac{1}{2\pi}\cos^{-1}(x) & y > 0 \\
      1 - \frac{1}{2\pi}\cos^{-1}(x) & y < 0
    \end{cases}.
  \] The continuity of $f^{-1}$ follows from the continuity of $\cos^{-1}$
  and the equality of $f^{-1}$ at the ``handoff'' points.\\
  Then \[
    f(f^{-1}(x, y)) = \begin{cases}
      (\sin(0), \cos(0)) = (\sin(2\pi), \cos(2\pi)) = (0, 0) & (x, y) = (0, 0) \\
      (\cos(\pi), \sin(\pi)) = (-1, 0) & (x, y) = (-1, 0) \\
      (\cos(\cos^-1(x)), \sin(\cos^-1(x))) = (x, \sqrt{1 - x^2}) = (x, y) & y > 0 \\
      (\cos(-\cos^{-1}(x)), \sin(-\cos^{-1}(x))) = (x, y) & y < 0
    \end{cases}.
  \] Similarly \[
    f^{-1}(f(t)) = \begin{cases}
      f^{-1}(0, 0) = (0 \sim 1)              & t =  (0 \sim 1) \\
      f^{-1}(-1, 0) = 1/2                    & t =  1/2 \\
      f^{-1}(\cos(2\pi t), \sin(2\pi t)) = t & t\in (0, 1/2) \\
      f^{-1}(\cos(2\pi t), \sin(2\pi t)) = t & t\in (1/2, 1)
    \end{cases}.
  \]

  Thus $f$ is a homeomorphism, so
   the two spaces are homeomorphic.
\end{proof}

% -----------------------------------------------------
% Third problem
% -----------------------------------------------------
\pagebreak

\begin{problem}{3}
  Prove that $S^1$, with either topology considered above, is a topological
  manifold.
\end{problem}

\begin{proof} Consider the quotient topology $[0, 1]$
  \begin{enumerate}[(i)]
    \item (Hausdorff)\\
    Let $a, b$ be distinct points in $[0, 1]/(0 \sim 1)$.\\
    \textbf{Case 1:} $a, b \not= (0 \sim 1)$.
    Then take $r = |a - b|/2$. \[
      B_r(a) \cap B_r(b) = \emptyset.
    \]
    \textbf{Case 2:} Without loss of generality, assume $a = (0 \sim 1)$.
    Let $r = \min(b, 1 - b)/2$. Then \[
      B_r(a) \cap B_r(b) = \emptyset.
    \]
    \item (Second countable)\\
    A countable basis for $[0, 1]/(0 \sim 1)$ is: \[
      \{ B_r(x) \cap (0, 1)\ |\ r \in \mathbb{Q}, x \in \mathbb{Q}\} \cup \{ [0, r) \cup (1-r, 1]\ |\ r \in \mathbb{Q}\}.
    \]

    \item (Locally Euclidean) \\
      Assume that $x \not= (0 \sim 1)$. Then the ball with radius
      $r = \min(x, 1 - x)$ maps in the obvious way to
      $B_r(x) \subset (0, 1) \subset \mathbb{R}$.

      Assume that $x = (0 \sim 1)$.
      Then the open set $[0, 1/4) \cup (3/4, 1]$
      is homeomorphic to $(0, 1/2)$ via \[
        f(x) = \begin{cases}
          x - 3/4 & x \in (3/4, 1) \\
          1/4     & x = (0 \sim 1) \\
          x + 1/4 & x \in (0, 1/4)
        \end{cases}.
      \] Where $f$ is continuous and has two sided inverse \[
        f^{-1}(p) = \begin{cases}
          p + 3/4 & p \in (0, 1/4) \\
          (0 \sim 1) & p = 1/4 \\
          p - 1/4 & p \in (1/4, 1/2) \\
        \end{cases}.
      \]
  \end{enumerate}
\end{proof}

% -----------------------------------------------------
% Fourth problem
% -----------------------------------------------------
\pagebreak

\begin{problem}{4}
  Show that the derivative of a function
  $f: \mathbb{R}^n \rightarrow \mathbb{R}^m$
  if it exists at a point $a \in \mathbb{R}^m$ is unique.
\end{problem}

\begin{proof}
  I will prove this starting only from the definition that if the derivative of
  a function exists at a point $a \in \mathbb{R}^n$ then the derivative is a
  linear map $L: \mathbb{R}^n \rightarrow \mathbb{R}^m$ satisfying \[
    \lim_{\vec{h} \rightarrow 0}
      \frac{|f(a + \vec{h}) - f(a) - L(\vec{h})|}{|\vec{h}|} = 0.
  \]

  Suppose for the sake of contradiction that there is another linear map
  $L_2: \mathbb{R}^n \rightarrow \mathbb{R}^m$ that satisfies the above property.
  By the triangle inequality \[
    0 = \lim_{\vec{h} \rightarrow 0}
      \frac{|f(a + \vec{h}) - f(a) - L(\vec{h})|}{|\vec{h}|} +
      \frac{|-f(a + \vec{h}) + f(a) + L_2(\vec{h})|}{|\vec{h}|}
      \geq \lim_{\vec{h} \rightarrow 0} \frac{|L_2(\vec{h}) - L(\vec{h})|}{|\vec{h}|}
      \geq 0.
  \] Thus \[
    \lim_{\vec{h} \rightarrow 0} \frac{|L_2(\vec{h}) - L(\vec{h})|}{|\vec{h}|} = 0.
  \]
  Notice that the limit does not depend on $\vec{h}$ being small.
  Let $\vec{h} = \varepsilon\vec{v}$. By linearity \[
    \frac{|L_2(\vec{h}) - L(\vec{h})|}{|\vec{h}|}
    = \frac{|\varepsilon L_2(\vec{v}) - \varepsilon L(\vec{v})|}{|\varepsilon\vec{v}|}
    = \frac{| L_2(\vec{v}) -  L(\vec{v})|}{|\vec{v}|}.
  \] Thus if the the ratio vanishes in the limit, it must vanish everywhere.
  So $L_2 = L$ for all $\vec{v}$, and the derivative is unique.
\end{proof}

% -----------------------------------------------------
% Fifth
% -----------------------------------------------------
\pagebreak

\begin{problem}{5}
  Produce, with proofs, examples of the following topological spaces which are
  not topological manifolds: \begin{enumerate}[a)]
    \item A space $X$ which is locally Euclidean and second countable,
      but not Hausdorff.
    \item A space $X$ which is Hausdorff and second countable, but not locally
      Euclidean.
  \end{enumerate}
\end{problem}

\begin{proof} \text{} \\
  \begin{enumerate}[a)]
    \item Let $X = \mathbb{R} \cup \{\star\}$,
    and let the topology $\mathcal{T}$ on $X$ be defined by \[
      \mathcal{T}
        = \mathcal{T}_{\mathbb{R},\text{std}} \cup
        \{ U \setminus \{0\} \cup \{ \star \}\ |
          \ U \in \mathcal{T}_{\mathbb{R},\text{std}} \text{ and } 0\in U
        \}.
    \]
    \textbf{Locally Euclidean:} Suppose that $x \not= \star$.
    Then the open ball of radius $r > 0$ centered at $x$ in $X$ is homeomorphic to
    $(-r, r) \in \mathcal{T}_{\mathbb{R},\text{std}}$ in the obvious way.\\
    Suppose then that $x = \star$,
    then the open ball of radius $1$ centered at $\star$ is homeomorphic to $(-1, 1)$ via \[
      f(x) = \begin{cases}
        0 & x = \star \\
        x & \text{otherwise}
      \end{cases} \text{ with inverse }
      f^{-1}(x) = \begin{cases}
        \star & x = 0 \\
        x & \text{otherwise}
      \end{cases}.
    \]
    \textbf{Second countable:} A countable basis for $\mathcal{T}$ is \[
      \{B_r(x)\ |\ r \in \mathbb{Q}, x \in \mathbb{Q} \cup \{\star\}\}.
    \]
    \textbf{Not Hausdorff:} For every open set which contains $0$ or $\star$
    contains points near $0$, so $0$ and $\star$
    do not have distinct neighborhoods. \\
    Let $U \in \mathcal{T}$ be a set that contains $0$.
    Then there exists some $\varepsilon_0 > 0$
    such that $B_{\varepsilon_0}(0) \in U$.
    At the same time, let $V \in \mathcal{T}$ be a set that contains $\star$.
    By construction, there exists some $\varepsilon_1 > 0$
    such that $B_{\varepsilon_1}(0) \setminus \{ 0\} \in V$.

    Thus for all such open sets $U$ and $V$, the point
    $(\varepsilon_0 + \varepsilon_1)/2 \in U \cap V$.
    So $0$ and $\star$ have no disjoint neighborhoods.

    \item
      Let $X = [0, 1]$
      with the induced topology from $\mathbb{R}$ (with the standard topology).\\
      \textbf{Hausdorff:} This is inherited from the standard topology on $\mathbb{R}$.\\
      \textbf{Second countable:} This is also inherited from the standard topology on $\mathbb{R}$.\\
      \textbf{Not locally Euclidean:}
        X is compact, which is not true of any open subset of $\mathbb{R}$.
        Thus $X$ cannot be homeomorphic to any open subset of $\mathbb{R}$,
        because this property is preserved under continuous maps.
  \end{enumerate}
\end{proof}

% -----------------------------------------------------
% Sixth
% -----------------------------------------------------
\pagebreak

\begin{problem}{6}
  Let $h$ be a continuous real-valued function on
  $S^1 = \{x^2 + y^2 = 1 \subset \mathbb{R}^2\}$ satisfying
  $h(0, 1) = h(1, 0) = 0$ and $h(-x_1, -x_2) = -h(x_1, x_2)$.
  Define a function $f: \mathbb{R}^2 \rightarrow \mathbb{R}$ by \[
    f(x) = \begin{cases}
      ||x||h(x/||x||) & x \not= 0 \\
      0 & x = 0
    \end{cases}
  \]
  \begin{enumerate}[a)]
    \item Show that $f$ is continuous at $(0, 0)$,
      that the partial derivatives of $f$ at $(0, 0)$ are defined,
      and that more generally all directional derivatives of $f$ are defined.

    \item Show that $f$ is not differentiable at $(0, 0)$
      except if $h$ is identically zero.
  \end{enumerate}
\end{problem}

\begin{proof} \text{\\}
  \begin{enumerate}[a)]
    \item $S^1$ is a compact subspace of $R^2$, so the continuity of $h$ implies
    that the image of $S^1$ under $h$ is bounded by some $N$.
    Let $\varepsilon > 0$. Then for all $x$ with $||x|| < \varepsilon/N$, \[
      ||f(x)|| < ||x||N < \varepsilon.
    \] So $f$ is continuous at $0$. \\ \\
    The existence of partial derivatives follows from the existence of
    directional derivatives, so it is sufficient to show that all directional
    derivatives are defined. \[
      df(0)(\vec{a})
        = \lim_{k \rightarrow 0} \frac{f(0 + k\vec{a}) + f(0)}{k}
        = \lim_{k \rightarrow 0} \frac{f(k\vec{a})}{k}
        = \lim_{k \rightarrow 0} \frac{
          ||k\vec{a}|| h\left(
            \frac{k\vec{a}}{||k\vec{a}||}
          \right)
        }{k}
        = ||\vec{a}|| h\left(\frac{\vec{a}}{||\vec{a}||}\right).
    \]
    \item
    If $f$ is differentiable at a point, all derivatives agree with the linear map.
    Because $df(0)(\vec{e}_1) = df(0)(\vec{e}_2) = 0$, this means that the
    directional derivative in the direction
    $\vec{a} = a_1 \vec{e}_1 + a_2 \vec{e}_2$ must vanish:
    \[
      df(0)(\vec{a}) = a_1 df(0)(\vec{e}_1) + a_2 df(0)(\vec{e}_2) = 0a_1 + 0a_2 = 0.
    \]
    This implies that if $f$ is differentiable at $(0, 0)$, then for all $\vec{a}$, \[
      df(0)(\vec{a}) = ||\vec{a}|| h\left(\frac{\vec{a}}{||\vec{a}||}\right) = 0.
    \]
    Thus $h$ (and consequently $f$) must be identically $0$.
  \end{enumerate}
\end{proof}

\end{document}
