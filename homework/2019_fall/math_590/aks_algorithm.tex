\documentclass{article}

\usepackage{amsmath, amssymb}
\usepackage{booktabs, tabularx} % For text alignment on cover page
\usepackage{epsfig}             % For \includegraphics command

\newcolumntype{C}{>{\centering\arraybackslash}X}


\begin{document}

\title{AKS Algorithm}
\author{Peter Kagey}
\date{Friday, September 27, 2019}

\maketitle

% -----------------------------------------------------
% First problem
% -----------------------------------------------------
The idea for this document is to write a high level overview of the AKS
algorithm, which provides a convincing \textit{heuristic} for why PRIME is in P
in the number of bits of $n$.
\section{Primes and binomial coefficients}
The idea uses the fact that when $n \geq 2$ and $\gcd(a, n) = 1$, then $n$ is
prime if and only if \[
  (x+a)^n \equiv x^n + a \pmod n
\] since all binomial coefficients are divisible by $n$ if and only if $n$ is
prime. However $(x+a)^n$ has a linear number of terms, so it takes exponential
time (with respect to $\log(n)$) to compute this power.

An improvement is to instead calculate these polynomials over
$\mathbb Z[x]/(x^r - 1)$ for some $r$ which is polynomial in $\log(n)$, because
this computation is quick in general.

\subsection{Power of a binomial over a quotient ring.}
For example, if we want to compute \[
  (x + 6)^{13} \pmod{x^3-1, 13}
\] we instead compute \begin{align*}
  (x + 6) &\equiv (x + 6) \pmod{x^3-1, 13} \\
  (x + 6)^2 &\equiv x^2 + 12x + 10 \pmod{x^3-1, 13} \\
  (x + 6)^4 &\equiv (x^2 + 12x + 10)^2 \equiv 8 x^2 + 7 x + 7 \pmod{x^3-1, 13} \\
  (x + 6)^8 &\equiv (8 x^2 + 7 x + 7)^2 \equiv 5 x^2 + 6 x + 5 \pmod{x^3-1, 13}
\end{align*}
Since $13 = 1 + 4 + 8$, we can write \begin{align*}
  (x + 6)^5 &\equiv (x + 6)(8 x^2 + 7 x + 7) \equiv 3 x^2 + 10 x + 11 \pmod{x^3-1, 13} \\
  (x + 6)^{13} &\equiv (5 x^2 + 6 x + 5)(3 x^2 + 10 x + 11) \equiv x + 6 \equiv x^{13} + 6 \pmod{x^3-1, 13},
  \end{align*} as expected.
And the number of steps this requires is polynomial in $r$, which is itself
polynomial in $\log(n)$.
\\
Simple enough. But testing just one value of $a$ and $r$ can result in false
positives. In order to get rid of false positives, it's necessary to choose a
list of $\set{a_1, a_2, \hdots, a_m}$. Moreover, if this algorithm is to be
polynomial time, where $m$ must be polynomial with respect to $\log(n)$.

\subsection{Choosing $r$.}
Let given some $r$ and $a$, with $\gcd(a, r) = 1$, define \[
  o_r(n) = \min\set{k \in \mathbb N : n^k \equiv 1 \pmod r},
\] the order of $a$ modulo $r$.
\\
By Lemma 4.3, for large $n$ we can find an $r \leq \log^5(n)$
(that is, polynomial in $\log(n)$) such that $o_r(n) > \log^2 n$.

\subsection{Two things to check.}
Now, for $n > 5690034$, we just need to \begin{enumerate}[(i)]
  \item trial divide a set of primes that is polynomial in $\log(n)$, namely check that $a \nmid n$ for all $1 < a \leq r$, and
  \item check that $(x + a)^{n} \equiv x^n + a \pmod{x^r-1, n}$ for all
    $a \leq \lfloor\log(n)\sqrt{\phi(r)}\rfloor$.
\end{enumerate}
Of course,
if (i) fails, then $n$ has a nontrivial divisor, so $n$ is composite.
If (i) fails then $n$ is composite by the properties of binomial coefficients
discussed above.
If (ii) succeeds, it is less obvious that $n$ is prime.

\end{document}
