\documentclass{article}

\usepackage{amsmath, amssymb}
\usepackage{booktabs, tabularx} % For text alignment on cover page
\usepackage{epsfig}             % For \includegraphics command

\newcolumntype{C}{>{\centering\arraybackslash}X}


\begin{document}

\title{Math 574: Homework 1}
\author{Peter Kagey}
\date{Monday, September 23, 2019}

\maketitle

% -----------------------------------------------------
% First problem
% -----------------------------------------------------
\begin{problem}{1} Let $V$ be a vector space over $F$ of dimension $n$, with
  $V_1$ and $V_2$ subspaces of $V$.
  \begin{enumerate}[(a)]
    \item Prove that $V_1 + V_2$ is a subspace.
    \item Prove that $\dim(V_1 + V_2) = \dim V_1 + \dim V_2 - \dim(V_1 \cap V_2)$.
  \end{enumerate}
\end{problem}

\begin{proof} ~
  \begin{enumerate}[(a)]
    \item It is enough to show that $V_1 + V_2$ is closed under vector addition
    and scalar multiplication, because the other vector space properties are
    inherited from $V$. Closure with respect to addition follows from the
    commutativity of addition: \[
      \underbrace{(v_1 + v_2)}_{V_1 + V_2} +
      \underbrace{(v'_1 + v'_2)}_{V_1 + V_2} =
      \underbrace{(v_1 + v'_1)}_{V_1}
      \underbrace{(v_2 + v'_2)}_{V_2},
    \] and closure with respect to scalar multiplication follows from
    distributivity: \[
      \underbrace{a(v_1 + v_2)}_{V_1 + V_2} =
      \underbrace{av_1}_{V_1}
      +
      \underbrace{av_2}_{V_2}.
    \]

    \item % Not very clear.
    The dimension of $V_1 + V_2$ is the cardinality of its basis.
    If $V_1$ has basis $\set{u_i}$, $V_2$ has basis $\set{w_i}$, and
    $V_1 \cap V_2$ has basis $\set{v_i}$ then each $v_i$ can be written as both
    $v_i = \sum_k a_ku_k$ and
    $v_i = \sum_k b_kw_k$, where at least one $a_k$ and at least one $b_k$ are nonzero.
    Then solving for one such $b_k$ in terms of the other $a_i$s and $b_i$s
    allows for $b_k$ to be removed from the basis. This can be done inductively
    with each element the basis $\set{v_i}$ of $V_1 \cap V_2$ removing one
    element of the spanning set, ultimately resulting in a set of linearly
    independent vectors, so \[
      \dim(V_1 + V_2) = \dim V_1 + \dim V_2 - \dim(V_1 \cap V_2)
    \] as desired.
  \end{enumerate}
\end{proof}
% -----------------------------------------------------
% Second problem
% -----------------------------------------------------
\begin{problem}{2} ~
  \begin{enumerate}[(a)]
    \item Prove that $V = V_1 \oplus V_2$ if and only if $n = n_1 + n_2$ and $V_1 \cap V_2 = 0$.
    \item Prove that $V = V_1 \oplus V_2$ if and only if each $v \in V$ can be
    written uniquely as $v_1 + v_2$.
  \end{enumerate}
\end{problem}

\begin{proof} ~
  \begin{enumerate}[(a)]
    \item ~\\ $(\Longrightarrow)$ Assume $V = V_1 \oplus V_2$. Then by definition,
    $V = V_1 + V_2$ and $V_1 \cap V_2 = 0$. So by Problem 1 (b), \[
      \underbrace{\dim(V)}_{n} =
      \underbrace{\dim(V_1)}_{n_1} +
      \underbrace{\dim(V_2)}_{n_2} +
      \underbrace{\dim(V_1 \cap V_2)}_{0}.
    \]
    $(\Longleftarrow)$ Assume $n = n_1 + n_2$ and $V_1 \cap V_2 = 0$.
    Then any basis of $V_1$ and any basis of $V_2$ are disjoint, call them
    $\set{u_i}$ and $\set{w_i}$ respectively, and their (disjoint) union forms
    a basis for $V$: $\set{u_i} \cup \set{w_i}$. Since any vector $v \in V$ can
    be written in terms of its basis elements, \[
      v = \underbrace{\sum_i a_iu_i}_{V_1} + \underbrace{\sum_i b_iw_i}_{V_2}.
    \]
    \item ~\\
    $(\Longrightarrow)$ Assume $V = V_1 \oplus V_2$. Then by definition,
    $V = V_1 + V_2$ so each $v$ can be written as $v_1 + v_2$. Moreover, this
    sum is unique because if $v = v_1' + v_2'$, since $V_1 \cap V_2 = 0$,
    $v_1' - v_1 = 0$ and $v_2' - v_2 = 0$.
    \\~\\
    $(\Longleftarrow)$ Assume that each $v \in V$ can be uniquely written as
    $v_1 + v_2$, then $V \subseteq V_1 + V_2$ and $V_1 + V_2 \subseteq V$
    (since they are subspaces). Next, arguing by contrapositive, suppose that
    there exists a nonzero $v \in V_1 \cap V_2$, then \[
      \underbrace{(v_1 - u)}_{V_1} + \underbrace{(v_2 + u)}_{V_2} = v,
    \] so the choice of vectors are not unique.
  \end{enumerate}
\end{proof}
% -----------------------------------------------------
% Third problem
% -----------------------------------------------------
\begin{problem}{3} %
  Let $\fn T V V$ be a linear transformation with $\dim V = n$.
  \begin{enumerate}[(a)]
    \item Prove that $\rank(T) = \rank(T^2)$ if and only if $V = T(V) \oplus \ker(T)$.
    \item
  \end{enumerate}
\end{problem}

\begin{proof}
  The rank of $T$ is $\dim T(V)$. % In any case, $\ker(T) \cap T(V) = 0$
  \begin{enumerate}[(a)]
    \item ~\\ % Questionable!
      $(\Longrightarrow)$ Assume $\rank(T) = \rank(T^2)$.
      Let $\set{u_i}$ be a basis for $T(V)$. By the hypothesis, $\set{T(u_i)}$
      has the same dimension and is also a basis for $T(V)$. Since all basis
      elements of $\ker(T)$ are sent to $0$ under $T$,
      $\ker(T) \cap T(V) = 0$ and moreover, every vector can be written as
      $v_1 + v_2$ with $v_1$ in the kernel and $v_2$ in the image.
      \\~\\
      $(\Longleftarrow)$ Assume $V = T(V) \oplus \ker(T)$. Let $\set{u_i}$
      be a basis for the image of $T$ and $\set{w_i}$ be a basis for the kernel
      of $T$, so that for each $v \in V$ \[
        v = \sum_{i=1}^{m} a_iu_i + \sum_{i=1}^{n-m} b_iw_i
      \] then \[
        T(v) = T\bigl(\sum_{i=1}^{m} a_iu_i + \sum_{i=1}^{n-m} b_iw_i\bigr) = \sum a_iT(u_i) + \sum b_i\underbrace{T(w_i)}_0.
      \] Since $T(V) \cap \ker(T) = 0$, $\sum a_iu_i$ is in $\ker(T)$ if and
      only if $a_1 = \hdots = a_m = 0$. Therefore $\set{T(u_i)}$ forms a basis for
      $T(V)$ and \[
        \rank(T) = \dim\set{u_1, \hdots, u_m} = \dim\set{T(u_1), \hdots, T(u_m)} = \rank(T^2).
      \]
    \item If $T^2 = T$, then clearly $\rank(T) = \rank(T^2)$, so
    $V = \ker(T) \oplus T(V)$. Furthermore, every element in the image is fixed
    under a successive transformation, so $T(V) = \set{v \in V \mid T(v) = v}$.
  \end{enumerate}
\end{proof}

% -----------------------------------------------------
% Fourth problem
% -----------------------------------------------------
\begin{problem}{4} %
  Let $V$ be the space of complex valued $C^\infty$ functions on $\R$, and let
  $\fn D V V$ be differentiation. \begin{enumerate}[(a)]
    \item Find all eigenvalues and eigenvectors for $D$.
    \item Find $\ker(D^n)$.
  \end{enumerate}
\end{problem}

\begin{proof} ~
  \begin{enumerate}[(a)]
    \item This is equivalent to solving the differential equation
    $D(f(x)) = af(x)$. By the existence and uniqueness of ODEs, the only
    solution to this equation is $b\exp(ax)$, and $D(b\exp(ax)) = ab\exp(ax)$,
    which has eigenvalue $a$.
    \item Firstly, note that the only functions $f$ such that $D f = 0$ are the
    constant functions, $f(x) = c_1$. By induction with the hypothesis that
    $\ker(D^n)$ consists of polynomials of degree $n-1$. Certainly $\ker(D^1)$
    consists of $0$-degree polynomials, so assuming that
    $\ker(D^{n-1}) = \set{a_{n}x^{n-1} + a_{n-1}x^{n-2} + \hdots a_{2}x + a_1}$,
    and integrating,
    \begin{align*}
      \ker(D^n) &= \set{\frac{a_{n}}{n}x^{n} + \frac{a_{n-1}}{n-1}x^{n-1} + \hdots \frac{a_{2}}{2}x^2 + a_1x + a_0 \,\middle|\, a_i \in \C} \\
      &= \set{b_{n}x^{n} + b_{n-1}x^{n-1} + \hdots b_{2}x^2 + b_1x + b_0 \,\middle|\, b_i \in \C}
    \end{align*} gives the set of polynomials of degree $n$.
  \end{enumerate}
\end{proof}
% -----------------------------------------------------
% Fifth problem
% -----------------------------------------------------
\begin{problem}{5} % p 613

\end{problem}

\begin{proof} ~
  \begin{enumerate}[(a)]
    \item ~\\
    $(\Longrightarrow)$ Arguing by contrapositive, assume that $A_1$ has a nonzero
    kernel (that is, $A_1$ is not invertible), and let $u \in \ker(A_1)$ be a
    nonzero vector.
    Then \[
      (A_1 \otimes A_2)(u \otimes v) = (\underbrace{A_1(u)}_0 \otimes A_2(v)) = 0,
    \] so $A_1 \otimes A_2$ has a nonzero kernel and therefore is not invertible.
    \\
    $(\Longleftarrow)$ Assume $A_1$ and $A_2$ are invertible. Then \begin{align*}
      (\inv{A_1} \otimes \inv{A_2})(A_1 \otimes A_2)(u \otimes v)
      &= (\inv{A_1} \otimes \inv{A_2})(A_1(u) \otimes A_2(v))\\
      &= (\inv{A_1}(A_1(u)) \otimes \inv{A_2}(A_2(v))) \\
      &= (u,v),
    \end{align*}
    so $\inv{(A_1 \otimes A_2)} = (\inv{A_1} \otimes \inv{A_2})$.
    \item This follows almost directly by looking at the Kronecker product.
    Suppose that $a_{ii}$ gives the $i$th entry on the main diagonal of $A_1$.
    Then reading down the main diagonal of the Knocker product gives \begin{align*}
      \trace(A_1 \otimes A_2)
      &= a_{11}\trace(A_2) + a_{22}\trace(A_2) + \hdots + a_{nn}\trace(A_2) \\
      &= (a_{11} + a_{22} + \hdots + a_{nn})\trace(A_2) \\
      &= \trace(A_1)\trace(A_2),
    \end{align*}
    as desired.
  \end{enumerate}
\end{proof}
\end{document}
