\documentclass{article}

\usepackage{amsmath, amssymb}
\usepackage{booktabs, tabularx} % For text alignment on cover page
\usepackage{epsfig}             % For \includegraphics command

\newcolumntype{C}{>{\centering\arraybackslash}X}


\begin{document}

\title{Math 574: Homework 2}
\author{Peter Kagey}
\date{Monday, October 14, 2019}

\maketitle

% -----------------------------------------------------
% First problem
% -----------------------------------------------------
\begin{problem}{1} ~
  \begin{enumerate}[(a)]
    \item Show that $A, B \in M_3(K)$ are similar if and only if they have the
    same minimal and characteristic polynomials.
  \end{enumerate}
\end{problem}

\begin{proof} ~
  $A$ and $B$ are similar if and only if they have the same Jordan
  normal form, so it is sufficient to compare the minimal and characteristic
  polynomials of matrices in Jordan normal form.
  \begin{enumerate}[(a)]
    \item A $3 \times 3$ matrix can have one of three Jordan normal forms \[
    A_1 = \begin{bmatrix}
      \lambda_1 & 0 & 0 \\
      0 & \lambda_2 & 0 \\
      0 & 0 & \lambda_3
    \end{bmatrix},\
    A_2 = \begin{bmatrix}
      \lambda_1 & 1 & 0 \\
      0 & \lambda_1 & 0 \\
      0 & 0 & \lambda_2
    \end{bmatrix},\ \text{ or }
    A_3 = \begin{bmatrix}
      \lambda_1 & 1 & 0 \\
      0 & \lambda_1 & 1 \\
      0 & 0 & \lambda_1
    \end{bmatrix}.
    \]
    Notice that a matrix $A$ \begin{enumerate}[(i)]
      \item similar to $A_1$ if and only if $(\lambda_1 - x)$ divides $m_{A}(x)$ exactly once,
      \item similar to $A_2$ if and only if $(\lambda_1 - x)$ divides $m_{A}(x)$ exactly twice, and
      \item similar to $A_3$ if and only if $(\lambda_1 - x)$ divides $m_{A}(x)$ exactly three times.
    \end{enumerate}
    \item Let \[
      A = \begin{bmatrix}
        1 & 1 & 0 & 0 \\
        0 & 1 & 0 & 0 \\
        0 & 0 & 1 & 1 \\
        0 & 0 & 0 & 1
      \end{bmatrix}
      \hspace{1cm}\text{and}\hspace{1cm}
      B = \begin{bmatrix}
        1 & 0 & 0 & 0 \\
        0 & 1 & 0 & 0 \\
        0 & 0 & 1 & 1 \\
        0 & 0 & 0 & 1
      \end{bmatrix}.
    \]
    Both $A$ and $B$ have the same minimal and characteristic polynomials
    \begin{alignat*}{2}
      m_A(x) &= m_B(x) &= (1-x)^2 \\
      p_A(x) &= p_B(x) &= (1-x)^4,
    \end{alignat*}
    but $A$ and $B$ are not similar because they have different Jordan
    canonical forms.
  \end{enumerate}
\end{proof}
% -----------------------------------------------------
% Second problem
% -----------------------------------------------------
\begin{problem}{2} ~
  Fix $A \in M_n(K)$ and let $C(A) = \set{B : BA = AB}$.
\end{problem}

\begin{proof} ~
  \begin{enumerate}[(a)]
    \item Suppose $A$ is cyclic, that is $p_A(x) = m_A(x)$, and moreover \[
      A^n = c_0I + c_1A + \hdots + c_{n-1}A^{n-1}
    \] and the Jordan normal form of $A$ is a single Jordan block.
    Now look at what commutes \[
      \begin{bmatrix}
        \lambda & 1 & \ & \ & \ \\
         \ & \lambda & 1 & \ & \ \\
         \ &  \ & \lambda & 1 & \ \\
        \lambda & 1 & \ & \ & \ \\
        \lambda & 1 & \ & \ & \
      \end{bmatrix}
    \]
    % Notice that $B$ commutes with $A$ if and only if $B$ commutes with $J_A$: \[
    %   BAU = BUU^{-1}AU = ABU
    % \]
    \item First consider $A$ in Jordan canonical form.
  \end{enumerate}
\end{proof}
% -----------------------------------------------------
% Third problem
% -----------------------------------------------------
\begin{problem}{3} ~
  \begin{enumerate}[(a)]
    \item Show that if $N$ is nilpotent then $N^n = 0$.
    \item
    \item How many similarity classes of $5 \times 5$ nilpotent matrices are there?
  \end{enumerate}
\end{problem}

\begin{proof} ~
  \begin{enumerate}[(a)]
    \item If $N^d = 0$, then there exists some $d' \leq n$ such that
    $N^{d'} = 0$ because the minimal polynomial $m_N(x) | x^d$, so the minimal
    polynomial is of the form $m_N(x) = x^{d'}$ with $d' <= n$, since the
    minimal polynomial has degree less than or equal to $n$.
    \item
    \item By (a), the characteristic polynomial of a $5 \times 5$ nilpotent
    matrix is $p(x) = x^5$, so in Jordan canonical form, $a_{ii} = 0$. Thus
    the Jordan canonical form of $A$ has
    zeros along the diagonal, with possibly some ones on the superdiagonal: \[
      \begin{bmatrix}
        0 & * & 0 & 0 & 0 \\
        0 & 0 & * & 0 & 0 \\
        0 & 0 & 0 & * & 0 \\
        0 & 0 & 0 & 0 & * \\
        0 & 0 & 0 & 0 & 0
      \end{bmatrix}.
    \] Thus similarity classes of nilpotent matrices are in bijection with the
    seven partitions of $5$ via the size of the Jordan blocks: \begin{align*}
        &5 \\
        &4 + 1 \\
        &3 + 2 \\
        &3 + 1 + 1 \\
        &2 + 2 + 1 \\
        &2 + 1 + 1 + 1\\
        &1 + 1 + 1 + 1 + 1
    \end{align*}
  \end{enumerate}
\end{proof}

% -----------------------------------------------------
% Fourth problem
% -----------------------------------------------------
\begin{problem}{4} %
\end{problem}

\begin{proof} ~
\end{proof}
% -----------------------------------------------------
% Fifth problem
% -----------------------------------------------------
\begin{problem}{5} % p 613
\end{problem}

\begin{proof} ~
  \begin{enumerate}[(a)]
    \item   Let $A = U^{-1} J_A U$, where $J_A$ is the Jordan canonical form of $A$, which
      can be written $J_A = D_A + N_A$, with $D_A$ the diagonal entries of $J_A$,
      and $N_A$ the superdiagonal entries of $J_A$. \[
        \underbrace{
          \begin{bmatrix}
            \lambda_1 & * & \ & \ & \ \\
            \ & \lambda_2 & * & \ & \ \\
            \ & \ & \lambda_3 & \ddots & \ \\
            \ & \ & \ & \ddots & * \\
            \ & \ & \ & \ & \lambda_n
          \end{bmatrix}
        }_{J_A}
        = \underbrace{
          \begin{bmatrix}
            \lambda_1 & 0         & \         & \      & \ \\
            \         & \lambda_2 & 0         & \      & \ \\
            \         & \         & \lambda_3 & \ddots & \ \\
            \         & \         & \         & \ddots & 0 \\
            \         & \         & \         & \      & \lambda_n
          \end{bmatrix}
        }_{D_A}
        = \underbrace{
          \begin{bmatrix}
            0 & * & \ & \      & \ \\
            \ & 0 & * & \      & \ \\
            \ & \ & 0 & \ddots & \ \\
            \ & \ & \ & \ddots & * \\
            \ & \ & \ & \      & 0
          \end{bmatrix}
        }_{N_A}
      \]
      Notice that $p_{N_A}(x) = x^n$, so by Cayley-Hamilton, $N_A^n = 0$ and $N_A$.
      is nilpotent.
      Next, notice that \begin{align*}
        A
        &= U^{-1} J_A U \\
        &= U^{-1} (D_A + N_A) U \\
        &= U^{-1} D_A U + U^{-1} N_A U
      \end{align*} where $U^{-1} D_A U$ is clearly diagonalizable, and where
      $U^{-1} N_A U$ is nilpotent because \begin{align*}
        (U^{-1} N_A U)^n
          &= \underbrace{(U^{-1} N_A U)(U^{-1} N_A U)\hdots(U^{-1} N_A U)}_{n} \\
          &= U^{-1} \underbrace{N_A^n}_0 U \\
          &= 0.
      \end{align*}
  \end{enumerate}
\end{proof}
\end{document}
