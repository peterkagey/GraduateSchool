\documentclass{article}

\usepackage{amsmath, amssymb}
\usepackage{booktabs, tabularx} % For text alignment on cover page
\usepackage{epsfig}             % For \includegraphics command

\newcolumntype{C}{>{\centering\arraybackslash}X}


\begin{document}

\title{Math 574: Homework 2}
\author{Peter Kagey}
\date{Monday, October 14, 2019}

\maketitle

% -----------------------------------------------------
% First problem
% -----------------------------------------------------
\begin{problem}{1} ~
  \begin{enumerate}[(a)]
    \item Show that $A, B \in M_3(K)$ are similar if and only if they have the
    same minimal and characteristic polynomials.
  \end{enumerate}
\end{problem}

\begin{proof} ~
  $A$ and $B$ are similar if and only if they have the same Jordan
  normal form, so it is sufficient to compare the minimal and characteristic
  polynomials of matrices in Jordan normal form.
  \begin{enumerate}[(a)]
    \item A $3 \times 3$ matrix can have one of three Jordan normal forms \[
    A_1 = \begin{bmatrix}
      \lambda_1 & 0 & 0 \\
      0 & \lambda_2 & 0 \\
      0 & 0 & \lambda_3
    \end{bmatrix},\
    A_2 = \begin{bmatrix}
      \lambda_1 & 1 & 0 \\
      0 & \lambda_1 & 0 \\
      0 & 0 & \lambda_2
    \end{bmatrix},\ \text{ or }
    A_3 = \begin{bmatrix}
      \lambda_1 & 1 & 0 \\
      0 & \lambda_1 & 1 \\
      0 & 0 & \lambda_1
    \end{bmatrix}.
    \]
    Notice that a matrix $A$ \begin{enumerate}[(i)]
      \item similar to $A_1$ if and only if $(\lambda_1 - x)$ divides $m_{A}(x)$ exactly once,
      \item similar to $A_2$ if and only if $(\lambda_1 - x)$ divides $m_{A}(x)$ exactly twice, and
      \item similar to $A_3$ if and only if $(\lambda_1 - x)$ divides $m_{A}(x)$ exactly three times.
    \end{enumerate}
    \item Let \[
      A = \begin{bmatrix}
        1 & 1 & 0 & 0 \\
        0 & 1 & 0 & 0 \\
        0 & 0 & 1 & 1 \\
        0 & 0 & 0 & 1
      \end{bmatrix}
      \hspace{1cm}\text{and}\hspace{1cm}
      B = \begin{bmatrix}
        1 & 0 & 0 & 0 \\
        0 & 1 & 0 & 0 \\
        0 & 0 & 1 & 1 \\
        0 & 0 & 0 & 1
      \end{bmatrix}.
    \]
    Both $A$ and $B$ have the same minimal and characteristic polynomials
    \begin{alignat*}{2}
      m_A(x) &= m_B(x) &= (1-x)^2 \\
      p_A(x) &= p_B(x) &= (1-x)^4,
    \end{alignat*}
    but $A$ and $B$ are not similar because they have different Jordan
    canonical forms.
  \end{enumerate}
\end{proof}
% -----------------------------------------------------
% Second problem
% -----------------------------------------------------
\begin{problem}{2} ~
  Fix $A \in M_n(K)$ and let $C(A) = \set{B : BA = AB}$.
\end{problem}

\begin{proof} ~
  \begin{enumerate}[(a)]
    % \item Suppose $A$ is cyclic, that is the Jordan normal form $J_A = U^{-1}AU$
    % is a composed of Jordan blocks with distinct eigenvalues.
    % Now consider each Jordan block:
    % \\
    % Let $B' = U^{-1}BU$ (with the same $U$ as above) and notice that \begin{align*}
    %   AB = BA \Leftrightarrow
    %   A\underbrace{UU^{-1}}_IB = B\underbrace{UU^{-1}}_IA
    %   \Leftrightarrow \underbrace{U^{-1}AU}_{J_A} \underbrace{U^{-1}BU}_{B'} = \underbrace{U^{-1}BU}_{B'}\underbrace{U^{-1}AU}_{J_A}.
    % \end{align*}
    % Now look at what commutes with $J_A$ \[
    %   \underbrace{
    %     \begin{bmatrix}
    %       \lambda & 1 & \ & \ & \ \\
    %        \ & \lambda & 1 & \ & \ \\
    %        \ &  \ & \ddots & \ddots & \ \\
    %        \ & \ & \ & \lambda & 1  \\
    %        \ & \ & \ & \ & \lambda  \\
    %     \end{bmatrix}
    %   }_{J_A}
    %   \underbrace{
    %     \begin{bmatrix}
    %       b_{11} & b_{12} & \hdots & b_{1n} \\
    %       b_{21} & b_{22} & \hdots & b_{2n} \\
    %       b_{31} & b_{32} & \hdots & b_{3n} \\
    %       \vdots & \vdots & \ddots & \vdots \\
    %       b_{n1} & b_{n2} & \hdots & b_{nn}
    %     \end{bmatrix}
    %   }_{B'} = \lambda J_A + \begin{bmatrix}
    %     b_{21} & b_{22} & \hdots & b_{2n} \\
    %     b_{31} & b_{32} & \hdots & b_{3n} \\
    %     \vdots & \vdots & \ddots & \vdots \\
    %     b_{n1} & b_{n2} & \hdots & b_{nn} \\
    %     0 & 0 & \hdots & 0
    %   \end{bmatrix}
    % \] and similarly, \[
    %   \underbrace{
    %     \begin{bmatrix}
    %       b_{11} & b_{12} & \hdots & b_{1n} \\
    %       b_{21} & b_{22} & \hdots & b_{2n} \\
    %       b_{31} & b_{32} & \hdots & b_{3n} \\
    %       \vdots & \vdots & \ddots & \vdots \\
    %       b_{n1} & b_{n2} & \hdots & b_{nn}
    %     \end{bmatrix}
    %   }_B
    %   \underbrace{
    %     \begin{bmatrix}
    %       \lambda & 1 & \ & \ & \ \\
    %        \ & \lambda & 1 & \ & \ \\
    %        \ &  \ & \ddots & \ddots & \ \\
    %        \ & \ & \ & \lambda & 1  \\
    %        \ & \ & \ & \ & \lambda  \\
    %     \end{bmatrix}
    %   }_{J_A}
    %   = \lambda J_A + \begin{bmatrix}
    %     0 & b_{11} & \hdots & b_{1(n-1)} \\
    %     0 & b_{21} & \hdots & b_{2(n-1)} \\
    %     \vdots & \vdots & \ddots & \vdots \\
    %     0 & b_{n1} & \hdots & b_{n(n-1)}
    %   \end{bmatrix}
    % \]
    % So $a_{21} = a_{31} = \hdots = a_{(n-1)1} = 0$ and $a_{n1} = a_{n2} = \hdots = a_{n(n-1)} = 0$,
    % And otherwise, $a_{ij} = a_{(i+1)(j+1)}$, so if $B'$ commutes with $J_A$ then
    % $B'$ is of the form \[
    %  B' = \begin{bmatrix}
    %     c_1 & c_2 & c_3 & \hdots & c_n \ \\
    %     \ & c_1 & c_2 & \hdots & c_{n-1}\ \\
    %     \ &  \ & \ddots & \ddots & \vdots \\
    %     \ & \ & \ & c_1 & c_2  \\
    %     \ & \ & \ & \ & c_1  \\
    %  \end{bmatrix}
    % \] since $b_1, b_2, \hdots, b_n$ can be chosen arbitrarily, $B'$ has dimension $n$.
    % So clearly $B = UB'U^{-1}$ also has dimension $n$.
    % \item When $A$ is a general matrix, the same general argument holds, except
    % for that $J_A$ now may have zeros on the super diagonal. 1
    \item Firstly, suppose $f(x) \in K[x]$. Then \begin{align*}
      Af(A) &= A(c_0I + c_1A + \cdots + c_nA^n) \\
        &= c_0A + c_1A^2 + \cdots + c_nA^{n+1} \\
        &= (c_0I + c_1A + \cdots + c_nA^n)A
    \end{align*} so $f(A) \in C(A)$.
    Conversely, suppose $v$ is a cyclic vector for $A$, that is
    $\set{v, Av, A^2v, \cdots, A^{n-1}v}$ is a basis for $M_n(K)$, meaning we can write \[
      Bv = c_0v + c_1Av + \cdots + c_nA^nv = f(A)v.
    \]
    Moreover, we can write $B(A^kv) = f(A)(A^kv)$, and since ${A^k}_{k=0}^{n-1}$
    is a basis, $B = f(A)$ because it $B$ and $f(A)$ send every basis vector to
    the same place.
    \\
    Lastly, $\dim C(A) = n$ because the map \[
      (c_0, c_1, \hdots, c_{n-1}) \mapsto c_0I + c_1A + \cdots + c_{n-1}A^{n-1} = f(A) = B
    \]
    is surjective since the minimal polynomial is degree $n$.
    \item Suppose that $A$ is not necessarily cyclic. Notice that the set
    $\set{f(A) : f(x) \in k[x]}$ is a subspace of $C(A)$ since it is closed
    under addition and scalar multiplication. However,
    $\dim\set{f(A) : f(x) \in k[x]}$ is the degree of the minimal polynomial.
    \\
    Let $J_A$ denote the Jordan normal form of $A$, $J_A = U^{-1}AU$. Next
    let $B' = U^{-1}BU$ (with the same $U$ as above) and notice that
    $\dim(C(A)) = \dim(C(J_A))$: \begin{align*}
      AB = BA \Leftrightarrow
      A\underbrace{UU^{-1}}_IB = B\underbrace{UU^{-1}}_IA
      \Leftrightarrow \underbrace{U^{-1}AU}_{J_A} \underbrace{U^{-1}BU}_{B'} = \underbrace{U^{-1}BU}_{B'}\underbrace{U^{-1}AU}_{J_A}.
    \end{align*}
    So it is enough to consider the Jordan normal form of $A$, thus apply (a)
    blockwise, since each block is cyclic.
  \end{enumerate}
\end{proof}
% -----------------------------------------------------
% Third problem
% -----------------------------------------------------
\begin{problem}{3} ~
  \begin{enumerate}[(a)]
    \item Show that if $N$ is nilpotent then $N^n = 0$.
    \item Show that $\dim \ker N$ is the number of Jordan blocks in its Jordan canonical form.
    \item How many similarity classes of $5 \times 5$ nilpotent matrices are there?
  \end{enumerate}
\end{problem}

\begin{proof} ~
  \begin{enumerate}[(a)]
    \item If $N^d = 0$, then there exists some $d' \leq n$ such that
    $N^{d'} = 0$ because the minimal polynomial $m_N(x) | x^d$, so the minimal
    polynomial is of the form $m_N(x) = x^{d'}$ with $d' <= n$, since the
    minimal polynomial has degree less than or equal to $n$.
    \item Because the minimal polynomial is of the form $m_N(x) = x^{d'}$,
    the Jordan canonical form of $N$ must have all zeros on the diagonal.
    Now it is enough to go block by block, and show that each block has nullity
    of exactly $1$, namely \[
      \begin{bmatrix}
        0 & 1 & \ & \ & \ & \\\
        \ & 0 & 1 & \ & \ & \\\
        \ & \ & \ddots & \ddots & \ & \ \\
        \ & \ & \ & 0 & 1 & \ \\
        \ & \ & \ & \ & 0 & 1\\
        \ & \ & \ & \ & \ & 0\\
      \end{bmatrix}\begin{bmatrix}
        a_{11} \\ a_{21} \\ \vdots \\ a_{(m-1)1} \\ a_{m1}
      \end{bmatrix} = \begin{bmatrix}
        a_{21} \\ a_{31} \\ \vdots \\ a_{m1} \\ 0
      \end{bmatrix}
    \] so $a_11$ is a free variable and the dimension of the kernel of any block is $1$.
    By induction on the number of Jordan blocks, the dimension of the kernel is
    equal to the number of blocks.
    \item By (a), the characteristic polynomial of a $5 \times 5$ nilpotent
    matrix is $p(x) = x^5$, so in Jordan canonical form, $a_{ii} = 0$. Thus
    the Jordan canonical form of $A$ has
    zeros along the diagonal, with possibly some ones on the superdiagonal: \[
      \begin{bmatrix}
        0 & * & 0 & 0 & 0 \\
        0 & 0 & * & 0 & 0 \\
        0 & 0 & 0 & * & 0 \\
        0 & 0 & 0 & 0 & * \\
        0 & 0 & 0 & 0 & 0
      \end{bmatrix}.
    \] Thus similarity classes of nilpotent matrices are in bijection with the
    seven partitions of $5$ via the size of the Jordan blocks: \begin{align*}
        &5 \\
        &4 + 1 \\
        &3 + 2 \\
        &3 + 1 + 1 \\
        &2 + 2 + 1 \\
        &2 + 1 + 1 + 1\\
        &1 + 1 + 1 + 1 + 1
    \end{align*}
  \end{enumerate}
\end{proof}

% -----------------------------------------------------
% Fourth problem
% -----------------------------------------------------
\begin{problem}{4} %
  Let $\fn T V V$ be a linear transformation, and fix $v \in V$.
  \begin{enumerate}[(a)]
    \item Show that there is a unique polynomial $g(x)$ so that $h(T)v = 0$ if
    and only if $g | h$ and $g | m_T$.
    \item Show that the degree of $G$ is the dimension of the span of
    $\set{v, Tv, T^2v, \hdots}$.
  \end{enumerate}
\end{problem}

\begin{proof} First suppose that $\dim V = n$.
  \begin{enumerate}[(a)]
    \item
    First notice that $\set{v, Tv, T^2v, \hdots, T^nv}$ cannot be linearly
    independent, because it consists of $n + 1$ vectors.
    Now take the largest $k$ such that $\set{v, Tv, T^2v, \hdots, T^kv}$
    \textit{is} then, in particular, \[
      h(T)v = c_{k+1}T^{k+1}v + c_{k}T^{k}v + \cdots + c_{1}Tv + c_0v = 0.
    \] This forces $c_{k+1}$ to be nonzero, so let $g(x) = h(x)/c_n$, so that
    $g(x)$ is monic. \\
    Notice that there cannot be a different polynomial $f$ of equal or lower
    degree than $g$,
    otherwise the set $\set{v, Tv, T^2v, \hdots, T^kv}$ would be linearly
    dependent.
    Thus every polynomial $h(x)$ such that $h(T)v = 0$ is a multiple of $g(x)$.
    \item By construction, $\set{v, Tv, T^2v, \hdots, T^kv}$ is the maximal
    linearly independent set. Moreover $T^mv$ can be written as a linear
    combination of this linearly independent set for all $m > k$, so applying
    this procedure inductively, the aforementioned set is a basis for
    $\set{v, Tv, T^2v, \hdots}$.
  \end{enumerate}
\end{proof}
% -----------------------------------------------------
% Fifth problem
% -----------------------------------------------------
\begin{problem}{5} % p 613
\end{problem}

\begin{proof} ~
  \begin{enumerate}[(a)]
    \item   Let $A = U^{-1} J_A U$, where $J_A$ is the Jordan canonical form of $A$, which
      can be written $J_A = D_A + N_A$, with $D_A$ the diagonal entries of $J_A$,
      and $N_A$ the superdiagonal entries of $J_A$. \[
        \underbrace{
          \begin{bmatrix}
            \lambda_1 & * & \ & \ & \ \\
            \ & \lambda_2 & * & \ & \ \\
            \ & \ & \lambda_3 & \ddots & \ \\
            \ & \ & \ & \ddots & * \\
            \ & \ & \ & \ & \lambda_n
          \end{bmatrix}
        }_{J_A}
        = \underbrace{
          \begin{bmatrix}
            \lambda_1 & 0         & \         & \      & \ \\
            \         & \lambda_2 & 0         & \      & \ \\
            \         & \         & \lambda_3 & \ddots & \ \\
            \         & \         & \         & \ddots & 0 \\
            \         & \         & \         & \      & \lambda_n
          \end{bmatrix}
        }_{D_A}
        = \underbrace{
          \begin{bmatrix}
            0 & * & \ & \      & \ \\
            \ & 0 & * & \      & \ \\
            \ & \ & 0 & \ddots & \ \\
            \ & \ & \ & \ddots & * \\
            \ & \ & \ & \      & 0
          \end{bmatrix}
        }_{N_A}
      \]
      Notice that $p_{N_A}(x) = x^n$, so by Cayley-Hamilton, $N_A^n = 0$ and $N_A$.
      is nilpotent.
      Next, notice that \begin{align*}
        A
        &= U^{-1} J_A U \\
        &= U^{-1} (D_A + N_A) U \\
        &= U^{-1} D_A U + U^{-1} N_A U
      \end{align*} where $U^{-1} D_A U$ is clearly diagonalizable, and where
      $U^{-1} N_A U$ is nilpotent because \begin{align*}
        (U^{-1} N_A U)^n
          &= \underbrace{(U^{-1} N_A U)(U^{-1} N_A U)\hdots(U^{-1} N_A U)}_{n} \\
          &= U^{-1} \underbrace{N_A^n}_0 U \\
          &= 0.
      \end{align*}
    \item
  \end{enumerate}
\end{proof}
\end{document}
