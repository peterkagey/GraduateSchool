\documentclass[12 pt]{article}
\pagestyle{empty}
\addtolength{\topmargin}{-0.9in}
\addtolength{\textheight}{1.9in}
\addtolength{\oddsidemargin}{-0.7in}
\addtolength{\textwidth}{1.4in}
\newcommand{\D}{\displaystyle}
\usepackage{amsmath}

\begin{document}
	\begin{center}
		\textbf{\hfill Math 226/229 -- Quiz} \\
	\end{center}
	\medskip

	\noindent
	\textbf{Week 10 Quiz Solutions.} \hfill
	\vspace{.1in}
	%\textbf{Discussion: 10 am   11 am}\
	\hspace*{0.2in}
	\medskip
	\noindent
  \begin{enumerate}
		\item (\textit{5 points})
		Use the second derivatives test to identify local maxima, local minima, and
		saddle points for the function \[
			f(x, y) = \frac{x^3}{3} + 8x + 3y^2 - 6xy.
		\]
    \\~\\
    \textbf{Solution.}
    \\
		First find the critical points of $f$, by setting the gradient equal to zero
		\begin{align*}
				\nabla f(x,y) = \langle x^2 + 8 - 6y, 6y - 6x \rangle = \langle 0, 0 \rangle.
		\end{align*}
		By the second equation this occurs when $x = y$. By substituting in the
		first equation, this gives $x^2 - 6x + 8 = (x-2)(x-4) = 0$. So the critical
		points occur at $(2, 2)$ or  $(4, 4)$.
		By the second derivative test \begin{align*}
			D &= f_{xx}(x, y)f_{yy}(x, y) - [f_{xy}(x, y)]^2 \\
				&= (2x)(6) - (-6)^2 \\
				&= 12x - 36.
		\end{align*}
		When $(x,y) = (2, 2)$, $D = -12 < 0$, so this is a saddle point. When
		$(x, y) = (4, 4)$, $D = 12 > 0$ and $f_{xx} = 8$, so this is a local
		minimum.
		\item (\textit{5 points})
		Use Lagrange multipiers to find the extreme values of the above function $f(x, y)$
		subject to the constraint that $x - y = 2$.
		\\~\\
    \textbf{Solution.}
		Name the constraint $g(x, y) = x - y$. Then we have \begin{align*}
			\nabla f(x, y) &= \lambda \nabla g(x, y) \\
			\langle x^2 + 8 - 6y, 6y - 6x \rangle &= \lambda\langle 1, -1 \rangle
		\end{align*} which gives the following system of equations \begin{align}
			x^2 + 8 - 6y &= \lambda \\
			6y - 6x &= -\lambda \\
			x - y &= 2
		\end{align}
		Combining equations (2) and (3) gives that $\lambda = 12$.
		Then solving for $y$ in equation (3) and using equation (1) gives
		\begin{align*}
			x^2 + 8 - 6(x-2) &= 12 \\
			x^2 - 6x + 8 &= 0 \\
			(x - 2)(x - 4) &= 0.
		\end{align*}
		So $x = 2$ or $x = 4$, corresponding to the points $(2, 0)$ and $(4, 2)$
		respectively. The function $f$ does not have a global maximum subject to
		the constraint $x - y = 2$, but since \[
		f(2, 0) = \frac 83 + 16 = \frac{56}{3} > f(4, 2) = \frac{64}{3} + 32 + 12 - 48 = \frac{52}{3}\]
		$(2, 0)$ is a local max and $(4, 2)$ is a local min.
  \end{enumerate}
\end{document}
