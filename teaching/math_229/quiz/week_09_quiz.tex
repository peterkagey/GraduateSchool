\documentclass[12 pt]{article}
\pagestyle{empty}
\addtolength{\topmargin}{-0.9in}
\addtolength{\textheight}{1.9in}
\addtolength{\oddsidemargin}{-0.7in}
\addtolength{\textwidth}{1.4in}
\newcommand{\D}{\displaystyle}
\usepackage{amsmath}

\begin{document}
  \begin{center}
    \textbf{\hfill Math 226/229 -- Quiz} \\
  \end{center}
  \medskip

  \noindent
  \textbf{Name}\ \rule{3.5in}{.4pt} \hfill
  \vspace{.1in}
  \hspace*{0.2in}
  \begin{itemize}
    \item \textbf{Write as neatly as you can!}
    \item \textbf{No calculators are allowed.}
    \item \textbf{You must show your work to obtain full credit.}
  \end{itemize}

	\medskip
  \noindent

  \begin{enumerate}
    \item (\textit{5 points})
    Suppose $f(1,2,3)=9, f_x(1,2,3)=\sqrt{17}, f_y(1,2,3) = \ln 5, f_z(1,2,3) = e^2$. Use this information to approximate $f(0.99, 2.01, 2.98)$.
		\vspace{2in}
		\item (\textit{5 points})
		The temperature $T$ at a location $(x, y, z)$ is given by $T(x,y,z) = 100e^{-(x^2+y^2/4+z^2/9)}$. At $(1,2,3)$, what is the direction in which $T$ increases most rapidly? What is the direction of most rapid decrease in T?
  \end{enumerate}
\end{document}
