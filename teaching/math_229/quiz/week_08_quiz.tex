\documentclass[12 pt]{article}
\pagestyle{empty}
\addtolength{\topmargin}{-0.9in}
\addtolength{\textheight}{1.9in}
\addtolength{\oddsidemargin}{-0.7in}
\addtolength{\textwidth}{1.4in}
\newcommand{\D}{\displaystyle}
\usepackage{amsmath}

\begin{document}
  \begin{center}
    \textbf{\hfill Math 226/229 -- Quiz} \\
  \end{center}
  \medskip

  \noindent
  \textbf{Name}\ \rule{3.5in}{.4pt} \hfill
  \vspace{.1in}
  \hspace*{0.2in}
  \begin{itemize}
    \item \textbf{Write as neatly as you can!}
    \item \textbf{No calculators are allowed.}
    \item \textbf{You must show your work to obtain full credit.}
  \end{itemize}

	\medskip
  \noindent

  \begin{enumerate}
    \item (\textit{5 points})
    Given the function \[
      f(x, y, z) = \frac{x}{3y^2 + yz}
    \] compute $f_{yx} (1, 2, 3)$ in whichever order you want.
		\vspace{2in}
		\item (\textit{5 points})
		Use the \textbf{chain rule} to determine
    $\displaystyle \frac{\partial T}{\partial \theta}$ and
    $\displaystyle \frac{\partial^2 T}{\partial \theta \partial r}$
    when $r = 2$ and $\theta = \pi/3$
    given that \begin{align*}
      T &= xe^y, \\
      x &= r\cos\theta, \text{ and}\\
      y &= r\sin\theta.
    \end{align*}
  \end{enumerate}
\end{document}
