\documentclass[12 pt]{article}
\pagestyle{empty}
\addtolength{\topmargin}{-0.9in}
\addtolength{\textheight}{1.9in}
\addtolength{\oddsidemargin}{-0.7in}
\addtolength{\textwidth}{1.4in}
\newcommand{\D}{\displaystyle}
\usepackage{amsmath}

\begin{document}
	\begin{center}
		\textbf{\hfill Math 226/229 -- Quiz} \\
	\end{center}
	\medskip

	\noindent
	\textbf{Week 3 Quiz Solutions.} \hfill
	\vspace{.1in}
	%\textbf{Discussion: 10 am   11 am}\
	\hspace*{0.2in}
	\medskip
	\noindent
	\begin{enumerate}
    \item (\textit{5 points})
    \begin{enumerate}
	  \item Find a vector $\vec{v}$ of the form $\langle c, 2c, c-1\rangle$ (for some constant c)
      such that $\vec{v} \cdot \langle 3, 2, 1\rangle = 15$
	  \item Find the vector projection of $\vec{v}$ onto the vector $\langle -2, 1, 2\rangle$
	\end{enumerate}
    \textbf{Solution.}
    \begin{enumerate}
      \item We compute the dot product using the coordinates \begin{align*}
        3(c) + 2(2c) + 1(c-1) &= 15 \\
        8c-1 &= 15 \\
        c &= 2 \\
        \vec{v} &= \langle 2, 4, 1 \rangle
      \end{align*}
      \item We can use the projection formula, \[
        \operatorname{proj}_a(b) = \left(\frac{a \cdot b}{|a|}\right)\frac{a}{|a|},
      \] with $a = \langle -2, 1, 2\rangle$ and $b = \langle 2, 4, 1 \rangle$ giving \[
        \operatorname{proj}_{\langle -2, 1, 2\rangle}(\vec v)
        = \left(\frac{-4 + 4 + 2}{\sqrt{9}}\right)\frac{\langle -2, 1, 2\rangle}{\sqrt{9}}
        = \frac{2}{9}\langle -2, 1, 2\rangle
      \]
    \end{enumerate}
		\item (\textit{5 points})
		\begin{enumerate}
			\item Two forces $\vec{F}$ and $\vec{G}$ act on a wagon. $\vec{F}$ has magnitude $4$ N and acts at an angle of $60^\circ$ from the \textbf{negative} $x$-axis. $\vec{G}$ has magnitude  $2\sqrt2$ N and acts at an angle of $45^\circ$ from the \textbf{postive} $x$-axis. Find the vector components of the net force.
			\item If this wagon is pulled 500 meters in the \textbf{positive} $x$ direction by this net force, find the total work done by the net force on the wagon.
		\end{enumerate}
    \textbf{Solution.}
    \begin{enumerate}
      \item Writing the vectors as magnitude and direction gives \begin{alignat*}{2}
        \vec{F}
          &= 4\text{ N}\langle-\cos(60^\circ), -\sin(60^\circ)\rangle
          &&= \langle -2, -2\sqrt 3 \rangle\text{ N} \\
        \vec{G}
          &= 2\sqrt 2\text{ N}\langle\cos(45^\circ), \sin(45^\circ)\rangle
          &&= \langle 2, 2 \rangle\text{ N} \\
        \vec{F} + \vec{G}
          &= \langle 0, 2-2\sqrt 3\rangle\text{ N} \\
      \end{alignat*}
      \item The formula for work is $W = \vec F_\text{net} \cdot \vec d$. In this case, $\vec F_\text{net} = \vec{F} + \vec{G}$. \[
        W =
        \langle 0  \text{ N}, 2-2\sqrt3\text{ N}\rangle \cdot
        \langle 500\text{ m}, 0        \text{ m}\rangle
        = 0 \text{ Nm}.
      \]
    \end{enumerate}
  \end{enumerate}
\end{document}
