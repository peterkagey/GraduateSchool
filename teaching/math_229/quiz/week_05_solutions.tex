\documentclass[12 pt]{article}
\pagestyle{empty}
\addtolength{\topmargin}{-0.9in}
\addtolength{\textheight}{1.9in}
\addtolength{\oddsidemargin}{-0.7in}
\addtolength{\textwidth}{1.4in}
\newcommand{\D}{\displaystyle}
\usepackage{amsmath}

\begin{document}
	\begin{center}
		\textbf{\hfill Math 226/229 -- Quiz} \\
	\end{center}
	\medskip

	\noindent
	\textbf{Week 5 Quiz Solutions.} \hfill
	\vspace{.1in}
	%\textbf{Discussion: 10 am   11 am}\
	\hspace*{0.2in}
	\medskip
	\noindent
  \begin{enumerate}
    \item (\textit{5 points})
    Find a parametric equations of the tangent line to the curve
    $\vec r(t)=\langle\frac{\pi}{t}, \cos^{2}(t),e^t\rangle$ at the point
    $(1,1,e^\pi)$.
    \\~\\
    \textbf{Solution.}
    \\
    First, by solving for the $z$ coordinate and checking the others, it is
    clear that when $t = \pi$, $\vec r(\pi) = (1, 1, e^\pi)$.
    Then, it is enough to find the direction of a tangent vector at this point
    \begin{align*}
      \vec{r'}(t) &= \langle -\pi t^{-2}, -2\cos(t)\sin(t), e^t\rangle \\
      \vec{r'}(\pi) &= \langle -1/\pi, 0, e^\pi\rangle.
    \end{align*}
    Using a point on the line and its direction, the parametric equations
    of the tangent line can be written \begin{align*}
      x &= 1 - t/\pi \\
      y &= 1 \\
      z &= e^\pi + e^\pi t.
    \end{align*}
		\item (\textit{5 points})
		\begin{enumerate}
			\item Let $\vec{r}(t)=\langle0,t,t^2\rangle$ and
      $\vec{s}(u)=\langle u\sin(\pi u),u\cos(\pi u),u\rangle$.
      Find the intersection points of $\vec{r}(t)$ and $\vec{s}(u)$.
      \\~\\
      \textbf{Solution.}
      \begin{align}
        0   &= u\sin(\pi u) \\
        t   &= u\cos(\pi u) \\
        t^2 &= u
      \end{align}
      \begin{itemize}
        \item By the first equation, $u = 0$ or $\sin(\pi u) = 0$, so $u$ must
        be an integer.
        \item This means that, in the second equation, $\cos(\pi u) = \pm 1$, so
        $t = u$ or $t = -u$.
        \item Substitution into the third equation gives
        $u = (\pm u)^2 = u^2$ so $u = 0$ or $u = 1$.
        \item By the third equation, if $u = 0$ then $t = 0$.
        \item By the second equation, if $u = 1$ then $t = -1$.
      \end{itemize}
      Checking these values, \begin{align*}
        \vec s(0) &= \langle 0, 0, 0 \rangle = \vec r(0) \\
        \vec s(1) &= \langle 0, -1, 1 \rangle = \vec r(-1).
      \end{align*}

			\item Which (if any) of these intersection points are collision points?
      \\~\\
      \textbf{Solution.}
      \\
      Collisions happens when $\vec r(t) = \vec s(t)$.
      In this case, this only happens when $t = u = 0$.
		\end{enumerate}
  \end{enumerate}
\end{document}
