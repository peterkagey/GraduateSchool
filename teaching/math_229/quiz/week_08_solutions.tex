\documentclass[12 pt]{article}
\pagestyle{empty}
\addtolength{\topmargin}{-0.9in}
\addtolength{\textheight}{1.9in}
\addtolength{\oddsidemargin}{-0.7in}
\addtolength{\textwidth}{1.4in}
\newcommand{\D}{\displaystyle}
\usepackage{amsmath}

\begin{document}
	\begin{center}
		\textbf{\hfill Math 226/229 -- Quiz} \\
	\end{center}
	\medskip

	\noindent
	\textbf{Week 7 Quiz Solutions.} \hfill
	\vspace{.1in}
	%\textbf{Discussion: 10 am   11 am}\
	\hspace*{0.2in}
	\medskip
	\noindent
  \begin{enumerate}
		\item (\textit{5 points})
		Given the function \[
			f(x, y, z) = \frac{x}{3y^2 + yz}
		\] compute $f_{yx} (1, 2, 3)$ in whichever order you want.
    \\~\\
    \textbf{Solution.}
    \\
		It is easiest to compute the derivative with respect to $x$ first,
		and then compute the derivativ with respect to $y$: \begin{align*}
		f_xy(x, y, z) &= \frac{\partial}{\partial y}\left[
			\frac{\partial f}{\partial x}
		\right] \\
		&= \frac{\partial}{\partial y}\left[
			\frac{1}{3y^2 + yz}
		\right] \\
		&= \frac{-6y - z}{(3y^2 + yz)^2}
		\end{align*}
		So evaluating at $(1, 2, 3)$ gives \[
			\frac{-6(2) - 3}{(3(4) + 6)^2} = \frac{-15}{18^2}.
		\]
		\\~\\
		\item (\textit{5 points})
		Use the \textbf{chain rule} to determine
		$\displaystyle \frac{\partial T}{\partial \theta}$ and
		$\displaystyle \frac{\partial^2 T}{\partial \theta \partial r}$
		when $r = 2$ and $\theta = \pi/3$
		given that \begin{align*}
			T &= xe^y, \\
			x &= r\cos\theta, \text{ and}\\
			y &= r\sin\theta.
		\end{align*}
    \textbf{Solution.}
		\\
		By the chain rule, \begin{align*}
			\frac{\partial T}{\partial \theta}
			&= \frac{\partial T}{\partial x}\cdot\frac{\partial x}{\partial \theta}
			+ \frac{\partial T}{\partial y}\cdot\frac{\partial y}{\partial \theta}\\
			&= e^y(-r \sin\theta) + xe^y(r\cos\theta) \\
			&= e^{r \sin\theta}(-r \sin\theta) + (r\cos\theta)e^{r\sin\theta}(r\cos\theta) \\
			&= -ye^y + x^2e^{y}
		\end{align*}
		Evaluated at $r = 2$ and $\theta = \pi/3$ (and thus $x=1$ and $y=\sqrt3$) gives \[
			-\sqrt3e^{\sqrt3} + e^{\sqrt3} = e^{\sqrt3}(1- \sqrt3).
		\]
		Denote $\displaystyle \frac{\partial T}{\partial\theta} = T_\theta$. Then by
		switching the order of differentiation,
		\begin{align*}
			\frac{\partial^2 T}{\partial \theta \partial r}
			&= \frac{\partial T_\theta}{\partial r} \\
			&= \frac{\partial T_\theta}{\partial x}\cdot\frac{\partial x}{\partial r}
			+ \frac{\partial T_\theta}{\partial y}\cdot\frac{\partial y}{\partial r} \\
			&= (2xe^y)(\cos\theta) + (-e^y - ye^y + x^2e^y)(\sin\theta).
		\end{align*}
		Again evaluated at $r = 2$ and $\theta = \pi/3$
		(and thus $x=1$ and $y=\sqrt3$) gives \[
			e^{\sqrt3} + (-e^{\sqrt3} - \sqrt3e^{\sqrt3} + e^{\sqrt3})\frac{\sqrt3}2
			= -\frac{e^{\sqrt3}}2.
		\]
		\\
  \end{enumerate}
\end{document}
