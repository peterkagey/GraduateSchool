\documentclass[12 pt]{article}
\pagestyle{empty}
\addtolength{\topmargin}{-0.9in}
\addtolength{\textheight}{1.9in}
\addtolength{\oddsidemargin}{-0.7in}
\addtolength{\textwidth}{1.4in}
\newcommand{\D}{\displaystyle}
\usepackage{amsmath}

\begin{document}
  \begin{center}
    \textbf{\hfill Math 226/229 -- Quiz} \\
  \end{center}
  \medskip

  \noindent
  \textbf{Name}\ \rule{3.5in}{.4pt} \hfill
  \vspace{.1in}
  \hspace*{0.2in}
  \begin{itemize}
    \item \textbf{Write as neatly as you can!}
    \item \textbf{No calculators are allowed.}
    \item \textbf{You must show your work to obtain full credit.}
  \end{itemize}

	\medskip
  \noindent

  \begin{enumerate}
    \item (\textit{5 points})
    Use the second derivatives test to identify local maxima, local minima, and
    saddle points for the function \[
      f(x, y) = \frac{x^3}{3} + 8x + 3y^2 - 6xy.
    \]
		\vspace{2in}
		\item (\textit{5 points})
		Use Lagrange multipiers to find the extreme values of the function \[
      f(x, y) = \frac{x^3}{3} + 8x + 3y^2 - 6xy
    \]
    subject to the constraint that $x - y = 2$.
  \end{enumerate}
\end{document}
