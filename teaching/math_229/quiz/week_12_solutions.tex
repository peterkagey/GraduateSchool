\documentclass[12 pt]{article}
\pagestyle{empty}
\addtolength{\topmargin}{-0.9in}
\addtolength{\textheight}{1.9in}
\addtolength{\oddsidemargin}{-0.7in}
\addtolength{\textwidth}{1.4in}
\newcommand{\D}{\displaystyle}
\usepackage{amsmath, pgfplots}

\begin{document}
	\begin{center}
		\textbf{\hfill Math 226/229 -- Quiz} \\
	\end{center}
	\medskip

	\noindent
	\textbf{Week 12 Quiz Solutions.} \hfill
	\vspace{.1in}
	%\textbf{Discussion: 10 am   11 am}\
	\hspace*{0.2in}
	\medskip
	\noindent
  \begin{enumerate}
		\item (\textit{7 points})
      Let $E$ be the region of the cone $x^2 + y^2 = z^2$ that is
      between the $xy$-plane and the plane $z = 1$.
      The density of the solid is given by $f(x, y, z) = x^2 + y^2 + z^2$.
      \begin{enumerate}
        \item Set up \textbf{but do not evaluate} the integral to find the mass of $E$ using cylindrical coordinates.
        \item Set up \textbf{but do not evaluate} the integral to find the mass of $E$ using spherical coordinates.
      \end{enumerate}
    \textbf{Solution.}\\
			\begin{enumerate}
				\item Substituting $r^2$ for $x^2 + y^2$ in the integrand gives \[
					\int\int\int_E f(x,y,z)\,dV = \int_{\theta=0}^{2\pi}\int_{r=0}^1\int_{z=r}^1 r^3 + rz^2 \,dz\, dr\, d\theta
				\]
				\item We know that $z = \rho\sin(\phi) \leq 1$ so $\rho \leq \frac{1}{\sin\phi}$. Substituting $\rho^2$ for $x^2 + y^2 + z^2$ in the integrand gives: \[
					\int_{\theta=0}^{2\pi}\int_{\phi=0}^{\pi/4}\int_{\rho=0}^{1/\sin\phi} \rho^4\sin\phi \,d\rho\, d\phi\, d\theta
				\]
			\end{enumerate}
		\pagebreak
		\item (\textit{3 points}) Find the tangent plane to the surface
		parameterized by \begin{align*}
			x &= u + v \\
			y &= 2u \\
			z &= uv - 1
		\end{align*} at the point $(1,2,-1)$.
    \\~\\
    \textbf{Solution.}\\
		Write the parameterization as $\vec r(u, v) = \langle u + v, 2u, uv - 1 \rangle$.
		Then the partial derivatives with respect to $u$ and $v$ are \begin{align*}
			\vec r_u(u, v) &= \langle 1, 2, v \rangle \\
			\vec r_v(u, v) &= \langle 1, 0, u \rangle.
		\end{align*}
		In particular, the only solution to $\vec r(u, v) = \langle 1, 2, -1 \rangle$
		is when $u = 1$ and $v = 0$.
		The normal vector to the tangent plane is given by
		\[
			r_u(1, 0) \times r_v(1, 0) = \begin{bmatrix}
				\hat i & \hat j & \hat k \\
				1 & 2 & 0 \\
				1 & 0 & 1
			\end{bmatrix}
			= \langle 2, -1, -2 \rangle.
		\]
		Therefore the plane through $(1, 2, -1)$ is \[
			2(x - 1) - (y - 2) - 2(z + 1) = 0.
		\]
  \end{enumerate}
\end{document}
