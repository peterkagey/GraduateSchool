\documentclass[12 pt]{article}
\pagestyle{empty}
\addtolength{\topmargin}{-0.9in}
\addtolength{\textheight}{1.9in}
\addtolength{\oddsidemargin}{-0.7in}
\addtolength{\textwidth}{1.4in}
\newcommand{\D}{\displaystyle}
\usepackage{amsmath}

\begin{document}
  \begin{center}
    \textbf{\hfill MATH 040: Quiz 1} \\
  \end{center}
  \medskip

  \noindent
  \textbf{Name}\ \rule{3.5in}{.4pt} \hfill
  \vspace{.1in}
  \hspace*{0.2in}
  \begin{itemize}
    \item \textbf{Write as neatly as you can!}
    \item \textbf{No calculators are allowed.}
    \item \textbf{You must show your work to obtain full credit.}
  \end{itemize}

	\medskip
  \noindent

  \begin{enumerate}
    \item (\textit{5 points})
    Determine whether the four points (0, 12), (1, 9), (4,0) and (-1, 15) are
    on the same line. \\
    \textbf{Hint: use the first two points to come up with the equation for a
    line, and see if the other two points satisfy the equation.}

		\pagebreak
		\item (\textit{5 points}) Find all values of $x$ that satisfy the inequality \[
      0.3x + 2 \geq \frac{2}5x - 5
    \]
    \vspace{3in}\\
    \item (\textit{1 point, extra credit})
    Suppose Ann, Bea, and Caitlin split a $\$12$ pizza, but they don't finish it.
    If Ann eats $1/8$ of the pizza, Bea eats $1/4$ and Caitlin eats $3/8$,
    how should they split the cost of the pizza so that what everyone pays is
    proportional to what they ate?
  \end{enumerate}

\end{document}
