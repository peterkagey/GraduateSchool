\documentclass[12 pt]{article}
\pagestyle{empty}
\addtolength{\topmargin}{-0.9in}
\addtolength{\textheight}{1.9in}
\addtolength{\oddsidemargin}{-0.7in}
\addtolength{\textwidth}{1.4in}
\newcommand{\D}{\displaystyle}
\usepackage{amsmath}

\begin{document}
  \begin{center}
    \textbf{\hfill MATH 040: Quiz 3} \\
  \end{center}
  \medskip

  \noindent
  \textbf{Name}\ \rule{3.5in}{.4pt} \hfill
  \vspace{.1in}
  \hspace*{0.2in}
  \begin{itemize}
    \item \textbf{Write as neatly as you can!}
    \item \textbf{No calculators are allowed.}
    \item \textbf{You must show your work to obtain full credit.}
  \end{itemize}

	\medskip
  \noindent

  \begin{enumerate}
    \item (\textit{5 points})
    The number of books sold in a small town in 2015 was $32\,000$. In 2010,
    $28\,000$ books were sold in this town. Let $y$ be the number of books sold since $2010$. (i.e., $x=0$ represents the year $2010$.)

    Assuming the relationship between the number of years since 2010 and the
    number of books sold is linear, \textbf{Write a linear equation} modeling
    the number of books sold in terms of the year $x$.

    [Hint: The line must pass through the points $(0, 28\,000)$ and $(5, 32\,000)$.] \\

		\pagebreak
		\item (\textit{5 points})
    Consider the line given by equation $y = 2x + 0.17$. Find the equation of
    the line which is parallel to this line, and passes through the point
    $(0, -0.5)$.
	  \\
	   Also find the equation of the line that is perpendicular to the first line,
     and goes through the point $(0, 1.13)$.
    \vspace{3in}\\
    \item (\textit{1 point, extra credit})
    Suppose seller A sells a car at $\$1000$, and seller B sells the same car
    at a price that is $8\%$ higher than the price at seller A's.
    You want to sell the same car, at a price that is $5\%$ less than that of
    seller B. What is that price?
  \end{enumerate}
\end{document}