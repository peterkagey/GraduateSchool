\documentclass{article}

\usepackage[margin=1in]{geometry}
\usepackage{amsmath,amsthm,amssymb}
\usepackage{bbm, enumerate, tikz}
\usepackage{multicol}

\newenvironment{problem}[2][Problem]{\begin{trivlist}
\item[\hskip \labelsep {\bfseries #1}\hskip \labelsep {\bfseries #2.}]}{\end{trivlist}}
\newenvironment{note}[1][Note.]{\begin{trivlist}
\item[\hskip \labelsep {\bfseries #1}]}{\end{trivlist}}

\begin{document}

\title{Fall 2015: Complex Analysis Graduate Exam}
\author{Peter Kagey}

\maketitle

% -----------------------------------------------------
% First problems
% -----------------------------------------------------
\begin{problem}{1}
  Evaluate the integral \[
    \int_0^\infty \frac{\sin^2 x}{x^2}\, dx,
  \] being careful to justify your answer.
\end{problem}

\begin{proof}
\end{proof}

% -----------------------------------------------------
% Second problem
% -----------------------------------------------------
\pagebreak

\begin{problem}{2}
  Determine the number of roots of $f(z) = z^9 + z^6 + z^5 + 8z^3 + 1$ inside the annulus $1 < |z| < 2$.
\end{problem}

\begin{proof}
  By Rouch\'e's Theorem, if $g$ is analytic and $|f - g| < |f|$ along a simple
  curve, then both $f$ and $g$ have the same number of zeros inside this region.
\\~\\
  \textbf{Case 1: $|z| = 2$} First, consider the curve $|z| = 2$, and the
  function $g = z^9$. Now along this curve, by the triangle inequality \begin{align*}
    |f(z) - g(z)| &= |z^6 + z^5 + 8z^3 + 1| \\
    &\leq |z^6| + |z^5| + |8z^3| + |1| \\
    &= 2^6 + 2^5 + 8(2^3) + 1 \\
    &= 161.
\end{align*}
Similarly by the triangle inequality \begin{align*}
  |f(z)| &= |z^9 + z^6 + z^5 + 8z^3 + 1| \\
  &\geq |z^9| - |z^6| - |z^5| - |8z^3| - |1| \\
  &= 2^9 - 161 \\
  &= 351.
\end{align*}
Thus $|f-g| \leq 161 < 351 \leq |f|$, and $f$ has the same number of zeros as
$g$ when $|z| < 2$. Clearly $g$ has all nine zeros inside this region at $z = 0$,
so $f$ has nine zeros in $|z| < 2$
\\~\\
\textbf{Case 2: $|z| = 1$} Now, consider the curve $|z| = 1$, and the
function $g = 8z^3$. Along this curve, by the triangle inequality \begin{align*}
  |f(z) - g(z)| &= |z^9 + z^6 + z^5 + 1| \\
  &\leq |z^9| + |z^6| + |z^5| + |1| \\
  &= 4.
\end{align*}
Similarly by the triangle inequality \begin{align*}
|f(z)| &= |z^9 + z^6 + z^5 + 8z^3 + 1| \\
&> |8z^3| - |z^9| - |z^6| - |z^5| - |1| \\
&= 4.
\end{align*} ($^*$ I'm not sure of a nice way to show that this inequality is strict.)\\
Thus $|f-g| \leq 4 < |f|$, and $f$ has the same number of zeros as
$g$ when $|z| < 1$. Clearly $g$ has three zeros inside this region at $z = 0$,
so $f$ has three zeros in $|z| < 1$, and by the inclusion-exclusion principle,
$f$ has $9 - 3 = 6$ zeros inside the annulus $1 < |z| < 2$.
\end{proof}
% -----------------------------------------------------
% Third problem
% -----------------------------------------------------
\pagebreak

\begin{problem}{3}
  Suppose that $f$ is holomorphic on the open unit disk
  $\mathbb{D}=\{ z \in \mathbb{C} : |z| < 1 \}$
  and suppose that for $z \in \mathbb{D}$, one has $\mathfrak{R}(f(z)) > 0$ and
  $f(0) = 1$. Prove that \[
    |f(z)| \leq \frac{1 + |z|}{1 - |z|}
  \]  for all $z \in \mathbb{D}$.
\end{problem}

\begin{proof}
\end{proof}

% -----------------------------------------------------
% Fourth problem
% -----------------------------------------------------
\pagebreak

\begin{problem}{4}
  For $a_n = 1 - \frac{1}{n^2}$, let \[
    f(z) = \prod_{n=1}^\infty \frac{a_n - z}{1 - a_n z}.
  \]
  \begin{enumerate}[(a)]
    \item Show that $f$ defines a holomorphic function on the unit disk
      $\mathbb{D} = \{ z \in \mathbb{C} : |z| < 1 \}$.
    \item Prove that $f$ does not have an analytic continuation to any larger
      disk $\{ z \in \mathbb{C} : |z| < r \}$ for some $r > 1$.
  \end{enumerate}
\end{problem}

\begin{proof} $ $
  \begin{enumerate}[(a)]
    \item
    \item Suppose that $f$ did have an analytic continuation on
      $B_{1 + \varepsilon}(0)$. Then by continuity, \[
        f(1) = \lim_{z \rightarrow 1} f(z) = \lim_{n \rightarrow \infty} f(1 - 1/n^2) = 0,
      \] because $f$ has zeros at $a_n$.
      However, $f$ has poles at $n^2/(n^2 - 1)$, so taking the limit from the
      right along the real axis cannot equal zero because there are arbitrarily
      large values to the right of $1$. This is a contradition, so $f$ cannot
      have an analytic continuation on $B_{1+\varepsilon}(0)$.
  \end{enumerate}
\end{proof}

\end{document}
