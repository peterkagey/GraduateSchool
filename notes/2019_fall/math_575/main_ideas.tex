\documentclass{article}

\usepackage[margin=1in]{geometry}
\usepackage{amsmath,amsthm,amssymb}
\usepackage{bbm, enumerate}

\newenvironment{problem}[2][Problem]{\begin{trivlist}
\item[\hskip \labelsep {\bfseries #1}\hskip \labelsep {\bfseries #2.}]}{\end{trivlist}}
\newenvironment{note}[1][Note.]{\begin{trivlist}
\item[\hskip \labelsep {\bfseries #1}]}{\end{trivlist}}

\begin{document}

\title{Matrix Analysis: Main ideas}
\author{Peter Kagey}

\maketitle

\section{Definitions}
\subsection{Companion matrix}
A \textbf{companion matrix} to a monic polynomial \[
  p(x) = a_0 + a_1x + \dots + a_{n-1}x^{n-1} + x^n
\] is the $n \times n$ square matrix \[
  \begin{bmatrix}
    0 & 0 & \hdots & 0 & -a_0 \\
    1 & 0 & \hdots & 0 & -a_1 \\
    0 & 1 & \hdots & 0 & -a_2 \\
    \vdots & \vdots & \ddots & \ddots & \vdots \\
    0 & 0 & \hdots & 1 & -a_{n-1}
  \end{bmatrix}
\] which has the property that its characteristic and minimal polynomials are $p(x)$.

\subsection{Cyclic matrix}
An $n \times n$ matrix over a field $\mathbb F$ is called a
\textbf{cyclic matrix} if there exists a vector
$\vec v$ such that $\{\vec v, A\vec v, \dots, A^{n-1}\vec v\}$ is a basis for
$\mathbb F^n$.

\subsection{Normal matrix}
A matrix $A$ is called \textbf{normal} if it commutes with its conjugate
transpose, that is $A^*A = AA^*$.

\subsection{Tensor product}
...

\subsection{Unitary matrix}
A matrix $U$ is called \textbf{unitary} if its conjugate transpose $U^*$ is also its
inverse, that is $U^*U = I$.

% ---------------------------
\section{Decompositions}

\subsection{Schur decomposition}
Each $A \in M_n(\mathbb C)$ can be written \[
  A = QUQ^{-1}
\] where $Q$ is unitary and $U$ is upper triangular.
\subsection{Singular Value Decomposition}
Let $M$ be an $m \times n$ complex matrix. Then $M$ can be written \[
  M = U \Sigma V^*
\] where $U$ and $V$ are unitary matrices.

\section{Theorems}
\subsection{Special cases of normal matrices}
All unitary, Hermitian, and skew-Hermitian complex matrices are normal.
All orthogonal, symmetric, and skew-symmetric real matrices are normal.
\subsection{Spectral theorem}
A matrix $A$ is normal if and only if it is unitarily similar to a diagonal
matrix, that is \[
  A = U^*DU
\] for some unitary matrix $U$.

\subsection{Hermitian matrices have real eigenvalues}

\subsection{Perron theorem}
Any positive $n \times n$ matrix has a simple eigenvalue $\lambda_1 = r$ such that
all other eigenvalues are strictly smaller than $r$. Moreover, there exists
a corresponding eigenvector $v$ such that all components are positive.
\end{document}

% 1.  Basic  notions (vector space, basis, linear independence, rank, linear transformation, etc).
%
% 2.  Classification of matrices (equivalent  linear transformations) by row equivalence (row reduced echelon form), two sided equivalence, rank + nullity theorem
%
% 3.  Basic notions of quotient spaces and tensor products.
%
% 4.  diagonalization and triangulizations of matrices (and commuting sets of matrices).
%
% 5.  eigenvalues, eigenspaces, characteristic polynomial, minimal polynomial, Cayley-Hamilton theorem
%
% 6.  Companion matrices, rational canonical form, Jordan canonical form, primary decomposition, cyclic matrices and linear transformations.
%
% 7.  Classification of quadratic forms over fields of characteristic not 2 and alternating forms in any characteristic. Sylvester's theorem.
%
% 8.  Inner product spaces, self adjoint operators,  spectral theorem for self adjoint operators and normal matrices.
%
% 9.  Positive definite Hermitian matrices.
%
% 10.  Basic properties of normed vector spaces (and the basic examples);
% the c.
%
% 11.  Perron's theorem and variations;  stochastic matrices.
%
% 12.  matrix exponential, polynomials in matrices and linear transformations, limits in normed vector spaces.