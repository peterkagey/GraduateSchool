\documentclass{article}

\usepackage[margin=1in]{geometry}
\usepackage{amsmath,amsthm,amssymb}
\usepackage{bbm,enumerate,mathtools,mathrsfs}
\usepackage[hidelinks]{hyperref}
\usepackage{tikz}
\usetikzlibrary{matrix, arrows}

\newenvironment{definition}[1][Definition.]{
  \begin{trivlist} \item[\hskip \labelsep {\bfseries #1}]
}{\end{trivlist}}

\newenvironment{example}[1][Example.]{
  \begin{trivlist} \item[\hskip \labelsep {\bfseries #1}]
}{\end{trivlist}}

\newenvironment{note}[1][Note.]{
  \begin{trivlist} \item[\hskip \labelsep {\bfseries #1}]
}{\end{trivlist}}

\newenvironment{theorem}[1][Theorem.]{
  \begin{trivlist} \item[\hskip \labelsep {\bfseries #1}]
}{\end{trivlist}}

\newenvironment{exercise}[1][Exercise.]{
  \begin{trivlist} \item[\hskip \labelsep {\bfseries #1}]
}{\end{trivlist}}

\newcommand{\set}[1]{\{ #1 \}}
\newcommand{\ang}[1]{\langle #1 \rangle}
\newcommand{\paren}[1]{\left( #1 \right)}
\newcommand{\fn}[3]{#1 \colon #2 \rightarrow #3}

\begin{document}

\title{Math 510B Notes}
\author{Peter Kagey}
\date{Monday, January 14, 2019}

\maketitle

\begin{theorem} (recall from 2019-01-11) \\
  If $D$ is a UFD then $D[x]$ is also a UFD.
\end{theorem}
\begin{theorem}[Lemma.]
  If $f$ factors in $K[x]$ then it factors in $D[x]$. Namely,
  suppose $D$ is a UFD with field of fractions $K$, and assume $f(x) \in D[X]$
  is primitive. If $f(x) = g(x)h(x) \in K[x]$ then there exists a factorization
  $f(x) = g_2(x)h_2(x)$ with $g_2, h_2 \in D[x]$, where $g(x) = \alpha g_2(x)$
  and $h(x) = \beta h_2(x)$ with $\alpha, \beta \in K$.
\end{theorem}
\begin{proof}[Proof of lemma.] The polynomials $g$ and $h$ can be written as \[
    g(x) = \sum_{i=0}^n \paren{\frac{a_i}{b_i}} x^i,
    \hspace{1cm}
    h(x) = \sum_{i=0}^m \paren{\frac{c_i}{d_i}} x^i
  \] with $a_i, b_i, c_i, d_i \in D$. Then let $b = \prod_{i=0}^n b_i$ and
  $d = \prod_{i=0}^m d_i$ so that $b\cdot g(x) = g_1(x) \in D[x]$, and
  $bd\cdot f(x) = g_1(x)h_1(x)$. Since $f$ is primitive, taking the content of
  both sides, $C(bd \cdot f) = bd \approx C(g_1)\cdot C(h_1)$.
  Thus \begin{alignat*}{2}
    bd\cdot f(x)
    &= g_1(x)h_1(x) && \\
    &= C(g_1)g_2(x)\cdot C(h_1)h_2(x) \hspace{1cm }&& \text{where $g_2$ and $h_2$ are primitive}\\
    &\approx bd g_2(x)h_2(x) && \text{where $g_2h_2$ is primitive by Gauss}
  \end{alignat*} so by the cancellation property, $f(x) = ug_2(x)h_2(x)$ where
  $u$ is a unit.
\end{proof}

\begin{proof}[Proof of theorem.]
  Choose $f(x) \in D[x]$. \\
  \textbf{Existence.} \\
  Write $f(x) = C(f)f_1(x)$ where $f_1$ is primitive with $\deg(f_1) \geq 1$,
  so that $f(x) \not\in D$. Since $D$ is a UFD, $C(f)$ can be factored in $D$,
  $C(f) = p_1\cdots p_k$.
  \\~\\
  If $f_1$ is irreducible, then $f(x)$ factors as $p_1\cdots p_k f_1(x)$.
  \\~\\
  If $f_1$ is not irreducible, then $f_1(x) = g(x)h(x)$ with the degree of
  $g$ and $h$ strictly less than $f_1$, so by induction on the degree of
  polynomials, $g_1$ and $h_1$ are products of irreducibles, so $f$ factors as
  $f(x) = p_1 \cdots p_k h_1(x) \cdots h_n(x) g_1(x) \cdots g_m(x)$
  \\~\\
  \textbf{Uniqueness.} \\
  Assume that $f$ can be factored as both \begin{align*}
    f(x) &= c_1 \cdots c_m p_1(x) \cdots p_n(x) \\
    &= d_1 \cdots d_r q_1(x) \cdots q_s(x),
  \end{align*} where $c_i, d_i$ are prime and $p_i(x), q_i(x)$ are irreducible.
  Further, without loss of generality, move the content of each irreducible
  polynomial to the coefficient. Then
  $p_1(x) \cdots p_n(x)$ and $q_1(x) \cdots q_s(x)$ are primitive so
  $c_1 \cdots c_m \approx d_1 \cdots d_r$ and $c_i \approx d_i$ after
  relabeling. Therefore
  $p_1(x) \cdots p_n(x) \approx q_1(x) \cdots q_s(x) \in D[x]$.
  Consider these terms over the field of fractions $K[x]$, then $p_i(x), q_i(x)$
  are irreducible in $K[x]$ since they're irreducible in $D[x]$ (by the lemma.)
  Then the uniqueness of factorizations in $K[x]$ implies $n=s$ and
  $p_i(x) \approx q_i(x)$ in $K[x]$.
  \\~\\
  So for all $i$, $p_i(x) = \frac{a_i}{b_i}q_i(x)$, so
  $b_i p_i(x) = a_i q_i(x)$ and thus $C(b_i p_i(x)) = C(a_i q_i(x))$ and
  $b_i \approx a_i$. Thus $\frac{a_i}{b_i} = \frac{a_i}{u a_i} = u^{-1}$, which
  is a unit in $D$. Therefore polynomial parts are unique.
\end{proof}
\begin{theorem} (Eisenstein's irreducibility criteria for UFDs) \\
  Let $D$ be a UFD then $f(x) = a_0 + \hdots + a_n x^n \in D[x]$ is irreducible
  in $K[x]$ if there exists some prime $p \in D$ such that
  \begin{enumerate}
    \item the prime divides all but the leading coefficient,
    $p \mid a_0, \hdots p \mid a_{n-1}$, but $p \nmid a_n$, and
    \item the prime divides the constant term only once, $p^2 \nmid a_0$.
  \end{enumerate}
\end{theorem}
\begin{proof}
  By Gauss's lemma, if $f$ factors in $K[x]$ it factors in $D[x]$, so assume
  that \[
    f(x) = g(x)h(x) = (b_0 + \hdots + b_k x^k)(c_0 + \hdots + c_\ell x^\ell).
  \]
  Since $a_0 = b_0c_0$ and $p^2 \nmid a_0$, either $p \nmid b_0$ or
  $p \nmid c_0$, so assume without loss of generality that $p \nmid b_0$.
  \\
  Next consider the map $\fn{\phi}{D[x]}{D/\ang p[x]}$ which reduces all
  coefficients mod $p$ \[
    \phi(f(x))
    = \overline f(x)
    = \overline a_n x^n
    = (\overline b_0 + \hdots + \overline b_k x^k)
    (\overline c_0 + \hdots + \overline c_\ell x^\ell)
  \] where $b_0 \neq 0$, so $x \nmid \overline g(x)$. This means
  $x^n \mid h(x)$, so $l = n$ and $k = 0$, and thus $\overline g$ is constant.
  Therefore $f(x)$ has only trivial factorizations.
\end{proof}
\end{document}
