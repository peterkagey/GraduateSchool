\documentclass{article}

\usepackage[margin=1in]{geometry}
\usepackage{amsmath,amsthm,amssymb}
\usepackage{bbm,enumerate,mathtools,mathrsfs}
\usepackage[hidelinks]{hyperref}
\usepackage{tikz}
\usetikzlibrary{matrix, arrows}

\newenvironment{definition}[1][Definition.]{
  \begin{trivlist} \item[\hskip \labelsep {\bfseries #1}]
}{\end{trivlist}}

\newenvironment{example}[1][Example.]{
  \begin{trivlist} \item[\hskip \labelsep {\bfseries #1}]
}{\end{trivlist}}

\newenvironment{note}[1][Note.]{
  \begin{trivlist} \item[\hskip \labelsep {\bfseries #1}]
}{\end{trivlist}}

\newenvironment{theorem}[1][Theorem.]{
  \begin{trivlist} \item[\hskip \labelsep {\bfseries #1}]
}{\end{trivlist}}

\newenvironment{exercise}[1][Exercise.]{
  \begin{trivlist} \item[\hskip \labelsep {\bfseries #1}]
}{\end{trivlist}}

\newcommand{\set}[1]{\{ #1 \}}
\newcommand{\ang}[1]{\langle #1 \rangle}
\newcommand{\fn}[3]{#1 \colon #2 \rightarrow #3}

\begin{document}

\title{Math 510B Notes}
\author{Peter Kagey}

\maketitle

\begin{definition}
  Let $R$ be a commutative domain with unity. Then $R$ is called Euclidean if it
  has a ``division algorithm''. This is, there exists
  $\fn{\phi}{R - \set 0}{\mathbb N}$ satisfying \begin{enumerate}
    \item $\phi(a) \leq \phi(ab)$ if $ab \neq 0$, and
    \item $a = qb + r$ with $\phi(r) < \phi(b)$ for some $q, r \in R$ if $a, b \neq 0$.
  \end{enumerate}
\end{definition}

\begin{example}[Examples.] $ $
  \begin{enumerate}
    \item If $R = \mathbb Z$, then $\phi(a) = |a|$.
    \item If $R = k[x]$, then $\phi(f) = \deg(f)$
  \end{enumerate}
\end{example}

\begin{theorem}[Lemma.]
  If $R$ is Euclidean then $R$ is a PID.
\end{theorem}
\begin{proof}
  Need to show any ideal $I \subset R$ is principal.
  First, if $I = \ang 0$, we're done. Otherwise $I$ contains a nonzero element.
  Pick such an element $b \neq 0$ such that $\phi(a)$ is minimal. If $a$ is
  another nonzero element, then $a = qb + r$ where $\phi(r) < \phi(b)$, so
  $r = 0$. Thus $b = qa \in \ang a = I$.
\end{proof}

\begin{example}
  Let $F = \mathbb Q(\sqrt m)$, and let
  $\mathcal O_F = \set{a \in F : a \text{ is integral over } \mathbb Z}$.
  \begin{enumerate}
    \item If $m \cong 2, 3 \mod 4$, then
    $\mathcal O_F = \mathbb Z \oplus \mathbb Z(\sqrt m)$.
    \item if $m \cong 1 \mod 4$, then
    $\mathcal O_F = \mathbb Z \oplus \mathbb Z(1/2 + \sqrt m / 2)$.
  \end{enumerate}
\end{example}
\begin{note}
  An element $a \in \mathbb Q(\sqrt m)$ is integral over $\mathbb Z$ if there
  exists $\alpha_i \in \mathbb Z$ such that
  $a^k + \alpha_{k-1}a^{k-1} + \hdots + \alpha_0 = 0$
\end{note}
\begin{note}
  A048981 gives the twenty one values of $m$ such that $\mathcal O_F$ is Euclidean.
\end{note}
\begin{theorem}[Lemma.]
  Let $R$ be a PID, then greatest common divisors exist, and given $a, b \neq 0$ and
  $d = \gcd(a, b)$ (...?)
\end{theorem}
% TODO: Write proof.
\begin{proof} Omitted.
\end{proof}
\begin{theorem}[Corollary]
  If $R$ is Euclidean is it a PID, so it has greatest common divisors as usual.
\end{theorem}
\begin{theorem}
  Let $R$ be an integral domain. Then $R$ is a UFD if and only if
  \begin{enumerate}[(a)]
    \item $R$ has an ascending chain condition on principal ideals.
    (That is, every chain $\ang{a_1} \subseteq \ang{a_2} \subseteq \hdots$
    is eventually constant.)
    \item Irreducible elements are prime. (i.e. if $p|ab$ then $p|a$ or $p|b$.)
  \end{enumerate}
\end{theorem}
\begin{proof} $ $ \\
  ($\Longrightarrow$) Assume $R$ is a UFD.\\
  \textbf{Proof of (a).} First note that for any $a, b\in R$,
  $\ang a \subseteq \ang b$ if and only if $b | a$. So suppose there is a chain
  of principal ideals $\ang{a_1} \subseteq \ang{a_2} \subseteq \hdots$;
  since $a_{i+1} | a_i$, we can write $a_{i+1} = p_1\hdots p_n$ and write
  $a_i = up_{j_1}\hdots p_{j_k}$ where $u$ is a unit and $k \leq n$. Therefore
  the number of prime factors of the generators weakly decreases, and so the
  chain must eventually stop or become constant.
  \\~\\
  \textbf{Proof of (b).} Assume $a$ is irreducible, and assume $a|bc$ where
  $b = p_1\cdots p_r$ and $c = q_1 \cdots q_s$; that is, there exists $x \in R$
  such that $x a = bc = p_1\cdots p_r q_1 \cdots q_s$. Since $a$ is irreducible,
  $a = up_i$ or $a = u q_i$, so either $a|b$ or $a|c$.
  \\~\\
  ($\Longleftarrow$) Assume (a) and (b).\\
  \textbf{Existence.}
  Let $\mathcal S = \set{a \in R : a \text{ is not the product of irreducible polynomials}}$.
  Then assume for the sake of contradiction that $a \in \mathcal S$ is chosen
  so that $\ang a$ is maximal among the ideals $\ang b$, which can be done by
  (1). But since $a \in \mathcal S$, $a$ is not irreducible
  (or else is could be written as the one-term product $a$) so it factors as
  $a = a_1\cdots a_k$. But since $a \in \mathcal S$ was chosen so that $\ang a$
  is maximal, and $\ang a \subset \ang{a_i}$, $a_i \not\in \mathcal S$, and so
  can be written as a product of irreducible elements, and thus $a$ can be
  written as a product of irreducible elements. Thus $a \not\in \mathcal S$ so
  $\mathcal S = \emptyset$.
  \\~\\
  \textbf{Uniqueness.}
  Say $a = q_1 \hdots q_s = p_1 \hdots p_r$ where $p_i$ and $q_i$ are
  irreducible. By (2) this means $p_i$ and $q_i$ are prime, so since $p_1 | a$,
  $p_1 | q_1 \hdots q_s \hdots q_s$. In particular, after relabeling,
  $q_1 = u_1p_1$. By the cancellation property, it follows that
  $q_2 \hdots q_s = u_1p_2 \hdots p_r$. By induction, it follows that $s=r$ and
  $q_i = u_ip_i$ for all $i$ with $u_i$ unit.
\end{proof}
\end{document}
