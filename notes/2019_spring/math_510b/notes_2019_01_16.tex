\documentclass{article}

\usepackage[margin=1in]{geometry}
\usepackage{amsmath,amsthm,amssymb}
\usepackage{bbm,enumerate,mathtools,mathrsfs}
\usepackage[hidelinks]{hyperref}
\usepackage{tikz}
\usetikzlibrary{matrix, arrows}

\newenvironment{definition}[1][Definition.]{
  \begin{trivlist} \item[\hskip \labelsep {\bfseries #1}]
}{\end{trivlist}}

\newenvironment{example}[1][Example.]{
  \begin{trivlist} \item[\hskip \labelsep {\bfseries #1}]
}{\end{trivlist}}

\newenvironment{note}[1][Note.]{
  \begin{trivlist} \item[\hskip \labelsep {\bfseries #1}]
}{\end{trivlist}}

\newenvironment{theorem}[1][Theorem.]{
  \begin{trivlist} \item[\hskip \labelsep {\bfseries #1}]
}{\end{trivlist}}

\newenvironment{exercise}[1][Exercise.]{
  \begin{trivlist} \item[\hskip \labelsep {\bfseries #1}]
}{\end{trivlist}}

\newcommand{\set}[1]{\{ #1 \}}
\newcommand{\ang}[1]{\langle #1 \rangle}
\newcommand{\paren}[1]{\left( #1 \right)}
\newcommand{\fn}[3]{#1 \colon #2 \rightarrow #3}
\newcommand{\inv}[1]{#1^{-1}}

\begin{document}

\title{Math 510B Notes}
\author{Peter Kagey}
\date{Wednesday, January 16, 2019}

\maketitle

\begin{example}[Examples.] ~
  \begin{enumerate}
    \item Let $p$ be prime in $\mathbb Z$, then $x^n - p \in \mathbb Z[x]$ is
    irreducible in $\mathbb Q[x]$.
    \item Let $f(x, y) = y^4 - xy^3 - x^2y^2 + x \in \mathbb Z[x]$. Then write
    $\mathbb Q[x, y] = D[y]$ where $D = \mathbb Q[x]$. Since $x$ is prime in
    $D$, $f(x, y)$ is irreducible in $\mathbb Q[x, y]$
    \item Let $f(x, y)$ be irreducible in $\mathbb Q[x, y]$, then
    $g(x, y, z) = z^k + f(x, y)$ is irreducible in $\mathbb Q[x, y, z]$.
    \item Let $f(x, y) = x^2 + y^2 - 1 \in \mathbb R[x, y] \cong R[x][y]$. This
    can be factored as $y^2 + (x+1)(x-1)$. Then using $x+1$ (or $x-1$) as the
    prime, $f(x, y)$ is prime by Eisenstein.
    \item Let $f(x, y, z) = z^3 + x^2 + y^2 + 1$.
    Then let $p = x^2 + y^2 - 1 \in \mathbb R[x, y]$, so $z^3 + p$ is
    irreducible.
    \item Let $g(x, y, z) = x^2z^3 - y^2z + xyz - x^2 y$. Looking at this as a
    polynomial in $z$, this factors as
    $g(x, y, z) = x^2z^3 + (xy - y^2)z - x^2y$, so this is irreducible with
    prime $y$.
  \end{enumerate}
\end{example}
\begin{example}
  Let $D$ be a UFD and $p \in D$ a prime, then $D/\ang p$ might not be a UFD.
  For example, let $D = \mathbb Z[x]$ and $p = x^2 - 10$ which is irreducible in
  $D$, and thus prime. \[
    \frac{D}{\ang p} \cong \frac{\mathbb Z[x]}{\ang{x^2 - 10}} \cong \mathbb Z(\sqrt{10})
  \] which is not a UFD; for example, $9 = 3^2 = (\sqrt{10} + 1)(\sqrt{10} - 1)$.
\end{example}
\begin{theorem}[Lemma.]
  Assume $k$ is an algebraically closed field (e.g. $\mathbb C$).
  The maximal ideals of $k[x]$ are principal, generated by $(x-a)$ for some
  $a \in k$. In particular, the ideal $M_a := \ang{x - a}$ is the kernel of
  the specialization map $\fn{s_a}{k[x]}{k}$ which sends $f(x) \mapsto f(a)$ and
  thus $M_a \mapsto \ang 0$. Thus there exists a $1-1$ correspondence between
  maximal ideals of $k[x]$ and the set $k$.
\end{theorem}
\begin{proof}
  The ring $R = k[x]$ is a PID so $M = \ang{p(x)}$ for some $p(x)$ and $M$ is
  maximal so $M$ is a prime ideal, thus $p(x)$ is irreducible in R. Since $k$ is
  closed, $p(x) = x - a$ for some $a \in k$ so $M = \ang{x - a}$.
\end{proof}
\begin{theorem} (Hilbert's Nullstellensatz over $\mathbb C$, weak form) \\
  The maximal ideals of $\mathbb C[x_1, \hdots, x_n]$ are of the form
  $M_{\overline a} = \ang{x_1 - a_1, \hdots, x_n - a_n}$ for some
  $\overline a = (a_1, \hdots, a_n)$. Thus, there exists a 1-1 correspondence
  between maximal ideals of $\mathbb C[x_1, \hdots, x_n]$ and
  $\overline a \in \mathbb C^n$, where $M_{\overline a}$ is the kernel of the
  map $f(x_1, \cdots, x_n) \mapsto f(a_1, \cdots, a_n)$.
\end{theorem}
\begin{theorem}[Sublemma.]
  Any polynomial $f$ can be written as \[
    f(\widetilde x) = f(\overline a) + \sum_{i} c_i (x_i - a_i) + \sum_{i, j} c_{ij} (x_i - a_i)(x_j - a_j) + \hdots
  \] where the expression is finite because $f$ is a polynomial.
\end{theorem}
\begin{example}
  In $\mathbb C[x,y]$, let $\overline a = (1, -2)$ and $f(x, y) = x^2 + xy + 2$.
  Then let $x = u + 1$ and $y = v - 2$, so that \begin{align*}
    f(x, y) &= (u + 1)^2 + (u + 1)(v - 2) + 2 \\
    &= 1 + v + u^2 + uv \\
    &= 1 + (y + 2) + (x-1)^2 + (x-1)(y + 2)
  \end{align*}
\end{example}
\begin{proof}
  The proof will proceed in two steps: \begin{enumerate}[(1)]
    \item The kernel $\ker(s_{\bar a}) = M_{\bar a}$, and this ideal is maximal in
    $\mathbb C[\widetilde x]$.
    \item Every maximal ideal of $\mathbb C[\widetilde x]$ is of the form
    $M_{\bar a}$  for some $\bar a \in \mathbb C^n$.
  \end{enumerate}
  \textit{Proof of (1).}
  Notice that $\fn {s_{\bar a}}{\mathbb C[\widetilde x]}{\mathbb C}$ is a
  surjective ring homomorphism, and so
  $\mathbb C[\widetilde x]/\ker(s_{\bar a}) \cong \mathbb C$ by the first
  isomorphism theorem for rings. Since the quotient is
  a field, $\ker(s_{\bar a})$ is maximal.
  Also, $M_{\bar a} \subset \ker(s_{\bar a})$ because each generator
  $s_{\bar a}(x_i - a_i) = 0$.
  \\ (...)
  % TODO: Finish this.
  \\~\\
  \textit{Proof of (2).} Let $M$ be maximal ideal so that
  $k = \mathbb C[\widetilde x]/M$ is a field, let $\fn{\pi}{\mathbb C[\widetilde x]}{K}$
  be the usual quotient $x_i \mapsto \overline x_i$, and let
  $\pi_i = \fn{\pi|_{\mathbb C[x_i]}}{\mathbb C[x_i]}{K}$
  be the restriction to functions of polynomials in $\mathbb C[x_i]$.
  \\~\\
  Notice that $\ker(\pi_i) \neq 0$ for all $i$, because otherwise $\pi_i$ is
  injective.
  \begin{note} % TODO, why is $\widetilde\phi$ injective?
    For any integral domain with fraction field $F$ any injection $\fn \phi R K$
    can be extended to an injection $\fn {\widetilde\phi} F K$ by
    $\widetilde\phi(a/b) = \phi(a)\inv\phi(b)$
  \end{note}
  (...)
  % TODO: Finish this.
\end{proof}

\begin{example}[Non-example.]
  $\mathbb R[x]/(x^2 + 1) \cong \mathbb C$, so $(x^2 + 1)$ is maximal in
  $\mathbb R[x]$, but not of the form $X=(x - a)$ because $\mathbb R$ is not
  algebraically closed.
\end{example}

\begin{note}
  Suppose that $K[x_1, \hdots, x_n]$. Then Hilbert's Nullstellensatz weak form
  runs into trouble if \begin{enumerate}
    \item $K$ is not algebraically closed, (If $\overline K$ is the closure of
    $K$, then there is a correspondence between maximal ideals in
    $K[x_1, \hdots, x_n]$ and points in $\overline K^n$.) or,
    \item if $K$ is countable. (Harder, must use more sophisticated methods.)
  \end{enumerate}
\end{note}
\end{document}
