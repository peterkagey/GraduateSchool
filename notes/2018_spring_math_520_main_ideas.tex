\documentclass{article}

\usepackage[margin=1in]{geometry}
\usepackage{amsmath,amsthm,amssymb}
\usepackage{bbm, enumerate}

\newenvironment{problem}[2][Problem]{\begin{trivlist}
\item[\hskip \labelsep {\bfseries #1}\hskip \labelsep {\bfseries #2.}]}{\end{trivlist}}
\newenvironment{note}[1][Note.]{\begin{trivlist}
\item[\hskip \labelsep {\bfseries #1}]}{\end{trivlist}}

\begin{document}

\title{Complex Analysis: Main ideas}
\author{Peter Kagey}

\maketitle

\begin{enumerate}
  \item \textbf{Cauchy-Riemann Equations}\\
    Let $f(z) = u(x, y) + iv(x, y)$ then $f$ is holomorphic if and only if $
      u_x = v_y \text{ and } u_y = -v_x.
    $\\
    \textit{Note:} To remember which sign is which, check with
    $(x + iy)^2 = (x^2 - y^2) + i(2xy)$
  \item \textbf{Cauchy's Integral Theorem}\\
    Let $f$ be a holomorphic on a simply connected open set $\Omega$, then the
    integral along any closed curve $\gamma \in \Omega$ vanishes: \[
      \oint_\gamma f\,dz = 0.
    \]
  \item \textbf{Cauchy's Integral Formula}\\
    Let $f$ be a holomorphic on a simply connected open set $\Omega$, and let
    $\gamma \in \Omega$ be a simple closed curve around $z_0$. Then $f(z_0)$ is
    uniquely determined by the values on the boundary of the curve \[
      f(z_0) = \frac{1}{2\pi i}\oint_\gamma \frac{f(\xi)}{\xi - z_0}\,d\xi,
    \] and by differentiating under the integral \[
      f^{(n)}(z_0) = \frac{n!}{2\pi i}\oint_\gamma \frac{f(z)}{(z - z_0)^{n+1}}\,dz.
    \]
  \item \textbf{Defintion: Residue}
    The residue of $f$ at $z_0$ is the coefficient of $(z - z_0)^{-1}$ in the
    Taylor expansion of $f$ about $z_0$.
  \item \textbf{Residue Theorem}\\
    Let $\Omega$ be a ``nice'' region, and $f$ is holomorphic on
    $\Omega \setminus \{z_0\}$. Then \[
      \frac{1}{2\pi i}\oint_\gamma f(z) dz = \operatorname{Res}_{z_0}(f).
    \] where $\gamma$ is a nice curve in $\Omega$.

  \item \textbf{Schwarz Lemma}\\
    Assume that $f\colon \mathbb{D} \rightarrow \mathbb{D}$ is a holomorphic map
    such that $f(0) = 0$.\\~\\
    Then $|f(z)| \leq |z|$ for all $z \in \mathbb{D}$ and $|f'(0)| \leq 1$. Also, if
    $|f(z)| = |z|$ for some $z \not= 0$ or if $|f'(0)| = 1$, then $f$ is a rotation.

  \item \textbf{Rouch\'e's Theorem}\\
    Let $\gamma$ be a curve in $\Omega$ homologous to $0$ with winding number
    at most $1$. Then if $f$ and $g$ are analytic and satisfy the inequality
    $|f - g| < |f|$ for all points on $\gamma$, $f$ and $g$ enclose the same
    number of zeroes.

  \item \textbf{Argument principle}\\
    Suppose $f$ is a meromorphic function inside and on some closed contour $C$
    with no zeros or poles on $C$, then \[
      \oint_C\frac{f'(z)}{f(z)}dz = 2\pi i(N - P)
    \] where $N$ is the number of zeros and $P$ is the number of poles inside
    $C$.

  \item \textbf{Definition: Normal Family} (Ahlfors, p. 220)\\
    A family $\mathfrak F$ is said to be normal in $\Omega$ if every sequence
    $\{f_n\}$ of functions $f_n \in \mathfrak F$ contains a subsequence which
    converges uniformly on every compact subset of $\Omega$.
\end{enumerate}

\end{document}
