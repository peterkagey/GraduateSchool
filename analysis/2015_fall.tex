\documentclass{article}

\usepackage[margin=1in]{geometry}
\usepackage{amsmath,amsthm,amssymb}
\usepackage{bbm, enumerate}

\newenvironment{problem}[2][Problem]{\begin{trivlist}
\item[\hskip \labelsep {\bfseries #1}\hskip \labelsep {\bfseries #2.}]}{\end{trivlist}}
\newenvironment{note}[1][Note.]{\begin{trivlist}
\item[\hskip \labelsep {\bfseries #1}]}{\end{trivlist}}

\begin{document}

\title{Fall 2015: Real Analysis Graduate Exam}
\author{Peter Kagey}

\maketitle

% -----------------------------------------------------
% First problem
% -----------------------------------------------------
\begin{problem}{1}
  Prove that for almost all $x \in [0, 1]$, there are at more finitely many
  rational numbers with reduced form $p/q$ such that $q \geq 2$ and
  $|x - p/q| < 1/(q \log q)^2$.
  (Hint: consider intervals of length $2/(q log q)^2$ centered at rational points $p/q$.)
\end{problem}

\begin{proof}
\end{proof}

% -----------------------------------------------------
% Second problem
% -----------------------------------------------------
\pagebreak

\begin{problem}{2}
  Suppose that the real-valued function $f(x)$ is nondecreasing on the interval $[0, 1]$.
  Prove that there exists a sequence of continuous functions $f_n(x)$ such that
  $f_n \rightarrow f$ pointwise on this interval.
\end{problem}

\begin{proof}
\end{proof}

% -----------------------------------------------------
% Third problem
% -----------------------------------------------------
\pagebreak

\begin{problem}{3}
  Let $(X, \mu)$ be a finite measure space. Assume that a sequence of integrable
  functions $f_n$ satisfies $f_n \rightarrow f$ in measure, where $f$ is
  measurable. Assume that $f_n$ satisfies the following property:
  For every $\varepsilon > 0$ there exists $\delta > 0$ such that \[
    \mu(E) \leq \delta \Longrightarrow \int_E |f_n| d\mu \leq \varepsilon.
  \]
  Prove that $f$ is integrable and that \[
    \lim_n \int_X |f_n - f| d\mu = 0.
  \]
\end{problem}

\begin{proof}
\end{proof}

% -----------------------------------------------------
% Fourth problem
% -----------------------------------------------------
\pagebreak

\begin{problem}{4}
  Consider the following two statements about a function $f: [0, 1] \rightarrow \mathbb{R}$:
  \begin{enumerate}[(i)]
    \item $f$ is continuous almost everywhere
    \item $f$ is equal to a continuous function $g$ almost everywhere.
  \end{enumerate}
  Does (i) imply (ii)? Prove or give a counterexample.
  Does (ii) imply (i)? Prove or give a counterexample.
\end{problem}

\begin{proof}
\end{proof}

\end{document}
