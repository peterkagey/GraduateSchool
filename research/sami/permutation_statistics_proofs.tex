\documentclass{article}

\usepackage[margin=1in]{geometry}
\usepackage{amsmath,amsthm,amssymb}
\usepackage{array,multirow}
\usepackage{bbm, enumerate}
\usepackage[yyyymmdd,hhmmss]{datetime}

\let\originalleft\left
\let\originalright\right
\renewcommand{\left}{\mathopen{}\mathclose\bgroup\originalleft}
\renewcommand{\right}{\aftergroup\egroup\originalright}

\newcommand{\n}[1]{\multicolumn{1}{|c|}{#1}}

\newtheorem{theo}{Theorem}[subsection]
\newtheorem{definition}[theo]{Definition}
\newtheorem{conjecture}[theo]{Conjecture}
\newtheorem{lemma}[theo]{Lemma}
\newtheorem{theorem}[theo]{Theorem}
\newtheorem{datatable}[theo]{Table}
\newtheorem{example}[theo]{Example}
\newtheorem{note}[theo]{Note}

\begin{document}

% Notes, use multline

\title{Permutation statistics}
\author{Peter Kagey}
\date{\today\ at \currenttime}

\maketitle
\section{Fixed points}
Let $F(n,m)$ denote the number of permutations $\pi \in S_n$ with exactly $m$
fixed points. Then $F(1,1) = 1$ and $F(1,m) = 0$ for $m \neq 1$.
For $n > 1$, $F$ satisfies the following recurrence relation \begin{align}
  F(n, m) =
    \underbrace{F(n - 1, m-1)}_{\text{(A)}} +
    \underbrace{(n - m - 1)F(n - 1, m)}_{\text{(B)}} +
    \underbrace{(m + 1)F(n-1, m + 1)}_{\text{(C)}}
\end{align} where \begin{enumerate}[(A)]
  \item Append an $n$ to the end of a word of length $n-1$ with $m-1$ fixed
  points. This increases both the length and the number of fixed points by one.
  \item Choose one of the $n - m - 1$ non-fixed points of a word of length $n-1$ and $m$
  fixed points, and replace it with $n$, append the chosen letter to the end.
  This increases the length by one while preserving the number of fixed points.
  \item Choose one of the $m + 1$ fixed points of a word of length $n-1$ and $m+1$
  fixed points, replace it with $n$, append the chosen letter to the end. This
  decreases the number of fixed points by one and increases the length by one.
\end{enumerate}
\subsection{Take 1 (didn't work, skip to next subsection)}
\begin{theorem}
  Let $F^{(a)}(n, m)$ denote the number of permutations $\pi \in S_n$ such that
  $\pi$ has $m$ fixed points and $\pi(1) = a$, and
  let $E[\pi_{\text{fix},1}^{n,m}]$ denote the expected value of $\pi$ for
  $\pi \in S_n$ with exactly $m$ fixed points, then for $m \neq n - 1$.
  \begin{align}
    E[\pi_{\text{fix},1}^{n,m}] := \frac{1}{F(n,m)}\sum_{a=1}^n aF^{(a)}(n, m) = \frac{n - m + 2}{2}
  \end{align}
  NOTE: Fix division by $0$ when $m = n - 1$.
\end{theorem}
\begin{proof}
  By multiplying both sides (when $m \neq n - 1$), it is equivalent to show that \begin{align}
    F(n,m)E[\pi_{\text{fix},1}^{n,m}] = \sum_{a=1}^n aF^{(a)}(n, m).
  \end{align}
  I'll start by showing that both $F^{(1)}$ and $F^{(n)}$ with $b \neq 1$
  satisfy essentially similar recurrences to $F$, namely
  \begin{align}
    F^{(1)}(n, m) &=
      \underbrace{F^{(1)}(n - 1, m-1)}_{\text{(A)}} +
      \underbrace{(n - m - 1)F^{(1)}(n - 1, m)}_{\text{(B)}} +
      \underbrace{mF^{(1)}(n-1, m + 1)}_{\text{(C')}} \\
    &=
      \underbrace{F(n-1,m-1)}_{\text{(D)}} \\
    F^{(n)}(n, m) &=
    \underbrace{F^{(n)}(n - 1, m-1)}_{\text{(A)}} +
    \underbrace{(n - m - 2)F^{(n)}(n - 1, m)}_{\text{(B')}} +
    \underbrace{(m + 1)F^{(n)}(n-1, m + 1)}_{\text{(C)}} \\
    &= \underbrace{F(n-1,m) - F^{(1)}(n-1, m)}_{\text{(E)}} + \underbrace{F^{(1)}(n-1, m+1)}_{\text{(F)}}
  \end{align} where (A) has the exact same construction as above, and
  \begin{enumerate}[(A')]
    \item[(B')]
    Choose one of the $n - m - 2$ non-fixed points
    \textit{that is not the first letter} of a word of length $n-1$ and $m$
    fixed points, and replace it with $n$, append the chosen letter to the end.
    This increases the length by one while preserving the number of fixed points.
    \item[(C')] Choose one of the $m$ fixed points
    \textit{that is not the first letter} of a word of length $n-1$ and $m+1$
    fixed points, replace it with $n$, append the chosen letter to the end. This
    decreases the number of fixed points by one and increases the length by one.
    \item[(D)] Increase every letter by $1$ and prepend a $1$, this preserves
    the number of fixed points.
    \item[(E)] Take a permutation not starting with $1$, and replace its first
    letter with $n$, and put the previous first letter at the end of the word.
    \item[(F)] Take a permutation starting with $1$, and replace its first
    letter with $n$, and put the $1$ at the end of the word.
  \end{enumerate}
  % By induction [show base case], \begin{align}
  %   F(n,m)E[\pi_{\text{fix},1}^{n,m}]
  %   % \left(
  %   %   F(n - 1, m-1) +
  %   %   (n - m - 1)F(n - 1, m) +
  %   %   (m + 1)F(n-1, m + 1)
  %   % \right)E[\pi_{\text{fix},1}^{n,m}] =
  %   &= \left(
  %     \sum_{a=1}^n F^{(a)}(n,m)
  %   \right)E[\pi_{\text{fix},1}^{n,m}] \\
  %   &= \left(
  %     F^{(1)}(n,m) + (n-1)F^{(n)}(n,m)
  %   \right)E[\pi_{\text{fix},1}^{n,m}]
  % \end{align}
  % On the right hand side \begin{align}
  %   \sum_{a=1}^n aF^{(a)}(n, m)
  %   &= F^{(1)}(n,m) + \frac{(n+2)(n-1)}{2}F^{(n)}(n,m)
  % \end{align}
  Now \begin{align}
    F(n,m)E[\pi_{\text{fix},1}^{n,m}]
    &= \sum_{a=1}^n aF^{(a)}(n, m) \\
    &= F^{(1)}(n, m) + \frac{(n+2)(n-1)}{2}F^{(n)}(n,m) \\
    &= F(n-1, m-1) + \frac{(n+2)(n-1)}{2}\left(F(n-1,m) - F^{(1)}(n-1, m) + F^{(1)}(n-1, m+1)\right) \\
    &= F(n-1, m-1) + \frac{(n+2)(n-1)}{2}\left(F(n-1,m) - F(n-2, m-1) + F(n-2, m)\right) \\
  \end{align}
  Put everything on the right hand side into terms of $F(n-2, *)$,
  \begin{align}
    \sum_{a=1}^n aF^{(a)}(n, m)
    &= F(n-1, m-1) + \frac{(n+2)(n-1)}{2}\left(F(n-1,m) - F(n-2, m-1) + F(n-2, m)\right) \\
    &= F(n - 2, m-2) + (n - m - 1)F(n - 2, m-1) + mF(n-2, m) \\
    &\hspace{1cm}+ \frac{(n+2)(n-1)}{2}\left(
      % Expanded
      % F(n-2, m-1) + (n - m - 2)F(n - 2, m) + (m + 1)F(n-2, m+1)
      % - F(n-2, m-1) + F(n-2, m)
      (n - m - 1)F(n - 2, m) + (m + 1)F(n-2, m+1)
    \right) \\
    &= F(n - 2, m-2)\\
    &\hspace{1cm}+ (n - m - 1)F(n - 2, m-1)\\
    &\hspace{1cm}+ \left(\frac{(n+2)(n-1)(n-m-1)}{2} + m\right)F(n-2, m)\\
    &\hspace{1cm}+ \left(\frac{(n+2)(n-1)(m+1)}{2}\right)F(n-2, m+1)
  \end{align}
  Rewrite $F(n,m)$ in terms of $F(n-2, *)$,
  \begin{align}
    F(n,m) &=
    F(n - 1, m-1) +
    (n - m - 1)F(n - 1, m) +
    (m + 1)F(n-1, m + 1) \\
    &= % F(n-1, m-1)
    F(n-2, m-2) + (n - m - 1)F(n - 2, m-1) + m F(n-2, m) \\
    &\hspace{1cm}+ %(n - m - 1)F(n - 1, m)
    (n-m-1)\left(F(n-2, m-1) + (n-m-2)F(n-2, m) + (m+1)F(n-2, m+1)\right)\\
    &\hspace{1cm}+ %(m + 1)F(n-1, m + 1)
    (m+1)\left(F(n-2, m) + (n-m-3)F(n-2, m+1) + (m+2)F(n-2, m+2)\right)\\
    &= % regrouping
    F(n-2, m-2) \\
    &\hspace{1cm}+ 2(n-m-1)F(n-2, m-1) \\
    &\hspace{1cm}+ (2 m + 1 + (n - m - 1)(n - m - 2))F(n-2, m) \\
    &\hspace{1cm}+ 2(1+m)(n-m-2)F(n-2, m+1) \\
    &\hspace{1cm}+ (m+1)(m+2)F(n-2, m+2)
  \end{align}
  The problem, $E[\pi_{\text{fix},1}^{n,m}] = (n - m + 2)/2$, and the algebra doesn't
  obviously work out.
\end{proof}
\subsection{Take 2}
  For $n > 1$, $F$ satisfies the following recurrence relation \begin{align}
    F^{(1)}(n, m) &= F(n-1, m-1) \\
    F^{(a)}(n, m) &=
      F^{(a)}(n-1, m-1) +
      \underbrace{(n - m - 2)F^{(a)}(n - 1, m)}_{\text{(B')}} +
      (m + 1)F^{(a)}(n-1, m + 1)
  \end{align} where the first recurrence comes from incrementing all of the letters and prepending $1$ and \begin{enumerate}[(B)]
    \item[(B')] Choose one of the $n - m - 1 - 1$ non-fixed points \textit{that is not the first letter} of a word of length $n-1$ and $m$
    fixed points, and replace it with $n$, append the chosen letter to the end.
    This increases the length by one while preserving the number of fixed points.
  \end{enumerate}
  \begin{theorem}
  \end{theorem}
  \begin{proof}
    Note that $F^{(a)}(n, m) = 0$ for $a > n$.
    \begin{align}
      F(n,m)E[\pi_{\text{fix},1}^{n,m}]
      &= \sum_{a=1}^n aF^{(a)}(n, m) \\
      &= F(n-1, m-1) \\
      &\hspace{1cm} - F^{(1)}(n-1, m-1) + \sum_{a=1}^{n-1} a F^{(a)}(n-1, m-1) \\
      &\hspace{1cm} + (n - m - 2)\left(-F^{(1)}(n-1, m)   + \sum_{a=1}^{n-1} a F^{(a)}(n-1, m)\right) \\
      &\hspace{1cm} + (m + 1)    \left(-F^{(1)}(n-1, m+1) + \sum_{a=1}^{n-1} a F^{(a)}(n-1, m+1)\right)
    \end{align}
      Now use the recurrence (from Take 1) that \begin{align}
      F^{(1)}(n, m) = F^{(1)}(n - 1, m-1) +
        (n - m - 1)F^{(1)}(n - 1, m) +
        mF^{(1)}(n-1, m + 1)
      \end{align}
    \begin{align}
      \sum_{a=1}^n aF^{(a)}(n, m)
      &= \sum_{a=1}^{n-1} a F^{(a)}(n-1, m-1) \\
      &\hspace{1cm} + (n - m - 2)\left(\sum_{a=1}^{n-1} a F^{(a)}(n-1, m)\right) + F^{(1)}(n-1, m)\\
      &\hspace{1cm} + (m + 1)\left(\sum_{a=1}^{n-1} a F^{(a)}(n-1, m+1)\right) -F^{(1)}(n-1, m+1)
    \end{align}
    By the induction hypothesis, these sums equal $F(n-1, m-1)E[\pi_{\text{fix},1}^{n-1,m-1}]$, $F(n-1, m)E[\pi_{\text{fix},1}^{n-1,m}]$, and $F(n-1, m + 1)E[\pi_{\text{fix},1}^{n-1,m + 1}]$ respectively, so
    \begin{align}
      \sum_{a=1}^n aF^{(a)}(n, m)
      &= \frac{n-m+2}{2}F(n-1, m-1) + \frac{(n-m-2)(n-m+1)}{2}F(n-1, m) \\
      &\hspace{1cm}
      + \frac{(m+1)(n-m)}{2}F(n-1, m+1)
      + F^{(1)}(n-1, m)
      - F^{(1)}(n-1, m+1)
    \end{align}
    %   &= F(n-1, m-1) - F^{(1)}(n-1, m-1) - F^{(1)}(n - 1, m) - F^{(a)}(n-1, m + 1) \\
    %   &+ F(n-1, m-1)E[\pi_{\text{fix},1}^{n-1,m-1}] +
    %     (n - m - 2)F(n - 1, m)E[\pi_{\text{fix},1}^{n-1,m}] +
    %     (m + 1)F(n-1, m+1)E[\pi_{\text{fix},1}^{n-1,m+1}]\\
    %   &= F(n-1, m-1) - F^{(1)}(n-1, m-1) - F^{(1)}(n - 1, m) - F^{(a)}(n-1, m + 1) \\
    %   &+ \frac12 (n-m+2) F(n-1, m-1) +
    %     \frac12 (n-m+1)(n-m-2)F(n - 1, m) +
    %     \frac12 (n-m)(m+1)F(n-1, m+1)\\
    % \end{align}
  \end{proof}

\subsection{Take 3} (Mostly copy/pasted from Take 2.)
  For $n > 1$, $F$ satisfies the following recurrence relation \begin{align}
    F^{(1)}(n, m) &= F(n-1, m-1) \\
    F^{(a)}(n, m) &=
      F^{(a)}(n-1, m-1) +
      \underbrace{(n - m - 2)F^{(a)}(n - 1, m)}_{\text{(B')}} +
      (m + 1)F^{(a)}(n-1, m + 1) \\
    F^{(n)}(n, m) &= F(n-1, m) - F(n-2,m-1) + F(n-2, m)
  \end{align} where the first recurrence comes from incrementing all of the letters and prepending $1$, the last comes from Take 1, and \begin{enumerate}[(B)]
    \item[(B')] Choose one of the $n - m - 1 - 1$ non-fixed points \textit{that is not the first letter} of a word of length $n-1$ and $m$
    fixed points, and replace it with $n$, append the chosen letter to the end.
    This increases the length by one while preserving the number of fixed points.
  \end{enumerate}
  \begin{theorem}
  \end{theorem}
  \begin{proof}
    Note that $F^{(a)}(n, m) = 0$ for $a > n$.
    \begin{equation}
      \begin{split}
        \sum_{a=1}^n aF^{(a)}(n, m) =
        F(n-1, m-1) &-F^{(1)}(n-1, m-1) + \sum_{a=1}^{n-1} a F^{(a)}(n-1, m-1) \\
        + (n - m - 2)&\left(-F^{(1)}(n-1, m) + \sum_{a=1}^{n-1} a F^{(a)}(n-1, m)\right) \\
        + (m + 1)&\left(-F^{(1)}(n-1, m+1) + \sum_{a=1}^{n-1} a F^{(a)}(n-1, m+1)\right)
      \end{split}
    \end{equation}
      Now use the recurrence (from Take 1) that \begin{align}
      F^{(1)}(n, m) = F^{(1)}(n - 1, m-1) +
        (n - m - 1)F^{(1)}(n - 1, m) +
        mF^{(1)}(n-1, m + 1)
      \end{align}
    \begin{align}
      \sum_{a=1}^n aF^{(a)}(n, m)
      &= \sum_{a=1}^{n-1} a F^{(a)}(n-1, m-1) \\
      &\hspace{1cm} + (n - m - 2)\left(\sum_{a=1}^{n-1} a F^{(a)}(n-1, m)\right) + F^{(1)}(n-1, m)\\
      &\hspace{1cm} + (m + 1)\left(\sum_{a=1}^{n-1} a F^{(a)}(n-1, m+1)\right) -F^{(1)}(n-1, m+1)
    \end{align}
    By the induction hypothesis, these sums equal $F(n-1, m-1)E[\pi_{\text{fix},1}^{n-1,m-1}]$, $F(n-1, m)E[\pi_{\text{fix},1}^{n-1,m}]$, and $F(n-1, m + 1)E[\pi_{\text{fix},1}^{n-1,m + 1}]$ respectively, so
    \begin{align}
      \sum_{a=1}^n aF^{(a)}(n, m)
      &= \frac{n-m+2}{2}F(n-1, m-1) + \frac{(n-m-2)(n-m+1)}{2}F(n-1, m) \\
      &\hspace{1cm}
      + \frac{(m+1)(n-m)}{2}F(n-1, m+1)
      + F^{(1)}(n-1, m)
      - F^{(1)}(n-1, m+1)
    \end{align}
    %   &= F(n-1, m-1) - F^{(1)}(n-1, m-1) - F^{(1)}(n - 1, m) - F^{(a)}(n-1, m + 1) \\
    %   &+ F(n-1, m-1)E[\pi_{\text{fix},1}^{n-1,m-1}] +
    %     (n - m - 2)F(n - 1, m)E[\pi_{\text{fix},1}^{n-1,m}] +
    %     (m + 1)F(n-1, m+1)E[\pi_{\text{fix},1}^{n-1,m+1}]\\
    %   &= F(n-1, m-1) - F^{(1)}(n-1, m-1) - F^{(1)}(n - 1, m) - F^{(a)}(n-1, m + 1) \\
    %   &+ \frac12 (n-m+2) F(n-1, m-1) +
    %     \frac12 (n-m+1)(n-m-2)F(n - 1, m) +
    %     \frac12 (n-m)(m+1)F(n-1, m+1)\\
    % \end{align}
  \end{proof}
\pagebreak
\section{Descents}
\subsection{1-Descents}
  Let $D(n, m)$ denote the number of permutations on $S_n$ with $m$ descents,
  and let $D^{(a)}(n, m)$ denote the number of permutations starting with the
  letter $a$ on $S_n$ with $m$ descents.
  Then the important recurrence relations hold for $n > 1$: \begin{align}
    D(n, m)
    = \underbrace{(n - m)D(n-1, m-1)}_\text{(A)}
    + \underbrace{(m + 1)D(n-1, m)}_\text{(B)}
  \end{align}
  where \begin{enumerate}[(A)]
    \item Increase the number of descents by one by appending $n$ to the
    beginning or any non-descent. There are $n - 1 - m + 1$ such positions.
    \item Preserve the number of descents by appending $n$ between any descent,
    or at the end. There are $m + 1$ ways of doing this.
  \end{enumerate}
  Similarly, for $n > 1$: \begin{align}
    D^{(n)}(n, m) &= D(n - 1, m - 1) \\
    D^{(a)}(n, m)
    &= \underbrace{(n - m - 1)D^{(a)}(n - 1, m - 1)}_\text{(C)}
    + \underbrace{(m + 1)    D^{(a)}(n - 1, m)    }_\text{(D)} \text{ for } a \in [n-1].
  \end{align} where the first recurrence comes from prepending $n$
  which increments the number of descents, and \begin{enumerate}
    \item[(C)] Increase the number of descents by placing $n$ between
    any non-descent. There are $n - 1 - m$ such positions.
    \item[(D)] Preserve the number of descents by appending $n$ between
    any descent, or at the end. There are $m + 1$ ways of doing this.
  \end{enumerate}
  \begin{definition}
    Let $E[\pi_\text{des}^{m, n}]$ denote the expected value of the first letter of a permutation $\pi \in S_n$ with $m$ descents. That is \begin{align}
      E[\pi_\text{des}^{m, n}] = \frac{\sum_{a=1}^n a D^{(a)}(n, m)}{D(n, m)}.
    \end{align}
  \end{definition}
  \begin{theorem}
    The expected value of the first letter of a permutation $\pi \in S_n$ with $m$ descents is $m + 1$ \begin{align}
      E[\pi_\text{des}^{m, n}] =  m + 1
    \end{align} for $m \in \{0, 1, 2, \dots, n - 1\}$.
  \end{theorem}
  \begin{proof}
    By induction on $n$ with induction hypothesis $
      D(n, m)E[\pi_\text{des}^{m, n}] = \sum_{a=1}^n a D^{(a)}(n, m),
    $ and base case clear for $n = 2$.
    \begin{align}
      \sum_{a=1}^n a D^{(a)}(n, m) &= n D^{(n)}(n, m) + \sum_{a=1}^{n-1} a D^{(a)}(n, m) \\
      &= n D(n-1, m-1) + \sum_{a=1}^{n-1} a\left((n-m-1)D^{(a)}(n-1, m-1) + (m + 1)D^{(a)}(n - 1, m) \right)\\
      &= n D(n-1, m-1)
        + (n-m-1) \sum_{a=1}^{n-1} aD^{(a)}(n-1, m-1)
        + (m + 1) \sum_{a=1}^{n-1} aD^{(a)}(n - 1, m)\\
      \end{align}
      By the induction hypothesis, these sums are $D(n-1,m-1)E[\pi_\text{des}^{m-1,n-1}] = mD(n-1,m-1)$ and $D(n-1,m)E[\pi_\text{des}^{m,n-1}] = (m + 1)D(n-1,m)$ respectively, so
      \begin{align}
      % &= n D(n-1, m-1) + (n - m - 1)mD(n-1,m-1) + (m+1)^2D(n-1,m) \\
      \sum_{a=1}^n a D^{(a)}(n, m)
      &= \underbrace{(mn - m^2 - m - n)}_{(n-m)(m+1)}D(n-1,m-1) + (m+1)^2D(n-1,m)
      \\
      &= (m + 1)
      % \underbrace{
        \left((n-m)D(n-1,m-1) + (m+1)D(n-1,m)\right)
      % }_{D(n,m)}
      \\
      &= (m + 1)D(n, m).
    \end{align}
    Thus it directly follows that \begin{align}
      E[\pi_\text{des}^{m, n}] = \frac{\sum_{a=1}^n a D^{(a)}(n, m)}{D(n, m)} = \frac{(m + 1) D(n, m)}{D(n, m)} = m + 1.
    \end{align}
  \end{proof}
  \section{Cycles}
  \begin{definition}
    Let $\operatorname{cyc}_k(\pi)$ denote the number of $k$-cycles in $\pi$.
  \end{definition}
  \begin{definition}
    Let $C_k(n,m)$ be the number of permutations $\pi \in S_n$ such that
    $\operatorname{cyc}_k(\pi) = m$.
  \end{definition}
  \begin{lemma}
    \label{cyc1Recurrence}
    $C_k^{(1)}(n,m) = C_k(n-1, m)$ for all $k \geq 2$.
  \end{lemma}
  \begin{proof}
    Writing $\pi$ as a word, consider the map
    $\pi_1\pi_2\dots\pi_n \mapsto (\pi_2 - 1)\dots(\pi_n - 1)$. Since
    $\pi_1 = 1$, the inverse map is clear.
  \end{proof}
  \begin{lemma}
    \label{allSame}
    $C_k^{(2)}(n,m) = \cdots = C_k^{(n)}(n,m)$.
  \end{lemma}
  \begin{proof}
    It is enough to show that $C_k^{(a)}(n,m) = C_k^{(b)}(n,m)$ for all
    $a, b > 1$. Since the permutations under consideration do not fix $1$,
    conjugation by $(ab)$ is an isomorphism which takes all words starting
    with $a$ to words starting with $b$ without changing the cycle structure.
  \end{proof}
  \begin{lemma}
    \label{cycRecurrenceWithFixedBeginning}
    For all $2 \leq a \leq n$, \begin{align}
      C_k^{(a)}(n,m) = \frac{C_k(n, m) - C_k(n-1, m)}{n - 1}.
    \end{align}
  \end{lemma}
  \begin{proof}
    Since \begin{align}
      C_k(n, m) = C_k^{(1)}(n, m) + C_k^{(2)}(n, m) + \dots + C_k^{(n)}(n, m)
    \end{align} using Lemma \ref{allSame}, for all values $2 \leq a \leq n$,
    this can be rewritten as \begin{align}
      C_k(n, m) = C_k^{(1)}(n, m) + (n-1)C_k^{(a)}(n, m)
    \end{align} solving for $C_k^{(a)}(n, m)$ and using the substitution from Lemma
    \ref{cyc1Recurrence} gives the desired result: \begin{align}
      C_k^{(a)}(n, m) = \frac{C_k(n, m) - C_k(n-1, m)}{n - 1}.
    \end{align}
  \end{proof}
  \begin{note}
    \label{basecase}
    It appears that $C_k(n,0)$ is given by the expansion of the exponential 
    generating function \begin{equation}
      \frac{\exp(-x^k/k)}{(1-x)},
    \end{equation} and moreover, it appears that \begin{equation}
      C_k(n,0) = \sum_{i=0}^{\lfloor n/k \rfloor} \frac{n! (-1)^i}{i!\,k^i} = A122974(n,k).
    \end{equation} These appear in the OEIS as
    \begin{align}
      C_1(n,0) &= A000166(n) \\
      C_2(n,0) &= A000266(n) \\
      C_3(n,0) &= A000090(n) \\
      C_4(n,0) &= A000138(n) \\
      C_5(n,0) &= A060725(n) \\
      C_6(n,0) &= A060726(n)
    \end{align}
  \end{note}
  \begin{theorem}
    \label{cycRecurrence}
    For all $k > 0, m > 0$ \begin{equation}
      mC_k(n, m) = (k-1)!\binom{n}{k}C_k(n-k, m-1).
    \end{equation}
  \end{theorem}
  \begin{proof}
    As an abuse of notation, let $C_k(n, m) = \{ \pi \in S_n \mid \operatorname{cyc}_k(\pi) = m\}$.
    Then consider the two sets, whose cardinalities match the left- and
    right-hand sides of the equation above:
    \begin{align}
      S^{L}_{n,m,k} &= \{ (\pi, c) \mid \pi \in C_k(n, m), c \text{ a distinguished } k\text{-cycle of } \pi \} \\
      S^{R}_{n,m,k} &= \{ (\sigma, d) \mid \pi \in C_k(n-m, m-1), d \text{ an } n\text{-ary necklace of length } k\}
    \end{align}
    The first set, $S^{L}_{n,m,k}$, is constructed by taking a permutation in
    $C_k(n,m)$ and choosing one of its $m$ $k$-cycles to be distinguished, so
    $S^{L}_{n,m,k} = mC_k(n,m)$.

    In second set, $S^{R}_{n,m,k}$, the two parts of the tuple are independent.
    There are $C_k(n-k, m-1)$ choices for $\sigma$ and $(k-1)!\binom{n}{k}$
    choices for $d$. Thus $S^{R}_{n,m,k} = (k-1)!\binom{n}{k}C_k(n-k, m-1)$.

    Now, consider the map $\phi \colon S^{L}_{n,m,k} \rightarrow S^{R}_{n,m,k}$
    which in cycle notation does the following \begin{equation}
      (\pi_1\pi_2 \dots \pi_\ell, \pi_1) \mapsto (\pi'_2 \dots \pi'_\ell, \pi_1)
    \end{equation} where $\pi'_i$ is $\pi_i$ after relabeling.

    By construction, $\sigma$ has one fewer $k$-cycle and $k$ fewer letters
    than $\pi$.
  \end{proof}
  \begin{example}
    Suppose $\pi = (18)\mathbf{(37)}(254)$ in cycle notation with $(37)$ distinguished.
    Then \begin{align}
      ((18)(37)(254), (37)) \mapsto ((16)(243), (37))
    \end{align} under this bijection.
  \end{example}
  \begin{theorem}
    \label{expectedValueCyc2}
    For $k > 1$, the expected value of a permutation $\pi \in S_n$ with $m$ $k$-cycles is
    given by \begin{align}
      E_{n,m}^{\text{cyc}_k} = \frac n2\left(1 - \frac{C_k(n-1,m)}{C_k(n,m)}\right) + 1.
    \end{align}
  \end{theorem}
  \begin{proof}
    By definition, \begin{align}
      E_{n,m}^{\text{cyc}_k} = \frac{
        \displaystyle \sum_{a = 1}^n aC_k^{(a)}(n, m)
      }{
        C_k(n, m)
      }.
    \end{align} Using Lemma \ref{allSame}, we can consolidate all but the first term of
    the numerator \begin{align}
      \sum_{a = 1}^n aC_k^{(a)}(n, m) &=
      C_k^{(1)}(n,m) +
      \sum_{a = 2}^n aC_k^{(n)}(n, m) \\
      &= C_k^{(1)}(n,m) + C_k^{(n)}(n, m)\sum_{a = 2}^n a \\
      &= C_k^{(1)}(n,m) + \frac{(n-1)(n+2)}{2} C_k^{(n)}(n, m) \\
    \end{align}
    Now using the recurrences in Lemmas \ref{cyc1Recurrence} and
    \ref{cycRecurrenceWithFixedBeginning} \begin{align}
      \sum_{a = 1}^n aC_k^{(a)}(n, m) &=
      C_k(n-1,m) + \frac{(n-1)(n+2)}{2}\left(
        \frac{C_k(n, m) - C_k(n-1, m)}{n - 1}
      \right) \\
      &= \left(\frac{n}{2} + 1\right) C_k(n,m) - \frac n2C_k(n-1,m).
    \end{align}
    Lastly, dividing by the numerator yields the result \begin{align}
      E_{n,m}^{\text{cyc}_k}
      = \frac{
        \left(\frac{n}{2} + 1\right) C_k(n,m) - \frac n2C_k(n-1,m)
      }{
        C_k(n, m)
      }
      = \frac n2\left(1 - \frac{C_k(n-1,m)}{C_k(n,m)}\right) + 1.
    \end{align}
  \end{proof}
  \begin{note}
    Why does this proof fail when $k = 1$?
  \end{note}
  \begin{note}
    In some sense, this theorem is all we could hope for, since between 
    Note \ref{basecase} and Theorem \ref{cycRecurrence} 
    this recurrence is easy to compute.
  \end{note}
  \subsection{$2$-cycles}
  \begin{note}
    $C_2(n,m) = A114320(n, m)$
  \end{note}
  \begin{definition}
    $A161936(n)$ gives the number of direct isometries that are derangements 
    of the $(n-1)$-dimensional facets of an $n$-cube.
    (A direct isometry is a rotation of the hypercube.)
  \end{definition}
  \begin{definition}
    $A000354(n)$ gives the expansion of the exponential generating function \begin{equation}
      \frac{\exp(-x)}{1 - 2x}.
    \end{equation} which is also the number of $(n-1)$-dimensional facet derangements for 
    the $n$-dimensional hypercube.
  \end{definition}
  \begin{note}
    Heuristically, about half of the isometries which derange faces should be
    direct isometries, and the other half should involve a reflection.
  \end{note}
  \begin{conjecture}
    \begin{align}
      E_{n,m}^{\text{cyc}_2} = \begin{cases}
        \displaystyle\frac{n+1}{2} + \frac{(-1)^{n/2-m}}{2 A000354(\frac n2 - m)} & n \text{ is even}\\
        \displaystyle\frac{n+1}{2} & \text{otherwise}
      \end{cases}.
    \end{align}
  \end{conjecture}
  \begin{conjecture}
    \begin{align}
      E_{n,m}^{\text{cyc}_2} = \begin{cases}
        \displaystyle\frac n 2 + \frac{A161936(\frac n2-m)}{A000354(\frac n2-m)} & n \text{ is even}\\
        \displaystyle\frac n 2 + \frac 1 2 & \text{otherwise}
      \end{cases}
    \end{align} extending the domain of A161936 so that $A161936(0) := 1$.
  \end{conjecture}
  \begin{conjecture}
    This is equivalent to the conjecture that \begin{equation}
      \frac{A161936(n-m)}{A000354(n-m)} + n\frac{C_2(2n-1,m)}{C_2(2n,m)} = 1
    \end{equation} and \begin{equation}
      (2n+1)C_2(2n,m) = C_2(2n+1,m).
    \end{equation}
  \end{conjecture}
  \begin{note}
    The ``odd'' part of the conjecture follows from Theorem
    \ref{expectedValueCyc2} and the fact that $nC_2(n-1,m) = C_2(n,m)$ when $n$ is odd.
    However, I don't know how to show the latter part, and ideally I'd like a bijective proof.
    % the latter of which itself follows from an essentially similar proof to
    % the one above (distinguishing $1$-cycles instead of $2$-cycles.)
  \end{note}
  \subsection{$3$-cycles}
  \begin{conjecture}
    \begin{equation}
      E_{n,m}^{\text{cyc}_3} = \begin{cases}
        \displaystyle\frac{n+1}{2} + \frac{(-1)^{n/3-m}}{2 A000180(\frac n3 - m)} & 3 \mid n \\
        \displaystyle\frac{n+1}{2} & \text{otherwise}
      \end{cases}.
    \end{equation}
  \end{conjecture}
  \begin{note}
    $A000180$ begins $1, 2, 13, 116, 1393, 20894, 376093, 7897952, 189550849, \dots$
  \end{note}
  \begin{note}
    Does this relate to wreath products the way the $2$-cycle version relates 
    to hypercube derangements?
  \end{note}
  \subsection{$k$-cycles}
  \begin{conjecture}
    For $k > 1$,
    \begin{equation}
      E_{n,m}^{\text{cyc}_k} = \begin{cases}
        \displaystyle\frac{n+1}{2} + \frac{(-1)^{n/k-m}}{2 A320032(\frac nk - m, k)} & k \mid n \\
        \displaystyle\frac{n+1}{2} & \text{otherwise}
      \end{cases},
    \end{equation} where $A320032(n,k)$ is the expansion of the exponential 
    generating function \begin{equation}
      \frac{\exp(-x)}{1 - kx}.
    \end{equation}
  \end{conjecture}
  \begin{note}
    These conjectures allow us to compute $C_k^{(1)}(n,k)$ and $C_k^{(\alpha)}(n,k)$ for $\alpha \neq 1$.
  \end{note}
\end{document}
